\documentclass{beamer}

% \usepackage{beamerthemesplit} // Activate for custom appearance

\title[Crossover w/ 3-species]{BEC-BCS Crossover with the Feshbach Resonance for a Three-Hyperfine-Species Model}
\author[Guojun Zhu]{Guojun Zhu}
\institute{University of Illinois at Urbana-Champaign}
\date{Dec. 15th, 2011}
%\usetheme[language=english,
%titlepagelogo=logopolito, bullet=circle, pageofpages=of, titleline=true, color=blue
%]{TorinoTh}
%\rel{Anthony J. Leggett}
%\usetheme{PaloAlto}
\usetheme{Hannover}
\usecolortheme{whale}
\AtBeginSection[]
{
  \begin{frame}
    \frametitle{Table of Contents}
    \tableofcontents[currentsection]
  \end{frame}
}
\begin{document}

\frame{\titlepage}

\section[Outline]{}
\frame{\tableofcontents}

\section{Introduction}
\subsection{Overview of the Beamer Class}
\frame
{
  \frametitle{Features of the Beamer Class}

  \begin{itemize}
  \item<1-> Normal LaTeX class.
  \item<2-> Easy overlays.
  \item<3-> No external programs needed.      
  \end{itemize}
}

\begin{frame}
        \frametitle{`Hidden higher-order concepts?'}
        \begin{itemize}[<+->]
        \item The truths of arithmetic which are independent of PA in some 
        sense themselves `{contain} essentially {\color{blue}{hidden higher-order}},
         or infinitary, concepts'???
        \item `Truths in the language of arithmetic which \ldots
        \item That suggests stronger version of Isaacson's thesis. 
        \end{itemize}
\end{frame}
\section{Mean-field result}
\frame{}
\section{Excitation Mode}
\frame{}
\section{Conclusion}
\frame{}
\end{document}
