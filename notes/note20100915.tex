\subsection{Single channel problem}
From \eef{eq:20100909:fullgap}, the single-channel equation is 
\begin{equation}
\frac{2F_{\vk}}{\sqrt{1-4 F_{\vk}^2}} \xi^{ab}_{\vk}+U_{\vk\vk'}F_{\vk'}=0
\end{equation}
The factor $\nth{\sqrt{1-4 F_{\vk}^2}}$ gives the many-body correction comparing to two-body \sch equation.  This factor is significant when $F_{\vk}$ is large.  In BCS, this is near Fermi surface;  in BEC, all $F_{\vk}$ is small, and therefore, the equation is closer to \sch  equation.  This is understandable because that there are much less overlapping for BEC, where molecule (overlapping) is small.  

In each equation of \eef{eq:20100909:fullgap},  the first term clearly involves the pauli exclusion of the channel itself as well as from the other channel;  the second terms seems to be in higher order due to statistics;  the third term is the normal inner-channel interaction; and the last term is the inter-channel interaction.  In the first term of the second equation,  $F_{\vk}$ should be larger than $G_{\vk}$ in low-k ($k_{F}$)and indicates that the inter-channel exclusion is more important than inner-channel effect.  This makes sense as open channel weight is larger in low k.  

\subsection{}
A term as $Y_{\vk\vk'}F_{\vk'}$ is fine in BEC side, but in BCS side, $F_{\vk}$ significantly differs from $v_{\vk}$ in low k and this terms seems significantly smaller than the case in two-body. 

In the first equation of \eef{eq:20100909:fullgap}, inter-channel interaction $Y_{\vk\vk'}G_{\vk'}$ gets large in resonance because it pulls the components also from high k of $G_{\vk'}$, while the pauli exclusion only comes from the single $G_{\vk}$, which seems always small.  

Key seems still in the two-body problem in this exact same representation and then we can renormalize it.  This serves two purpose: renormalize the high-k divergence and provide the comparison of two-body quantity $a_{s}$.

\subsection{}
It is energetically very disadvantageous to diviate from the resonant bound state, furthermore, it is very small in k-space as the bound-state is relatively small.  Therefore, as the first order approximation, we drop the second term and the denominator in the first term of the second equation (close-channel) of \eef{eq:20100909:fullgap}\footnote{ This is certainly OK for he BEC end where $F_{k}\ll1$ all the time.  It is less satisfactory, might be problematic,  when $F_{k}$ is close to maximum $\nth2$, this happens around Fermi energy in BCS limit, but the approximation is still OK for other places. So at least for bulk of the region of $G_{\vk}$ satisfies \eef{eq:20100915:gapb}}.  In the first equation, we drop the factor $\left(1+\sqrt{1-4 F_{\vk}^2-4 G_{\vk}^2}\right)^2$ in the second term for easier calculation, (no strong reason yet), the second term is relatively minor comparing to the first. We write down the approximated equations
\begin{subequations}\label{eq:20100915:gap}
\begin{gather}\label{eq:20100915:gapa}
\frac{2F_{\vk}}{\sqrt{1-4 F_{\vk}^2-4 G_{\vk}^2}} (\xi^{ab}_{\vk}+  G_{\vk}^2\eta)+U_{\vk\vk'}F_{\vk'}+Y_{\vk\vk'}G_{\vk'}=0\\
\label{eq:20100915:gapb}
{2G_{\vk}}(\xi^{ab}_{\vk}+\eta)+V_{\vk\vk'}G_{\vk'}+Y_{\vk\vk'}F_{\vk'}=0
\end{gather}
\end{subequations}
If $G_{\vk}=\alpha\phi^{0}_{\vk}$ as $\phi_{\vk}^{0}$ is the solution for the isolated close-channel \sch.  
\begin{equation}\label{eq:20100915:twobody}
{2\phi^{0}_{\vk}}(\epsilon_{\vk})+V_{\vk\vk'}\phi^{0}_{\vk'}=-2E^{0}\phi^{0}_{\vk'}
\end{equation}
\eef{eq:20100915:gapb} becomes
\begin{equation}
G_{\vk}=\frac{Y_{\vk\vk'}F_{\vk'}}{2(E^{0}-\eta+\mu)}
\end{equation}
Plug this back into the last term of \eef{eq:20100915:gapa}, we have 
\begin{equation}
\frac{2F_{\vk}}{\sqrt{1-4 F_{\vk}^2-4 G_{\vk}^2}} (\xi^{ab}_{\vk}+  G_{\vk}^2\eta)+U_{\vk\vk'}F_{\vk'}+\frac{Y_{\vk\vk'}Y_{\vk'\vk''}}{2(E^{0}-\eta+\mu)}F_{\vk''}=0
\end{equation}
Now considering the last two terms has weak dependecy in low k, we can set 
\begin{equation}\label{eq:20100915:gap1}
\frac{2F_{\vk}}{\sqrt{1-4 F_{\vk}^2-4 G_{\vk}^2}} \xi^{ab}_{\vk}+  G_{\vk}^2\eta)\equiv\Delta_{\vk}=-(U_{\vk\vk'}F_{\vk'}+\frac{Y_{\vk\vk'}Y_{\vk'\vk''}}{2(E^{0}-\eta+\mu)}F_{\vk''}
\end{equation}
We can express $F_{\vk}$ according to $\Delta_{\vk}$,  (ignore the higher order of $G_{\vk}$)
\begin{equation}\label{eq:20100915:F}
F_{\vk}=\frac{\Delta_{\vk}}2\sqrt{\frac{(1-4G_{\vk}^{2})}{(\xi^{ab}_{\vk}+  G_{\vk}^2\eta)^{2}+\Delta_{\vk}^{2}}}
\end{equation}
Put it back into the second half of gap equation (\ref{eq:20100915:gap1}), 
\begin{equation}\label{eq:20100915:onechannel}
\Delta_{\vk}=-\sum_{\vk'}\br{U_{\vk\vk'}+\frac{Y_{\vk\vk''}Y_{\vk''\vk'}}{2(E^{0}-\eta+\mu)}}\frac{\Delta_{\vk}}2\sqrt{\frac{(1-4G_{\vk}^{2})}{(\xi^{ab}_{\vk}+  G_{\vk}^2\eta)^{2}+\Delta_{\vk}^{2}}}
\end{equation}
At low k, $\Delta_{\vk}$ has weak dependency on k and we can take $\Delta_{\vk'}$ out of the summation,  then go through the same procedure as in \cite{Leggett,Fetter} to renormalize the equation
\begin{equation}\label{eq:20100915:renormGap}
\nth{\tilde{t_{0}}(\mu)}=\sum_{\vk'}\sqrt{(1-4G_{\vk'}^{2})}
\br{\nth{\epsilon_{\vk'}}-\nth{\sqrt{{(\xi^{ab}_{\vk}+  G_{\vk}^2\eta)^{2}+\Delta_{\vk}^{2}}}}}
\end{equation}
Noted that $\tilde{t_{0}}(\mu)$ is the zero-energy scattering amplitude for a reduced density of energy by factor $\sqrt{(1-4G_{\vk'}^{2})}$.  
\begin{gather}
\tilde{t_{0}}(\mu)=\br{1-\tilde{U}\tilde{ K}}^{-1}\tilde{U}\label{eq:20100915:t0}\\
\tilde{U}_{\vk\vk'}=\nth{2} \br{U_{\vk\vk'}+\frac{Y_{\vk\vk''}Y_{\vk''\vk'}}{2(E^{0}-\eta+\mu)}}\label{eq:20100915:tu}\\
\tilde{K}=\frac{\sqrt{1-4G_{\vk}^{2}}}{\epsilon_{\vk}}\delta_{\vk\vk'}\label{eq:20100915:tk}
\end{gather}
This is a little odd as the two-body quantity $t_{0}$ is affected by the many-body quantity $G_{\vk}$.  Nevertheless, it seems OK to simply use the two-body close-channel bound-state wave function as $G_{\vk}$ for the lowest order approximation.   Of course, in higher order, especially, the BCS case, the open-channel is significantly different from the two-body solution and therefore the weight is significantly different, from \eef{eq:20100915:gapb}, this affect the weight of close-channel $G_{\vk}$.  Unlike the many-body equation \eef{eq:20100915:renormGap}, where only low-k is interesting and therefore $G_{\vk}$ can be regarded as constant and small, the summation should go all the way to high-k.  As \eef{eq:20100915:tu}, $\tilde{U}$ is the unmodified, the resonance position is probably not modified, but the width of resonance is changed.  



\subsection{Tony's comment\label{sec:20100915:tony}}
Tony suggested that in \eef{eq:20100915:renormGap} we only need to care about the $G_{\vk'}$ where the difference in summation is substantial.  

He mentioned that in normal BEC, below $T_{C}$ the fluctuation of paritcle is very large for gand cannonical emsemble.  That might have to do with the problem in Sec \ref{sec:20100915:deltaN}.
Hementioned a paper \cite{Politzer} about it.  In atomic BEC, fluctuation is large for grand-canonical ensemble.  
