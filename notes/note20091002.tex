\section{note of 2009.10.02}
In two-body problem, what is the essential part for close-channel state?  Bound state, or what?  In normal treatment, such as Tony's \cite{Leggett}, the only thing matter is the feedback from close-channel to open channel.  And it ceases to have any importance after be capsulated into one parameter.  It is the case not only in the Feshbach resonance, but also the shape resonance.  The only thing that really useful for Feshbach is that the detuning is tunable and the feedback is tunable. 

So key features of close-channel state:
\begin{description}
	\item[-]bound state, loose or tight?
	\item[-]large level spacing between different levels. Or is it important?
	
\end{description}

In Tony's approach, only this $\kappa$ affect things. And maybe it is the only thing matter for close channel.  So is it possible to just tune other things in bound state to make the same $\kappa$, then you have the same effect?  
