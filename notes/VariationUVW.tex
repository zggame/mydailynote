\subsection{Variation method with the u,v,w}
We start with three hyperfine species a, b, c, where a is the common species in two channel. (a,b) is the open channel while (a,c) is the close channel. And the Hamiltonian is written in the form 
\begin{equation}
\begin{split}
 H=&\sum_\vk\epsilon^a_\vk{}a^+_\vk{}a^{}_\vk+\sum_\vk\epsilon^b_\vk{}b^+_\vk{}b^{}_\vk+\sum_\vk\epsilon^c_\vk{}c^+_\vk{}c^{}_\vk\\
  &+\nth{2}\sum_{\vk\vk'}U_{\vk\vk'}a^+_\vk{}b^+_{-\vk}{}b^{}_{-\vk'}a^{}_{\vk'}
	+\nth{2}\sum_{\vk\vk'}V_{\vk\vk'}a^+_\vk{}c^+_{-\vk}{}c^{}_{-\vk'}a^{}_{\vk'}\\
 &+\nth{2}\sum_{\vk\vk'}Y_{\vk\vk'}a^+_\vk{}b^+_{-\vk}{}c^{}_{-\vk'}a^{}_{\vk'}
	+\nth{2}\sum_{\vk\vk'}Y^*_{\vk\vk'}a^+_{\vk'}{}c^+_{-\vk'}{}b^{}_{-\vk}a^{}_{\vk}
\end{split} 
\end{equation}
By the Hermition condition we have 
\begin{equation}
 U_{\vk'\vk}=U^*_{\vk\vk'},\qquad{} V_{\vk'\vk}=V^*_{\vk\vk'}
\end{equation}
  We start from the ansatz as 
\begin{equation}\label{eq:ansatz}
 \ket{\Psi}=\prod_\vk\br{u_\vk+v_\vk{}a^\dg_\vk{}b^\dg_{-\vk}+w_\vk{}a^\dg_\vk{}c^\dg_{-\vk}}\ket{0}
\end{equation}
Here we require $\abs{u_\vk}^2+\abs{v_\vk}^2+\abs{w_\vk}^2=1$ for normalization.  For all the interaction term, there are two types of contribution,
for example, 
\begin{equation*}
\av{U_{\vk\vk'}a^\dg_\vk{}b^\dg_{-\vk}{}b^{}_{-\vk'}a^{}_{\vk'}}
=\sum_{\vk}U_{\vk\vk}\abs{v_\vk}^2+\sum_{\vk\neq\vk'}U_{\vk\vk'}v^{}_{\vk'}u^*_{\vk'}u^{}_\vk{}v^*_\vk
\end{equation*}
The first term is the Hatree term and the second term is more interesting pairing term.



And the free energy is 
\begin{equation}
 \begin{split}
  &F\equiv\av{H-\mu{}N}\\
    =&\sum(\xi^a_\vk+\xi^b_\vk)\abs{v_\vk}^2+\sum(\xi^a_\vk+\xi^c_\vk)\abs{w_\vk}^2\\
    &+\nth2\sum_{\vk}U_{\vk\vk}\abs{v_\vk}^2+\nth2\sum_{\vk\neq\vk'}U_{\vk\vk'}v^{}_{\vk'}u^*_{\vk'}u^{}_\vk{}v^*_\vk\\
    &+\nth2\sum_{\vk}V_{\vk\vk}\abs{w_\vk}^2
      +\nth2\sum_{\vk\neq\vk'}V_{\vk\vk'}w^{}_{\vk'}u^*_{\vk'}u^{}_\vk{}w^*_\vk\\
    &+\nth2\sum_{\vk}Y_{\vk\vk}w^{}_{\vk}v^*_\vk{}
      +\nth2\sum_{\vk\neq\vk'}Y_{\vk\vk'}w^{}_{\vk'}{u^{*}_{\vk'}}v^*_\vk{}u^{}_\vk\\
    &+\nth2\sum_{\vk}Y^*_{\vk\vk}w^*_{\vk}v^{}_{\vk}{}
      +\nth2\sum_{\vk\neq\vk'}Y^*_{\vk\vk'}w^*_{\vk}{u^{}_{\vk}}v^{}_{\vk'}{}u^{*}_{\vk'}
 \end{split}
\end{equation}
Where 
\begin{equation*}
 \xi^a_\vk=\epsilon^a_\vk-\mu^a,\qquad\xi^b_\vk=\epsilon^b_\vk-\mu^b,\qquad\xi^c_\vk=\epsilon^c_\vk-\mu^b
\end{equation*}
The chemical potential is added to make sure the $n_a=n_b+n_c=\nth{2}n$.
 I drop the Hatree term as this in some sense just shift the chemical potentials as it only relates to the density.  (\emph{Not sure still valid in the two-channel problems, especially for the close-channel.})  Also, I ignore the summation on the second only goes through $\vk\neq\vk'$ as the correction is in the higher order. 
 
Now introduce the parameters $\theta_\vk$, $\phi_\vk$ to include the normalization condition.  
\begin{equation}
 u_{\vk}=\cos\theta_\vk,\qquad{} v_{\vk}=\sin\theta_\vk\cos\phi_\vk,\qquad{}
 w_{\vk}=\sin\theta_\vk\sin\phi_\vk
\end{equation}
We take them as real quantities as those that minimize the free energy are real.(?) Now the free energy can be written as 

\begin{equation}
 \begin{split}
  F=&\sum\xi^{ab}_\vk\sin^2{\theta_\vk}\cos^2\phi_\vk+\sum\xi^{ac}_\vk\sin^2{\theta_\vk}\sin^2\phi_vk\\
    &+\nth2\sum_{\vk\vk'}U_{\vk\vk'}\cos\theta_{\vk'}\sin\theta_{\vk'}\cos\phi_{\vk'}\cos\theta_\vk{}\sin\theta_\vk\cos\phi_\vk\\
    &+\nth2\sum_{\vk\vk'}V_{\vk\vk'}\cos\theta_{\vk'}\sin\theta_{\vk'}\sin\phi_{\vk'}\cos\theta_\vk{}\sin\theta_\vk\sin\phi_\vk\\
    &+\nth2\sum_{\vk\vk'}Y_{\vk\vk'}\cos\theta_{\vk'}\sin\theta_{\vk'}\cos\phi_{\vk'}\cos\theta_\vk{}\sin\theta_\vk\sin\phi_\vk\\
    &+\nth2\sum_{\vk\vk'}Y^*_{\vk\vk'}\cos\theta_{\vk'}\sin\theta_{\vk'}\sin\phi_{\vk'}\cos\theta_\vk{}\sin\theta_\vk\cos\phi_\vk\\
    =&\nth4\sum_\vk\xi^{ab}_\vk(1-\cos2\theta_\vk)(1+\cos2\phi_\vk)+\nth4\sum_\vk\xi^{ac}_\vk(1-\cos2\theta_\vk)(1-\cos2\phi_\vk)\\
    &+\nth{8}\sum_{\vk\vk'}U\sin2\theta_\vk\cos\phi_\vk\sin2\theta_{\vk'}\cos\phi_{\vk'}+\nth{8}\sum_{\vk\vk'}V\sin2\theta_\vk\sin\phi_\vk\sin2\theta_{\vk'}\sin\phi_{\vk'}    \\
    &+\nth{4}\sum_{\vk\vk'}Y\sin2\theta_\vk\cos\phi_\vk\sin2\theta_{\vk'}\sin\phi_{\vk'}    
 \end{split}
\end{equation}
We assume that $Y=Y^*$. From the above equation, we can differentiate it with respect to $\theta_\vk$ and $\phi_\vk$ and set the derivative as 0, therefore minimize free energy. 
\begin{align}
0=&\pdiff{F}{\theta_\vk}\notag\\
 =&\nth{2}\sin2\theta_\vk\mbr{\xi^{ab}_\vk(1+\cos2\phi_\vk)+\xi^{ac}_\vk(1-\cos2\phi_\vk)}\notag\\
 &+\nth2\sum_{\vk'}\cos2\theta_\vk\sin2\theta_{\vk'}\mbr{U\cos\phi_\vk\cos\phi_{\vk'}+V\sin\phi_\vk\sin\phi_{\vk'}+Y\sin(\phi_{\vk'}+\phi_\vk)}\\
 0=&\pdiff{F}{\phi_\vk}\notag\\
 =&-\nth{2}(\xi^{ab}_\vk-\xi^{ac}_\vk)\sin2\phi_\vk(1-\cos2\theta_\vk)\notag\\
 &-\nth4\sum_{\vk'}\sin2\theta_\vk\sin2\theta_{\vk'}\mbr{U\sin\phi_\vk\cos\phi_{\vk'}-V\cos\phi_\vk\sin\phi_{\vk'}-Y\cos(\phi_{\vk'}+\phi_\vk)}
\end{align}
  These two set of equations (for each $\vk$, but decoupled) fully determine the wave-function. 
We introduce two quantities:
\begin{align}
\Delta_\vk^U&=\sum_{\vk'}\sin2\theta_{\vk'}(U_{\vk\vk'}\cos\phi_{\vk'}+Y_{\vk\vk'}\sin\phi_{\vk'})\label{eq:gap1}\\
\Delta_\vk^V&=\sum_{\vk'}\sin2\theta_{\vk'}(V_{\vk\vk'}\sin\phi_{\vk'}+Y_{\vk\vk'}\cos\phi_{\vk'})\label{eq:gap2}
\end{align} 
As we can see eqs.(\ref{eq:gap1},\ref{eq:gap2}) is indeed very similar to the structure of two-body Schr\"{o}‌dinger equation.
If we ‌introduce the Zeeman energy detuning between two channels
\begin{equation}
\eta=\xi^{ab}-\xi^{ac}
\end{equation}
These equations can be written into a more compact form
\begin{align}
\tan2\theta_\vk&=-\frac{\cos\phi_\vk\Delta^U_\vk+\sin\phi_\vk\Delta^V_\vk}{2\xi^{ab}_\vk+2\eta\cos^2\phi_\vk}\label{eq:tan1}\\
\tan\theta_\vk&=-\frac{\sin\phi_\vk\Delta^U_\vk-\cos\phi_\vk\Delta^V_\vk}{2\eta\sin2\phi_\vk}\label{eq:tan2}
\end{align} 


Furthermore, when $k\rightarrow\infty$, from eq. (\ref{eq:tan1}), we see that $\theta\rightarrow0$ as $1/\xi$;  and from eq. (\ref{eq:tan2}), we see that $\sin\phi_\vk\Delta^U=\cos\phi_\vk\Delta^V$ as we assume $\Delta$ varies slowly over $\vk$.  We have 
\begin{align*}
\cos^2\phi_\vk&=\frac{{\Delta^U}^2}{{\Delta^V}^2+{\Delta^U}^2}\\
\sin^2\phi_\vk&=\frac{{\Delta^V}^2}{{\Delta^V}^2+{\Delta^U}^2}
\end{align*}
And 
\[\tan2\theta_\vk=-\frac{({{\Delta^V}^2+{\Delta^U}^2})^{1/2}}{2\xi^{ab}_\vk}\]

Eqs. (\ref{eq:gap1},\ref{eq:gap2}) are similar as those derived from the Green's function method in the eq. (\ref{eq:gapMatrix}), and should be able to renormalized in the similar fashion $\Delta=T(F-G\Delta)$, where in our notation $F=(\cos\theta_\vk\sin\theta_\vk\cos\phi_\vk,\cos\theta_\vk\sin\theta_\vk\sin\phi_\vk)$.  The problem, however, is that the eqs. (\ref{eq:tan1},\ref{eq:tan2}) does not present a simple analytic solution about $\theta$ and $\phi$ in terms of $\Delta$ and therefore the gap equations cannot be written into a simple renormalized  equation although they are implicit functions of $\Delta$. It is a rather cumbersome process to calculate the integration in renormalized gap equations.  Furthermore, information of the close-channel is close to a simple one-level bound-state is not fully incorporated and the equation might be simpler and has a nicer form if I can manage to do that.  

Another point as pointed out before is  the $2\times2$ T-matrix, which should relate to $a_s$ of both side, in the open channel, it should be the background one.  The tuning parameter of the resonance should be in $G$.  What exactly is the off-diagonal elements in T and how do they work in two-body problem?  This needs to be worked out explicitly for clear compaing.    
