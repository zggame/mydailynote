\subsection{Hamiltonian with collective mode}
The reduced BCS hamiltonian does not sustain a collective mode.  A non-zero-central-momentum component has to be added into the hamiltonian in order to have Anderson-Bogoliubov mode.  On the other hand, in real-space, a term such as $g\bar\psi(r)\bar\psi(r)\psi(r)\psi(r)$ actually has non-zero-central-momentum in momentum space and is enough to give collective mode.  It can be interpret as the Fourier transform into real-space of the s-wave component of the interaction.  

\subsection{Path integral for collective mode} 
It seems that Hubbard-Stratonovich transformation \cite{Altland} can be used to introduce a dual-bosonic-fields $(\Delta^{F},\;\Delta^{G})$. And instead a simple $2\times2$ spinor Gorkov representation, it is a $3\times3$ representation $(\bar\psi^{a},\;\psi^{b},\;\psi^{c})$, similar to the canonical transformation employed to find the fermionic excitation mode (Sec \ref{sec:fermionicExcitation}).  But first, I need to reproduce the result of variation method from saddle-point (mean-field) of this method, (employing the similar approximation?) and then maybe the second order is going to give me a way to find collective mode. 

\subsection{Tony's comment}
He thought that my break-up in the perturbation term of Eq. (\ref{eq:canonical:secular2}) is fine.  He suggested me to find the correction form of eigenvectors (i.e. the form of Bogoliubov transformation, or how the quasiparticle is corrected). 