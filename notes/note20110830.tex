\subsection{Questions}
In Bogliubov transformation, the mixture of creators and annihilators means pairing, or just non-number-conservation?

--It definitely indicates the number is not conserved and we need grand-canonical ensemble.  With a non-trivial coefficient, it is probably indicates the pairing as well. 

Is the $1/\epsilon_{\vk}$ the real behavior of wave-function or correlation?  Is that just the extrapolation from low-energy?  Is the divergence in integral the result of non-decay in interaction (contact type) only? 

--The wave function asymptotically approach $1\epsilon_{\vk}$ generally.  So this problem is more of the over-simplification of interaction as contact.  The proper decay in high-momentum of interaction solves the problem. 

\subsection{Tony's comment about mesure two-body quantities, such as $a_{s}$ in many-body context	}
In many-body context, two-body quantities is limited by many-body scale.  Imaging measuring $a_{s}$ that is larger than inter-particle distance $a_{0}$, this is probably difficult and limit by many-body scale $a_{0}$.  So it is probably equally difficult to measure the precise resonant position in many-body context.  It is limited by many-body scale.  

\subsection{Narrow resonance treatment agnostic to the fact that system is in superfluidity}
Although both of my approaches (path-integral and anstaz) based on the superfluidity, i.e., pairing, the idea of separate the scale of close-channel seems not really related to this fact.  Therefore, it is probably also possible to develop normal theory with the similar correction and procedure.  

\subsection{Tony's comment on group meeting}
For shift $\mu$ on detuning, I talked about in many-body system, energy is often calculated from Fermi energy instead of zero.  Tony asked in what range around Fermi surface, fermions participate?

It seems as the theory I developed is a zero-temperature one.  This range is more about interaction related.  In finite temperature, this is probably in order of $k_{c}T$.  But I am not sure, what interaction range I should put here. 

\subsection{rescaled mean-field equations of single-channel crossover}
If we rescaled everything with $k_{F}$ ($E_{F}$). The number equation in 3D becomes
\begin{equation}
\frac{2}{3}=\int dx\, x^{2}(1-\frac{x^{2}-x_{\mu}^{2}}{\sqrt{(x^{2}-x_{\mu}^{2})^{2}+x_{\Delta}^{4}}})
\end{equation}
The gap equation is 
\begin{equation}
\nth{a_{s}k_{F}}=-\frac{2}{\pi}\int dx\,\br{\frac{x^{2}}{\sqrt{(x^{2}-x_{\mu}^{2})^{2}+x_{\Delta}^{4}}}-1}
\end{equation}
Here 
\begin{equation}
x=\frac{k}{k_{F}},\qquad {}
\frac{\hbar^{2}(x_{\mu}k_{F})^{2}}{2m}=\mu,\qquad
\frac{\hbar^{2}(x_{\Delta}k_{F})^{2}}{2m}=\Delta
\end{equation}


\subsection{dimension in the two-body Feshbach resonance }
In Leggett's analysis, or my thesis Chapter 3.  $\phi_{0}\sim L^{-1/2}$ while $\chi\sim 1$ ($\chi_{c}\sim1$), because of the different normalization.  $\chi$ or $\chi_{c}$ are both normalized to 1 for $r=0$ and this make sure their short-range parts have the same normalization.  $K(r,r')\sim E\cdot{L^{-1}}$, $\kappa (\text{or } \mathcal{K})\sim E$ 