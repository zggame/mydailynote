% !TeX root =thesis.tex
% 
In this thesis, we studied the narrow resonance, especially the case where two channels share the same species.  In general, for narrow resonance without shared species, the main correction is that the weight in close channel.  Two number equations exists, one for open channel and one for close channel.  The number equation resembling the single channel is for the density of open-channel, as in  \cite{GurarieNarrow}.  

However, when there is common species,  Pauli exclusion between two channels need to be considered carefully for narrow resonance.  Overall, our treatment go along the spirit of universality.  For the dilute system with short-range potential, such as dilute ultracold Alkali gas, the short-range wave function is essentially unchanged from two-body  to many-body.  And this particular feature serves formally as the boundary condition.  On the other hand, the rest of the wave function is essentially free where no two particles is in short-range.  Different from the universality treatment is that the resonance, which makes the system, even the two body part, extremely sensitive to the change in small relative weight between two channels and therefore requires extra cares.  In Feshbach resonance, the close-channel bound state, $\phi_{0}$, in resonance is actually not sensitive to resonance, therefore, we apply the simple boundary condition on it. On the other hand, open-channel direct interaction is dominated by the indirect interaction mediated by close-channel, therefore, the small change in the close-channel bound-state weight \emph{does} affect the short-range wave function in open-channel and the boundary condition it follows, as $a_{s}$. 

When the close channel bound state is in the order of   inter-particle distance, it is a genuine three-species many-body problem and no obvious solution is available.  However, when this bound-state is much smaller than inter-particle distance, this ratio ($r_{c}/a_{0}$) serves as the expanding coefficient and we can extracted the effect perturbatively.  In essence, we can ignore the many-body effects between these close-channel bound-states, while only taking into consideration of the Pauli exclusion between channels.  A few new parameters needs to be introduced and can be calibrated from experiments.  System's mean field property can still be determined by gap equations and number equations as in single-channel problem.  The excitation modes are also close to the original single-channel result with correction in order of $r_{c}/a_{0}$.

In essence, there are three different regimes for the many-body energy scale $E_F$ when comparing with two-body physics.  

First, when $E_F$ is much smaller than the characteristic energy $\delta_c$ (Sec. \ref{sec:intro:twobody}), around resonance, close-channel weight is negligible, and we can ignore close-channel at all and just take the open-channel interaction varies according to s-wave scattering length $a_s$. 

Second, when $E_F$ is larger than $\delta_c$, but smaller than the close-channel bound state binding energy $E_b$, or the bare detuning $\eta$, close-channel weight is significant and cannot be ignored.  We have one extra condition $n_{\text{open}}+n_{\text{close}}=n_{\text{total}}$. Also, we need to take into account that in the resonance formula of $a_s$, shifting should count from Fermi sea instead of zero as in two-body physics.  Furthermore,  This is extensively treated in \cite{GurarieNarrow} as well as in this thesis.  Furthermore, Pauli exclusion between two channels needs to be taken into account when they share common species.  Nevertheless, in the low-momentum, close-channel bound state has very small weight and  open-channel dominates.  therefore, we can still treat the Pauli exclusion between channels perturbatively with the expansion over ratio $E_F/\eta$.  This is the emphasis of the current thesis.  

Third, when $E_F$ larger than binding energy $E_b$ or bare detuning $\eta$, we have a pure three species many-body problem. Such region can be achieved in two ways, a very large $E_F$, which indicate a very dense system, and it is hard to image that  the original dilute alkali gas model still applies at all. Various approximation in the model probably breaks down beforehand, such as pseudo-potential $a_s$ replacement. On the other hand, it requires a very small $\eta$, which indicates a true three species many-body problem, which is unclear how to solve.   