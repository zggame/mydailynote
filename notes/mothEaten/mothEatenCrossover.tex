\documentclass{article}
\usepackage{fullpage}
\usepackage{amsfonts}
\usepackage{amssymb}
\usepackage{amsmath}
\usepackage{hyperref}
\usepackage{graphicx}% Include figure files
\usepackage{epstopdf}
\usepackage{subfig}

%\usepackage{setspace}\doublespace
\author{Monique Combescot \and Guojun Zhu}
\title{Moth-eaten effect on BEC-BCS pairing}
\newcommand{\vk}{\ensuremath{\mathbf{k}}}
\newcommand{\vK}{\ensuremath{\mathbf{K}}}
\providecommand{\vr}{\ensuremath{\mathbf{r}}}
%\newcommand{\vec}[1]{\ensuremath{\mathbf{#1}}}

\newcommand{\gk}{\ensuremath{{g}(\mathbf{k})}}

\newcommand{\vp}{\ensuremath{\mathbf{p}}}
\newcommand{\gp}{\ensuremath{{g}(\mathbf{p})}}

\newcommand{\vq}{\ensuremath{\mathbf{q}}}

\newcommand{\Fo}{\ensuremath{\mathbf{F_0}}}


\newcommand{\E}{\ensuremath{\mathbf{E}}}
\newcommand{\A}{\ensuremath{\mathbf{A}}}
\newcommand{\J}{\ensuremath{\mathcal{J}}}

\newcommand{\ket}[1]{\ensuremath{\left|#1\right>}}
\newcommand{\bra}[1]{\ensuremath{\left<#1\right|}}

\newcommand{\twoe}{\ensuremath{2\epsilon_\vk-\E_1}}

\newcommand{\nth}[1]{\ensuremath{\frac{1}{#1}}}

\newcommand{\br}[1]{\ensuremath{\left(#1\right)}}
\newcommand{\mbr}[1]{\ensuremath{\left[#1\right]}}
\newcommand{\bbr}[1]{\ensuremath{\left\{#1\right\}}}


\newcommand{\tk}{\ensuremath{\tilde{k}}}

\newcommand{\kp}{\ensuremath{\ket{\Psi}}}

\newcommand{\av}[1]{\ensuremath{\bigl<{#1}\bigr>}}
\newcommand{\avs}[3] {\av{#1{\lvert{#2}\rvert}#3}}
\newcommand{\avv}[2][\nu] {\avs{#1}{#2}{#1}}
\newcommand{\avt}[2]{\av{{#1}|{#2}}}
\newcommand{\avtu}[1]{\av{T_\tau#1}}

\newcommand{\Bop}{\ensuremath{\mathbf{B_0^+}}}
\newcommand{\Bmp}{\ensuremath{\mathbf{B_m^+}}}
\newcommand{\Bnp}{\ensuremath{\mathbf{B_n^+}}}
\newcommand{\Bo}{\ensuremath{\mathbf{B_0}}}
\newcommand{\Bopn}{\ensuremath{\mathbf{{B_0^+}^n}}}
\newcommand{\Bon}{\ensuremath{\mathbf{{B_0}^n}}}


\newcommand{\zmatrix}{\ensuremath{\br{\begin{smallmatrix}0&0\\0&0\end{smallmatrix}}}}
\newcommand{\fmtrx}[4]{\ensuremath{\br{\begin{smallmatrix}#1&#2\\#3&#4\end{smallmatrix}}}}
\newcommand{\smtrx}[6]{\ensuremath{\br{\begin{smallmatrix}#1&#2\\#3&#4\\#5&#6\end{smallmatrix}}}}

\newcommand{\vz}{\ensuremath{v^{\beta\alpha}_{\vk,\vk}}}


\providecommand{\abs}[1]{\ensuremath{\lvert{#1}\rvert}}

\newcommand{\sg}[1][1]{\ensuremath{\sigma_\frac{#1}{2}}}

\newcommand{\rhof}{\ensuremath{\rho(\ef)}}
\newcommand{\omt}{\ensuremath{\tilde{\Omega}}}
\newcommand{\cht}{\ensuremath{\tilde{\chi_0}}}
\newcommand{\Atl}{\ensuremath{\abs{A}^{2l}}}
\newcommand{\ef}{\ensuremath{\epsilon_F}}

\newcommand{\lca}{\ensuremath{\ln\br{1+\frac{\cht}{\alpha}}}}

\newcommand{\com}[2]{\ensuremath{\mbr{#1,#2}}}
\newcommand{\D}{\ensuremath{\mathit{D}}}
\newcommand{\dg}{\ensuremath{\dagger}}
\newcommand{\nG}{\ensuremath{\hat{\mathcal{G}}^{-1}}}

\providecommand{\lvk}{\ensuremath{1/\vk_F}}
\providecommand{\hm}{\ensuremath{\frac{\hbar^2}}{2m}}
\providecommand{\pdiff}[2]{\ensuremath{\frac{\partial{#1}}{\partial{#2}}}}
\providecommand{\dpdiff}[2]{\ensuremath{\frac{\partial^2{#1}}{\partial{{#2}^2}}}}

\providecommand{\H}{\ensuremath{\mathcal{H}}}
\providecommand{\wt}[1]{\widetilde{#1}}

\providecommand{\eef}[1]{Eq. (\ref{#1})}

\providecommand{\sch}{{Schr\"{o}dinger }}

\providecommand{\sgn}{\ensuremath{\text{sgn}}}
\newcommand{\Arctg}{\ensuremath{\text{Arctg}}}

\providecommand{\comm}[1]{\textit{\scriptsize \uwave{(#1)}}}
\renewcommand{\emph}[1]{\textbf{#1}}
\newcommand{\td}{{\ensuremath{{(2D)}}}}
\newcommand{\sd}{{\ensuremath{{(3D)}}}}
\begin{document}
\maketitle
\numberwithin{equation}{section}
\section{The system}
We consider $N$ fermion pairs $(\alpha,\beta)$ governed by the hamiltonian
\begin{equation}
H=H_{0}+V
\end{equation}
where $H_0$ is the kinetic energy 
\begin{equation}
H_0=\sum_{\vk}\epsilon_\vk(a^\dagger_\vk{}a^{}_\vk+b^\dagger_\vk{}b^{}_\vk)
\label{eq:}
\end{equation}
and $V$ is interaction
\begin{equation}
V=-v\sum_{\vk\vk'}\omega_{\vk'}\omega_\vk\beta^\dagger_{\vk'}{}\beta^{}_\vk
\label{eq:}
\end{equation}
and $\beta^\dagger_{\vk}=a^\dagger_{\vk}b^\dagger_{-\vk}$ creates a zero momentum pair while $\omega_\vk=1$ for $0<\epsilon_\vk<\Omega$ and zero otherwise.  
\section{One pair}
The single pair energy $E_1$ follows from Cooper equation
\begin{equation}
1=v\sum_{\vk}\frac{\omega_\vk}{2\epsilon_\vk-E_1}\equiv{}v\,S(E_1)
\label{eq:}
\end{equation}
\subsection{2D systems}
In 2D, the density of state is constant, so that 
\begin{equation}
S^{(\text{2D})}(E<0)=\rho\int_0^{\Omega}\frac{d\epsilon}{2\epsilon-E}=\frac{\rho}{2}\ln\left(\frac{2\Omega-E}{-E}\right)
\label{eq:}
\end{equation}
$S^{(\text{2D})}(E)$ diverges when $E\rightarrow{}0_{-}$ while it goes to zero as $\rho\Omega/(-E)$ when $E$ goes to $-\infty$. A bound state with $E_1<0$, tus exists no matter how weak $v$ is. It is given by 
\begin{equation}
E_1^{(\text{2D})}=-\frac{2\sigma}{1-\sigma}\Omega
\label{eq:}
\end{equation}
where $\sigma=e^{-2/\rho{v}}$
\subsection{3D systems}
In 3D, the density of states increases as $\sqrt{\epsilon}$. Let us write it as 
\begin{equation}
\rho(\epsilon)=\rho\sqrt{\epsilon/\Omega}
\label{eq:}
\end{equation}
where $\rho$ is the density of state at the potential threshold. We then have

\section{N pairs}
Richardson \cite{Richardson1} and Gaudin \cite{gaudin} have shown that the energy of N fermion pairs interacting through the BCS reduced potential is exactly given by $E_N=R_1+\cdots+R_N$ where the $R_i$'s follow from $N$ coupled nonlinear equations
\begin{equation}
 1=v\sum_\vk\frac{w_\vk}{2\epsilon_\vk-R_i}+\sum_{j\neq{}i}\frac{2v}{R_i-R_j}\qquad\text{for}\;i=(1,\cdots,N)
\end{equation}
\subsection{2D systems}
We have shown that when the density of states is constant, as in standard BCS superconductivity set also in 2D systems, a compact form of $E_N$ can be derived. It leads to 
\begin{equation}
 E^(2D)_N=N\,E^{2D}_1+\frac{N(N-1)}{\rho}\frac{1+\sigma}{1-\sigma}
\end{equation}
within under extensive terms in $(N/\rho)^n$ with $n\leq2$

This gives the condensation energy in $N$ pairs, i.e., the difference between the $N$-pair energy without and with potention, as 
\begin{equation}\label{eq:E2D}
\begin{split}
 \mathcal{E}^{(2D)}_N&=E_N^{\td}(v=0)-E_N^\td\equiv{}N\epsilon_N^\td\\
&=N(1-\frac{N-1}{N_\Omega})\frac{2\sigma}{1-\sigma}\Omega
\end{split}
\end{equation}
where $N_\Omega=\rho\Omega$ is the number of states in the potential layer. 

The condensation energy per pair $\epsilon^\td_N=\mathcal{E}^\td_N/N$ thus decreases linearly with N. This decrease is due to a moth-eaten effect induced by Pauli blocking on the number empty states feeling the potential and thus available to from the $N$-pair bound state.  In the limit of a number of pairs filling the potential layer, $N=N_\Omega$, the condensation energy reduceds to 
\begin{equation}
 \mathcal{E}^\td_{N_\Omega}=\Omega[\frac{2\sigma}{1-\sigma}]
\end{equation}
so that the condensation energy per pair tends to zero as $\epsilon^\td_{N_\Omega}=[2\sigma/(1-\sigma)]/\rho$ in the large sample limit: for $N=N_\Omega$, the system has no more freedom to construct a lower energy ground state, the condensation energy per pair thus have to cancel.  
\subsection{3D system}
A similar compact expression of the $N$-pair energy has not been derived so far in the case of a $\sqrt{\epsilon}$ density of states. Let us first qualitatively understand the behavior of the $N$ pair energy when $N$ increases. 

It is clear that the same moth eaten effect must bring the condensation energy per pair down to zero when $N$ approach $N_\Omega$, this effect fundamentally linked to the Pauli exclusion principle being unaffected by an increase of space dimensionality. 

\subsubsection{}
For potential $v$ lower than the threshold value $v_{th}(1)=1/\rho$ for the existence of one-pair bound state, $\mathcal{E}_N^\sd$ is equal to zero for $N=1$ and also for $N=N_\Omega$.  It is however clear that $\mathcal{E}_N^\sd$ cannot stay equal to zero for all $N$ between $1$ and $N_\Omega$ because for $N$ equal to a praction of $N_\Omega$, the density of states is essentially constant.  So that we can at least freeze a fraction of the $N$ electrons as in standard BCS superconductivity with a frozen core: a finite condensation thus exists no matter how weak $v$ is. 
%\tableofcontents

We can have a lower boundary of this condensation energy per pair by freezing $N_0$ of these $N$ electrons.  The remaining $N-N_0$ pairs enjoy the potential attraction in a region where the density of state stays finite, between $\rho\sqrt{\epsilon_{F_0}/\Omega}$ and $\rho$.  The condensation energy of these $(N-N_0$ pairs would then read according to Eq. (\ref{eq:E2D})
\begin{equation}\label{eq:E2D}
 \overline{\mathcal{E}}_{N-N_0}=(N-N_0)(1-\frac{N-N_0-1}{N_\Omega})\frac{2\bar\sigma}{1-\bar\sigma}\Omega
\end{equation}
where $\bar{\sigma}=e^{-2/{\bar{\rho}v}}$ with $\bar\rho$ being average density of states in the potential layer above the frozen core. This gives a condensation energy per pair equal to $\overline{\mathcal{E}}_{N-N_0}/N$.  By taking $\bar\rho$ as independant of $N_0$, we find that this condensation energy per pair is maximized at $N-N_0=N_\Omega/2$, which of course implies to have $N$ larger than $N_\Omega/2$. 
\bibliography{../citation}
%\bibliographystyle{apsrmp4-1long}
\bibliographystyle{apalike}

\end{document}
