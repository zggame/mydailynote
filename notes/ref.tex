\subsection{BCS}
\subsubsection{Old BCS Asmpototic Solution}
\begin{description}
 \item [\cite{BogoliubovColl}] First one get Anderson-Bogoliubov mode in BCS.  (Eq. (4.16))  It also states that non-zero central momentum components are necessary for collective mode(sec. 4.2).  It linearized the Fr\"{o}hlich Hamiltonian with non-zero expectation of the normal expectation as well as abnormal ones (keep the non-zero central momentum quantities). 
  \item [\cite{AndersonBCS}]Get bosonic mode, Anderson-Bogoliubov mode, in BCS with  equation of motion method.  One has to introduce none-zero central momentum interaction to reduced BCS hamiltonian to get it.  The linearization is specific for BCS ansatz, i.e., non-zero mean-field value for anonymous pair expectation. In the final result for excitation, the excitation also expanded around only zero-central-momentum and others are approximated by zero-central-momentum quantities, i.e., BCS ground state value. 
   
   $\bar{A}$ is the time reversal of $A$ (opposite momentum and opposite spin).
	 
\item[\cite{Rickayzen}] Use the same equation of motion method as \cite{AndersonBCS}, but with a canonical transformation first.  So the resulting equation is much simpler. 
\item[ \cite{BcsExact}] with useful citations to others.        
 \end{description}

\subsection{Feshbach Resonance}
\subsubsection{Basics}
\subsubsection{Narrow}
\begin{description}
  \item [\cite{JacksonNarrow}]Nice study with two-channel model and how to renomalize it. 
	
 \end{description}
\subsection{Miscellaneous}
\subsection{}
\begin{description}
\item[\cite{Politzer}]He discuss the difference of BEC fluctuation between in canonical ensemble and in grand-canonical ensemble.  In grand-canonical ensemble that the fluctuation of ground-state can be as large as $N$.  However, in canonical ensemble, the fluctuation of ground state is equal to that of all excited state and stays finite all the time.  But other quantities seem to be fine between two framework.  
\end{description}
