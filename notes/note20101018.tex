\subsection{Shift of $\mu=0$ from universality point $a_s=\infty$ in single channel problem}
This can be somehow understood with Fermi sea. Universality point $a_s=\infty$ marks the point where zero-energy pair starts to form the bound-state.  However, in many-body Fermi-sea, there are plenty of higher energy pairs that are still in scattering state thus prevent chemical potential goes under zero, where the system is more like two-fermion-bound-molecule system.  This only happens when zero-energy scattering amplitude $a_s$ sufficient negative and it is more energetically favorable to form bound-state for higher-energy pairs.  We expect that this shift should goes proportional to fermi energy, which is the case.  

\subsection{Case with Open-Channel Intrachannel Interaction}
When there is open-channel intrachannel interaction $U_{\vk\vk'}$, the universality point is not exact at the point where close-channel level at zero of open-channel ($\eta=E^0$), but shifted by $\kappa=\delta\mu_B\Delta_B$ (see \cite{Leggett}, also see Sec. \ref{sec:LeggettAndMessiah}).  Furthermore, open-channel dominates around universality point with range $\delta_c$.  $delta_c\equiv\frac{\kappa^2}{\hbar^2/m_ra_{bg}^2}$ can be much smaller than $\kappa$ if $\kappa\ll\hbar^2/m_ra_{bg}^2$ (not necessarily the experimental cases).  In such a case, there are possibility that Fermi energy $E_F$ larger than $\delta_c$ but smaller than $\kappa$.  And this needs special care.  The region where close-channel bound-state is below open-chanel zero $\eta-E^0<0$ (extreme extreme BEC) and close-channel obviously dominates is not relatively trivial and note very interesting.  the interesting region is all in $\eta>E^0$.  

\subsection{Shizhong's comment}
He suggested me to look into \eef{eq:20100915:GF} for the relation between $\Delta$ and $\alpha$. Multiply both side with $\phi^0_k$.  LHS is $\alpha$, RHS $\avs{\phi^0}{Y}{F}$ relates to the two-body inter-channel coupling as $F_k$ should be in the same shape as two-body open-channel in short range except the normalization.  

He also suggested me to look into \eef{eq:20100909:fullgap} and maybe define some new parameter to absorb the $\sqrt{1-4 F_{\vk}^2-4 G_{\vk}^2}$ in denominator.  

\subsection{$\alpha$}
In two-body, close-channel weight, $\alpha_{2}$, determines s-wave scattering length $a_{s}$.  One obvious limit is that $\alpha\rightarrow1$ at BEC limit.  

Unlike \cite{Leggett} Appendix 4A, where open-channel normalization is always $\chi(r)\rightarrow1-r/a_{s}$\footnote{This normalization ensures that the $\chi(0)=1$.  We assume the short-range of the wave function is the same, or varies little except normalization. By fixing $\chi(0)=1$, we ensure the short-range part takes the same normalization.  }, here we always have the total number fixed, i.e., open channel weight ($F_{k}$) keeps shrinking from BCS-end to BEC-end. 

The region we are interested is where $|a_{s}|$ is substantially larger than $|a_{bg}|$.  However, this does not imply that close-channel takes more weight all the time, but does that mean the combination $Y_{kk'}\chi^{c}_{k'}$ is always large? The coupling strength, $Y_{kk'}$ plays some role in the relation between $\Delta$ and $\alpha$; in two-body level, it plays similar role between $\alpha$ and $a_{s}$, how to relate them?

\subsection{$\alpha$ in two-body}
In two-body, the close-channel coefficient is proportional to $1/(e-\tilde{\delta})$ (See (4.A.11) in \cite{Leggett}.  This is fine for the region $\mu<0$. At this region, what we really want is the bound state where $E<0$ and at BEC extreme, close-channel dominates and $E$ follow $\mu$.  So this coefficient is very large, indicating that close-channel indeed dominates. In this region, the scattering state ($E=0$) is actually the other branch and as the detuning $\tilde{\delta}$ increases, close-channel coefficient decreases.  