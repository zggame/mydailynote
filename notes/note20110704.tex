Tan's universal papers\cite{Tan2008-1,Tan2008-2} basically just match a free part into boundary condition ($\eta$ integral).  And he split the space into short-range part ($I(\epsilon)$) and long-range free part ($D(\epsilon)$).  And finally, many body wave function in free part ($D(\epsilon)$) just need to match the Bethe-Pierels condition, i.e., $1/r-1/a$, the s-wave scattering condition.  If we took the close-channel part is small and put all things into the $I(\epsilon)$, the open-channel match this boundary condition.  How the narrow resonance plays here? This interpretation leads only to the broad resonance. To get to narrow resonance, we have to relax 


In Tan's work and Combescot's following work\cite{CombescotTan}, the limiting is for all wave function in s-wave scattering state. \emph{What happens for the real case where the weight in potential range $r_{c}$ is not negligible?} The theory breaks down or not?  What does exact the short-range potential mean here?

(Could it be just universal long-wave length behavior? )

One thing somewhat misleading in these papers is that the ``high-momentum'' limit $C/k^{4}$.  This \emph{high-momemtum} is not really to infinite, it is capped by the $1/r_{c}$, only for momentum lower enough that insensitive to the structure within potential.  However, when taking $a_{s}$ limit, $r_{c}\rightarrow0$, this limit can be formally taken as infinity and we can call it \emph{high-momentum}.  Nevertheless, it is ``high'' comparing to many-body characteristic.  


