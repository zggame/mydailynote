\subsection{}
One simple attempt is to use Eq. (\ref{eq:100816:F0N},\ref{eq:100816:BcsBlock}),  with $N=1$.  This is the coherent superposition of lots of modified BCS ansatz, where one $k$ state is blocked by the close-channel.  And furthermore, we can image to extend $N$ to large number but still much smaller than the total number for situation like the broad-resonance.  

A close-look at Eq. (\ref{eq:100816:BcsBlock}) shows that this is one term of the expansion of the three-species ansatz
\begin{equation}\tag{\ref{eq:ansatz}}
 \ket{\Psi}=\prod_\vk\br{u_\vk+v_\vk{}a^\dg_\vk{}b^\dg_{-\vk}+w_\vk{}a^\dg_\vk{}c^\dg_{-\vk}}\ket{0}
\end{equation}
The expansion of the above equation includes infinite number of different $\ket{F_{0}}$ with different particle-numbers and composition.  So there are a few questions in mind:

\begin{itemize}
\item Is the coherent nature of them important?
\item If it is easier to handle Eq. (\ref{eq:100816:BcsBlock}), how the distribution of different particle-numbers plays in the problem?  What kind of the distribution is it? 
\item For $N>1$, the renormalization is complicated as in the coboson handling.  How to treat them? 
\end{itemize}

Or we can expand Eq. (\ref{eq:100816:F0N}) further to something like 
\begin{equation}\label{eq:100823:F0N}
\ket{F_{0}}=\sum\alpha_{(k_1,k_{2},...)}\beta^{\dg}_{k_{1}}\beta^{\dg}_{k_{2}}\cdots\ket{0}
\end{equation}
where $\beta^{\dg}_{k}=a^{\dg}_{k}c^{\dg}_{-k}$.
The above equation is easier in terms of the renormalization.  

\subsection{}
When the close-channel bound-state, which might or might not be the eigenstate of the bare close-channel potential, is relatively small comparing to the inter-particle distance, one can image that an expansion of close-channel component as described above over \eef{eq:ansatz} takes a much large range in k-space ($1/a^{c}$).   So, if we are interesting only in the momentum range up-to or not too much larger than fermi momentum, \eef{eq:100823:F0N} is dominated by the one-pair term.  