% !TeX root =thesis.tex

\chapter{Introduction}
Feshbach resonance is widely used in the ultracold alkali gas experiments as an indirect way to tune interaction strength.  This unique ability gives physicist the opportunity to study the evolution of many-body system under change of interaction strength, which connects different systems originally under different frameworks.  Particularly for the fermion system, there was long theoretically work about uniform treatment over BEC and BCS \cite{Eagles,LeggettCrossover,Nozieres,RanderiaBEC}, and Feshbach resonance within the ultracold alkali gas provide the perfect grounds to verify it.  Within the lowest order, the theory works remarkably well, the experiments fit it qualitatively.  

Here we actually try to look into the idiosyncratic of the Feshbach resonance as comparing to a real ``Simple'' knob on the interaction strength.  Within the two-body description of Feshbach resonance, there is a parameter $\delta_c$ describing how close to resonance is necessary to have substantial weight in close-channel.  Combining this with a many-body problem, a simple question is how this energy scale compares to a typical many-body energy scale, Fermi energy.  If Fermi energy is much smaller comparing to $\delta_c$, (\emph{broad resonance}), close-channel can be safely ignored at many-body level and the problem can be well-described as a two-species fermion system with tunable interaction.  On the contrary, when Fermi energy is larger than or comparable to $\delta_c$, close-channel cannot be ignored in many-body level.  Nevertheless, another important thing in the problem is that we are dealing with a Feshbach resonance with a relatively tight-bound close-channel state, which is much smaller comparing to other many-body scale.  Therefore, we are never dealing with a true three(four)-species Fermion system, which probably requires quite different techniques to solve.

To complicate the problem even further, the real experiments system often has one common species.  Pauli exclusion prevents the same species occupy both channels.  This effect has no counter-part in two-body problem, and will be one of the central problems in this study. 

Roughly speaking, many-body nature brings two effects.  The first one closely associates with Fermi energy.  At low temperature, most fermions are inactive and only fermions close to fermi surface can participate, therefore, energy often needs to be measured from Fermi sea instead of zero as in two-body situation.  It always relates to Fermi Energy $E_F$.  This effect has been extensively studied in \cite{GurarieNarrow}.

The second effect is unique for the three-species problem.  Phase spaces of two channels are overlapped because of the Pauli exclusion from the common species. This effect is controlled by overlapping of states in two channels. We can estimate it roughly.  In the close-channel, bound-state is relatively small, and binding energy $E_b$ is close to absolute energy difference between two channels, $\eta$.  On the other hand, fermions in open-channel occupies the lowest states in the momentum space and therefore spread out in space.  The typical energy scale would be $E_F$.  Therefore a ratio $E_F/\eta$ would control such effect. 