\subsection{Tony's Comment}
\begin{itemize}
\item Adding Pauli Principle into the common approach of Feshbach where close-channel is simple bosons as perturbation maybe.  
\item In my approach, take the Pauli principle within close-channel away, what happens?
\item  He feels that the BEC side of the single-channel many-body ansatz should reduce to 2-body solution fairly straight-fowardly.  BCS might have some problems as it crosses from bound-state to no bound-state.  
\item  Given $\delta\ll\delta_{c}\ll\epsilon_{F}$, can we draw any concrete conclusion?
\item Write down the situations in two extremes.  
\item In the BEC limit( say $1/k_{F}a_{s}>10$), can we reduce things into the something like real molecular description, such as Pitaskii energy, Bogoliubov description, etc?  In both single channel and two-channel? 
\end{itemize}
