%%
%% This is file `thesis-ex.tex',
%% generated with the docstrip utility.
%%
%% The original source files were:
%%
%% uiucthesis2009.dtx  (with options: `example')
%% 
\def\fileversion{v2.25a} \def\filedate{2009/10/10}
%% Package and Class "uiucthesis2009" for use with LaTeX2e.
\documentclass[edeposit,fullpage,prequest]{uiucthesis2009}
\usepackage{amsfonts}
\usepackage{amssymb}
\usepackage{amsmath}
\usepackage{hyperref}
\usepackage{graphicx}% Include figure files
\usepackage{epstopdf}
\usepackage{subfig}
\usepackage{makeidx}
\usepackage[page,toc,title,titletoc]{appendix}


%\usepackage{comment}
%%\includecomment{mycomment}
%\specialcomment{mycommentzhu} {\begingroup\ttfamily\footnotesize}{\endgroup}
%%\excludecomment{mycomment}

%%adding comment, use the next line for disable comment
\newcommand{\mycomment}[1]{\textit{#1}}
%\newcommand{\mycomment}[1]{}

\newcommand{\vk}{\ensuremath{\mathbf{k}}}
\providecommand{\vr}{\ensuremath{\mathbf{r}}}
%\newcommand{\vec}[1]{\ensuremath{\mathbf{#1}}}

\newcommand{\gk}{\ensuremath{{g}(\mathbf{k})}}

\newcommand{\vp}{\ensuremath{\mathbf{p}}}
\newcommand{\gp}{\ensuremath{{g}(\mathbf{p})}}

\newcommand{\vq}{\ensuremath{\mathbf{q}}}

\newcommand{\Fo}{\ensuremath{\mathbf{F_0}}}


\newcommand{\E}{\ensuremath{\mathbf{E}}}
\newcommand{\A}{\ensuremath{\mathbf{A}}}
\newcommand{\J}{\ensuremath{\mathcal{J}}}

\newcommand{\ket}[1]{\ensuremath{\left|#1\right>}}
\newcommand{\bra}[1]{\ensuremath{\left<#1\right|}}

\newcommand{\twoe}{\ensuremath{2\epsilon_\vk-\E_1}}

\newcommand{\nth}[1]{\ensuremath{\frac{1}{#1}}}

\newcommand{\br}[1]{\ensuremath{\left(#1\right)}}
\newcommand{\mbr}[1]{\ensuremath{\left[#1\right]}}
\newcommand{\bbr}[1]{\ensuremath{\left\{#1\right\}}}


\newcommand{\tk}{\ensuremath{\tilde{k}}}

\newcommand{\kp}{\ensuremath{\ket{\Psi}}}

\newcommand{\av}[1]{\ensuremath{\bigl<{#1}\bigr>}}
\newcommand{\avv}[2][\nu]{\av{#1{\lvert{#2}\rvert}#1}}
\newcommand{\avt}[2]{\av{{#1}|{#2}}}
\newcommand{\avtu}[1]{\av{T_\tau#1}}

\newcommand{\Bop}{\ensuremath{\mathbf{B_0^+}}}
\newcommand{\Bmp}{\ensuremath{\mathbf{B_m^+}}}
\newcommand{\Bnp}{\ensuremath{\mathbf{B_n^+}}}
\newcommand{\Bo}{\ensuremath{\mathbf{B_0}}}
\newcommand{\Bopn}{\ensuremath{\mathbf{{B_0^+}^n}}}
\newcommand{\Bon}{\ensuremath{\mathbf{{B_0}^n}}}


\newcommand{\zmatrix}{\ensuremath{\br{\begin{smallmatrix}0&0\\0&0\end{smallmatrix}}}}
\newcommand{\fmtrx}[4]{\ensuremath{\br{\begin{smallmatrix}#1&#2\\#3&#4\end{smallmatrix}}}}
\newcommand{\smtrx}[6]{\ensuremath{\br{\begin{smallmatrix}#1&#2\\#3&#4\\#5&#6\end{smallmatrix}}}}

\newcommand{\vz}{\ensuremath{v^{\beta\alpha}_{\vk,\vk}}}
%\newcommand{\fz}{

\providecommand{\abs}[1]{\ensuremath{\lvert{#1}\rvert}}

\newcommand{\sg}[1][1]{\ensuremath{\sigma_\frac{#1}{2}}}

\newcommand{\rhof}{\ensuremath{\rho(\ef)}}
\newcommand{\omt}{\ensuremath{\tilde{\Omega}}}
\newcommand{\cht}{\ensuremath{\tilde{\chi_0}}}
\newcommand{\Atl}{\ensuremath{\abs{A}^{2l}}}
\newcommand{\ef}{\ensuremath{\epsilon_F}}

\newcommand{\lca}{\ensuremath{\ln\br{1+\frac{\cht}{\alpha}}}}

\newcommand{\com}[2]{\ensuremath{\mbr{#1,#2}}}
\newcommand{\D}{\ensuremath{\mathit{D}}}
\newcommand{\dg}{\ensuremath{\dagger}}

\providecommand{\lvk}{\ensuremath{1/\vk_F}}
\providecommand{\hm}{\ensuremath{\frac{\hbar^2}}{2m}}
\providecommand{\pdiff}[2]{\ensuremath{\frac{\partial{#1}}{\partial{#2}}}}
\providecommand{\dpdiff}[2]{\ensuremath{\frac{\partial^2{#1}}{\partial{{#2}^2}}}}

\providecommand{\H}{\ensuremath{\mathcal{H}}}
\providecommand{\wt}[1]{\widetilde{#1}}


%\newenvironment{unsure}
%	{\begin{itshape}}% begin code
%	{\end{itshape}}%                    end code
\newenvironment{unsure}{}{}



\begin{document}


\title{BEC-BCS Crossover with Feshbach Resonance for Three-Hyperfine-Species Model}
\author{Guojun Zhu}
\department{Physics}
\schools{B.S., University of Science and Technology of China, 2001\\
         M.S, University of Illinois at Urbana-Champaign, 2003}
\phdthesis
\advisor{Anthony Leggett}
\degreeyear{2011}
\committee{Professor Gordon Baym\\Professor Brian DeMarco\\Professor Anthony Leggett\\Professor Scott Willenbrock}
\maketitle

\frontmatter

%
\begin{abstract}
The BEC-BCS crossover problem was intensively studied both theoretically and experimentally largely thanks to the Feshbach resonance via which the effective interaction between alkali atoms is tuned.  In the Feshbach resonance, there is one characteristic energy scale, $\delta_c$. The closed-channel component only dominates when the detuning from resonance is more than $\delta_c$.  When the many-body scale (e.g. the Fermi energy, $E_{F}$) is larger than $\delta_c$, the closed channel is important enough to be included in the many-body theory.  Furthermore, when two channels share the same hyperfine species as in the two most available experiment systems (${}^6\text{Li}$ and ${}^{40}\text{K}$), the Pauli exclusion between two channels also needs to be taken into consideration of the many-body theory.  The current  thesis addresses this problem in details. A set of gap equations and number equations  are derived at the mean-field level.  The fermionic and bosonic excitation spectrums are also derived within this model.  In all the cases, assuming the bound-state of the closed-channel in resonance is much smaller than inter-particle distance, as well as the s-wave scattering length, $a_s$, we find that  the basic conclusion in the broad resonance is still valid. In every case, correction is  in the  order of ratio between the Fermi energy and the Zeeman energy difference between two channels, $E_F/\eta$.  
\end{abstract}

%% Create a dedication in italics with no heading, centered vertically
%% on the page.
\begin{dedication}
To Jie, Ethan and Chloe
\end{dedication}

%% Create an Acknowledgements page, many departments require you to
%% include funding support in this.
\chapter*{Acknowledgments}
First I would like to thank my adviser, Prof. Anthony J. Leggett.  This thesis would not be possible without his guidance and patience.  I would like to thank Prof. Monique Combescot from Paris especially for her help, kindness and lots of invaluable advise.  Also, I would like to thank Dr. Wei-Cheng Lee, Dr. Shizhong Zhang, Dr. Parag Ghosh,  Douglas
Packard  for their many discussions and suggestions.  Finally I would thank my wife and my family for the constant support and affection.  

%% The thesis format requires the Table of Contents to come
%% before any other major sections, all of these sections after
%% the Table of Contents must be listed therein (i.e., use \chapter,
%% not \chapter*).  Common sections to have between the Table of
%% Contents and the main text are:
%%
%% List of Tables
%% List of Figures
%% List Symbols and/or Abbreviations
%% etc.

\tableofcontents
%\listoftables
%\listoffigures

%% Create a List of Abbreviations. The left column
%% is 1 inch wide and left-justified
%\chapter{List of Abbreviations}
%
%\begin{symbollist*}
%\item[w.f.] wave function.
%\end{symbollist*}

%% Create a List of Symbols. The left column
%% is 0.7 inch wide and centered
\chapter{List of Symbols}

\begin{symbollist}[0.7in]
%\item[$\psi$] Wave function.
\item[$\phi_{i}$] Bound-state wave functions of the isolated close-channel, especially, $\phi_{0}$ is the one in resonance. (Page \pageref{eq:pathInt2:phi})
\item[$\Delta_{1,2}$] Order parameters for two channels. (Page \pageref{eq:pathInt2:identity})
\item[$h_{1\vk}$, $ h_{2\vk}$] (Open, closed) (anomalous) two-body correlation in a many-body system. (Page \pageref{eq:pathInt2:h2})
\item[$a_{s}$, $a_{s}^{(o)}$] (Open channel) s-wave scattering length. Subscript ${}^{(o)}$ is dropped when there is no ambiguity. (Page \pageref{sec:intro:as})
\item[$\eta$] Absolute detuning between two channels. Notice that $\eta=0$ is not where $a_{s}$ diverges. (Page \pageref{eq:intro:ham})
\item[$E_{b}$, $E_{b}^{(i)}$] Binding energy of the $i^{th}$ two-body (bound) eigen-wave-function in the isolated close-channel.  Superscript ${}^{(i)}$ is dropped when referring to the one in resonance. (Page \pageref{eq:intro:sch2})
\item[$\kappa$] Momentum scale of the resonant bound-state in the isolated close-channel.  $\kappa^{2}/2m=E_{b}$.
\item[$\delta_{c}$] Energy scale of the  detuning from resonance where the closed-channel takes substantial weight. (Page \pageref{eq:intro:deltaC})
\item[$a_{0}$] Average inter-particle distance, $a_{0}k_{F}\sim1$.
\item[$r_{c}$] Potential range.  All  potentials are zero outside $r_{c}$.
\item[$a_{c}$] Characteristic size of the bound state in resonance ($\phi_{0}$) if the close-channel is isolated. It is proportional  to the inverse of $\kappa$.
\item[$k_{F}$] Fermi momentum. $k_{F}=\hbar(3\pi^{2}n)^{1/3}$ in 3D.
\item[$\gamma_{i\vk}$] Correction of the fermionic excitation spectrum over $\pm{}E_{\vk}$ and $\epsilon_{\vk}+\eta$. (Page \pageref{eq:pathInt2:xiExpand})

\item[$E_{\vk}$] $E_{\vk}$ corresponds the elementary fermionic excitation energy for $\vk$ in the single channel.  $E_{\vk}=\sqrt{\epsilon_{\vk}^{2}+\Delta^{2}}$, where $\epsilon_{\vk}=\hbar^{2}k^{2}/2m$ is the kinetic energy. In two-channel model, it is defined similarly, $E_{\vk}=\sqrt{\epsilon_{\vk}^{2}+\Delta_{1}^{2}}$. However, it is only the zeroth order   energy for the two elementary fermionic excitation modes. (Page \pageref{eq:pathInt:G0})
\item[$\alpha$]  Ratio of the closed-channel correlation $h_{2\vk}$ to the normalized two-body closed-channel bound state wave function, $\phi_{\vk}$ at high momentum.  $h_{2\vk}\xrightarrow{\text{high momentum}}\alpha\phi_{\vk}$. (Page \pageref{eq:pathInt2:hphi})
\item[$n_{o}$, $n_{c}$] Density of the open-channel (closed-channel) atoms. 
\item[$\mathcal{V}_{0}$]  Total volume of the system. 
\end{symbollist}

\mainmatter
% !TeX root =thesis.tex

\chapter{Introduction}
Feshbach resonance is widely used in the ultracold alkali gas experiments as an indirect way to tune interaction strength.  This unique ability gives physicist the opportunity to study the evolution of many-body system under change of interaction strength, which connects different systems originally under different frameworks.  Particularly for the fermion system, there was long theoretically work about uniform treatment over BEC and BCS \cite{Eagle,LeggettCrossover,Nozieres,RanderiaBEC}, and Feshbach resonance within the ultracold alkali gas provide the perfect grounds to verify it.  Within the lowest order, the theory works remarkably well, the experiments fit it qualitatively.  

Here we actually try to look into the idiosyncratic of the Feshbach resonance as comparing to a real ``Simple'' knob on the interaction strength.  Within the two-body description of Feshbach resonance, there is a parameter $\delta_c$ describing how close to resonance is necessary to have substantial weight in close-channel.  Combining this with a many-body problem, a simple question is how this energy scale compares to a typical many-body energy scale, Fermi energy.  If Fermi energy is much smaller comparing to $\delta_c$, (\emph{broad resonance}), close-channel can be safely ignored at many-body level and the problem can be well-described as a two-species fermion system with tunable interaction.  On the contrary, when Fermi energy is larger than or comparable to $\delta_c$, close-channel cannot be ignored in many-body level.  Nevertheless, another important thing in the problem is that we are dealing with a Feshbach resonance with a relatively tight-bound close-channel state, which is much smaller comparing to other many-body scale.  Therefore, we are never dealing with a true three(four)-species Fermion system, which probably requires quite different techniques to solve.

To complicate the problem even further, the real experiments system often has one common species.  Pauli exclusion prevents the same species occupy both channels.  This effect has no counter-part in two-body problem, and will be one of the central problems in this study. 

Roughly speaking, many-body nature brings two effects.  The first one closely associates with Fermi energy.  At low temperature, most fermions are inactive and only fermions close to fermi surface can participate, therefore, energy often needs to be measured from Fermi sea instead of zero as in two-body situation.  It always relates to Fermi Energy $E_F$.  This effect has been extensively studied in \cite{GurarieNarrow}.

The second effect is unique for the three-species problem.  Phase spaces of two channels are overlapped because of the Pauli exclusion from the common species. This effect is controlled by overlapping of states in two channels. We can estimate it roughly.  In the close-channel, bound-state is relatively small, and binding energy $E_b$ is close to absolute energy difference between two channels, $\eta$.  On the other hand, fermions in open-channel occupies the lowest states in the momentum space and therefore spread out in space.  The typical energy scale would be $E_F$.  Therefore a ratio $E_F/\eta$ would control such effect. 

Alternatively, we can make a rough estimation on simple double-fermion molecule gas.  If we assume the molecule size is $r_{c}$ and the total number is $N$.  If assume further that the bound-state is close-to-threshold.  We will have wave function $A/(k^{2}+\kappa^{2})$, where $\hbar^{2}\kappa^{2}/2m=E_{b}$, (see Appendix \ref{sec:pathInt2:short-range}) and $\sum_{k=0}^{1/r_{c}}\psi^{2}\sim{}N$, now if we considering only particles within range $E_{F}\ll\kappa$, the total particle in it is roughly $N\cdot(k_{F}r_{c})$.  this number can be much smaller than $N$. Put this into the perspective of two-channel problem, the low-momentum is still dominated by open-channel component even when total weight of close-channel is comparable or higher than it of open-channel because close-channel is mostly in high-momentum state.    

\section{Two-body Feshbach resonance problem}
Here we will briefly review the Feshbach resonance in two-body setup, we will mostly follow treatment in \cite{Leggett}.
\section{Single-channel BEC-BCS crossover}
Here we will briefly review the BEC-BCS crossover in single channel.   Here the interaction strength is tuned along the crossover region.  This applies to the broad-resonance, where the close-channel weight is negligible and serves only to modify the effective strength in open-channel.  We will mostly follow treatment in \cite{Leggett}.
%\chapter{Path-Integral approach}


% !TeX root = subNote.tex

\subsection{Path integral approach for two-channel}
For two-channel problem, we write down the Hamiltonian as
\begin{equation}
\begin{split}
H&=\int{d^{d}r}\bigg\{\sum_{j}\bar\psi_{j}\mbr{\nth{2m}(-i\nabla)^{2}-\mu+\eta_{j}}\psi_{j}\\
	&\qquad-U\bar\psi_{a}(r)\bar\psi_{b}(r)\psi_{b}(r)\psi_{a}(r)-V\bar\psi_{a}(r)\bar\psi_{c}(r)\psi_{c}(r)\psi_{a}(r)\\
	&\qquad-\mbr{Y\bar\psi_{a}(r)\bar\psi_{b}(r)\psi_{b}(r)\psi_{a}(r)+h.c.}
	\bigg\}\\
 &=\int{d^{d}r}\bigg\{\sum_{j}\bar\psi_{j}\mbr{\nth{2m}(-i\nabla)^{2}-\mu+\eta_{j}}\psi_{j}
 	-(\bar\psi\bar\psi)\mtrx{U&Y\\Y^{*}&V}(\psi\psi)
\end{split}
\end{equation}
Here $\eta_{j}$ is the Zeeman energy of the specific hyperfine species.  a is the common species of two channels, (a,b) is the open channel and (a,c) is the close channel.  We can introduce a unitary transformation $Q$ (mixing two channels) to diagonalize the interaction matrix into diagonal matrix $A$.
\begin{equation}
Q^{\dg}AQ=\mtrx{U&Y\\Y^{*}&V}\equiv{}\tilde{U}
\end{equation}
The finite temperature action is 
\begin{equation}\label{eq:pathInt2:actionFermi}
S(\bar\psi,\psi)=\int^{\beta}_{0}d\tau\int{d^{d}r}\mbr{\sum_{j}\bar\psi_{j}(\partial_\tau-\nth{2m}\nabla^{2}-\mu+\eta_{j})\psi_{j}
-(\bar\psi\bar\psi)Q^{\dg}AQ(\psi\psi)}
\end{equation}

Now $(\psi\psi)$ is column vector and $(\bar\psi\bar\psi)$ is row vector
\begin{equation*}
(\bar\psi\bar\psi)=\mtrx{\bar\psi_{a}\bar\psi_{b}&\bar\psi_{a}\bar\psi_{c}}
\qquad(\psi\psi)=\mtrx{\psi_{b}\psi_{a}\\\psi_{c}\psi_{a}}
\end{equation*}

Similar as in Sec. \ref{sec:pathInt}, we can try to work on the two-channel problem.  Here, the bosonic field is a 2-component vector   and start from a fat identity
\begin{equation}\label{eq:pathInt2:indetity}
1=\int{D(\Delta,\bar\Delta)}\exp(-\int{dx}\Delta^{\dg}A^{-1}\Delta)
\end{equation}
here $x$ is four-coordinator,  $\int{dx}=\int^{\beta}_{0}d\tau\int{d^{d}r}$.  All the constant is absorbed into measure of functional integral of $D(\Delta,\bar\Delta)$.

\[
\Delta^{\dg}=(\bar\Delta_{1},\bar\Delta_{2})\qquad\Delta=\begin{pmatrix}\Delta_{1}\\\Delta_{2}\end{pmatrix}
\]
We can make a shift in $\Delta$
\begin{equation}
\Delta\longrightarrow\Delta-A\,Q(\psi\psi)
\end{equation}
Write it into the matrix form
\begin{equation*}
\mtrx{\Delta_{1}\\\Delta_{2}}\longrightarrow
	\mtrx{\Delta_{1}\\\Delta_{2}}-\mtrx{A_{11}&0\\0&A_{22}}\mtrx{Q_{11}&Q_{12}\\Q_{21}&Q_{22}}
	\mtrx{\psi_{b}\psi_{a}\\\psi_{c}\psi_{a}}
\end{equation*}
\begin{equation*}
\mtrx{\bar\Delta_{1},\bar\Delta_{2}}\longrightarrow
	\mtrx{\bar\Delta_{1},\bar\Delta_{2}}-
	\mtrx{\bar\psi_{a}\bar\psi_{b}&\bar\psi_{a}\bar\psi_{c}}
	\mtrx{Q_{11}^{*}&Q_{21}^{*}\\Q_{12}^{*}&Q_{22}^{*}}\mtrx{A_{11}&0\\0&A_{22}}
\end{equation*}

Note that in principle, $\bar{\Delta}_{i}$ is not simple complex conjugate of $\Delta_{i}$ as they are related to fermion field $\psi$ and $\bar\psi$, which are both Grassman numbers and is not relevant to each other.  But in our solution, we actually take them as complex conjugate.  Now the fat identity Eq. \ref{eq:pathInt2:indetity} becomes 
\begin{equation}
1=\int{D(\Delta,\bar\Delta)}\exp\big\{-\int{dx}
	[\Delta^{\dg}A^{-1}\Delta-(\bar\psi\bar\psi)Q^{\dg}\Delta-\bar\Delta{Q}(\psi\psi)+(\bar\psi\bar\psi)Q^{\dg}AQ(\psi\psi)]\big\}
\end{equation}
And we have 
\begin{equation}
\exp[\int{dx}(\bar\psi\bar\psi)Q^{\dg}AQ(\psi\psi)]
=\int{D(\Delta,\bar\Delta)}\exp\big\{-\int{dx}
	[\Delta^{\dg}A^{-1}\Delta-(\bar\psi\bar\psi)Q^{\dg}\Delta-\bar\Delta{Q}(\psi\psi)]\big\}
\end{equation}
This is ready to be apply to the original action in Eq. \ref{eq:pathInt2:actionFermi}, 
\begin{equation}\label{eq:pathInt2:actionMix}
S_{\tau}(\bar\Delta,\Delta,\bar\psi_{i},\psi_{i})=\int^{\beta}_{0}d\tau\int{d^{d}r}\bbr{\sum_{j}\bar\psi_{j}(\partial_\tau-\nth{2m}\nabla^{2}-\mu+\eta_{j})\psi_{j}
+[\Delta^{\dg}A^{-1}\Delta-(\bar\psi\bar\psi)Q^{\dg}\Delta-\bar\Delta{Q}(\psi\psi)}
\end{equation}
We can introduce a spinor space similar to Nambu spinor representation in single-channel superconductivity.  
\begin{equation}
\bar\Psi=\mtrx{\bar\psi_{a}&\psi_{b}&\psi_{c}}\qquad\Psi=\mtrx{\psi_{a}\\\bar\psi_{b}\\\bar\psi_{c}}
\end{equation}
the action can be rewritten in a more compact form
\begin{equation}\label{eq:pathInt2:actionMix}
S(\bar\Delta,\Delta,\bar\psi_{i},\psi_{i})=\int^{\beta}_{0}d\tau\int{d^{d}r}
	\mbr{\bar\Delta{}A^{-1}\Delta-\bar\Psi\mathcal{G}^{-1}\Psi}
\end{equation}
where 
\begin{equation}
\mathcal{G}^{-1}=
\begin{pmatrix}
-\partial_{\tau}+\nth{2m}\nabla^{2}+\mu-\eta_{a}&Q^{*}_{11}\Delta_{1}+Q^{*}_{21}\Delta_{2}&Q^{*}_{12}\Delta_{1}+Q^{*}_{22}\Delta_{2}\\
Q^{}_{11}\bar\Delta_{1}+Q^{}_{21}\bar\Delta_{2}&-\partial_{\tau}-\nth{2m}\nabla^{2}-\mu+\eta_{b}&0\\
Q^{}_{12}\bar\Delta_{1}+Q^{}_{22}\bar\Delta_{2}&0&-\partial_{\tau}-\nth{2m}\nabla^{2}-\mu+\eta_{c}
\end{pmatrix}
\end{equation}
The action is now bilinear to $\Psi$ and we can integrate it out formally
\begin{equation}
S(\bar\Delta,\Delta)=\int{dx}
	\br{\bar\Delta{}A^{-1}\Delta-\tr\ln{G}^{-1}}
\end{equation}
This form can be simplified by introducing mixture within two channels.
\begin{equation}
D\equiv\mtrx{D_{1}\\D_{2}}=Q^{\dg}\Delta
\end{equation}

We furthermore assume $\eta_{a}=\eta_{b}=0$, $\eta_{c}=\eta$. In frequency-momentum space, 
\begin{equation}\label{eq:pathInt2:nG}
\mathcal{G}^{-1}=i\omega_{n}I+
\begin{pmatrix}
-\xi_{k}&D_{1}&D_{2}\\
\bar{D}_{1}&+\xi_{k}&0\\
\bar{D}_{2}&0&+\xi_{k}+\eta
\end{pmatrix}
\end{equation}
here $\xi_{k}=\nth{2m}k^{2}-\mu$; and 
\begin{equation*}
\bar\Delta=\bar{D}\,Q^{\dg}\qquad\Delta=Q\,D
\end{equation*}
\[
\bar\Delta{}A^{-1}\Delta=\bar{D}Q^{\dg}A^{-1}QD=\bar{D}\tilde{U}^{-1}D
\]
Now we can change the functional variable into $D(\bar{D})$ 
\begin{equation}\label{eq:pathInt2:actionD}
S(\bar{D},D)=\int{dx}\br{\bar{D}\tilde{U}^{-1}D-\tr\ln{G}^{-1}}
\end{equation}
Comparing Eq. \ref{eq:pathInt2:nG} with Eq. \ref{eq:canonical:M} in Bogoliubov transformation treatment, we  see that $D_{1,2}$ have simple physical meaning in the mean-field expression of $\nG_{0}$ and therefore are more  direct counterpart of order parameter $\Delta$ in single-channel problem. 



\subsubsection{Diagonalize Green's function, Fermionic mode and Bogoliubov transform\label{sec:diagonalGreen}}
Here we will use the approach in Sec. \ref{sec:diagonalizeGreen1} for our problem.  In current problem, we need to diagonalize a $3\times3$ matrix Eq. \ref{eq:pathInt2:nG}, in another word, we need to figure out the Bogoliubov canonical transformation and the quasiparticle spectrum for this problem.   This involves solving a cubic equation, and an exact solution exists in principle.  However,  it offers little intuition in write the exact result, instead we content to find the approximated result when spectrum does not change too much from the broad-resonance, where the only effect of close-channel is modifying the effective interaction of open-channel. 


Within the assumption that spectrum  deviates not too much from na\"{i}ve broad-resonance solution, we  break down the unitary transformation into two 
pieces. 
\begin{equation}
B_k=L_k^{\dg}T_k^{\dg}G_k^{-1}T_kL_k
\end{equation} 
Here $T$ and $L$ are both unitary transformation.  We take $T$ as the transformation at lowest order, i.e., when we can ignore Pauli exclusion between channels. 
\begin{equation}
T_k=\mtrx{u_k&v_k&0\\-v_k&u_k&0\\0&0&1}
\end{equation}
Here we define 
\begin{gather}
E_{\vk}=(\xi_{\vk}^{2}+D_{1}^{2})^{1/2}\\
v_{\vk}^{2}=1-u_{\vk}^{2}=\nth{2}\br{1-\frac{\xi_{\vk}}{E_{\vk}}}
\end{gather}

In the broad-resonance, $L$ is simply identity matrix.  Here we will try to approximate it to the first order correction due to Pauli exclusion between two-channel.  
Apply $T_k$ onto $G^{-1}$, we have 
\begin{equation}\label{eq:pathInt2:G2}
T_k^{\dg}G_k^{-1}T_k=i\omega_nI+\mtrx{-E_k&0&u_kD_2\\0&+E_k&v_kD_2\\u_kD_2&v_kD_2&+\xi_k+\eta}
\end{equation}
We regard the off-diagonal elements as perturbation.  This matrix can then be diagonalized with another unitary transformation $L_{\vk}$ within the first order of $D_{i}^{2}/(E\eta)$
\begin{equation}
\begin{split}
B_{\omega_{n},\vk}&=i\omega_{n}I-
	\mtrx{\xi_{1}{}_{\vk}&0&0\\0&\xi_{2}{}_{\vk}&0\\0&0&\xi_{3}{}_{\vk}}\\
	&\approx{}i\omega_{n}I+
	\mtrx{-E_{\vk}-\frac{D_{1}^{2}}{\eta}&0&0\\
	0&E_{\vk}-\frac{D_{2}^{2}}{\eta}&0\\0&0&\xi_{\vk}+\eta+\frac{D_{1}^{2}+D_{2}^{2}}{2\eta}}
%	&=
%	\mtrx{i\omega_{n}-E_{\vk}&0&0\\0&i\omega_{n}+E_{\vk}&0\\0&0&i\omega_{n}+\eta}
%	+\mtrx{-\frac{D_{1}^{2}}{\eta}&0&0\\0&-\frac{D_{2}^{2}}{\eta}&0\\0&0&+\frac{D_{1}^{2}+D_{2}^{2}}{2\eta}}\\
%	&\equiv{}B^{(0)}_{\vk}+B^{(1)}_{\vk}
\end{split}	
\end{equation}
\begin{equation}\label{eq:pathInt2:L1}
L_{\vk}\approx{}I+
\mtrx{0&-\frac{D_{1}{}D_{2}{}}{4E^{2}_{\vk}}&u_{\vk}\\
\frac{D_{1}{}D_{2}{}}{4E^{2}_{\vk}}&0&v_{\vk}\\
-u_{\vk}&-v_{\vk}&0
}\frac{D_{2}{}}{\eta}
\equiv{}I+\delta_{k}\qquad
L^{\dg}_{\vk}=I-\delta_{\vk}
\end{equation}
Here we use $uv=D_{1}/2E$.  Please see Appendix \ref{sec:diagonalize} for details.  Note that $L$ and $L^{\dg}$ are unitary only to the first order of $D_{i}/\eta$
And now it is easy to express the Green's function
\begin{equation}
G_{\vk}=T_{\vk}L_{\vk}B_{\vk}^{-1}L_{\vk}^{\dg}T_{\vk}^{\dg}
\end{equation}
This is ready to be expanded over the perturbation in order of  $D_{i}^{2}/(E\eta)$ or $D_{i}/\eta$.  It is easy to see that all $\omega_{n}$ dependence concentrates on $B_{\vk}$, which simplifies the Matsubara frequency summation considerably.   
\begin{equation}\label{eq:pathInt2:Gexpand}
G_{k}=T_{\vk}B_{\omega_{n},\vk}^{-1}T_{\vk}^{\dg}+T_{\vk}\delta_{\vk}B_{\omega_{n},\vk}^{-1}T_{\vk}^{\dg}
	-T_{\vk}B_{\omega_{n},\vk}^{-1}\delta_{\vk}T_{\vk}^{\dg}=G_{\vk}^{(0)}+G_{\vk}^{(1)}
\end{equation}

From another point, we can interprate the above transform $T_\vk{}L_\vk$ as Bogoliubov canonical transformation and $B$ gives us spectrum of fermionic quasi-particle excitation.  
%We only need to keep the zeroth order term for $B_{\vk}^{-1}$ for first order expansion.  

\subsubsection{Mean field result}
 Use the same techniques as Eq. (\ref{eq:pathInt:diffTr}), we have two equations for $D_{1}$ and $D_{2}$,
 \begin{align}
\frac{\delta}{\delta{}D_{1}}:&\qquad&
(\tilde{U}^{-1})_{11}\bar{D}_{1}+(\tilde{U}^{-1})_{21}\bar{D}_{2}-\tr\mbr{{G_{0}}\cdot\cmtrx{0&1&0\\0&0&0\\0&0&0}}=0\\
\frac{\delta}{\delta{}D_{2}}:&\qquad&
(\tilde{U}^{-1})_{12}\bar{D}_{1}+(\tilde{U}^{-1})_{22}\bar{D}_{2}-\tr\mbr{{G_{0}}\cdot\cmtrx{0&0&1\\0&0&0\\0&0&0}}=0
 \end{align}


 If we take $D$ as real constant\footnote{Actually $D_{2}{_{\vk}}$ cannot be constant at high momentum.  However, for the momentum we are interested, i.e. the momentum lower or in the order of Fermi momentum, it slowly varies.  Therefore fairly it is reasonable to take it as constant.},     we can find the mean field result.  Usting Eq. (\ref{eq:pathInt2:Gexpand}),
 
 \begin{equation}\label{eq:pathInt2:G0}
 \begin{split}
 G_{0}=&
 \begin{pmatrix}
 {g_{1\,\vk}}&
-u_{k}v_{k}{g_{2\,\vk}}&0\\
-u_{k}v_{k}{g_{2\,\vk}}& {g_{3\,\vk}}&0\\
  0&0&\nth{i\omega_{n}-\xi_{3}{}_{\vk}}
 \end{pmatrix}\\
&+\frac{D_{2}}{\eta}
\begin{pmatrix}
\frac{D_{1}^{2}D_{2}}{4E_{\vk}^{3}}g_{2\,\vk}&\frac{D_{1}D_{2}\xi_{\vk}}{4E_{\vk}^{3}}g_{2\,\vk}&-g_{1\,\vk}+\nth{i\omega_{n}-\xi_{3}{}_{\vk}}\\
\frac{D_{1}D_{2}\xi_{\vk}}{4E_{\vk}^{3}}g_{2\,\vk}&-\frac{D_{1}^{2}D_{2}}{4E_{\vk}^{3}}g_{2\,\vk}&u_{k}v_{k}{g_{2\,\vk}}\\
-g_{1\,\vk}+\nth{i\omega_{n}-\xi_{3}{}_{\vk}}&u_{k}v_{k}{g_{2\,\vk}}&0
\end{pmatrix}
\end{split}
 \end{equation}
\begin{gather}
g_{1}{}_{\vk}=\frac{u_{\vk}^{2}}{i\omega_{n}-\xi_{1}{}_{\vk}}+\frac{v_{\vk}^{2}}{i\omega_{n}-\xi_{2}{}_{\vk}}\\
g_{2}{}_{\vk}=\nth{i\omega_{n}-\xi_{1}{}_{\vk}}-\nth{i\omega_{n}-\xi_{2}{}_{\vk}}\\
g_{3}{}_{\vk}=\frac{v_{\vk}^{2}}{i\omega_{n}-\xi_{1}{}_{\vk}}+\frac{u_{\vk}^{2}}{i\omega_{n}-\xi_{2}{}_{\vk}}
\end{gather}

 \begin{align*}
\tr\mbr{G_{0}\cdot\cmtrx{0&1&0\\0&0&0\\0&0&0}}&=
\sum_{\omega_{n}}\sum_{\vk}
	\mbr{(\nth{i\omega_{n}-\xi_{1}{}_{\vk}}-\nth{i\omega_{n}-\xi_{2}{_{\vk}}})
	(-\frac{D_{1}}{2E_{\vk}}+\frac{D_{1}D_{2}^{2}\xi_\vk}{4E_{\vk}^{3}\eta})}\\
	&=\sum_{\vk}(\frac{D_{1}}{2E_{\vk}}-\frac{D_{1}D_{2}^{2}\xi_\vk}{4E_{\vk}^{3}\eta})\\
\tr\mbr{G_{0}\cdot\cmtrx{0&0&1\\0&0&0\\0&0&0}}&=
\sum_{\omega_{n}}\sum_{\vk}
\mbr{\nth{i\omega_{n}-\xi_{3}{}_{\vk}}-
\frac{u_{\vk}^{2}}{i\omega_{n}-\xi_{1}{}_{\vk}}-\frac{v_{\vk}^{2}}{i\omega_{n}-\xi_{2}{}_{\vk}}}\frac{D_{2}}{\eta}\\
&=\sum_{\vk}\frac{D_{2}}{\eta}u_{\vk}^2
  \end{align*}
Here we take the interest only in $T=0$, so we only need to consider the negative frequencies ($\xi_{2\,\vk}$, $\xi_{3\,\vk}$) for summation of Matsubara frequency. Note that the second summation diverges badly in high-momentum. %this is because we take $D_2$ as constant.  It decreases at high-energy in the scale of $\eta$ and needs to be regularize carefully.  
We notice that this term is controlled by parameter $D_{2}/\eta$, this actually goes back to the fact that we only keep the first order in $L$ expansion (Eq. \eqref{eq:pathInt2:L1}), which is only valid for energy smaller or in order of Fermi energy.  In a more careful study, this term should be like $D_{2}/(\epsilon_{k}+\eta)$, is approximately $D_{2}/\eta$ when the interesting region  is lower or at the Fermi energy.   We can reestablish the $F_{k}\propto1/\epsilon_{k}$ if we retain all terms in the expansion of $L$, i.e. inverting Green's function $G$ exactly.       Indeed it should be just proportional to simple bounded two-body solution of isolated close-channel, $\phi_{0\,\vk}$ at high-momentum, which is not of interest for the many-body problem. 
 Another interesting thing about this term is the $u_\vk$ factor, which is small below chemical potential $\mu$ in BCS side.  This shows the fact that the low momentum is filled mostly by open-channel and close-channel is crowded out.  However, this does not affect close-channel too much as it is much more extended in momentum space and its occupation over each level is low due to the smaller size of close-channel bound state.  With above result, we can rewrite gap equations as 
\begin{equation*}
\tilde{U}^{-1}D-\mtrx{\sum_{\vk}(\frac{D_{1}}{2E_{\vk}}-\frac{D_{1}D_{2}^{2}\xi_\vk}{4E_{\vk}^{3}\eta})\\\sum_{\vk}\frac{D_{2}}{\eta}u_{\vk}^2}=0
\end{equation*}
 We can multiply it with $\tilde{U}$ and we have 
\begin{equation}\label{eq:pathInt2:meanfield}
\mtrx{D_1\\D_2}=\mtrx{U&Y\\Y^{*}&V}\sum_{\vk}\mtrx{\frac{D_{1}}{2E_{\vk}}-\frac{D_{1}D_{2}^{2}\xi_\vk}{4E_{\vk}^{3}\eta}\\
\frac{D_{2}}{\eta}u_{\vk}^2}
\end{equation}
 This equation can be renormalized in a very similar fashion as variation method. Notice that the second term in the first component describes the Pauli exclusion between two channels.  

\subsection{Renormalizing mean-field equation}
We define two quantities for the summand in the mean-field equation Eq. \ref{eq:pathInt2:meanfield}.
\begin{gather}
F_{1\,\vk}=\frac{D_{1}}{2E_{\vk}}-\frac{D_{1}D_{2}^{2}\xi_\vk}{4E_{\vk}^{3}\eta}\\
F_{2\,\vk}=\frac{D_{2}}{\eta}u_{\vk}^2\label{eq:pathInt2:F2k}
\end{gather}
Considering the argument from last section, we modify equation of $F_2$ to the following,
\begin{equation}
\tilde{F}_{2\,\vk}=\frac{D_{2}}{\eta+2\epsilon_{\vk}}u_{\vk}^2\label{eq:pathInt2:F2kMod}
\end{equation}
And now $F_2$ has the same behavior at high momentum as $F_1$, actually this is the behavior we expected for $\epsilon_k<\eta$, it falls off even faster beyond energy scale of $\eta$, which is determined by the specific shape of close-channel potential.

We can rewrite the mean-field equation as
\begin{gather}
D_{1}=\sum_{\vk}(U F_{1\,\vk}+Y \tilde{F}_{2\,\vk})\label{eq:pathInt2:D2}\\
D_{2}=\sum_{\vk}(Y^{*} F_{1\,\vk}+V \tilde{F}_{2\,\vk})\label{eq:pathInt2:D2}
\end{gather}
Here we see the $F_{1\,\vk}$ and $\tilde{F}_{2}$  both go as $1/\epsilon_{\vk}$  at high-momentum.  %To resolve divergence in the summation of $F_{2\,\vk}$ we restore the momentum dependence on $D_{2\,\vk}$ and therefore $F_{2\,\vk}$.  
  And we can see $\frac{D_{2}}{\eta}$ is actually a good approximation at low-momentum, where kinetic energy is much smaller than $\eta$.  We can rewrite $\tilde{F}_{2}=\alpha\phi_{0\,\vk}u_{\vk}^{2}$. Eq. (\ref{eq:pathInt2:D2}) can be rewritten as
\begin{equation*}
\eta{}F_{2\,\vp}=\sum_{\vk}(Y^{*} F_{1\,\vk}+V F_{2\,\vk})\label{eq:pathInt2:D2}
\end{equation*}
comparing this to a two-body \sch equation
\begin{equation*}
-E_{0}\phi_{0\,\vp}=\epsilon_{\vp}\phi_{0\,\vp}-\sum_{\vk}V \phi_{0\,\vk}\label{eq:pathInt2:D2}
\end{equation*}
where $E_{0}$ is the binding energy of two-body bound state of isolated close-channel.   


\subsection{Collective mode}
%\subsection{Simple approach for collective mode}
With the above arrangement, it is easy to study the collective mode of the system, which corresponding the second order expansion over $D$. We introduce the deviation over $\nG=G_{0}^{-1}+K_{q}$, where $G_{0}$ is described by real constant $D_{1,2}$
\begin{equation}
K_\vq=\mtrx{0&\theta_{1\,\vq}&\theta_{2\,\vq}\\\theta_{1\,-\vq}^*&0&0\\\theta_{2\,-\vq}^*&0&0}
\end{equation}
Follow the same approach in single-channel (Sec. \ref{sec:collective1}), we need to calculate $\tr(\hat{G_{0}}\hat{K}\hat{G_{0}}\hat{K})$, 
\begin{equation}
\tr({G_{0\,k}}{K_{q}}{G}_{0\,k+q}{K_{-q}})=\Theta_{q}^{\dg}M_{q}\Theta_{q}
\end{equation}
where 
\begin{equation}
\Theta_{q}=\mtrx{\theta^{*}_{1\,\vq}\\\theta^{}_{1\,-\vq}\\\theta^{*}_{2\,\vq}\\\theta^{}_{2\,-\vq}}
\qquad
\Theta_{q}^{\dg}=\mtrx{\theta^{}_{1\,\vq}&\theta^{*}_{1\,-\vq}&\theta^{}_{2\,\vq}&\theta^{*}_{2\,-\vq}}
\end{equation}
At the lowest order, if we take $L_{\vq}=I$,
\begin{equation}
M^{(0)}_{q}=\sum_{\vk}
\begin{pmatrix}
-u_{\vk}^{2}u_{\vk+\vq}^{2}f_{1\,q}+v_{\vk}^{2}v_{\vk+\vq}^{2}f_{2\,q}&u_{\vk}v_{\vk}u_{\vk+\vq}v_{\vk+\vq}(f_{2\,q}-f_{1\,q})&0&0\\
u_{\vk}v_{\vk}u_{\vk+\vq}v_{\vk+\vq}(f_{2\,q}-f_{1\,q})&u_{\vk}^{2}u_{\vk+\vq}^{2}f_{2\,q}-v_{\vk}^{2}v_{\vk+\vq}^{2}f_{1\,q}&0&0\\
0&0&-u_{\vk}^{2}f_{3\,q}&0\\
0&0&0&u_{\vk+\vq}^{2}f_{4\,q}
\end{pmatrix}
\end{equation}
where
\begin{gather}
f_{1\,q}=\nth{i q_{l}+\xi_{1\,\vk}-\xi_{2\,\vk+\vq}}\\
f_{2\,q}=\nth{i q_{l}-\xi_{1\,\vk+\vq}+\xi_{2\,\vk}}\\
f_{3\,q}=\nth{i q_{l}+\xi_{1\,\vk}-\xi_{3\,\vk+\vq}}\\
f_{4\,q}=\nth{i q_{l}-\xi_{1\,\vk+\vq}+\xi_{3\,\vk}}
\end{gather}
The two channels are completely decoupled at this level and the upper quarter is exactly same as single channel case if we ignore the correction over quasi-particle spectrum.   We can also expand each $f_{i\,q}$ to get the first order about $D_{2}/\eta$. (We take $q=0$ for it. See Appendix \ref{sec:expansionM})
\begin{equation}
M^{(1a)}=\sum_{\vk}
\begin{pmatrix}
(1-\frac{D_{1}^{2}}{2E_{\vk}^{2}})\frac{D_{1}^{2}-D_{2}^{2}}{4E_{\vk}^{2}\eta}
&-\frac{D_{1}^{2}}{4E_{\vk}^{2}}\frac{D_{1}^{2}-D_{2}^{2}}{4E_{\vk}^{2}\eta}&0&0\\
-\frac{D_{1}^{2}}{4E_{\vk}^{2}}\frac{D_{1}^{2}-D_{2}^{2}}{4E_{\vk}^{2}\eta}
&(1-\frac{D_{1}^{2}}{2E_{\vk}^{2}})\frac{D_{1}^{2}-D_{2}^{2}}{4E_{\vk}^{2}\eta}&0&0\\
0&0&\nth{2}(1+\frac{\xi_{\vk}}{E_{\vk}})\frac{3D_{1}^{2}+D_{2}^{2}}{2\eta(E_{\vk}+\xi_{\vk}+\eta)}&0\\
0&0&0&\nth{2}(1+\frac{\xi_{\vk}}{E_{\vk}})\frac{3D_{1}^{2}+D_{2}^{2}}{2\eta(E_{\vk}+\xi_{\vk}+\eta)}
\end{pmatrix}
\end{equation}

To be consistent, we also need to expand $L_{q}$ to the first order of  $D_{2}/\eta$ as well.  
\begin{equation}
\begin{split}
M^{(1b)}=&\qquad\frac{D_{2}}{\eta}\sum_{\vk}\\
&\begin{pmatrix}
-\frac{D_{1}^{2}D_{2}\xi_{\vk}}{4E^{5}_{\vk}}&-\frac{D_{1}^{2}D_{2}\xi_{\vk}}{2E^{5}_{\vk}}
&\frac{D_{1}\xi_{\vk}}{4E_{\vk}^{3}}&\frac{D_{1}}{2E_{\vk}}(\nth{E_{\vk}+\xi_{\vk}+\eta}+\nth{2E_{\vk}})\\
-\frac{D_{1}^{2}D_{2}\xi_{\vk}}{4E^{5}_{\vk}}&-\frac{D_{1}^{2}D_{2}\xi_{\vk}}{2E^{5}_{\vk}}
&\frac{D_{1}}{2E_{\vk}}(\nth{E_{\vk}+\xi_{\vk}+\eta}+\nth{2E_{\vk}})&\frac{D_{1}\xi_{\vk}}{4E_{\vk}^{3}}\\
\frac{D_{1}\xi_{\vk}}{4E_{\vk}^{3}}&\frac{D_{1}}{2E_{\vk}}(\nth{E_{\vk}+\xi_{\vk}+\eta}+\nth{2E_{\vk}})
&-\frac{D_{1}^{2}D_{2}}{4E_{\vk}^{3}(E_{\vk}+\xi_{\vk}+\eta)}&0\\
\frac{D_{1}}{2E_{\vk}}(\nth{E_{\vk}+\xi_{\vk}+\eta}+\nth{2E_{\vk}})&\frac{D_{1}\xi_{\vk}}{4E_{\vk}^{3}}
&0&-\frac{D_{1}^{2}D_{2}}{4E_{\vk}^{3}(E_{\vk}+\xi_{\vk}+\eta)}\\
\end{pmatrix}
\end{split}
\end{equation}
The interesting thing here is that if we only takes the first order of $D_{2}/\eta$, the two-channel is still decoupled when we write down the secular equation.  And the correction of the elements at the upper-left corner ($\theta_{1\,\pm{q}}$) is the same and that will gives an small finite value for $\omega_{q}$ at $q=0$. \emph{This conclusion is however problemetic as it violates the f sum rule.  Phase flucturation should be a Goldstone mode without mass.(see Sec. \ref{sec:phaseFluctuation}) }

\subsubsection{Alternative Approach: Phase fluctuation mode \label{sec:phaseFluctuation}}
 Action of $D$, $S(\bar{D_i},D_i)$ Eq. \eqref{eq:pathInt2:nG} and Eq. \eqref{eq:pathInt2:actionD}, is invariant over the phase change of $D_{1\,\vk}$ and $D_{2\,\vk}$ simultaneously. We therefore conclude that we should have a massless (Goldstone) mode.  And particularly, this mode is related to the local phase invariance. 
\begin{equation*}
D_{i}(x)\rightarrow{}D_{i}(x)e^{i2\theta(x)}\qquad{}
\bar{D}_{i}(x)\rightarrow{}\bar{D}_{i}(x)e^{-i2\theta(x)}
\end{equation*}
This corresponding to the translation of fermionic operator $\psi$
\begin{equation*}
\psi_{i}(x)\rightarrow{}\psi_{i}(x)e^{i\theta(x)}\qquad{}
\bar{\psi}_{i}(x)\rightarrow{}\bar{\psi}_{i}(x)e^{-i2\theta(x)}
\end{equation*}
Note that the phase $\theta(x)$ is common for the different components. The rest three degree of freedom of $D_i$, magnitude variation of two $D_i$ and internal phase between two $D_i$, change the action accordingly, while the overall phase $\theta(x)$ leaves action invariant.  We here isolate this special degree of freedom while leave others at their mean-field values. With a phase shift, we can rewrite the action (taken mean-field value $D^{(0)}$) (here we closely follow Nagaosa\cite{Nagaosa})
\begin{subequations}
\begin{align}\label{eq:pathInt2:actionPhase}
S[\theta,\bar\psi_{i},\psi_{i}]=&S_0[\bar\psi_{i},\psi_{i}]+S_1[\theta,\bar\psi_{i},\psi_{i}]+S_2[\theta,\bar\psi_{i},\psi_{i}]\\
S_0[\bar\psi_{i},\psi_{i}]=&\int{dx}
\Big\{\sum_{j}\bar\psi_{j}(\partial_\tau-\nth{2m}\nabla^{2}-\mu+\eta_{j})\psi_{j}\nonumber\\
&\quad+[D^{(0)}{}^{\dg}\tilde{U}^{-1}D^{(0)}-(\bar\psi\bar\psi)D^{(0)}-{D^{(0)}}{}^{\dg}(\psi\psi)\Big\}\\
S_1[\theta,\bar\psi_{i},\psi_{i}]=&\int{dx}\sum_{j}\Big\{
   i\,\bar\psi_{j}(\partial_{\tau}\theta)\psi_{j}+\nabla\theta\cdot\nth{2mi}[\bar\psi_{j}\nabla\psi_{j}-(\nabla\bar{\psi}_{j}\psi_{j})]\Big\}\\
S_2[\theta,\bar\psi_{i},\psi_{i}]=&\int{dx}\sum_{j}\nth{2m}(\nabla\theta)^{2}\bar\psi_{j}\psi_{j}
\end{align}
\end{subequations}
Note that here $D^{(0)}$ is a constant 2-component vector, no longer functional variables.  Here we see that $S_{0}$ has the same form as before except it only takes the mean field value of $D$ and it is described by the same correlation $G_{0}$ (Eq. \eqref{eq:pathInt2:G0}, \eqref{eq:pathInt2:nG}).  We can regard $S_{1}$ and $S_{2}$ as perturbation for the so-called gradient expansion on $\nabla\theta$.  It is then obvious that $S_{1}$ is in the first order while $S_{2}$ is in the second order.  Use the same spinor representation as before, $S$ is bilinear of $\psi$ and therefore we can formally integrate out $\psi$. 
\begin{equation}
S[\theta]=const.+\ln\det\nG(\theta)
\end{equation}
 We write out the formal Green function according to above action (with respect to Nambu-like spinor)
\begin{subequations}
\begin{align}
\nG=&G_{0}^{-1}+K_{1}+K_{2},\\
K_{1\, k,k'}=
	&\nth{({\beta{}V)}^{1/2}}(\omega_n-\omega_{n'})\theta(k-k')\sigma_3+
		\nth{{(\beta{}V)}^{1/2}}i\frac{(\vk-\vk')\cdot(\vk+\vk')}{2m}\theta(k-k')\hat{1}\\
K_{2\, k,k'}=
	&\nth{2m}\sum_{q,q'}\nth{{\beta{}V}}(\vq\cdot\vq')\theta(q)\theta(q')\delta(q+q'+k-k')\sigma_3
\end{align}
\end{subequations}
where $G_{0}$ is the same as (Eq. \eqref{eq:pathInt2:G0}, \eqref{eq:pathInt2:nG}).  And like in the single channel, 
\begin{equation}
\sigma_3=\mtrx{1&0&0\\0&-1&0\\0&0&-1}
\end{equation}
and $\hat{1}$ is $3\times3$ identity matrix.  As the expansion in Eq. \ref{eq:pathInt:expand}, we can look for the expansion of $\nG$ over $K_{1,2}$.  
\begin{equation}\tag{\ref{eq:pathInt:expand}}
\tr\ln \hat{G}^{-1}=\tr\ln\hat{G_{0}}^{-1}+\tr(\hat{G_{0}}\hat{K})-\nth{2}\tr(\hat{G_{0}}\hat{K}\hat{G_{0}}\hat{K})+\cdots
\end{equation}
For the first order, $\tr(\hat{G_{0}}\hat{K})$, 
\begin{align}
\tr(\hat{G_{0}}\hat{K_1})=&\sum_{k}{G_{0\,k}K_{1\,k,k}}=0\\
\tr(\hat{G_{0}}\hat{K_2})=&\sum_{k}{G_{0\,k}K_{2\,k,k}}\nonumber\\
	=&-\nth{2m}\nth{{\beta{}V}}\sum_{k}\tr(\hat{G}_{0\,k}\sigma_3)\sum_{q}q^2\theta{q}\theta{-q}\nonumber\\
	=&-\frac{n}{2m}\sum_{q}q^2\theta{(q)}\;\theta{(-q)}
\end{align}
Here we use the fact $\nth{{\beta{}V}}\sum_{k}\tr(\hat{G}_{0\,k}\sigma_3)=n$. $\tr(\hat{G_{0}}\hat{K_2})$ is already in the second order of $\theta$, and we only need to keep the expansion of $K_2$ to this order. On the other hand, we have to go to the second order of $K_1$ for the second order of $\theta$. 
\begin{align}		
\tr(\hat{G_{0}}{K_1}\hat{G_{0}}{K_1})=&\sum_{k,q}\tr(\hat{G}_{0,k+q}K_{1\,k+q,k}\hat{G}_{0\,k}K_{1\,k,k+q})\\
=&\nth{{\beta{}V}}\sum_{k,q}\theta(q)\theta(-q)\Big[(-\omega_m^2)\tr(\hat{G}_{0,k+q}\sigma_3\hat{G}_{0\,k}\sigma_3)\nonumber\\
&\quad+\nth{m^2}\sum_{i,j}q_iq_j(k_i+\frac{q_i}{2})(k_j+\frac{q_j}{2})\tr(\hat{G}_{0,k+q}\hat{G}_{0\,k})\Big]\nonumber\\
\equiv&\sum_{q}\theta(q)\theta(-q)\big[-\pi^{(0)}(q)\omega_m^2+\sum_{ij}\pi^{(\perp)}_{ij}(q)q_iq_j\big]
\end{align}
\emph{Here we only interested in the low-energy behavior of the mode and therefore we only use $\pi(0)$.} Use the previous result of Sec. \ref{sec:diagonalGreen}, we can calculate the Green's function in the lowest and first order of $D_i/\eta$ in $\pi$.  After some long but straigh-forward algebra (see Sec. \ref{sec:calculatePi}), we find
\begin{gather}
\pi^{(0)}(0)\approx\sum_{\vk}\frac{D_{1}^{2}}{E_{\vk}^{3}}-(\frac{D_{1}^{2}+D_{2}^{2}}{2\eta})\sum_{\vk}\frac{D_{1}^{2}}{E_{\vk}^{4}}\\
\pi^{(\perp)}(0)=0
\end{gather}
Combine all these together
\begin{equation}
S[\theta]=\int{dx}\sum_{q}\theta(q)\theta(-q)\big[\nth{2}\pi^{(0)}(q)\omega_m^2-\frac{n}{2m}q^2\big]
\end{equation}
And this determines the velocity of the Anderson-Bogoliubov collective mode.  We see that its behavior is qualitatively same as  single-channel with some correction in the order of $D_{i}/\eta$.






\begin{subappendices}

\subsection{Diagonalize Matrix Eq. (\ref{eq:pathInt2:G2})\label{sec:diagonalize}}
We need to find a unitary transformation $L$ to diagonalize matrix 
\begin{equation*}
i\omega_{n}I+\mtrx{-E_k&0&u_kD_2\\0&+E_k&v_kD_2\\u_kD_2&v_kD_2&+\xi_k+\eta}
\end{equation*}
We drops all the $k$ subscripts in this section.  We notice that the first term is proportional to identity matrix and does not change by unitary transformation, we only need to concentrate for the second term.  We rescale all elements with $E$, 
\begin{equation*}
R=
\begin{pmatrix}
-1&0&y_1\\
0&1&y_2\\
y_1&y_2&t
\end{pmatrix}
\end{equation*}
The secular equation is 
\begin{equation}
(x^{2}-1)(x-t)-(y_{1}^{2}+y_{2}^{2})x+y_{1}^{2}-y_{2}^{2}=0
\end{equation}
We will assume at the zeroth order, the three eigenvalues are $-1$, $1$ and $t$.  Both $y_{1,2}$ and t are larger than 1, however, we will verify that given condition $y_{i}^{2}\ll{t}$, the correction is indeed small and the expansion is legit.  \emph{Indeed, this approximation is not as bad as it seems to be, close-channel component can still be smaller than open-channel at low-k (in order of $k_{F}$)  due to close-channel bound state is much smaller than inter-particle distance even when total close-channel is more than open-channel.  And here all the quantities are about low-k unless specifically noticed.} 
We expand the system to the first order of $y_{i}^{2}/{t}$, and find
\begin{equation}
\begin{array}{ccc}
x^{(0)}&\quad{}x^{(1)}&\quad{}Eigenvector\nonumber\\
-1&-\frac{y_{1}^{2}}{t}&\mtrx{1&\frac{y_{1}y_{2}}{2t}&-\frac{y_{1}}{t}}\\
1&-\frac{y_{2}^{2}}{t}&\mtrx{-\frac{y_{1}y_{2}}{2t}&1&-\frac{y_{2}}{t}}\\
t&\frac{y_{1}^{2}+y_{2}^{2}}{2t}&\mtrx{\frac{y_{1}}{t}&\frac{y_{2}}{t}&1}
\end{array}
\end{equation}
Now it is easy to write down the corresponding diagonal matrix and unitary transformation
\begin{equation}
B=i\omega_{n}I+E\mtrx{-1-\frac{y_{1}^{2}}{t}&0&0\\0&1-\frac{y_{2}^{2}}{t}&0\\0&0&t+\frac{y_{1}^{2}+y_{2}^{2}}{2t}}
\end{equation}
\begin{equation}
L=\mtrx{1&-\frac{y_{1}y_{2}}{2t}&\frac{y_{1}}{t}\\\frac{y_{1}y_{2}}{2t}&1&\frac{y_{2}}{t}\\-\frac{y_{1}}{t}&-\frac{y_{2}}{t}&1}
\end{equation}
Here $L$ is not exactly unitary transformation, it only unitary in the first order of  $y_{i}^{2}/{t}$ (or $D_{i}^{2}/(E\eta)$). We have 
\[
B=L^{\dg}RL+o(\frac{y_{i}^{2}}{t})
\]
Alternatively, we can write $L$ as 
\begin{equation}
L=I+
\mtrx{0&-\frac{D_{1}D_{2}}{4E^{2}}&u\\
\frac{D_{1}D_{2}}{4E^{2}}&0&v\\
-u&v&0
}\frac{D_{2}}{\eta}
\end{equation}
Here we use $uv=D_{1}/2E$


%\subsection{Expansion of $M$ matrix\label{sec:expansionM}}
\subsection{ Calculate $\pi^{(0)}$ and $\pi^{\perp}$\label{sec:calculatePi}}
Here we will calculate $\pi^{(0)}$ and $\pi^{\perp}$ to the first order of $D_i/\eta$ using the expansion of Green's function in Sec. \ref{sec:diagonalGreen}.

\begin{equation}\label{eq:pathInt2:pi0}
\begin{split}
\pi^{(0)}(0)=&\sum_k\tr(\hat{G}_{0\,k}\sigma_3\hat{G}_{0\,k}\sigma_3)\\
	\approx&\sum_k\tr\big(T_{\vk}B_{k}^{-1}T_{\vk}^{\dg}\sigma_3T_{\vk}B_{k}^{-1}T_{\vk}^{\dg}\sigma_3\big)\\
	&\quad+\tr\Big(T_{\vk}\delta_{\vk}B_{k}^{-1}T_{\vk}^{\dg}\sigma_3T_{\vk}B_{k}^{-1}T_{\vk}^{\dg}\sigma_3
	-T_{\vk}B_{k}^{-1}\delta_{\vk}T_{\vk}^{\dg}\sigma_3T_{\vk}B_{k}^{-1}T_{\vk}^{\dg}\sigma_3\\
	&\qquad+T_{\vk}B_{k}^{-1}T_{\vk}^{\dg}\sigma_3T_{\vk}\delta_{\vk}B_{k}^{-1}T_{\vk}^{\dg}\sigma_3
	-T_{\vk}B_{k}^{-1}T_{\vk}^{\dg}\sigma_3T_{\vk}B_{k}^{-1}\delta_{\vk}T_{\vk}^{\dg}\sigma_3\Big)\\
	=&\sum_k\tr\big(M_{k}\big)+2\tr\Big(\delta_{\vk}M_{k}-\delta_{\vk}M_{k}\Big)
\end{split}
\end{equation}
where 
\begin{equation}
M_{k}=T_{\vk}^{\dg}\sigma_3T_{\vk}B_{k}^{-1}T_{\vk}^{\dg}\sigma_3T_{\vk}B_{k}^{-1}
\end{equation}
Here we use the cyclical  property of the trace $\tr(AB)=\tr(BA)$.  
It is straightforward to calculate
\begin{equation*}
T_{\vk}^{\dg}\sigma_3T_{\vk}B_{k}^{-1}=
\begin{pmatrix}
{\frac{\xi_{\vk}}{E_{\vk}(i\omega_{k}-\xi_{1\,\vk})}}&\frac{D_{1}}{E_{\vk}(i\omega_{k}-\xi_{2\,\vk})}&0\\
{\frac{D_{1}}{E_{\vk}(i\omega_{k}-\xi_{1\,\vk})}}&-\frac{\xi_{\vk}}{E_{\vk}(i\omega_{k}-\xi_{2\,\vk})}&0\\
0&0&-\nth{i\omega_{k}-\xi_{3\,\vk}}\\
\end{pmatrix}
\end{equation*}
Now it is easy to calculate the first term
\begin{equation}
\begin{split}
\sum_k\tr\big(M_{k}\big)=&
\sum_{k}\mbr{
\frac{2D_{1}^{2}}{E_{\vk}^{2}(i\omega_{k}-\xi_{1\,\vk})(i\omega_{k}-\xi_{2\,\vk})}+
\br{\frac{\xi_{\vk}^{2}}{E_{\vk}^{2}(i\omega_{k}-\xi_{1\,\vk})^{2}}+\frac{\xi_{\vk}^{2}}{E_{\vk}^{2}(i\omega_{k}-\xi_{2\,\vk})^{2}}
-\nth{(i\omega_{k}-\xi_{3\,\vk})^{2}}}
}
\end{split}
\end{equation}
Only root $\xi_{2\,\vk}$ in the first term contributes in Matrubara frequency summation.
\begin{equation}\label{eq:pathInt2:pi0-1}
\sum_k\tr\big(M_{k}\big)=\sum_{\vk}\frac{2D_{1}^{2}}{E_{\vk}^{2}(\xi_{1\,\vk}-\xi_{2\,\vk})}
\approx\sum_{\vk}\frac{D_{1}^{2}}{E_{\vk}^{3}}-(\frac{D_{1}^{2}+D_{2}^{2}}{2\eta})\sum_{\vk}\frac{D_{1}^{2}}{E_{\vk}^{4}}
\end{equation}


For the lowest order of the second term in Eq. \eqref{eq:pathInt2:pi0}, we only need to take the lowest order of $B_{k}$
\begin{equation}
B_{k}=\mtrx{i\omega_{k}-E_{\vk}&0&0\\0&i\omega_{k}+E_{\vk}&0\\0&0&i\omega_{k}+\xi_{\vk}+\eta}
\end{equation}
It is easy to verify at this approximation
\begin{equation}\label{eq:pathInt2:pi0-2}
\tr\Big(\delta_{\vk}M_{k}-\delta_{\vk}M_{k}\Big)=0
\end{equation}
Combine Eq. \eqref{eq:pathInt2:pi0-2} and Eq. \eqref{eq:pathInt2:pi0-2}, we have 
\begin{equation}
\pi^{(0)}(0)\approx\sum_{\vk}\frac{D_{1}^{2}}{E_{\vk}^{3}}-(\frac{D_{1}^{2}+D_{2}^{2}}{2\eta})\sum_{\vk}\frac{D_{1}^{2}}{E_{\vk}^{4}}
\end{equation}

and it is actually exact for $\pi^{\perp}(0)$
\begin{equation}
\begin{split}
\pi^{\perp}(0)=&\sum_k\tr(\hat{G}_{0\,k}\hat{G}_{0\,k})\\
	=&\sum_k\tr\big(T_{\vk}L_{\vk}B_{k}^{-1}L_{\vk}^{\dg}T_{\vk}^{\dg}T_{\vk}L_{\vk}B_{k}^{-1}L_{\vk}^{\dg}T_{\vk}^{\dg}\big)\\
	=&\sum_k\tr\big(B_{k}^{-1}B_{k}^{-1}\big)\\
	=&\sum_{\vk,i}(\sum_{\omega_{k}}(i\omega_{k}-\xi_{i})^{-2})\\
	=&0
\end{split}
\end{equation}
\end{subappendices}
\chapter{Conclusions\label{ch:conclusion}}
% !TeX root =thesis.tex
% 
In this thesis, we studied the narrow resonance, especially the case where two channels share the same species.  In general, for narrow resonance without shared species, the main correction is that the weight in close channel.  Two number equations exists, one for open channel and one for close channel.  The number equation resembling the single channel is for the density of open-channel, as in  \cite{GurarieNarrow}.  

However, when there is common species,  Pauli exclusion between two channels need to be considered carefully for narrow resonance.  Overall, our treatment uses the spirit of universality.  For the dilute system with short-range potential, such as dilute ultracold Alkali gas, the short-range is essentially unchanged from two-body physics to many-body.  And this particular feature serves formally as the boundary condition.  On the other hand, the rest of the wave function, where no two particles in short-range is essentially free.  Different from the universality treatment is that the resonance, which makes the system, even the two body part, extremely sensitive to the change in small relative weight between two channels and therefore requires extra cares.  In Feshbach resonance, the close-channel bound state, $\phi_{0}$, in resonance is actually not sensitive to resonance, therefore, we apply the simple boundary condition on it. On the other hand, open-channel direct interaction is dominated by the indirect interaction mediated by close-channel, therefore, the small change in the close-channel bound-state weight \emph{does} affect the short-range wave function in open-channel and the boundary condition it follows, as $a_{s}$. 

When the close channel bound state is in the order of   inter-particle distance, it is really a three species many-body problem and no obvious solution is available.  However, when this bound-state is much smaller than inter-particle distance, this ratio ($r_{c}/a_{0}$) serves as the expanding coefficient and we can extracted the effect perturbatively.  In essence, we can ignore the many-body effects between these close-channel bound-states, while only taking into consideration of the Pauli exclusion between channels.  A few new parameters needs to be introduced and can be calibrated from experiments.  System's mean field property can still be determined by gap equations and number equations as in single-channel problem.  The excitation modes are also close to the original single-channel result with correction in order of $r_{c}/a_{0}$.

In essence, there are three different regimes for the many-body energy scale $E_F$ when comparing with two-body physics.  

First, when $E_F$ is much smaller than the characteristic energy $\delta_c$ (Sec. \ref{sec:intro:twobody}), around resonance, close-channel weight is negligible, and we can ignore close-channel at all and just take the open-channel interaction varies according to s-wave scattering length $a_s$. 

Second, when $E_F$ is larger than $\delta_c$, but smaller than the close-channel bound state binding energy $E_b$, or the bare detuning $\eta$, close-channel weight is significant and cannot be ignored.  We have one extra condition $n_{\text{open}}+n_{\text{close}}=n_{\text{total}}$. Also, we need to take into account that in the resonance formula of $a_s$, shifting should count from Fermi sea instead of zero as in two-body physics.  Furthermore,  This is extensively treated in \cite{GurarieNarrow} as well as in this thesis.  Furthermore, Pauli exclusion between two channels needs to be taken into account when they share common species.  Nevertheless, in the low-momentum, close-channel bound state has very small weight and  open-channel dominates.  therefore, we can still treat the Pauli exclusion between channels perturbatively with the expansion over ratio $E_F/\eta$.  This is the emphasis of the current thesis.  

Third, when $E_F$ larger than binding energy $E_b$ or bare detuning $\eta$, we have a pure three species many-body problem. Such region can be achieved in two ways, a very large $E_F$, which indicate a very dense system, and it is hard to image that this is still in the usual dilute alkali gas model. All kinds of other approximation probably break down beforehand, such as pseudo-potential $a_s$ replacement. On the other hand, it requires a very small $\eta$, which indicates a true three species many-body problem, which is unclear how to solve.   

\appendix

%\include{Appendix.tex}
% !TeX root =thesis.tex
\chapter{The BCS-type ansatz and the variation approach for the two-channel problem \label{ch:mean}}
We can also formulate the problem  using the  BCS-type ansatz.  It is not difficult to calculate  the expectation of the free energy and  the wave function that optimizes the free energy within the ansatz space.  The optimization process gives us gap equations, which determine the exact wave function with the constraint from number equations.   %with those common parameters as chemical potential, gap.  
We will see that this method yields  the mean-field solution consistent with the path-integral approach.  However, this method is mean-field by nature and it is difficult  to extend it to study the collective excitation of the system.  

We put the Hamiltonian Eq. \ref{eq:pathInt2:ham2} into the momentum space, and restore the momentum dependence of the interaction coefficients
\begin{equation}\label{eq:uvw:hamiltonian}
\begin{split}
 H=&\sum_\vk\epsilon^a_\vk{}a^+_\vk{}a^{}_\vk+\sum_\vk\epsilon^b_\vk{}b^+_\vk{}b^{}_\vk+\sum_\vk\epsilon^c_\vk{}c^+_\vk{}c^{}_\vk\\
  &-\sum_{\vk\vk'}U_{\vk\vk'}a^+_\vk{}b^+_{-\vk}{}b^{}_{-\vk'}a^{}_{\vk'}
	-\sum_{\vk\vk'}V_{\vk\vk'}a^+_\vk{}c^+_{-\vk}{}c^{}_{-\vk'}a^{}_{\vk'}\\
 &-\sum_{\vk\vk'}Y_{\vk\vk'}a^+_\vk{}b^+_{-\vk}{}c^{}_{-\vk'}a^{}_{\vk'}
	-\sum_{\vk\vk'}Y^*_{\vk\vk'}a^+_{\vk'}{}c^+_{-\vk'}{}b^{}_{-\vk}a^{}_{\vk}
\end{split} 
\end{equation}
By the hermition condition we have 
\begin{equation}
 U_{\vk'\vk}=U^*_{\vk\vk'},\qquad{} V_{\vk'\vk}=V^*_{\vk\vk'}
\end{equation}
  We introduce the BCS-type ansatz 
\begin{equation}\label{eq:ansatz}
 \ket{\Psi}=\prod_\vk\br{u_\vk+v_\vk{}a^\dg_\vk{}b^\dg_{-\vk}+w_\vk{}a^\dg_\vk{}c^\dg_{-\vk}}\ket{0}
\end{equation}
$\ket{0}$ is the particle vacuum state.  Here we require $\abs{u_\vk}^2+\abs{v_\vk}^2+\abs{w_\vk}^2=1$ for normalization.  This ansatz is closely modeled on the original BCS ansatz for superconductivity.  An alternative ansatz would be $\prod_\vk(u_\vk+v_\vk{}a^\dg_\vk{}b^\dg_{-\vk})(u^{'}_\vk+w_\vk{}a^\dg_\vk{}c^\dg_{-\vk})\ket{0}$, which is actually the same as Eq. \ref{eq:ansatz}, because  the cross term vanishes due to the Pauli exclusion of the common species.   Similarly to the original BCS ansatz, this ansatz does not have the fixed particle number.  We  will just require the expected value of number operator matches the total particle number. For all the interaction term, there are two types of contribution,
for example, 
\begin{equation*}
\av{U_{\vk\vk'}a^\dg_\vk{}b^\dg_{-\vk}{}b^{}_{-\vk'}a^{}_{\vk'}}
=\sum_{\vk}U_{\vk\vk}\abs{v_\vk}^2+\sum_{\vk\neq\vk'}U_{\vk\vk'}v^{}_{\vk'}u^*_{\vk'}u^{}_\vk{}v^*_\vk
\end{equation*}
The first term is the Hatree term and the second term is the more interesting pairing term.  


And the full free energy is 
\begin{equation}\label{eq:uvw:F}
 \begin{split}
  &F\equiv\av{H-\mu{}N}\\
    =&\sum(\xi^a_\vk+\xi^b_\vk)\abs{v_\vk}^2+\sum(\xi^a_\vk+\xi^c_\vk)\abs{w_\vk}^2\\
    &-\sum_{\vk}U_{\vk\vk}\abs{v_\vk}^2-\sum_{\vk\neq\vk'}U_{\vk\vk'}v^{}_{\vk'}u^*_{\vk'}u^{}_\vk{}v^*_\vk\\
    &-\sum_{\vk}V_{\vk\vk}\abs{w_\vk}^2
      -\sum_{\vk\neq\vk'}V_{\vk\vk'}w^{}_{\vk'}u^*_{\vk'}u^{}_\vk{}w^*_\vk\\
    &-\sum_{\vk}Y_{\vk\vk}w^{}_{\vk}v^*_\vk{}
      -\sum_{\vk\neq\vk'}Y_{\vk\vk'}w^{}_{\vk'}{u^{*}_{\vk'}}v^*_\vk{}u^{}_\vk\\
    &-\sum_{\vk}Y^*_{\vk\vk}w^*_{\vk}v^{}_{\vk}{}
      -\sum_{\vk\neq\vk'}Y^*_{\vk\vk'}w^*_{\vk}{u^{}_{\vk}}v^{}_{\vk'}{}u^{*}_{\vk'}
 \end{split}
\end{equation}
Where 
\begin{equation*}
 \xi^a_\vk=\epsilon^a_\vk-\mu^a,\qquad\xi^b_\vk=\epsilon^b_\vk-\mu^b,\qquad\xi^c_\vk=\epsilon^c_\vk-\mu^b
\end{equation*}
Two chemical potentials are added to make sure the $n_a=n_b+n_c=\nth{2}n$.  In principle, $\mu^{a}$ does not need to be equal to $\mu^{b}$  as there is no exchange/conversion between $(a)$ and $(b,c)$.  Nevertheless, we  set $\mu^{a}=\mu^{b}$ for simplicity and drop the superscript on chemical potential hereinafter. 
We will also drop the Hatree terms because they  only relate to the density and thus can be absorbed into the chemical potentials.   In the second summation, we ignore the fact that the summation only go through $\vk\neq\vk'$ as the correction due to this is in the higher order. 
 
 \section{Exact gap equations and number equations}
Introduce the new parameter $h_{1\vk}$ and $h_{2\vk}$, corresponding to the mean field value (equal time average) of the anomalous Green's functions,  (as $h_{1,2}$  in Sec \ref{sec:pathInt2:meanfield}), and solve $u_{\vk}$, $v_{\vk}$, $w_{\vk}$ with $h_{1\vk}$ and $h_{2\vk}$ (treat everything as real\footnote{It is not hard to prove that the parameters that optimize the free energy are real within an overall phase factor when the interactions are real. (See also footnote \ref{foot:pathInt2:real} in Chapter \ref{ch:path2}, page \pageref{foot:pathInt2:real}.)}
\begin{gather}
u_{\vk}^2+v^{2}_{\vk}+w^{2}_{\vk}=1\\
u_{\vk}v_{\vk}=h_{1\vk}\\
u_{\vk}w_{\vk}=h_{2\vk}
\end{gather}
One complication in solving above equations is that $u_\vk$, $v_\vk$ and  $w_\vk$ are all  monotonic functions of $k$, while $h_{1\vk}$ or $h_{2\vk}$ is not in the BCS end.  So we need to be careful when taking the square root.  We introduce $\sgn_k$ for such purpose.  
\begin{equation}
\begin{split}
u_{\vk}^2=&\frac{1}{2} \left(1+\sgn_{k}\sqrt{1-4 h_{1\vk}^2-4 h_{2\vk}^2}\right)\\
v^{2}_{\vk}=&\frac{2 h_{1\vk}^2}{1+\sgn_{k}\sqrt{1-4 h_{1\vk}^2-4 h_{2\vk}^2}}
=\frac{ h_{1\vk}^2}{2( h_{1\vk}^2+ h_{2\vk}^2)} \left(1-\sgn_{k}\sqrt{1-4 h_{1\vk}^2-4 h_{2\vk}^2}\right)\\\
w^{2}_{\vk}=&\frac{2 h_{2\vk}^2}{1+\sgn_{k}\sqrt{1-4 h_{1\vk}^2-4 h_{2\vk}^2}}
=\frac{h_{2\vk}^2}{2( h_{1\vk}^2+ h_{2\vk}^2)} \left(1-\sgn_{k}\sqrt{1-4 h_{1\vk}^2-4 h_{2\vk}^2}\right)
\end{split}
\end{equation}
{\label{foot:20100909:sgn}  $\sgn_k=1$  for the whole region in BEC cases, or for large $k$ in BCS cases. In BCS cases,  $\sgn_{k}=-1$ when $k$ is small.  In the single-channel crossover, $\sgn_{k}=\sgn(\epsilon_{k}-\mu)$, the turning point is at $u_\vk^2=1/2$ (zero chemical potential $\mu=0$).  However, in the two-channel crossover, this is more delicate.  The turning point is still $u_\vk^2=1/2$, but no longer exactly at $\mu=0$.  This sign  is very important in number equations Eqs. \ref{eq:20100909:number}.  These $\sgn_{k}$ functions are absorbed after we convert the formulas back into notations of  $(u_{\vk},\,v_{\vk}, \,w_{\vk})$ or $\xi_{k}$ as it is possible $\xi_{k}<0$.}


%Below, we adopt $\sgn_k=1$. (This works for BEC and most part of BCS; and in the final equations, the sign function is taken care of automatically by using $E_{\vk}$.  It is easy to verify that  we arrive at the same final result in the other cases where $\sgn_{k}=-1$. ) 
Derivatives over $h_{1\vk}$ are\footnote{Here we follow the convention that takes $h_{1\vk}$ and $h_{2\vk}$ as real}
\begin{equation}
\begin{split}
\pdiff{u_{\vk}^2}{h_{1\vk}}=&-\frac{2 h_{1\vk}}{\sqrt{1-4 h_{1\vk}^2-4 h_{2\vk}^2}}\sgn_{k}\\
\pdiff{v_{\vk}^2}{h_{1\vk}}=&\frac{2 h_{1\vk}\sgn_k}{\sqrt{1-4 h_{1\vk}^2-4 h_{2\vk}^2}}-\frac{8 h_{1\vk} h_{2\vk}^2\sgn_{k}}{\sqrt{1-4 h_{1\vk}^2-4 h_{2\vk}^2} \left(1+\sgn_{k}\sqrt{1-4 h_{1\vk}^2-4 h_{2\vk}^2}\right)^2}\\
\pdiff{w_{\vk}^2}{h_{1\vk}}=&\frac{8 h_{1\vk} h_{2\vk}^2\sgn_{k}}{\sqrt{1-4 h_{1\vk}^2-4 h_{2\vk}^2} \left(1+\sgn_{k}\sqrt{1-4 h_{1\vk}^2-4 h_{2\vk}^2}\right)^2}\\
=&\sgn_{k}\frac{h_{1\vk} h_{2\vk}^2 \left(1-\sgn_{k}\sqrt{1-4 h_{1\vk}^2-4 h_{2\vk}^2}\right)^2}{2\sqrt{1-4 h_{1\vk}^2-4 h_{2\vk}^2} (h_{1\vk}^2 + h_{2\vk}^2)^2}
\end{split}
\end{equation}
The second term is not as obvious, but can be obtained by $\pdiff{v_{\vk}^2}{h_{1\vk}}=-\pdiff{u_{\vk}^2}{h_{1\vk}}-\pdiff{w_{\vk}^2}{h_{1\vk}}$.   Similarly, the derivatives over $h_{2\vk}$ are
\begin{equation}
\begin{split}
\pdiff{u_{\vk}^2}{h_{2\vk}}=&-\frac{2 h_{2\vk}}{\sqrt{1-4 h_{1\vk}^2-4 h_{2\vk}^2}}\sgn_{k}\\
\pdiff{v_{\vk}^2}{h_{2\vk}}=&\frac{8 h_{1\vk} ^{2}h_{2\vk}\sgn_{k}}{\sqrt{1-4 h_{1\vk}^2-4 h_{2\vk}^2} \left(1+\sgn_{k}\sqrt{1-4 h_{1\vk}^2-4 h_{2\vk}^2}\right)^2}\\
=&\sgn_{k}\frac{h_{1\vk}^{2}h_{2\vk} \left(1-\sgn_{k}\sqrt{1-4 h_{1\vk}^2-4 h_{2\vk}^2}\right)^2}{2\sqrt{1-4 h_{1\vk}^2-4 h_{2\vk}^2} (h_{1\vk}^2 + h_{2\vk}^2)^2}\\
\pdiff{w_{\vk}^2}{h_{2\vk}}=&\frac{2 h_{2\vk}\sgn_k}{\sqrt{1-4 h_{1\vk}^2-4 h_{2\vk}^2}}-\frac{8 h_{1\vk} ^{2}h_{2\vk}\sgn_{k}}{\sqrt{1-4 h_{1\vk}^2-4 h_{2\vk}^2} \left(1+\sgn_{k}\sqrt{1-4 h_{1\vk}^2-4 h_{2\vk}^2}\right)^2}
\end{split}
\end{equation}
The gap equations are 
\begin{subequations}\label{eq:20100909:fullgap}
\begin{multline}
\frac{h_{1\vk}\sgn_k}{\sqrt{1-4 h_{1\vk}^2-4 h_{2\vk}^2}} \xi^{ab}_{\vk}+\frac{4 h_{1\vk} h_{2\vk}^2\sgn_{k}}{\sqrt{1-4 h_{1\vk}^2-4 h_{2\vk}^2} \left(1+\sgn_{k}\sqrt{1-4 h_{1\vk}^2-4 h_{2\vk}^2}\right)^2}\eta\\
-\sum_{\vk'}U_{\vk\vk'}h_{1\vk'}-\sum_{\vk'}Y_{\vk\vk'}h_{2\vk'}=0
\label{eq:20100909:fullgapa}
\end{multline}
\begin{multline}
\frac{h_{2\vk}\sgn_k}{\sqrt{1-4 h_{1\vk}^2-4 h_{2\vk}^2}} \xi^{ac}_{\vk}-\frac{4 h_{1\vk}^{2} h_{2\vk}\sgn_{k}}{\sqrt{1-4 h_{1\vk}^2-4 h_{2\vk}^2} \left(1+\sgn_{k}\sqrt{1-4 h_{1\vk}^2-4 h_{2\vk}^2}\right)^2}\eta\\
-\sum_{\vk'}V_{\vk\vk'}h_{2\vk'}-\sum_{\vk'}Y_{\vk\vk'}h_{1\vk'}=0
\label{eq:20100909:fullgapb}
\end{multline}
\end{subequations}
where $\eta=\epsilon^{ac}_{\vk}-\epsilon^{ab}_{\vk}$ is the bare Zeeman energy difference and is large than most energy scales, such as $E_{F}$.  It is close to  the binding energy of the closed-channel bound state when it is not too far away from the resonance point.   We see one  disadvantages of the ansatz method: unlike the path integral approach, where $h_{1\vk}$ and $h_{2\vk}$ have a very specific forms, Eqs. (\ref{eq:pathInt2:h1}, \ref{eq:pathInt2:h2}), and ready for further approximation and analysis,  we do not have much guide for these two quantities here and need to make some guess.  

We can define order parameters
\begin{gather}
\Delta_{1\vk}=\sum_{\vk'}U_{\vk\vk'}h_{1\vk'}+\sum_{\vk'}Y_{\vk\vk'}h_{2\vk'}\\
\Delta_{2\vk}=\sum_{\vk'}V_{\vk\vk'}h_{2\vk'}+\sum_{\vk'}Y_{\vk\vk'}h_{1\vk'}
\end{gather}
Both of them should have little $\vk$-dependence at low momentum for short-range interactions.  We will drop the $\vk$ subscripts in both $\Delta$ in the following. 
Two number equations can be obtained.  One for each channel (see footnote (\ref{foot:20100909:sgn}) for $\sgn_{k}$).
\begin{subequations}\label{eq:20100909:number}
\begin{gather}
N_{\text{open}}=2\sum_{\vk}v_{\vk}^{2}
	=\sum_{\vk}\frac{ h_{1\vk}^2}{( h_{1\vk}^2+ h_{2\vk}^2)} \left(1-\sgn_{k}\sqrt{1-4 h_{1\vk}^2-4 h_{2\vk}^2}\right)
	\label{eq:meanfield:opennum}\\
N_{\text{close}}=2\sum_{\vk}w_{\vk}^{2}
	=\sum_{\vk}\frac{ h_{2\vk}^2}{( h_{1\vk}^2+ h_{2\vk}^2)} \left(1-\sgn_{k}\sqrt{1-4 h_{1\vk}^2-4 h_{2\vk}^2}\right)
\end{gather} 
\end{subequations}


\section{Approximate solution of the mean field equations}
Here instead of solving these equations, we just demonstrate that they are consistent to the saddle point solution obtained in the path-integral methods up to the first order of correction $\zeta/E_{\vk}$.

Let us look at the closed-channel gap equation (Eq. \ref{eq:20100909:fullgapb}) first.  It can be rewritten as 
\begin{multline*}
\frac{h_{2\vk} \sgn_{k}}{\sqrt{1-4 h_{1\vk}^2-4 h_{2\vk}^2}}\br{2 \xi_{\vk}+\eta-\frac{4 h_{1\vk}^{2}}{\left(1+\sgn_{k}\sqrt{1-4 h_{1\vk}^2-4 h_{2\vk}^2}\right)^2}\eta}
-\sum_{\vk'}V_{\vk\vk'}h_{2\vk'}-\sum_{\vk'}Y_{\vk\vk'}h_{1\vk'}=0
\end{multline*}
We can put some terms back into the notation of $(u_{\vk},\,v_{\vk},\, w_{\vk})$.
\begin{gather}
\sgn_{k}\sqrt{1-4 h_{1\vk}^2-4 h_{2\vk}^2}=(u^{2}_{\vk}-v_{\vk}^{2}-w_{\vk}^{2})\\
1+\sgn_{k}\sqrt{1-4 h_{1\vk}^2-4 h_{2\vk}^2}=2u_{\vk}^{2}\label{eq:meanfield:1sqrt}\\
\frac{4 h_{1\vk}^{2} }{\left(1+\sgn_{k}\sqrt{1-4 h_{1\vk}^2-4 h_{2\vk}^2}\right)^2}=\frac{v_{\vk}^{2}}{u_{\vk}^{2}}
\end{gather}
We rewrite the closed-channel gap equation using these relations
\begin{equation*}
\frac{h_{2\vk}}{u^{2}_{\vk}-v_{\vk}^{2}-w_{\vk}^{2}}\br{2 \xi_{\vk}+\eta-\frac{v_{\vk}^{2}}{u_{\vk}^{2}}\eta}-\sum_{\vk'}V_{\vk\vk'}h_{2\vk'}-\sum_{\vk'}Y_{\vk\vk'}h_{1\vk'}=0
\end{equation*}
At high momentum ($\gg{}k_{F}$),  the third term in the parenthesis is negligible and the the denominator $u^{2}_{\vk}-v_{\vk}^{2}-w_{\vk}^{2}\approx1$.  This equation is then very similar to the two-body \sch equation (Eq. \ref{eq:intro:close}). For the closed-channel, the interaction, or the closed-channel bound state, $\phi_{0}$, has non-zero value for a range  in the momentum space much larger than $k_{F}$.  Thus we expect the closed-channel correlation $h_{2\vk}\propto\phi_{0}f(\vk)$ where $f(\vk)$ approach one at high momentum.  This is in fact the same argument as Eq. \ref{eq:pathInt2:hphif}  in the beginning of Chapter \ref{ch:path2}.  The next thing to notice is that the third term in the parenthesis might dominate the other two terms in  certain situation (low momentum, BCS) because the denominator  $u_{\vk}^{2}$ can be very small in such regions.  A nature way to compensate this big term is to set\footnote{$u_{\vk}^{2}$ in Chapter \ref{ch:path2} is defined to be $(\xi_{\vk}+E_{\vk})/2E_{\vk}$ and does not carry an obvious physical meaning as $u_{\vk}^{2}$ in this appendix. They are the same in the lowest order of our expansion ($\zeta/E_{\vk}$) for the narrow resonance. This does not affect our analysis because the closed-channel correlation is actually a first order quantity.} $f(\vk)=u_{\vk}^{2}$.  Now we arrive the same conclusion for the closed-channel correlation as in the path-integral approach (Eq. \ref{eq:pathInt2:hphi})
\begin{equation}\label{eq:meanfield:h2phi}
h_{2\vk}=\tilde\alpha{}\phi_{0\vk}u_{\vk}^{2}
\end{equation}
Note that $\tilde\alpha$ is not fixed yet and should not be identified as $\alpha$ in the path-integral approach (Eq. \ref{eq:pathInt2:hphi}) \textit{a prior}. The equation becomes,
\begin{equation*}
\frac{\tilde\alpha{}\phi_{0\vk}u_{\vk}^{2} }{u^{2}_{\vk}-v_{\vk}^{2}-w_{\vk}^{2}}\br{2 \xi_{\vk}+\eta-\frac{v_{\vk}^{2}}{u_{\vk}^{2}}\eta}-\sum_{\vk'}V_{\vk\vk'}\tilde\alpha{}\phi_{0\vk}u_{\vk}^{2}-\sum_{\vk'}Y_{\vk\vk'}h_{1\vk'}=0
\end{equation*}
Using the similar approach as in the path-integral method, we multiply the equation with $\phi_{0\vk}^{*}$ and integrate the equation in order to find $\tilde{\alpha}$. In the denominator of the first term $u^{2}_{\vk}\gg{}v_{\vk}^{2}$,$u^{2}_{\vk}\gg{}w_{\vk}^{2}$ in the most of the integral domains (up to order of $\kappa$($\eta$) in the momentum space).  The equation is approximately 
\begin{equation*}
\tilde\alpha{}\sum_{\vk}\frac{\phi_{0\vk}\phi_{0\vk}^{*}u_{\vk}^{2}}{u^{2}_{\vk}}\br{2 \xi_{\vk}+\eta}-\tilde\alpha\sum_{\vk\vk'}V_{\vk\vk'}{}\phi_{0\vk}\phi_{0\vk}^{*}u_{\vk}^{2}-\sum_{\vk\vk'}\phi_{0\vk}^{*}Y_{\vk\vk'}h_{1\vk'}=0
\end{equation*}
And it is not hard to find 
\begin{equation}
\tilde\alpha\approx\frac{\sum_{\vk\vk'}{\phi_{\vk}^{*}}{Y_{\vk\vk'}}{h_{1\vk'}}}{{-E_{b}+\eta-2\mu-\tilde\lambda_{1}}}
\end{equation}
 And $\tilde\lambda_{1}$ is simpler than Eq. \ref{eq:pathInt2:lambda1}
\begin{equation*}
\tilde\lambda_{1}(\eta)\equiv-\sum_{\vk\vk'}{\phi_{\vk}^{*}}{v_{\vk'}^{2}V_{\vk\vk'}}\phi_{\vk'}
\end{equation*}
But it should be equal to $\lambda_{1}$ from the path-integral approach at the lowest order of $\zeta$ because this term is the dominated one (see Sec. \ref{sec:pathInt2:lambda}). We will just use $\lambda_{1}$ hereinafter. $\tilde\alpha$ is the same as $\alpha$ in the path-integral approach if $h_{1\vk}$ here is the same as the one in the path-integral approach,  which we will show in the following. Particularly for low momentum, $\phi_{0\vk}\sim\nth{\eta}$ (see Appendix \ref{sec:pathInt2:short-range}) and it is not hard to find at low momentum
\begin{equation}\label{eq:meanfield:aphi}
\alpha\phi_{0\vk}\overset{{k\lesssim{k_F}}}{=}\frac{\Delta_2}{\eta}
\end{equation}

   Now let us check the open-channel gap equation.  We can rewrite it as 
   \begin{equation*}
   \frac{h_{1\vk} \sgn_{k}}{\sqrt{1-4 h_{1\vk}^2-4 h_{2\vk}^2}}\br{ \xi^{ab}_{\vk}+\frac{4 h_{2\vk}^2}{\left(1+\sgn_{k}\sqrt{1-4 h_{1\vk}^2-4 h_{2\vk}^2}\right)^2}\eta}=\Delta_{1}
   \end{equation*}
Again, using Eq. \ref{eq:meanfield:1sqrt}, we replace the denominator with $2u_\vk^2$.  Also we replace $h_{2\vk}$ with $\tilde\alpha{}\phi_{0\vk}u_{\vk}^{2}$ using Eq. \ref{eq:meanfield:h2phi}.  
 \begin{equation*}
   \frac{h_{1\vk}\sgn_{k}}{\sqrt{1-4 h_{1\vk}^2-4 h_{2\vk}^2}}\br{ 2\xi_{\vk}+({\tilde\alpha}\phi_{0\vk})^2\eta}=\Delta_{1}
   \end{equation*}
 At low momentum, the second term in the parenthesis can be simplified as $\Delta_2^2/\eta=\zeta$  using Eq. \ref{eq:meanfield:aphi}
   \begin{equation*}
   \frac{h_{1\vk}\sgn_{k}}{\sqrt{1-4 h_{1\vk}^2-4 h_{2\vk}^2}}\br{ 2\xi_{\vk}+\zeta}=\Delta_{1}
   \end{equation*}
   We can solve $h_{1\vk}$ in terms of $\Delta_1$.  
   \begin{equation*}
   h_{1\vk}^2=\frac{\Delta_1^2(1-4h_{2\vk}^2)}{(2\xi_{\vk}+\zeta)^2+4\Delta_1^2}
   \approx\frac{\Delta_1^2(1-4h_{2\vk}^2)}{4E_\vk^2+4\xi_\vk\zeta}
   \approx\frac{\Delta_1^2}{4E_\vk^2}(1-\frac{\xi_\vk\zeta}{E_\vk^2})(1-4\frac{\zeta}{\eta}u_\vk^4)
   \end{equation*}
   Here we take the low momentum value of $h_{2\vk}$ in the last step.  Actually this term is in the higher order than $\frac{\xi_\vk\zeta}{E_\vk^2}$, we can neglect this term for the first order. Now we have 
    \begin{equation*}
   h_{1\vk}^2\approx\frac{\Delta_1^2}{4E_\vk^2}(1-\frac{\xi_\vk\zeta}{E_\vk^2})
   \end{equation*}
   This is equivalent in the first order of $\zeta/E_\vk$ to the $h_{1\vk}$ in the path-integral approach (Eq. \ref{eq:pathInt2:h1}). 
   
   
It is easy to see that $h_{2\vk}^2$ is the higher order as $\zeta/\eta$ and $w_\vk\ll1$ all the time.  So we always have $h_{2\vk}\approx{}w_\vk$ except in low momentum.  However, the summation in the low momentum is very small, one order higher in $\zeta/E_{\vk}$.  The closed-channel number equation is then simply $N_{\text{closed}}=\sum{}w_\vk^2\approx{}\sum_{\vk}h_{2\vk}^2$, which is the same result in the path-integral approach as we discussed in Eq. \ref{eq:pathInt2:closeD2}.

We move to the open-channel number equation Eq. \ref{eq:meanfield:opennum}. We notice the factor $\frac{ h_{1\vk}^2}{( h_{1\vk}^2+ h_{2\vk}^2)} $ limit the summation into  low momentum.  Within this range,
\begin{equation*}
\frac{ h_{1\vk}^2}{( h_{1\vk}^2+ h_{2\vk}^2)} =1-\frac{h_{2\vk}^2}{( h_{1\vk}^2+ h_{2\vk}^2)}\approx1-\frac{h_{2\vk}^2}{h_{1\vk}^2}=1+O((\frac{\zeta}{E_{\vk}})^{2})
\end{equation*}
On the other hand, we have 
\begin{equation*}
\sqrt{1-4 h_{1\vk}^2-4 h_{2\vk}^2}\approx1-4h_{1\vk}^{2}+O((\frac{\zeta}{E_{\vk}})^{2})
	\approx\frac{\xi_{\vk}}{E_{\vk}}+\frac{\Delta_{1}^{2}}{4E_{\vk}^{3}}\zeta+O((\frac{\zeta}{E_{\vk}})^{2})
\end{equation*}
So we have the open-channel number equation 
\begin{equation}
N_{\text{open}}\approx\sum_{\vk}1-\frac{\xi_{\vk}}{E_{\vk}}-\frac{\Delta_{1}^{2}}{4E_{\vk}^{3}}\zeta
\end{equation}
This is consistent to the open-channel number equation derived from the path-integral approach  (Eq. \ref{eq:pathInt2:openNum}, where the second term is negligible comparing to the third term).

Thus we have shown that the mean-field solution of the path-integral approach can solve the gap equations and number equations of the ansatz method up to the first order of $\zeta/E_{\vk}$. 
%At the lowest order, the second term in both gap equations, Eq. \ref{eq:20100909:fullgap}, are negligible.  And to the lowest order, the first term in these two equations are equal $\Delta_{1}$ and $\Delta_{2}$ respectively.  











































%The number equation is (see footnote (\ref{foot:20100909:sgn}) for $\sgn_{k}$)
%\begin{equation}\label{eq:20100909:number}
%N=2\sum_{\vk}(v_{\vk}^{2}+w_{\vk}^{2})=\sum_{\vk}\left(1-\sgn_{k}\sqrt{1-4 h_{1\vk}^2-4 h_{2\vk}^2}\right)
%\end{equation} 
%where N is the number of atoms (twice  the number of pairs)
%
%\section{Approximating closed-channel with bound-state level}
%Let us look at \eef{eq:20100909:fullgapb}.  It is energetically very disadvantageous to deviate from the bound state as the binding energy and the difference from the next bound state is much  larger than Fermi energy.     This equation must close to  the solution for the isolated closed-channel \sch equation.  
%\begin{equation}\label{eq:20100915:twobody}
%{\phi^{0}_{\vk}}(2\epsilon^{ac}_{\vk})-V_{\vk\vk'}\phi^{0}_{\vk'}=-E_{b}\phi^{0}_{\vk'}
%\end{equation}
%This equation relates region (in k-space) much larger than $k_{F}$.  On the other hand, the open-channel 
%$h_{1\vk}$ is only substantial in region around or not too high above  $k_{F}$.  Therefore, as the first order approximation, we takes $h_{1\vk}=0$ for this equation. So we can drop the second term and the denominator in the first term ($h_{2\vk}\ll1$ all the time) of \eef{eq:20100909:fullgapb}\footnote{ This is certainly OK for he BEC end where $F_{k}\ll1$ all the time.  It is less satisfactory, might be problematic,  when $F_{k}$ is close to maximum $\nth2$, this happens around Fermi energy in BCS limit, but the approximation is still OK for other places. So at least for bulk of the region of $h_{2\vk}$ satisfies \eef{eq:20100915:gapb}}.  In  open-channel equation, \eef{eq:20100909:fullgapa}, we drop the factor $\left(1+\sqrt{1-4 h_{1\vk}^2-4 h_{2\vk}^2}\right)^2$ in the second term for easier calculation, (no strong reason yet, but this term can only takes value between 1 and 4, and the second term is relatively minor comparing to the first). We write down the approximated gap equations
%\begin{subequations}\label{eq:20100915:gap}
%\begin{gather}\label{eq:20100915:gapa}
%\frac{2h_{1\vk}}{\sqrt{1-4 h_{1\vk}^2-4 h_{2\vk}^2}} (\epsilon^{ab}_{\vk}-2\mu+  h_{2\vk}^2\eta)-U_{\vk\vk'}h_{1\vk'}-Y_{\vk\vk'}h_{2\vk'}=0\\
%\label{eq:20100915:gapb}
%{2h_{2\vk}}(\epsilon^{ab}_{\vk}-2\mu+\eta)-V_{\vk\vk'}h_{2\vk'}-Y_{\vk\vk'}h_{1\vk'}=0
%\end{gather}
%\end{subequations}
%Here we express $\xi^{ab}_{\vk}=\epsilon^{ab}_{\vk}-2\mu$, $\epsilon^{ac}_{\vk}=\epsilon^{ab}_{\vk}+\eta$.  And many-body closed-channel wave function $h_{2\vk}$ is proportional to two-body bound-state wave function $\phi_{\vk}^{0}$,  $h_{2\vk}=\alpha\phi^{0}_{\vk}$.  Comparing to $h_{2\vk}$ in path integral approach (Eq. \ref{eq:pathInt2:hphi} in Sec. \ref{sec:pathInt2:ren}), the many-body effect factor $u_{\vk}^{2}$ is missing.  It can be derived if taken into account the factor $\nth{{\sqrt{1-4 h_{1\vk}^2-4 h_{2\vk}^2}}}$ with more care.  However, contrast with a known structure of $h_{2\vk}$ there, we are quite blank with the behavior of $h_{2\vk}$, only in hinder sight, we realize this extra many-body factor, $u_{\vk}^{2}$.   Here we just stick to the lowest order and ignore this factor.  
%\eef{eq:20100915:gapb} becomes
%\begin{equation}\label{eq:20100915:GF}
%h_{2\vk}=-\frac{Y_{\vk\vk'}h_{1\vk'}}{2(E^{0}-\eta+2\mu)}
%\end{equation}
%Plug this back into the last term of \eef{eq:20100915:gapa}, we have 
%\begin{equation}
%\frac{2h_{1\vk}}{\sqrt{1-4 h_{1\vk}^2-4 h_{2\vk}^2}} (\xi^{ab}_{\vk}+  h_{2\vk}^2\eta)-U_{\vk\vk'}h_{1\vk'}+\frac{Y_{\vk\vk'}Y_{\vk'\vk''}}{2(E^{0}-\eta+2\mu)}h_{1\vk''}=0
%\end{equation}
%Now considering the last two terms has weak dependency in low k, we can set 
%\begin{equation}\label{eq:20100915:gap1}
%\frac{2h_{1\vk}}{\sqrt{1-4 h_{1\vk}^2-4 h_{2\vk}^2}} (\xi^{ab}_{\vk}+  h_{2\vk}^2\eta)\equiv\Delta_{\vk}=(U_{\vk\vk'}h_{1\vk'}-\frac{Y_{\vk\vk'}Y_{\vk'\vk''}}{2(E^{0}-\eta+2\mu)}h_{1\vk''})
%\end{equation}
%We can express $h_{1\vk}$ according to $\Delta_{\vk}$,  (ignore the higher order of $h_{2\vk}$)
%\begin{equation}\label{eq:20100915:F}
%h_{1\vk}=\frac{\Delta_{\vk}}2\sqrt{\frac{(1-4h_{2\vk}^{2})}{(\xi^{ab}_{\vk}+  h_{2\vk}^2\eta)^{2}+\Delta_{\vk}^{2}}}
%\end{equation}
%Put it back into the second half of gap equation (\ref{eq:20100915:gap1}), 
%\begin{equation}\label{eq:20100915:onechannel}
%\Delta_{\vk}=\sum_{\vk'}\br{U_{\vk\vk'}-\frac{Y_{\vk\vk''}Y_{\vk''\vk'}}{2(E^{0}-\eta+2\mu)}}\frac{\Delta_{\vk'}}2\sqrt{\frac{(1-4h_{2\vk'}^{2})}{(\xi^{ab}_{\vk'}+  h_{2\vk'}^2\eta)^{2}+\Delta_{\vk'}^{2}}}
%\end{equation}
%At low k, $\Delta_{\vk}$ has weak dependency on k and we can take $\Delta_{\vk'}$ out of the summation,  then go through the same procedure as in \cite{Leggett,Fetter} to renormalize the equations. 
%
%\section{Number equation}
%For number equation \eef{eq:20100909:number}, more approximation can be made in Feshbach resonance.  closed-channel component $h_{2\vk}$ is much extended in k-space due to the tight-binding, and $h_{1\vk}$ is significant up to the order of Fermi energy $E_{F}$.  It seems OK to assume that most weight of closed-channel lies beyond $E_{F}$, so when $\epsilon_{k}\gg{E_{F}}$, the integrand becomes simply $h_{2\vk}^{2}$, the summation gives $N_{close}$.  Therefore we have 
%\begin{equation}\label{eq:20101004:number}
%N_{open}=\sum_{\vk}{}^{'}\left(1-\sgn_{k}\sqrt{1-4 h_{1\vk}^2-4 h_{2\vk}^2}\right)-h_{2\vk}^{2}
%\end{equation} 
%Here the summation only goes up to the order of $E_{F}$.
%
%
%Put \eef{eq:20100915:F} into number equation \eef{eq:20100909:number}, we have\footnote{Here we drop $\sgn_{k}$ as $(\epsilon^{ab}_{\vk}-2\mu+  h_{2,0}^2\eta)$ takes care of the sign}
%\begin{equation}\label{eq:20101011:number}
%N=\sum{1-\sqrt{1-4h_{2\vk}^{2}}\frac{(\epsilon^{ab}_{\vk}-2\mu+  h_{2\vk}^2\eta)}{{\sqrt{{(\epsilon^{ab}_{\vk}-2\mu+  h_{2\vk}^2\eta)^{2}+\Delta^{2}}}}}}
%\end{equation}
%This equation also requires to take into account that $h_{2\vk}$ approaches 0 at high momentum to be convergent.
%\section{summary}
%Now all summation goes in order of Fermi energy $E_F$, so we can replace closed-channel component $h_{2\vk}$ with $h_{2,0}=h_{2\,\vk=0}$
%\begin{equation}
%h_{1\vk}=\frac{\Delta}2\sqrt{\frac{(1-4h_{2\vk}^{2})}{(\epsilon^{ab}_{\vk}-2\mu+  h_{2\vk}^2\eta)^{2}+\Delta^{2}}}
%\end{equation}
%Gap equation 
%\begin{equation}\label{eq:20101004:gapS}
%\nth{{t_{0}}(\mu)}=\sum_{\vk}
%\br{\nth{\epsilon_{\vk}}-\frac{\sqrt{(1-4h_{2\vk}^{2})}}{\sqrt{{(\epsilon^{ab}_{\vk}-2\mu+  h_{2\vk}^2\eta)^{2}+\Delta^{2}}}}}
%\end{equation} 
%\begin{gather}
%{t_{0}}(\mu)=\br{1-\tilde{U}\tilde{ K}}^{-1}\tilde{U}\label{eq:20101004:t011}\\
%\tilde{U}_{\vk\vk'}=\nth{2} \br{U_{\vk\vk'}-\frac{Y_{\vk\vk''}Y_{\vk''\vk'}}{2(E^{0}-\eta+2\mu)}}\label{eq:20101004:tu11}\\
%{K}=\nth{\epsilon_{\vk}}\delta_{\vk\vk'}\label{eq:20101004:tk11}
%\end{gather}
%Here Eqs. (\ref{eq:20101004:t011}-\ref{eq:20101004:tk11}) follows the same two-body formula for zero-energy T-matrix element.  However, the detuning is shifted by a many-body quantity $\mu$ that should be determined by solving gap equation with number equation.  
%And number equation
%\begin{equation}\tag{\ref{eq:20100909:number}}
%N=\sum_{\vk} \left(1-\sgn_{k}\sqrt{1-4 h_{1\vk}^2-4 h_{2\vk}^2}\right)
%\end{equation} 
%
%\subsection{$G_0$ and $\eta$}
%%In the full region, $\eta$ should be close to the binding-energy of the closed-channel bound-state all the time.  The sweep in detuning should always be smaller than $\eta$ in order.  So maybe it is fine to take it as constant of binding energy $E^0$.  However, this quantiy is a two-bdoy quantity and very specific for different case. 
%%
%%$G_0=\alpha\phi_0$, where $\phi_0$ is normlized wave function of two-body closed-channel bound-state, strictly two-body.  Also, it is different from case to case. Its Pauli exclusion effect only shows up in many-body physics. It is determined by the real size of closed-channel bound-state and therefore relates to binding energy $E^0$.  
%%
%%For the same $a_s$ or $t_0$, $G_0$ or $\eta$ can take different value for different resonance.  Maybe we can take it as a different exogenous parameter. But we still need to find a way to estimate/deduce it from measurable quantities.  It is possible that they only relate to many-body quantities (same number for different densities maybe?)  because they are pertinent in many-body physics.
%%
%%\subsection{more on $G_{k}$ and $\eta$}
%$\phi^{0}$ the w.f. of two-body closed-channel bound state varies in different Feshbach resonance and is a case-specific quantity.  Its relevant property is included in the general two-body Feshbach resonance formula's $\Delta{B}$.  However, its other properties does matter in narrow resonance problem.  Fortunately, this can be simplified if it is a close-to-threshold bound-state, that mostly is outside the potential.  Therefore, most weight of $\phi^{0}$ follows the simple form 
%\[
%\phi^{0}(r)=C\frac{e^{-\kappa{r}}}{r}
%\]
% where $\frac{\hbar^{2}\kappa^{2}}{2m}=E^{0}$ and $\kappa>0$.  $C^{2}=\frac{\kappa}{2\pi}$.  Convert it into k-space
% \begin{equation}\label{eq:20101011:phi}
%\phi^{0}_k=\nth{(2\pi)^{\frac32}}\int{d^{3}\vr\phi^{0}(r)e^{-i\vk\cdot\vr}}=\nth{\pi}\frac{\kappa}{k^{2}+\kappa^{2}}
%\end{equation}
%Now we can estimated $G_{k}^{2}\eta$. $G_{k}=\alpha\phi^{0}_{k}$ where 
%\[|\alpha|^{2}<n\sim{E_{F}^{3/2}}\]
%\[
%\abs{\phi^{0}_{k=0}}^{2}=\nth{\pi^{2}\kappa^{3}}\sim\nth{(E^{0})^{3/2}}
%\]
%And $\eta\sim{E^{0}}$, So 
%\[
%G_{k}^{2}\eta<G_{k=0}^{2}\eta<E_{F}(\frac{E_{F}}{E^{0}})^{1/2}
%\]
%And we have $E_{F}\ll{E^{0}}$, so this term is much smaller than $\mu\sim{E_{F}}$ in BCS side. For now, we just omit this term.  Considering $G_{k}^{2}\ll1$, we can expand $\sqrt{1-4G_{k}^{2}}$ in \eef{eq:20101011:number} and \eef{eq:20101004:gapS} 
%\begin{gather}
%\nth{{t_{0}}(\mu)}=\sum_{\vk}
%\br{\nth{\epsilon_{\vk}}-\frac{1}{\sqrt{{(\epsilon_{\vk}-2\mu)^{2}+\Delta^{2}}}}
%+\frac{2\alpha^{2}|\phi^{0}_{k}|^{2}}{\sqrt{{(\epsilon_{\vk}-2\mu)^{2}+\Delta^{2}}}}}\label{eq:20101011:gapa}
%\\
%N=\sum{1-\frac{(\epsilon_{\vk}-2\mu)}{{\sqrt{{(\epsilon_{\vk}-2\mu)^{2}+\Delta^{2}}}}}
%+2\alpha^{2}|\phi^{0}_{k}|^{2}\frac{(\epsilon_{\vk}-2\mu)}{{\sqrt{{(\epsilon_{\vk}-2\mu)^{2}+\Delta^{2}}}}}}
%\label{eq:20101011:gapb}
%\end{gather}
%%Here $\phi^{0}_{k}$ takes the form in \eef{eq:20101011:phi} and we have an extra exogenous parameter $\kappa$ ($\approx{}E^{0}$) (which seems to independent of the normal Feshbach resonance formula).
%%\begin{equation}
%%a_{s}=a_{bg}(1+\frac{\Delta{B}}{B-B_{0}})
%%\end{equation}
%%This seems to make sense that in Feshbach resonance, what is really matter is the coupling energy, which involves  the integration of $Y_{kk'}$ and $\alpha^{2}\phi^{0}$ ($\alpha^{2}$ is the closed-channel weight in two-body physics).  The resonance is strong even when $\alpha^{2}$ is small but if the coupling strength $Y_{kk'}$ is strong.  However, the Pauli exclusion only cares about the wave function or $\alpha$, not coupling strength.   Therefore, we do need one extra parameter to describe this.  
%%
%%A more convenient choice for $\phi^{0}$ is to normalized to total number N, i.e. $\sum|\phi^{0}_{k}|^{2}=N$.  So $0<|\alpha|^{2}<1$ is the ratio of atoms in closed-channel to total number.  For this normalization, we find 
%%\begin{equation}
%%\phi^{0}_{k}=\sqrt{8\pi\kappa\rho}\nth{k^{2}+\kappa^{2}}
%%\end{equation}
%%
%% \cite{shizhongSumRule} derived  (eq. (25))
%%\begin{equation}\label{eq:20101025:fr}
%%F(\vr)=\frac{m\Delta}{4\pi\hbar^{2}}\frac{1-r/a_{s}}{r}
%%\end{equation}
%%
%%we have the pre-factor $\frac{m\Delta}{4\pi\hbar^{2}}$ for the normalization and with \cite{Leggett} (Eq. (4.A.17)), we can express the relation between $\Delta$ and $\alpha$.  (assume $\kappa$ varies slowly with detuning)



\backmatter

\bibliography{../citation}
\bibliographystyle{apalike}

\end{document}
\endinput
%%
%% End of file `thesis-ex.tex'.
