% !TeX root =thesis.tex
% 
In this thesis, we study the narrow Feshbach resonance in  the three-species case where two channels share the same species.  

In general, for the narrow resonance without a shared species, the main correction to the single-channel result comes from the  extra counting of the atoms in the open-channel, which leads to the extra shift $2\mu$ in $\tilde{a}_{s}$, and the closed-channel, which leads to the extra number equation.  Two number equations exist, one for the open-channel and one for the closed-channel.  The open-channel number equation resembles the number equation of the single-channel model.

When there is a common species,  however, the  Pauli exclusion between two channels due to the common species in the three-species narrow resonance, calls for careful  consideration.  Our treatment follows the spirit of the so-called ``universality'' idea.  For a dilute system with a short-range potential, such as the dilute ultracold alkali gas, the short-range part of a two-body correlation does not significantly change from two-body  to many-body.  This particular feature justifies using the two-body quantities (e.g. the s-wave scattering length $a_{s}$) as the boundary condition for the many-body correlations.  The most part of the correlations, where no two particles are close in short-range, is essentially free from interaction. This long-range part of the correlations does change from two-body to many-body.  Unique in our problem is the involvement of the Feshbach resonance.  A resonance makes the system extremely sensitive to even small change in the relative weight between two channels. We use the fact that,  in the Feshbach resonance, the closed-channel bound state, $\phi_{0}$, as a short-range object,  is still not very sensitive to the resonance; hence, we can, within the universality idea,  apply a simple boundary condition to the closed-channel correlation. However, the indirect interaction mediated by the closed-channel dominates the direct interaction within the open-channel. Consequently, a small change in the closed-channel bound-state weight \emph{does} affect the short-range wave function in the open-channel, originally expressed through the boundary condition using the s-wave scattering length $a_{s}$. 

When the spatial extension of  the closed-channel bound state is  of the order of the   inter-particle distance, $a_{0}$, or even larger, the Feshbach resonance in the many-body context is a genuine three-species many-body problem and no simple solution is available.  By contrast, when the bound-state's spatial extension is much smaller than the interparticle distance, the ratio ($a_{c}/a_{0}$) serves as the expansion parameter and we can extract  the effect of the inter-channel Pauli exclusion perturbatively.  In essence, we can then  ignore the many-body effects within  closed-channel bound-states, while only taking into consideration  the Pauli exclusion between channels.  A few new parameters need to be introduced and can be calibrated from experiments, such as $\lambda_{1}$ (Eq. \ref{eq:pathInt2:lambda1}), $\lambda_{2}$ (Eq. \ref{eq:pathInt2:lambda2}).  Mean field properties can still be determined through gap equations and number equations similar as in the single-channel case.  The excitation modes are also close to the original single-channel result with correction of the order of $a_{c}/a_{0}$, where $a_{c}$ is the spatial extension of the uncoupled closed-channel  bound state and $a_{0}$ is the average interparticle distance.

We can distinguish three different regimes for the many-body energy scale $E_F$ when compared with the two-body physics.  
\begin{enumerate}
\item The regime where $E_F$ is much smaller than the characteristic energy $\delta_c$ (See Eq. \ref{eq:intro:deltaC} in Sec. \ref{sec:intro:twobody}). Around resonance, the closed-channel weight is negligible in this regime. We can then ignore the closed-channel completely and approximate the effective open-channel interaction by a  pseudo-potential characterized by the s-wave scattering length, $a_s$. 
\item The regime where  $E_F$ is larger than $\delta_c$, but smaller than the closed-channel bound state binding energy $E_b$, or the bare detuning $\eta$.  The closed-channel weight is significant in this regime and cannot be ignored, which means $n_{\text{open}}+n_{\text{close}}=n_{\text{total}}$. Also, we need to take into account the fact that, in the resonance formula of $a_s$, shifting should be counted from the Fermi surface instead of from zero as in two-body physics.    These effects, considered previously\cite{GurarieNarrow}, are  also carefully analyzed in this thesis.  Furthermore, the Pauli exclusion between two channels needs to be taken into account when these two channels share a common species.  Nevertheless, at  low momentum, where the open-channel has most of its weight, the closed-channel bound state only has very small weight.  Consequently, we can still treat the Pauli exclusion between channels perturbatively using the expansion coefficient  $E_F/\eta$. 
\item The regime where  $E_F$ is larger than the binding energy $E_b$ or the bare detuning $\eta$. We have a genuine three-species many-body problem. This regime can be achieved in two ways.  One way requires a very large $E_F$, which indicates a very dense system.  However, it would be hard to imagine that  the original dilute alkali gas model still applies in this case. Various approximations in the model would probably break down beforehand, such as the pseudo-potential approximation with $a_s$. The second way requires a very small $\eta$, which indicates a genuine three-species many-body problem.  This remains an open problem.   
\end{enumerate}
 

 



This thesis focuses on the second regime.  Careful analysis leads us to distinguish two related but different concepts, the total weight of the closed-channel  and the weight of the closed-channel in a particular momentum level.  
       Within our assumptions, namely, the spatial extension of a closed-channel bound state is much smaller than the interparticle distance, a very low occupation in low-momentum levels ($k\lesssim{}k_{F}$) of the closed-channel persists despite the large total weight of the closed-channel.     A suitable small parameter $E_F/\eta$ (or $a_0/a_c$), upon which a perturbation theory can be developed over a non-perturbative zeroth order solution, emerges with this discovery.  Pairing in many-body problems is known to be  non-perturbative and has to be handled with proper non-perturbative techniques.  Nonetheless, we can concentrate the non-perturbative part into the zeroth order  (broad resonance) solution, while treating the other corrections perturbatively.   The resulting theory  handles the two channels in steps self-consistently\footnote{Here by ``self-consistent'', we refer to the self-consistency in treating the close-channel according to zeroth order in the open-channel pairing.  The BCS-type treatment of pairing in the open-channel   is known to be not self-consistent. }. 
       
       In  summary, a two-channel model has a two-component order parameter $(\Delta_{1},\Delta_{2})$: one component for each channel.  The order parameter for the closed-channel can be determined by the number equation of the closed-channel (\eef{eq:pathInt2:closeD2})
\begin{equation}\tag{\ref{eq:pathInt2:closeD2}}
N_{close}\approx\sum_{\vk}\frac{\Delta_{2}^2}{(\xi_{\vk}+\eta)(2\xi_{\vk}+\eta)}
\end{equation}
 Where $\xi_{\vk}=\hbar^{2}k^{2}/2m-\mu$ and $\eta$ is the energy difference between two channels. The renormalized gap equation is given in \eef{eq:pathInt2:gapRenorm}
 \begin{equation}\tag{\ref{eq:pathInt2:gapRenorm}}
1=\frac{4\pi{\tilde{a}_{s}(\mu,\lambda_{1})}}{m}\sum(\nth{2E_{\vk}}-\nth{2\epsilon_{\vk}}-\frac{\Delta_{2}^{2}\xi_{\vk}}{4(\xi_{\vk}+\eta){E_{\vk}^{3}}})
	-\frac{\lambda_{2}}{\Delta_{1}}
\end{equation}
where $E_{\vk}=\sqrt{\xi_{\vk}^{2}+\Delta_{1}^{2}}$. Here, $\tilde{a}_{s}(\mu,\lambda_{1})$ is the two-body open-channel effective  s-wave scattering length with  an additional shifting $2\mu+\lambda_{1}$ as Eq. \ref{eq:pathInt2:asKshift}. 
\begin{equation}\tag{\ref{eq:pathInt2:asKshift}}
{a}_{s}=a_{\text{bg}}(1+\frac{\mathcal{K}}{\delta-2\mu-\lambda_{1}})
\end{equation}
  $\lambda_{1}$ and $\lambda_{2}$ are two new parameters describing the overlapping of the two channels, and they can be calibrated from  experiments (see Eqs. \ref{eq:pathInt2:lambda1} \ref{eq:pathInt2:lambda2}, and Sec. \ref{sec:pathInt2:lambda}).  The open-channel number equation is \eef{eq:pathInt2:openNum}
\begin{equation}\tag{\ref{eq:pathInt2:openNum}}
\begin{split}
N_{open}\approx\sum_\vk\mbr{\frac{E_\vk-\xi_\vk}{2E_\vk}(1+\frac{\Delta_{1}}{\eta}\zeta)-\frac{\Delta_{1}^{3}}{4E_\vk^{3}}\zeta
	}	
\end{split}
\end{equation}
Here $\zeta=\Delta_{2}^{2}/\Delta_{1}\eta\ll{}E_{\vk}$ appears in multiple places as the small expansion parameter.  The open-channel number equation and the associated gap equation need to be solved self-consistently to get  the mean field result ($\Delta_{1}$ and $\mu$).  One of the most noteworthy new phenomena is probably that the open-channel gap $\Delta_{1}$ saturates in the BEC side of  a narrow resonance. 

There are three fermionic excitation modes. Their spectrums with the first order correction due to the inter-channel Pauli exclusion are given by Eqs. \ref{eq:pathInt2:xiExpand}-\ref{eq:pathInt2:xiExpand3}
\begin{align}
E_{1\vk}&\approx{}E_{\vk}+u_{\vk}^{2}\Delta_{1}\zeta\tag{\ref{eq:pathInt2:xiExpand}}\\
E_{2\vk}&\approx{}E_{\vk}-v_{\vk}^{2}\Delta_{1}\zeta\tag{\ref{eq:pathInt2:xiExpand2}}\\
E_{3\vk}&\approx{}\epsilon_{\vk}+\eta+\frac{\zeta}{2}\Delta_{1}
\tag{\ref{eq:pathInt2:xiExpand3}}
\end{align}
With a two-component order parameter, the bosonic collective fluctuation modes are rich.  We explored two  modes about  phase fluctuations.  The two-component in-phase fluctuation is massless and the low-energy one.  It is similar to the Anderson-Bogoliubov modes in the single-channel problem with a small correction in the order of $\zeta$.  The new out-of-phase fluctuation is gapped and the minimum excitation energy is in the order of the pair-breaking energy ($\Delta_{1}$ in the BCS-like states, $\mu>0$ and $\sqrt{\mu^{2}+\Delta_{1}^{2}}$ in the BEC-like states, $\mu<0$).  

       In our approach, we take the broad resonance result (or the single-channel crossover) as our zeroth order solution, upon which the expansion is performed.  It is however known that the simple BCS-type pairing treatment is not adequate  to quantitatively describe the whole BEC-BCS crossover region.  Therefore the zeroth order solution used in this thesis (simple BCS type ansatz or saddle point) can be improved through further theoretical development.  Nevertheless, we expect the perturbative approach used here to build the narrow resonance from the single-channel crossover result to be still valid then.  Once the zeroth order solution (for a broad resonance or a single channel BEC-BCS crossover model) is patched over with whatever advancement, the correction of the narrow resonance in such a parameter regime, can still be obtained with a procedure similar as the one discussed in this thesis.  
       
       
\begin{unsure}
The theory we have developed here is for zero-temperature only. This limit simplifies the calculations considerably.  However, an extension to finite temperature along the same spirit should be possible.  The two-component order parameters should persist at low-temperature.  At higher temperature, these components  are likely to decay at different temperatures.  The open-channel order parameter, associated with a much lower energy scale (of the order of  the Fermi energy or even lower), should turn to zero first.  Then the system becomes a normal gas with a Feshbach resonance.  From the above discussion, many-body corrections due to the narrow resonance (both the intra- and inter-channel Pauli exclusion) seem to be agnostic to whether the system is in superfluid state or not.  Thus, we expect that these many-body corrections can be carried out in a similar fashion.   There should still be corrections over the detuning due to the chemical potential and the  inter-channel Pauli exclusion ($\lambda_{1}$) although a system is more likely to be a Fermi liquid in the BCS side and a gas of normal fermion-dimer-molecules on the BEC side in such a temperature.   
\end{unsure}
