% !TeX root =thesis.tex

\chapter{Two-channel many-body model\label{ch:path2}}
For narrow resonance, atoms have considerable weight in close-channel and the Pauli exclusion between two channels cannot be neglected.  A many-body framework needs to include both channels.  Before dive into the detailed calculation, let us make some rough estimation on scale and build the intuition.  The two-body problem is well understood (Sec. \ref{sec:intro:twobody}).  %Extended to many-body, three different types of Pauli exclusion requires to be considered: Pauli exclusion within open-channel, Pauli exclusion within close-channel and Pauli exclusion between two channels. Let us estimate the scale of each with an artificial question.  We  calculate how much is reduced in two-body wave function by one single type of Pauli exclusion when a new pair is added.  This quality is quite unphysical without any real measurable counterpart, but it serves as a starting point to build intuition. 
Let us discuss how much would change for each channels in many-body system.  Comparing to two-body system, a low-energy (cold) many-body system of fermions, mostly modifies the low energy part, around or below to Fermi energy; while leaves the high energy part intact.  A two-body open-channel wave function is almost like a free wave ($k=0$) with small kernel in $r_c$.  In momentum space, this wave function has most weight within Fermi momentum, and therefore completely modified by many-body effect. Next open-channel pair has to occupy the next available level in $k$ instead of the lowest energy two-body level.  This suggests that open-channel requires full many-body treatment.  The situation is quite different in close-channel.  Close-channel bound state\footnote{Here we want to stress the difference between the close-channel bound-state $\phi_{i}$ and the real bound state formed by two channels in BEC side.   The close-channel bound state $\phi_{i}$ is the eigenstate of hamiltonian of isolated close-channel, and is not a real eigenstate for the real space.  On the other hand, the full (two-channel) hamiltonian has eigenstate of bound-state at large detuning (in two-body context) or BEC-side (in many-body context).  The bound state solution for two-channel eigenstate has components both in open-channel and close-channel.  Often the open-channel weight is larger when close to resonance and therefore those bound-states are often called open-channel bound-states.  In this thesis, most time we only deal with close-channel bound-state, $\phi_{i}$.},  $\phi_{0}$ ,spreads mostly in $[0,1/a_{c}]$ range of momentum, only has tiny fraction within Fermi energy range if we assume the close-channel bound state is much smaller than inter-particle distance. The smallness of this overlap makes sure that many-body effects in close-channel can be  treated perturbatively.  As discussed in Sec. \ref{sec:intro:as} and Appendix \ref{sec:pathInt2:short-range}, the high momentum part of two-body correlation just follows the two-body wave function,  therefore, we can set correlation as 
\begin{equation}\label{eq:pathInt2:hphif}
h_{\vk}\sim\phi_{0\vk}f(\vk)
\end{equation}
 where $f(\vk)=0$ for $k\gg{k_{F}}$.  

%This is fairly straight-forward Pauli exclusion within open-channel.  Open-channel wave function is almost like a free wave ($k=0$) with small kernel in $r_c$.  The open-channel is like a Fermi sea with clear step at chemical potential in BCS side. It is less step-like in BEC  side, nevertheless, the lowest level ($k=0$) occupation is large and in the order $1$.   If adding a pair with the original wave-function, the normalization is simply going to be reduced by a factor in order of $1$.  Actually in BCS side, the next open-channel pair can only occupy the next $k$ level, which is nothing like the two-body wave function at all.  A many-body state is nothing like the simple exponential of two-body wave function.  Therefore, this type of Pauli exclusion calls for the most exact treatment, and indeed, we takes the anomalous correlation function, gap, and carry the non-perturbative (BCS) treatment all the way.
%
% Pauli exclusion within close-channel is just the opposite.  Given the assumption that the size of close-channel bound state, $a_c$, is much smaller than the average particle distance, $a_0$.  ????  This effect is in the order of $(a_c/a_0)^4$ (\cite{CobosonPhysicsReports}). ??????    We are going to ignore this effect most of time by taking close-close two-body correlation simply as two-body wave-function $\phi_0$ at high-momentum.  
% 
% The inter-channel Pauli exclusion sits in between of them.  If we take the open-channel as a simple Fermi gas (liquid).  The wave function of close-channel bound state $\phi_0$ spreads over $[0,1/a_c]$ in momentum space, a simple estimation of the overlap with open-channel Fermi sea is  $k_F^3(1/a_c)^{-3}\sim(a_c/a_0)^3\ll1$.  This effect is larger than the Pauli exclusion within close-channel but smaller than 1, therefore, we handle it perturbatively.  
%

Now let us get into the  calculation.  Path integral approach is particular suitable for this problem.   The Hubbard-Stratonovich transformation provides powerful tool to study non-trivial degree of freedom (order parameter) in the system.  It is more or less equivalent to other approaches at mean-filed level (Appendix \ref{ch:mean}). But it has great advantage to be easily extended to explore the fluctuation over the mean-field result.  

For two-channel problem, we write down the Hamiltonian as
\begin{equation}\label{eq:pathInt2:ham2}
\begin{split}
H&=\int{d^{d}r}\bigg\{\sum_{j}\bar\psi_{j}\mbr{\nth{2m}(-i\nabla)^{2}-\mu+\eta_{j}}\psi_{j}\\
	&\qquad-U\bar\psi_{a}(r)\bar\psi_{b}(r)\psi_{b}(r)\psi_{a}(r)-V\bar\psi_{a}(r)\bar\psi_{c}(r)\psi_{c}(r)\psi_{a}(r)\\
	&\qquad-\mbr{Y\bar\psi_{a}(r)\bar\psi_{b}(r)\psi_{c}(r)\psi_{a}(r)+h.c.}
	\bigg\}\\
 &=\int{d^{d}r}\bigg\{\sum_{j}\bar\psi_{j}\mbr{\nth{2m}(-i\nabla)^{2}-\mu+\eta_{j}}\psi_{j}
 	-(\bar\psi\bar\psi)\mtrx{U&Y\\Y^{*}&V}(\psi\psi)
\end{split}
\end{equation}
Here $\eta_{j}$ is the Zeeman energy of the specific hyperfine species.  ``a'' is the common species of two channels, (a,b) is the open channel and (a,c) is the close channel.  All the interactions ($U$, $V$, $Y$), are contact type, this simplifies Hubbard-Stratonovich transformation considerably.  It is plausible as we only study low-energy phenomenon for the short-range potential. 

One drawback of contact interaction is that it is flat in momentum space, in another word, it pairs not only zero central-momentum pairs, but also finite central-momentum pairs.  Nevertheless, the saddle point (mean-field) solution settles at the zero central-momentum pairing, therefore the mean-field solution coincides with that of  variation method derived from pairing of only opposite momentum atoms.  On the other hand, the collective mode (fluctuation of order parameter) emerges naturally from the included non-zero central-momentum interaction.  We do not need to introduce extra interaction term (as in \cite{AndersonBCS}) to study collective mode.   

There are three different types of Hilbert space in this chapter:  the coordinator space and its reciprocal momentum space (infinite dimension);   hyperfine spin, $a,\,b,\,c$ (3 dimension);  two channels $(a,b),\,(a,c)$ (2 dimension).  In principle, fermion field $\psi$ and $\bar\psi$ (and $\Delta$ and $\bar\Delta$ defined according to $\psi$ and $\bar\psi$),  both Grassman numbers, are independent to each other  and  not related as complex conjugate\footnote{Complex conjugate is not a well-defined concept for Grassman algebra}.  This is marked by using ``$\bar{\;}$''(bar) instead of normally used ``$^{\dg}$''(dagger) sign. In this chapter, $^{}\dg$ sign is reserved only for hermitian conjugate matrix in the later two finite-dimension spaces: hyperfine spins ($3\times3$ matrix) or channels ($2\times2$ matrix). 

We can introduce a unitary transformation $Q$ (mixing two channels) to diagonalize the interaction matrix into diagonal matrix $A$.
\begin{equation}
\begin{split}
Q^{\dg}AQ=\mtrx{U&Y\\Y^{*}&V}\equiv{}\tilde{U}\\
\tilde{U}\equiv\mtrx{U&Y\\Y^{*}&V}=Q\mtrx{A_{11}&0\\0&A_{22}}Q^{\dg}
\end{split}
\end{equation}
The finite temperature action is 
\begin{equation}\label{eq:pathInt2:actionFermi}
S(\bar\psi,\psi)=\int^{\beta}_{0}d\tau\int{d^{d}r}\mbr{\sum_{j}\bar\psi_{j}(\partial_\tau-\nth{2m}\nabla^{2}-\mu+\eta_{j})\psi_{j}
-(\bar\psi\bar\psi)Q^{\dg}AQ(\psi\psi)}
\end{equation}

Now $(\psi\psi)$ is column vector and $(\bar\psi\bar\psi)$ is row vector for channels
\begin{equation*}
(\bar\psi\bar\psi)=\mtrx{\bar\psi_{a}\bar\psi_{b}&\bar\psi_{a}\bar\psi_{c}}
\qquad(\psi\psi)=\mtrx{\psi_{b}\psi_{a}\\\psi_{c}\psi_{a}}
\end{equation*}

Similar as in Sec. \ref{sec:pathInt}, we can make Hubbard-Stratonovich transformation on the action.   Introduce a bosonic field as a 2-component vector   and start from the ``fat identity'', where  all the integral constant is absorbed into measure of functional integral of $D(\Delta,\bar\Delta)$ \cite{Altland}.
\begin{equation}\label{eq:pathInt2:indetity}
1=\int{D(\Delta,\bar\Delta)}\exp(-\int{dx}\Delta^{\dg}A^{-1}\Delta)
\end{equation}
\[
\Delta^{\dg}=(\bar\Delta_{1},\bar\Delta_{2})\qquad\Delta=\begin{pmatrix}\Delta_{1}\\\Delta_{2}\end{pmatrix}
\]
here $x$ is four-coordinator,  $\int{dx}=\int^{\beta}_{0}d\tau\int{d^{d}r}$. 

We can make a shift in $\Delta$
\begin{equation}
\Delta\longrightarrow\Delta-A\,Q(\psi\psi)
\end{equation}
Write it into the matrix form
\begin{equation*}
\mtrx{\Delta_{1}\\\Delta_{2}}\longrightarrow
	\mtrx{\Delta_{1}\\\Delta_{2}}-\mtrx{A_{11}&0\\0&A_{22}}\mtrx{Q_{11}&Q_{12}\\Q_{21}&Q_{22}}
	\mtrx{\psi_{b}\psi_{a}\\\psi_{c}\psi_{a}}
\end{equation*}
\begin{equation*}
\mtrx{\bar\Delta_{1},\bar\Delta_{2}}\longrightarrow
	\mtrx{\bar\Delta_{1},\bar\Delta_{2}}-
	\mtrx{\bar\psi_{a}\bar\psi_{b}&\bar\psi_{a}\bar\psi_{c}}
	\mtrx{Q_{11}^{*}&Q_{21}^{*}\\Q_{12}^{*}&Q_{22}^{*}}\mtrx{A_{11}&0\\0&A_{22}}
\end{equation*}

Note that  $\bar{\Delta}_{i}$ is not simple complex conjugate of $\Delta_{i}$ as they are related to $\psi\psi$ and $\bar\psi\bar\psi$. But in our following analysis, to the level of mean-field (saddle point) as well as simple Gaussian level of collective mode (only phase fluctuation), they are indeed complex conjugate (and real for saddle point).  We will treat them as complex conjugate afterward and use $\bar\Delta$ and $\Delta^{\dg}$ more or less arbitrarily.  Furthermore, $\Delta(\vr,\tau)$ carries the the same coordinator as $\psi(\vr,\tau)\psi(\vr,\tau)$ (or frequency-momentum in reciprocal space).  Now the ``fat identity'' Eq. \ref{eq:pathInt2:indetity} becomes 
\begin{equation}
1=\int{D(\Delta_{j},\bar\Delta_{j})}\exp\big\{-\int{dx}
	[\Delta^{\dg}A^{-1}\Delta-(\bar\psi\bar\psi)Q^{\dg}\Delta-\bar\Delta{Q}(\psi\psi)+(\bar\psi\bar\psi)Q^{\dg}AQ(\psi\psi)]\big\}
\end{equation}
And we have 
\begin{equation}
\exp[\int{dx}(\bar\psi\bar\psi)Q^{\dg}AQ(\psi\psi)]
=\int{D(\Delta,\bar\Delta)}\exp\big\{-\int{dx}
	[\Delta^{\dg}A^{-1}\Delta-(\bar\psi\bar\psi)Q^{\dg}\Delta-\bar\Delta{Q}(\psi\psi)]\big\}
\end{equation}
This is ready to be applied to the original action in Eq. \ref{eq:pathInt2:actionFermi}, 
\begin{equation}\label{eq:pathInt2:actionMix}
S_{\tau}(\bar\Delta,\Delta,\bar\psi_{i},\psi_{i})=\int^{\beta}_{0}d\tau\int{d^{d}r}\bbr{\sum_{j}\bar\psi_{j}(\partial_\tau-\nth{2m}\nabla^{2}-\mu+\eta_{j})\psi_{j}
+[\Delta^{\dg}A^{-1}\Delta-(\bar\psi\bar\psi)Q^{\dg}\Delta-\bar\Delta{Q}(\psi\psi)]}
\end{equation}
We can introduce a spinor space similar to Nambu spinor representation in single-channel superconductivity.  
\begin{equation}
\bar\Psi=\mtrx{\bar\psi_{a}&\psi_{b}&\psi_{c}}\qquad\Psi=\mtrx{\psi_{a}\\\bar\psi_{b}\\\bar\psi_{c}}
\end{equation}
The action can be rewritten in a more compact form
\begin{equation}\label{eq:pathInt2:actionMixCompact}
S(\bar\Delta,\Delta,\bar\psi_{i},\psi_{i})=\int^{\beta}_{0}d\tau\int{d^{d}r}
	\mbr{\Delta^{\dg}A^{-1}\Delta-\bar\Psi\mathcal{G}^{-1}\Psi}
\end{equation}
where 
\begin{equation}
\mathcal{G}^{-1}=
\begin{pmatrix}\label{eq:pathInt2:nGDelta}
-\partial_{\tau}+\nth{2m}\nabla^{2}+\mu-\eta_{a}&Q^{*}_{11}\Delta_{1}+Q^{*}_{21}\Delta_{2}&Q^{*}_{12}\Delta_{1}+Q^{*}_{22}\Delta_{2}\\
Q^{}_{11}\bar\Delta_{1}+Q^{}_{21}\bar\Delta_{2}&-\partial_{\tau}-\nth{2m}\nabla^{2}-\mu+\eta_{b}&0\\
Q^{}_{12}\bar\Delta_{1}+Q^{}_{22}\bar\Delta_{2}&0&-\partial_{\tau}-\nth{2m}\nabla^{2}-\mu+\eta_{c}
\end{pmatrix}
\end{equation}
$\nG$ is decoupled in frequency-momentum space between different momentum. 

\begin{equation}\label{eq:pathInt2:nGDeltaK}
\mathcal{G}^{-1}=
\begin{pmatrix}
i\omega_{n}-\xi_{k}-\eta_{a}&Q^{*}_{11}\Delta_{1}+Q^{*}_{21}\Delta_{2}&Q^{*}_{12}\Delta_{1}+Q^{*}_{22}\Delta_{2}\\
Q^{}_{11}\bar\Delta_{1}+Q^{}_{21}\bar\Delta_{2}&i\omega_{n}+\xi_{k}+\eta_{b}&0\\
Q^{}_{12}\bar\Delta_{1}+Q^{}_{22}\bar\Delta_{2}&0&i\omega_{n}+\xi_{k}+\eta_{c}
\end{pmatrix}
\end{equation}
here $\xi_{k}=\nth{2m}k^{2}-\mu$.\footnote{Here and afterward in the chapter, we take $\hbar=1$.} The diagonal elements of second line (${b}$) and third line (${c}$) corresponding to negative energy, because the particular choice in the Nambu spinor ($\bar\psi_{b}$ and $\bar\psi_{c}$).  The non-diagonal elements mix the different spins  and therefore lead to a number-non-conserved theory (mix-up between $\psi_{a}$ with $\bar{\psi}_{b,c}$).  

$\nG$ can be further simplified by introducing mixture within two channels.
\begin{equation}\label{eq:pathInt2:Ddef}
D\equiv\mtrx{D_{1}\\D_{2}}=Q^{\dg}\Delta
\end{equation}
\begin{equation*}
\bar\Delta=\bar{D}\,Q^{\dg}\qquad\Delta=Q\,D
\end{equation*}
It is not difficult to see that $D$ actually describe the off-diagonal coupling in fermionic field $\Psi$ and is the counterpart of order parameter, gap, in single-channel problem instead of $\Delta$ here.  We will see it  more clearly when  discussing mean-field solution. 

We furthermore assume $\eta_{a}=\eta_{b}=0$, $\eta_{c}=\eta$. (i.e., $\eta$ is the absolute Zeeman energy difference of two channels.) In frequency-momentum space, 
\begin{equation}\label{eq:pathInt2:nG}
\mathcal{G}^{-1}=i\omega_{n}I+
\begin{pmatrix}
-\xi_{k}&D_{1}&D_{2}\\
\bar{D}_{1}&+\xi_{k}&0\\
\bar{D}_{2}&0&+\xi_{k}+\eta
\end{pmatrix}
\end{equation}
 

and the first term $\bar\Delta{}A^{-1}\Delta$ in action Eq. (\ref{eq:pathInt2:actionMix}) becomes
\[
\bar\Delta{}A^{-1}\Delta=\bar{D}Q^{\dg}A^{-1}QD=\bar{D}\tilde{U}^{-1}D
\]


Now we can change the functional variable into $D(\bar{D})$ 
\begin{equation}\label{eq:pathInt2:actionMixD}
S(\bar{D},D,\bar\psi_{i},\psi_{i})=\int^{\beta}_{0}d\tau\int{d^{d}r}
	\mbr{\bar{D}\tilde{U}^{-1}D-\bar\Psi\mathcal{G}^{-1}\Psi}
\end{equation}
The action is now bilinear to $\Psi$ and we can integrate it out formally
\begin{equation}\label{eq:pathInt2:actionD}
S(\bar{D},D)=\int{dx}\br{\bar{D}\tilde{U}^{-1}D-\tr\ln\nG}
\end{equation}
Note that at this stage either $D(\vr,\tau)$ or $\Delta(\vr,\tau)$ is not necessarily homogenous in space or psudo-time.  



\section{Diagonalizing Green's function\label{sec:diagonalGreen}}
Eq. \eqref{eq:pathInt2:actionD} looks fairly innocent and compact.  Nevertheless, it has all the physics in it and is not as simple as it looks.  The major problem comes from term $\tr\ln\nG$, which includes  logarithm and trace over an infinite-dimension matrix.   All these operations are fairly straight-forward if we can diagonalize  Green's function (or its inverse) in a proper basis.    Actually, we have already off on a good start.  It is not hard to see $\nG$ is decoupled in frequency-momentum space (Eqs.  \ref{eq:pathInt2:nGDeltaK}, \ref{eq:pathInt2:nG}).  It is however mixed in the $3\times3$ hyperfine-spin space though.  The rest of this section is dedicated to diagonalize the $3\times3$ matrix in hyperfine-species space.  Note that in principle, the following discussion is not limited for constant $D$, but also applies to inhomogenous $D(\omega_{n},\vk)$ as well because the $3\times3$ hyperfine space is independent to the frequency-momentum space.   Here we use the approach in Sec. \ref{sec:diagonalizeGreen1} to diagonalize it.   
In current problem, we need to diagonalize a $3\times3$ matrix Eq. \ref{eq:pathInt2:nG}, in another word, we need to figure out the Bogoliubov canonical transformation over which the hamiltonian/action is diagonalized.   Eigen-problem of $3\times3$ matrix involves solving a cubic equation. An exact solution exists in principle.  However,  it offers little intuition in writing the exact result, instead we  find that  spectrum from the broad-resonance, where the only effect of close-channel is to modify the effective interaction of open-channel, serves a reasonable lowest order approximation and we proceeds to find the next order of correction over it (See Appendix \ref{sec:diagonalize} and \ref{sec:pathApp:consistency}). 


Within the assumption that spectrum  deviates not too much from na\"{i}ve broad-resonance solution, we  break down the unitary transformation into two steps $T$ and $L$. 
\begin{equation}\label{eq:pathInt2:B}
B_k=L_k^{\dg}T_k^{\dg}G_k^{-1}T_kL_k
\end{equation} 
Here $B_{k}$ is the diagonal matrix and $T$ and $L$ are both unitary transformation.  We take $T$ as the transformation at broad resonance, i.e., when we can ignore Pauli exclusion between channels. 
\begin{equation}
T_k=\mtrx{u_k&v_k&0\\-v_k&u_k&0\\0&0&1}
\end{equation}
Here we define 
\begin{gather}
v_{\vk}^{2}\equiv1-u_{\vk}^{2}\equiv\nth{2}\br{1-\frac{\xi_{\vk}}{E_{\vk}}}\\
E_{\vk}\equiv(\xi_{\vk}^{2}+D_{1}^{2})^{1/2}
\end{gather}

In the broad-resonance, where only the BCS pairing in open channel needs to be considered, $T$ is enough to diagonalize $G^{-1}$ and $L$ is simply identity matrix.  %Here we will try to approximate it to the first order correction due to Pauli exclusion between two-channel.  
In the narrow resonance, $T$ cannot diagonalize $G^{-1}$ because of the Pauli exclusion between channels, and therefore $L$ stands the extra correction due to Pauli exclusion in the canonical transformation. Basically, we seek the solution around the BCS ansatz ($T$ transform). 
Apply $T$ onto $G^{-1}$, we have 
\begin{equation}\label{eq:pathInt2:G2}
T_k^{\dg}G_k^{-1}T_k=i\omega_nI+\mtrx{-E_k&0&u_kD_2\\0&+E_k&v_kD_2\\u_kD_2&v_kD_2&+\xi_k+\eta}
\end{equation}
We regard the off-diagonal elements as perturbation.  This matrix can then be diagonalized with  unitary transformation $L_{\vk}$ within the first order of $D_{2}^{2}/(E\eta)$  (Please see Appendix \ref{sec:diagonalize} for details.)
\begin{equation}\label{eq:pathInt2:Bapprox}
\begin{split}
B_{\omega_{n},\vk}&=i\omega_{n}I-
	\mtrx{E_{1}{}_{\vk}&0&0\\0&-E_{2}{}_{\vk}&0\\0&0&-E_{3}{}_{\vk}}\\
	&\approx{}i\omega_{n}I-
	\mtrx{E_{\vk}+\frac{D_{2}^{2}u_{\vk}^{2}}{\eta}&0&0\\
	0&-\br{E_{\vk}-\frac{D_{2}^{2}v_{\vk}^{2}}{\eta}}&0\\0&0&-\br{\xi_{\vk}+\eta+\frac{D_{2}^{2}}{2\eta}}}
%	&=
%	\mtrx{i\omega_{n}-E_{\vk}&0&0\\0&i\omega_{n}+E_{\vk}&0\\0&0&i\omega_{n}+\eta}
%	+\mtrx{-\frac{D_{1}^{2}}{\eta}&0&0\\0&-\frac{D_{2}^{2}}{\eta}&0\\0&0&+\frac{D_{1}^{2}+D_{2}^{2}}{2\eta}}\\
%	&\equiv{}B^{(0)}_{\vk}+B^{(1)}_{\vk}
\end{split}	
\end{equation}
Here we choose the sign of $E_{1,2,3}$  to make their zeroth order positive.  Therefore, the Fermi sea of the redefined Bogoliubov quasiparticle is always filled up without extra particle-hole remapping.  We have 
\begin{align}\label{eq:pathInt2:xiExpand}
E_{1\vk}&\approx{}E_{\vk}+\frac{D_{2}^{2}u_{\vk}^{2}}{\xi_{\vk}+\eta}&\equiv{}&E_{\vk}+\gamma_{1\vk}\\
E_{2\vk}&\approx{}E_{\vk}-\frac{D_{2}^{2}v_{\vk}^{2}}{\xi_{\vk}+\eta}&\equiv{}&E_{\vk}+\gamma_{2\vk}
\label{eq:pathInt2:xiExpand2}\\
E_{3\vk}&\approx{}\xi_{\vk}+\eta-\frac{D_{2}^{2}}{2(\xi_{\vk}+\eta)}&\equiv{}&\xi_{\vk}+\eta+\gamma_{3\vk}
\label{eq:pathInt2:xiExpand3}
\end{align}

\begin{equation}\label{eq:pathInt2:L1}
L_{\vk}\approx{}I+
\mtrx{0&-\frac{D_{1}{}D_{2}{}}{4E^{2}_{\vk}}&u_{\vk}\\
\frac{D_{1}{}D_{2}{}}{4E^{2}_{\vk}}&0&v_{\vk}\\
-u_{\vk}&-v_{\vk}&0
}\frac{D_{2}{}}{\eta}
\equiv{}I+\delta_{k}\qquad
L^{\dg}_{\vk}=I-\delta_{\vk}
\end{equation}
Here we use $uv=D_{1}/2E$.    Note that $L$ and $L^{\dg}$ are unitary only to the first order of $D_{i}/\eta$
And now it is easy to express the Green's function as
\begin{equation}
G_{\vk}=T_{\vk}L_{\vk}B_{\vk}^{-1}L_{\vk}^{\dg}T_{\vk}^{\dg}
\end{equation}
This is ready to be expanded over the perturbation in order of  $D_{2}^{2}/(E\eta)$ or $D_{i}/\eta$.  It is easy to see that all $\omega_{n}$ dependence concentrates on $B_{\vk}$, which is linear in $\omega_{n}$  and simplifies the Matsubara frequency summation considerably.   
\begin{subequations}\label{eq:pathInt2:Gexpand}
\begin{gather}
G_{k}\approx{}T_{\vk}B_{\omega_{n},\vk}^{-1}T_{\vk}^{\dg}+T_{\vk}\delta_{\vk}B_{\omega_{n},\vk}^{-1}T_{\vk}^{\dg}
	-T_{\vk}B_{\omega_{n},\vk}^{-1}\delta_{\vk}T_{\vk}^{\dg}
	\equiv{}G_{\vk}^{(0)}+G_{\vk}^{(1)}\\
	G_{\vk}^{(0)}=T_{\vk}B_{\omega_{n},\vk}^{-1}T_{\vk}^{\dg}\\
	G_{\vk}^{(1)}=T_{\vk}\delta_{\vk}B_{\omega_{n},\vk}^{-1}T_{\vk}^{\dg}
	-T_{\vk}B_{\omega_{n},\vk}^{-1}\delta_{\vk}T_{\vk}^{\dg}
\end{gather}
\end{subequations}

%We only need to keep the zeroth order term for $B_{\vk}^{-1}$ for first order expansion.  
\section{Mean field equation \label{sec:pathInt2:meanfield}}
Use the same techniques as Eq. (\ref{eq:pathInt:diffTr}), we have two saddle point equations for $D_{1}$ and $D_{2}$ from Eq. \eqref{eq:pathInt2:actionD},
 \begin{align}
\frac{\delta}{\delta{}D_{1}}:&\qquad&
(\tilde{U}^{-1})_{11}\bar{D}_{1}+(\tilde{U}^{-1})_{21}\bar{D}_{2}-\tr\mbr{{G_{0}}\cdot\cmtrx{0&1&0\\0&0&0\\0&0&0}}=0
\label{eq:pathInt2:mf01}\\
\frac{\delta}{\delta{}D_{2}}:&\qquad&
(\tilde{U}^{-1})_{12}\bar{D}_{1}+(\tilde{U}^{-1})_{22}\bar{D}_{2}-\tr\mbr{{G_{0}}\cdot\cmtrx{0&0&1\\0&0&0\\0&0&0}}=0
\label{eq:pathInt2:mf02}
 \end{align}
 
 
  If we take $D$ as real constant,
  %\footnote{Actually $D_{2}{_{\vk}}$ cannot be constant at high momentum.  However, for the momentum we are interested, i.e. the momentum lower or in the order of Fermi momentum, it slowly varies.  Therefore  it is reasonable to take it as constant.}     
  we can find the mean field result. Eq. (\ref{eq:pathInt2:nG}) can be inverted to get $G$.  The inversion is quite tedious, but fortunately, we only need two elements of the $G$ matrix ($G_{0\, (21)}$ and $G_{0 \,(31)}$).  The final mean-field equations are (all $D_{i}$'s are taken as real, see Appendix \ref{sec:pathInt2:deriveMF} for detail) 
  \begin{equation}\label{eq:pathInt2:mf}
\mtrx{D_1\\D_2}=\mtrx{U&Y\\Y^{*}&V}\sum_{\vk}\mtrx{h_{1\vk}\\h_{2\vk}}
\end{equation}
  where 
  \begin{gather}
  h_{1\vk}=\av{\psi_{a,-\frac{\vk}{2}}\psi_{b,+\frac{\vk}{2}}}
  =D_{1}\frac{E_{1\,\vk}+\xi_{\vk}+\eta}{(E_{1\,\vk}+E_{2\,\vk})(E_{1\,\vk}+E_{3\,\vk})}\label{eq:pathInt2:h1}\\
  h_{2\vk}=\av{\psi_{a,-\frac{\vk}{2}}\psi_{c,+\frac{\vk}{2}}}
  =D_{2}\frac{E_{1\,\vk}+\xi_{\vk}}{(E_{1\,\vk}+E_{2\,\vk})(E_{1\,\vk}+E_{3\,\vk})}\label{eq:pathInt2:h2}
  \end{gather}


Comparing Eq. \ref{eq:pathInt2:mf} with gap equation for single-channel problem, we  see that $D_{1,2}$ are  direct counterpart of order parameter $\Delta$ in single-channel problem. 
If we simply take the lowest order of $\xi_{i}$, and ignore the close-channel, it is easy to identify $h_{1\vk}\approx{D_{1}}/(2E_{\vk})$ as the many-body wave function $F_{\vk}$ in single channel BEC-BCS crossover problem.  


At high-momentum, both $h_{1\vk}$ and $h_{2\vk}$ behave as $1/\epsilon_{\vk}$ and it diverges when summation is converted to integral in 3D.  This divergence can be mitigated by setting a high-momentum cutoff in integral or recognizing the decay in high momentum of interaction.  It is not unique in many-body and exists in the two-body physics if we take the high-momentum component of wave function simply as $1/\epsilon_{\vk}$ as well.  In the next section, we proceed to  remove divergence of summation in $h_{1\vk}$ and $h_{2\vk}$ by noting the same divergence in two-body wave function and manipulating accordingly. 
%It is interesting to look at Eq. \ref{eq:pathInt2:h2} more carefully, in BCS side, $\mu\approx{}E_{F}$, at low momentum ($k<k_{F}$), $\xi_k<0$ and $h_{2\,\vk}$ is close to 0; at higher momentum where $\epsilon_{\vk}>\mu$, we have $E_{2\,\vk}\approx\xi_{\vk}$, $E_{3\,\vk}\approx\xi_{\vk}+\eta$.  All these lead to $  h_{2\vk}\approx\frac{D_2}{(2\epsilon+\eta)}$, which coincides with two-particle wave function for such range ( momentum $k$ smaller than inverse of potential range $1/a_{c}$ as well as  characteristic of close-channel binding energy ``$\kappa$'', but larger than other scale, such as $k_{F}$, $1/a_{s}$).
%;  at very high momentum where $\epsilon_{\vk}>\eta, 1/a_c$, $D_2$ can no longer be treated as a constant and decays with energy, its specific form is determined by the specific short-range shape of potential. We expect the high-momentum normalization follows the middle-momentum normalization when comparing to two-body bound state ($D_2$ here is the normalization factor.  See sec \ref{sec:intro:as}).  Only in the low-momentum, it differs from the two-body wave function and that is where 
 
\subsection {Renormalization of mean field equation}
We can rewrite the mean-field equations \ref{eq:pathInt2:mf} as 
\begin{align}
D_{1\vp}&=\sum{}U_{\vp\vk}h_{1\vk}+\sum{}Y_{\vp\vk}h_{2\vk}\label{eq:pathInt2:mfopen}\\
D_{2\vp}&=\sum{}Y_{\vp\vk}h_{1\vk}+\sum{}V_{\vp\vk}h_{2\vk}\label{eq:pathInt2:mfclose}
\end{align}
The first thing to notice is that we restore the momentum dependence of interaction.  As pointed out previously, both $h_{i\vk}$ approach $1/\epsilon_{\vk}$ in high momentum and all the summation diverges at high momentum when converted to integral in 3D if we take the interaction as contact and pull them out of the summation.  However, we note that the real potential is not contact.  One way to remove the divergence is to  limit the summation/integral to a cutoff at high-momentum while keeping the interaction contact.  However an arbitrary cutoff is not desirable.  Alternatively, we can remove the divergence by restoring the momentum dependence in interaction and recognizing the decay for high-momentum in them.  The problem here is that those microscopic bare interaction is hard to pin down.      More readily available and observable is the low-energy effective interaction matrix elements.             In single channel problem, (Sec. \ref{sec:pathInt:meanfield}),  we  introduce a relation involving two-body s-wave scattering length $a_{s}$, Eq. \eqref{eq:pathInt:as}, which has the similar divergence in high-momentum.  We subtract it from the summation.  Now the difference behaves well in high-momentum,  and therefore a cutoff or bare momentum-dependent interaction is no longer necessary.  In principle, we can do the same in two channels and introduce the two-channel equivalent of s-wave scattering length, a $2\times2$ scattering amplitude matrix.  However, this $2\times2$ matrix is awkward in definition and not clear for the physical meaning.   Therefore, we adopt a two-step process with better physical intuition behind it.   

Besides the assumption of short-range potential, we makes another assumption:  the in-resonance two-body bound-state of isolated close-channel, $\phi$, is much smaller in size ($a_{c}$) than particle-particle distance ($a_{0}$), although larger than the potential range ($r_{c}$).     ($r_{c}\ll{}a_{c}\ll{}a_{0}$).  The problem would be a genuine 3-species problem and require other techniques if the close-channel bound-state is about the size  or even larger than  inter-particle distance. As discussed at the beginning of this chapter, in such assumption, the close-channel correlation follows its two-body bound-state in high momentum, and  we can write it as $h_{2\vk}\propto\phi_{0\vk}f(\vk)$ with factor $f(\vk)$ encapsulating all the many-body effects. This automatically cures the divergence involving $h_{2\vk}$ because $\phi_{0\vk}$ decays fast enough in high momentum and gives no singularity when integrated.   After $h_{2\vk}$ is taken care of, we remove the singularity of integral involving open-channel correlation $h_{1\vk}$ using the regular method by subtracting low-energy two-body interacting kernel $a_{s}$, which involves  the same high-momentum divergence as $h_{1\vk}$. 



 Let us look at the close-channel correlation $h_{2\vk}$ (Eq. \ref{eq:pathInt2:h2}) first. In the lowest order, we have $u_{\vk}^{2}\approx\frac{E_{1\,\vk}+\xi_{\vk}}{E_{1\,\vk}+E_{2\,\vk}}$, $h_{2\vk}$ can be rewritten into such form
\begin{equation}\label{eq:pathInt2:h2D2}
 h_{2\vk}\approx\frac{D_{2}}{(E_{1\,\vk}+E_{3\,\vk})}u_{\vk}^{2}
\end{equation}

%Within $r_{c}$, the  potential determine the specific shape of the wave-functions; while, at free region, $\mathcal{D}$ ($r\gg{r_{c}}$ see Sec. \ref{sec:intro:as}), system is ``free'', all the details of potential are encapsulated in a few parameters (for example, s-wave scattering length, $a_{s}$).       %    the behavior should be quite universal for these bound states except an overall normalization factor. 
  %Across the crossover, the internal structure of correlation ($h_{2\vk}$) at medium and high momentum ($k\gg{k_{F}}$) follows the two-body wave function ($\phi_{0\vk}$) does not change much, while the overall normalization does vary along tuning.   At low momentum, we can introduce a factor for  many-body effect as illustrated in Eq. \ref{eq:pathInt2:hphif}.   Eq. \ref{eq:pathInt2:h2} actually follows such form fairly well in the medium and low momentum.  In the lowest order, we have $u_{\vk}^{2}\approx\frac{E_{1\,\vk}+\xi_{\vk}}{E_{1\,\vk}+E_{2\,\vk}}$ and it serves as the many-body modifier.  At high momentum ($>1/r_{c}$), the mean field result is not really reliable as showed in the divergence of integral. $D_{2\vk}$ at high momentum is no longer a constant, but  has specific form determined by the specific form of  potential. 
 
 
 




%In three-species many-body problem, the above conclusion is still valid except at low-momentum (around or below Fermi energy) where Pauli exclusion between two channels is severe. 
We can see the $\nth{(E_{1\,\vk}+E_{3\,\vk})}\approx\nth{2\xi_{\vk}+\eta}$ is just the same as two-body wave function $\phi_{0}$ (when $k\ll\kappa$) besides normalization; while the extra factor $u_{\vk}^{2}$ is the many-body factor $f(\vk)$ which describes the Pauli exclusion between two channels.  At the low-momentum, open-channel weight is large, phase space left for close-channel is limited, this is shown mathematically as small $u_{\vk}^{2}$ factor.   At higher momentum ($k_{F}\ll{k}\ll1/r_{c}$), $u_{\vk}^{2}\approx1$ and close-channel wave-function just follow two-body counterpart with a different normalization factor.   It is not hard to see that $D_{2}$ is actually closely related to ``\emph{integrated contact intensity}'', $C$, in Tan's works (\cite{Tan2008-1,Tan2008-2}).  In his work, Tan concluded  the high-end of relative-momentum distribution asymptotically approaches  $C/k^{4}$.  Note that the high-end in his paper means momentum lower than $1/r_{c}$, but higher than any other scale ($1/a_{s}$, $1/a_{0}$).  In such scale, $u_{k}\approx1$ and especially, for close-to-threshold bound-state, $E_{b}\ll{}\frac{\hbar^{2}}{mr_{c}^{2}}$, so $C=\frac{D_{2}^{2}}{4m^{2}}$ for the close-channel.   
At even higher momentum ($>1/r_{c}$), the mean-field / saddle-point solution no longer applies.  And  two-body correlation, $h_{2\vk}$, should just follow two-body wave function (See Sec. \ref{sec:intro:as}).  Therefore, we replace $\frac{D_{2}}{(E_{1\,\vk}+E_{3\,\vk})}$ with two-body wave function $\phi_{0\vk}$ in Eq. \ref{eq:pathInt2:h2D2}, and we have 
\begin{equation}\label{eq:pathInt2:hphi}
h_{2\vk}=\alpha\phi^{(0)}_{\vk}u_{\vk}^{2}
\end{equation}
where $\phi^{(0)}$ is the normalized solution of  two-body \sch equation.
\begin{equation}\label{eq:pathInt2:phi}
-E_{b}^{(0)}\phi_{0\,\vp}=\epsilon_{\vp}\phi_{0\,\vp}-\sum_{\vk}V \phi_{0\,\vk}
\end{equation}
By replacing with the two-body wave function,we ignore the Pauli exclusion between different close-channel bound-state $\phi_{0}$, while keeping Pauli exclusion between channels as $u_{k}^{2}$.
This replacement also solves the renormalization problem automatically because the two-body wave function decays fast enough at high momentum and therefore remove the divergence in summation for $h_{2\vk}$.  


%The two-body wave function for isolated close-channel spreads in large momentum space and has small weight in low momentum (below $k_{F}$), and therefore, we expected $u_{\vk}^{2}$ has small correction only.  $D_{2}$ is the product of two factors: 1, the normalization ($A$) factor as in two-body wave function normalized to 1(Appendix \ref{sec:pathInt2:short-range}); 2, total number of particles in close-channel.  

%As discussed, $1/(2\xi_{\vk}+\eta)$ is the form of two-body close-channel wave function, $\phi_0$, at low momentum.  If we ignore the Pauli exclusion between close-channel bound states, the close-channel wave function $h_{2\vk}$ can be written as .
%\begin{equation}
%h_{2\vk}=\alpha\phi^{(0)}_{\vk}u_{\vk}^{2}+o(\frac{E_{F}}{\eta})
%\end{equation}
% where $\phi^{(0)}$ follows two-body \sch equation
%\begin{equation}\label{eq:pathInt2:phi}
%-E_{b}^{(0)}\phi_{0\,\vp}=\epsilon_{\vp}\phi_{0\,\vp}-\sum_{\vk}V \phi_{0\,\vk}
%\end{equation}


At the lowest order, $\phi_{0\vk}\sim\nth{\kappa^{2}+k^{2}}$ where $\kappa$ is the momentum scale related to binding energy or tuning, $\kappa^{2}/2m=E_{b}$.     In the many-body interesting region where momentum is not too larger than Fermi momentum, this quantity is actually very small and approximately a constant, $\nth{\kappa^{2}}$, because $\kappa{}\gg{}k_{F}$.  The part where the many-body factor $u_{\vk}^{2}$ significantly alters the wave function is actually rather small comparing to the whole spread of the wave function.  Within this range, the wave function before altering, $\phi_{0\vk}$, is small as well.   This fact applies across the full range of crossover.  At certain limit, $D_{2}$ ($\alpha$) may be large and close-channel weight is comparable or even larger than that of open-channel. However, this factor $\phi_{0\vk}$ is so small that the product of $h_{2\vk}$ is still much smaller than $1$.  
In summary, \emph{the full close-channel (including normalization) is always small in the low momentum region  even when total close-channel weight is large.}  This justifies our following perturbative treatment on the Pauli exclusion between two channels. 

%Let us look at the second equation Eq. \ref{eq:pathInt2:mfclose} first, as discussed before, $h_{2}$ is similar to $\phi^{(0)}$ in two-body wave function. More specifically

In the interesting region, biding energy $E_{b}^{(0)}$ is always close to absolute detuning $\eta$.   
Use Eq. \ref{eq:pathInt2:h2D2}, we can write 
\begin{equation*}
D_{2}\approx{}h_{2\vk}\frac{(\epsilon_{\vk}-2\mu+\eta)}{u_{\vk}^{2}}
\end{equation*}
Combine the above equation with Eq. \ref{eq:pathInt2:mfclose}, we get 
\begin{equation*}
h_{2\vk}\frac{(E_{1\vk}+E_{3\vk})}{u_{\vk}^{2}}=\sum{}Yh_{1\vk}+\sum{}Vh_{2\vk}
\end{equation*}
Multiply both sides with $\phi^{*}$ and integrate over the momentum, and using Eq. \ref{eq:pathInt2:hphi}
\begin{equation}
\alpha\br{-E_{b}+\eta-2\mu-\sum\abs{\phi_{\vk}}^{2}{(E_{\vk}-\xi_{\vk})}+\sum\abs{\phi_{\vk}}^{2}{v_{\vk}^{2}V}}=\sum{\phi_{\vk}^{*}}{Y}{h_{1\vk}}
\end{equation}
Comparing this to the two-body problem, detuning is shifted by many-body effects $\mu$ (mostly due to Pauli exclusion within open-channel) and $\lambda_{1}$ (mostly due to Pauli exclusion between channels).  
\begin{equation}\label{eq:pathInt2:alpha}
\alpha=\frac{\sum{\phi_{\vk}^{*}}{Y}{h_{1\vk}}}{\br{-E_{b}+\eta-2\mu-\lambda_{1}}}
\end{equation}
\begin{equation}\label{eq:pathInt2:lambda1}
\lambda_{1}(\eta)\equiv\sum\abs{\phi_{\vk}}^{2}{(E_{\vk}-\xi_{\vk})}-\sum\abs{\phi_{\vk}}^{2}{v_{\vk}^{2}V}
\end{equation}

$\lambda_{1}$ depends on detuning $\eta$ (through $E_{\vk}$, $v_{\vk}$), but the dependence is rather weak. We will discuss more detail/reasoning about it later in Sec \ref{sec:pathInt2:lambda}. 
Using Eq. \ref{eq:pathInt2:alpha}, we can express the $\alpha$ in $h_{2\vk}$ (Eq. \ref{eq:pathInt2:hphi}), and then plug it into Eq. \ref{eq:pathInt2:h1}
\begin{equation*}
D_{1}=\sum{}U{h_{1_{\vk}}}+\frac{\sum_{\vk\vk^{'}} Y{\phi_{\vk'}}{\phi_{\vk}}{Y}u_{\vk}^{2}{{h_{1\vk}}}}{\br{-E_{b}+\eta-2\mu-\lambda_{1}}}\end{equation*}
Comparing this with two-body problem, we can see the detuning part (denominator of the second term) is shifted by $2\mu+\lambda_{1}$ and there is an extra $u_{\vk}^{2}$ term introduced by many-body effect.  Nevertheless, none of these affect the high-momentum behavior, therefore, the equation can be normalized exactly as in two-body problem by introducing the long-wave-length s-scattering length $a_{s}$.  We rewrite the above equation
\begin{equation*}
D_{1}=\sum_{\vk}\br{U+\frac{\sum_{\vk'}Y{\phi_{\vk'}}{\phi_{\vk}}{Y}}{\br{-E_{b}+\eta-2\mu-\lambda_{1}}}}{{h_{1\vk}}}
	-\sum_{\vk}\frac{\sum_{\vk'}Y{\phi_{\vk'}}{\phi_{\vk}}{Y}}{\br{-E_{b}+\eta-2\mu-\lambda_{1}}}v_{\vk}^{2}{{h_{1_{\vk}}}}
\end{equation*}
The second term has no divergence at high momentum in 3D due to the extra $v_{k}^{2}$, actually this factor decreases quickly over ``gap'' $D_{1}$, therefore, the summation is essentially only over low-momentum.   Furthermore, considering the short-range nature, this term varies slowly over momentum.  

Multiply both side with $(1+GT)$, 
\begin{equation*}
(1+GT)D_{1}=Th_{1}-\lambda_{2}
\end{equation*}
and 
\begin{equation}
D_{1}=T\sum(h_{1}-GD_{1})-\lambda_{2}
=D_{1}\frac{4\pi{a_{s}}}{m}\sum(\frac{E_{1\,\vk}+\xi_{\vk}+\eta}{(E_{1\,\vk}+E_{2\,\vk})(E_{1\,\vk}+E_{3\,\vk})}-\nth{2\epsilon_{\vk}})
	-\lambda_{2}
\end{equation}
where 
\begin{equation}\label{eq:pathInt2:lambda2}
\lambda_{2}=(1+GT)\sum_{\vk}\frac{\sum_{\vk'}Y{\phi_{\vk'}}{\phi_{\vk}}{Y}}{\br{-E_{b}+\eta-2\mu-\lambda_{1}}}v_{\vk}^{2}{{h_{1\vk}}}
\end{equation}
collecting everything, we have the renormalized equation
\begin{equation}
1=\frac{4\pi{\tilde{a}_{s}}}{m}\sum(\frac{E_{1\,\vk}+\xi_{\vk}+\eta}{(E_{1\,\vk}+E_{2\,\vk})(E_{1\,\vk}+E_{3\,\vk})}-\nth{2\epsilon_{\vk}})
	-\frac{\lambda_{2}}{D_{1}}
\end{equation}
Note that $\tilde{a}_{s}$ corresponds to the two-body s-wave scattering length at detuning shifted by $2\mu+\lambda_{1}$.
Now we can expand the first term in the parentheses, using Eq. \ref{eq:pathInt2:xiExpand}, and keep in mind $E_{\vk}\ll{\eta}$ at low momentum where summation is about, we have\footnote{Here we used $u_{\vk}^{2}-v_{\vk}^{2}=\frac{\xi_{\vk}}{E_{\vk}}$}
\begin{equation}\label{eq:pathInt2:gapRenorm}
1=\frac{4\pi{\tilde{a}_{s}}}{m}\sum(\nth{2E_{\vk}}-\nth{2\epsilon_{\vk}}-\frac{D_{2}^{2}\xi_{\vk}}{4(\xi_{\vk}+\eta){E_{\vk}^{3}}})
	-\frac{\lambda_{2}}{D_{1}}
\end{equation}
The correction term does not have divergence in summation of high-momentum. 
In summary there are several difference of gap equation here comparing to single-channel problems:
\begin{enumerate}
\item\label{item:pathInt2:mu}The shift of $2\mu$ in detuning;
\item The extra shift of $\lambda_{1}$ in detuning;
\item The extra term $\lambda_{2}$ in Eq. \ref{eq:pathInt2:gapRenorm};
\item The extra term  in summation of Eq. \ref{eq:pathInt2:gapRenorm};
\end{enumerate}
Item \ref{item:pathInt2:mu} corresponds to the simple many-body effect  universal to both three and four species problem, and it is studied extensively by \cite{GurarieNarrow}; while the rest corrections are unique for Pauli-exclusion between channels in three-species problem.

Furthermore,  close look into $\lambda_1$ and $\lambda_2$ (Appendix \ref{sec:pathInt2:lambda}) reveals that they varies slowly with energy / momentum and is a function of density.  They describe  fundamental many-body effects, and do not have a counterpart in two-body problem.  

Both $\lambda_{1}$ and $\lambda_{2}$ involves the integral between open-channel wave function and close-channel wave function, (factor $v_{\vk}^{2}$ or $u_{\vk}^{2}$ describes mostly the open-channel wave function in the integral). The larger the overlap of two, the larger these two parameters is.  This has a very intuitive interpretation:  more overlap leads more severe inter-channel Pauli exclusion, which shows that $\lambda_{1}$ and $\lambda_{2}$ describe this effect between two  channels.  In our model, open-channel wave function is more or less spread all over the place, (even at BEC-side, the real bound-state is very loosely bound comparing to close-channel bound state), while close-channel wave function is very sensitive to binding energy, $E_{b}(\approx\eta)$.  When  close-channel  bound-state is closer to the threshold, i.e.,  bound state energy  is smaller, the close-channel bound-state is more spread out  in real space and has larger overlap with open-channel, therefore $\lambda_{1}$ and $\lambda_{2}$ are larger in such case.                           Nevertheless, $\lambda_1$ is much smaller than the  constant $\kappa$.  So the shift is not very large.  But it shifted with the change of density.  However, in the many-body system, no dramatic jump in physical quantities happens at resonance due to the crossover nature, and it is hard to observe this shift.  Nevertheless, this effect can show up at two-body type experiments.  


\subsection{Evaluation and estimation of $\lambda_{1}$ and $\lambda_{2}$\label{sec:pathInt2:lambda}}
It is obvious from the last section that most of the inter-channel Pauli exclusion is encapsulated in two parameters $\lambda_{1}$ and $\lambda_{2}$.   It is well-worthwhile to study them in more details.  $\lambda_{1}$ is defined as 
\begin{equation}\tag{\ref{eq:pathInt2:lambda1}}
\lambda_{1}(\eta)\equiv\sum\abs{\phi_{\vk}}^{2}{(E_{\vk}-\xi_{\vk})}-\sum\abs{\phi_{\vk}}^{2}{v_{\vk}^{2}V}
\end{equation}
$\phi_{\vk}$ is close to constant or varies slowly around or below Fermi momentum, $k_{F}$, because the close-channel bound state is much smaller than the inter-particle distance.  On the other hand, $E_{\vk}-\xi_{\vk}$ and $v_{\vk}^{2}$  are only non-zero in this region.  Therefore, it is a good approximation to replace $\phi_{\vk}$ with $\phi_{\vk=0}$ and take it out of the summation.  
\begin{equation*}
\lambda_{1}(\eta)=\sum\abs{\phi_{\vk}}^{2}{(E_{\vk}-\xi_{\vk})}-\sum\abs{\phi_{\vk}}^{2}{v_{\vk}^{2}V}
\end{equation*}

Use relationship $v_{k}=\frac{E_{\vk}-\xi_{\vk}}{2E_{\vk}}$, we can rewrite the above equation into 
\begin{equation*}
\lambda_{1}=\avs{\phi}{v_{\vk}^{2}(2E_{\vk}-V)}{\phi}
\end{equation*}
$v_{\vk}$ is close to zero when momentum much higher Fermi momentum, on this region, $E_{\vk}$ is much smaller comparing to potential energy $V$.  So $V\ket{\phi}\approx-\eta\ket{\phi}$, and $\phi\approx\frac{A}{\kappa^{2}}$Therefore, we can estimate this term 
\begin{equation}
\lambda_{1}=\frac{A^{2}}{\kappa^{2}}\sum{}v_{\vk}^{2}=n\frac{A^{2}}{\kappa^{2}}\sim{}D_{1}^{2}/\eta
\end{equation}







\begin{equation}\tag{\ref{eq:pathInt2:lambda2}}
\lambda_{2}=(1+GT)\frac{Y\ket{\phi}\bra{\phi}{Y}}{\br{-E_{b}+\eta-2\mu-\lambda_{1}}}v_{\vk}^{2}\ket{{h_{1}}}
\end{equation}
Here the argument is more or less the same as in $\lambda_{1}$, considering the short-range nature of $Y$, and $v_{k}^{2}$ introduces an extra $nA^{2}/\kappa^{2}\sim{}D_{1}^{2}/\eta$ factor.  

We can see, both of them depends on many-body effect through density.  Particularly, they depend on density of open-channel component linearly at the lowest order.  $\lambda_{i}(n_o)=\lambda_{i}^{(0)}n_o$.  When not too close to resonance, $n_o$ is close to total density.  And $\lambda_{i}$ can be measured at different densities,  $\lambda_{i}^{(0)}$ estimated then accordingly. 

In the replacement of $1/(E_{1\vk}+E_{3\vk})$ by $\phi_{k}$ in Eq. \ref{eq:pathInt2:hphi}, certain error is introduced by directly replacement.  We expect the error is in higher order.  They might lead a non-linear relationship between $\lambda_{i}$ and density $n$.  But we expect the non-linearity is weak and we can still include the Pauli exclusion in these two parameters $\lambda_{i}(n)$.



\subsection{Number equation}
We can derive two number equations for each channel.  
\begin{gather*}
\sum_{k}G_{22}e^{(-i\omega_n\delta_-)}=N_{open}\qquad
\sum_{k}G_{33}e^{(-i\omega_n\delta_-)}=N_{close}
\end{gather*}
Note that the Matsubara summation is formally diverged and we need to put in a small negative parts into the summation.  It is negative because $\Psi_2=\bar\psi_b$, $\Psi_3=\bar\psi_c$.  In the zero temperature Matsubara summation, we just need to considering the positive root, $E_{1\,\vk}$.  It is straightforward to find 
\begin{gather}
N_{open}=\sum_{\vk}\frac{(E_{1\,\vk}-\xi_{\vk})(E_{1\,\vk}+\xi_{\vk}+\eta)-D_2^2}{(E_{1\,\vk}+E_{2\,\vk})(E_{1\,\vk}+E_{3\,\vk})}\\
N_{close}=\sum_{\vk}\frac{(E_{1\,\vk}-\xi_{\vk})(E_{1\,\vk}+\xi_{\vk})-D_1^2}{(E_{1\,\vk}+E_{2\,\vk})(E_{1\,\vk}+E_{3\,\vk})}
=\sum_{\vk}\frac{E_{1\,\vk}^2-E_{\vk}^2}{(E_{1\,\vk}+E_{2\,\vk})(E_{1\,\vk}+E_{3\,\vk})}\label{eq:pathInt2:numClose}
\end{gather}
Let us look at the equation of close-channel first, if we expand $\xi_{i\,\vk}$, the lowest order is 
\begin{equation}
N_{close}=\sum_{\vk}\frac{\gamma_{1\,\vk}}{(E_{\vk}+\xi_{\vk}+\eta)}=\sum_{\vk}\frac{D_{2}^2u_{\vk}^{2}}{(\xi_{\vk}+\eta)(E_{\vk}+\xi_{\vk}+\eta)}
\end{equation}
This is consistent with Eq. \ref{eq:pathInt2:h2} and \ref{eq:pathInt2:h2D2} if we assume $N_{close}\approx\sum{h_{2}^{2}}$.  From appendix \ref{sec:pathApp:consistency}, we know the summand is much smaller than 1.  Nevertheless, the weight spread in a very large range of momentum ($\sim\eta$),  both of them has $D_{2}^{2}/\eta^{2}$ for summand. And the resulting sum can be in the order of total number $N$. Note the connection between $D_{2}$ and the ``Contact'' in theory of universality (\cite{Tan2008-1,Tan2008-2}).  The summand as  density of momentum is valid at momentum up to the scale $1/a_{c}$, and is mostly determined by two-body physics except an overall factor.  furthermore, we can renormalize it to include the higher-momentum contribution as the summation converges in 3D.  In the lowest order, this summation can be written as 
\begin{equation}
N_{close}\approx\sum_{\vk}\frac{D_{2}^2}{(\xi_{\vk}+\eta)(2\xi_{\vk}+\eta)}
\end{equation}
This equation has only one unknown parameter $D_{2}$.  Therefore, this equation can be used to estimate the $D_{2}$ from experiments.  

%Similar to the argument in gap equation, the summand  above is mostly useful at energy below $\eta$, where the component is small everywhere comparing to 1, however, the summation runs over high-momentum ($\sim\eta$), where $D_2$ should no longer be regarded as constant and summation can gives $N_{close}$ comparable to total number. 
% \mycomment{ This might not be correct statement.  Maybe it is just fine as it properly converges. } This summation shows that the close-channel state is small and therefore the summation runs up to high momentum, where mostly is determined by two-body physics.  


\begin{equation}
\begin{split}
N_{open}\approx&\sum_\vk\mbr{\frac{E_\vk-\xi_\vk}{2E_\vk}+\frac{\gamma_{1\vk}}{2E_\vk}
	-\frac{(E_\vk-\xi_\vk)(\gamma_{1\vk}+\gamma_{2\vk})}{4E_\vk^2}
	-\frac{(E_\vk-\xi_\vk)\gamma_{3\vk}}{2E_\vk(\xi_\vk+E_\vk+\eta)}
	-\frac{D_{2}^{2}}{2E_\vk(\xi_\vk+E_\vk+\eta)}}\\
	\approx&\sum_\vk\mbr{\frac{E_\vk-\xi_\vk}{2E_\vk}+\frac{{D_{2}^{2}(E_{\vk}+\xi_{\vk})}}{4E_\vk^{2}(\xi_{\vk}+\eta)}
	-\frac{(E_\vk-\xi_\vk)\xi_{\vk}D_{2}^{2}}{4E_\vk^3(\xi_{\vk}+\eta)}
	-\frac{D_{2}^{2}}{2E_\vk(\xi_\vk+E_\vk+\eta)}}	
\end{split}
\end{equation}
All terms are well behaved in 3D and does not need any renomalization. 
%\section{Mean field result}
 Use the same techniques as Eq. (\ref{eq:pathInt:diffTr}), we have two equations for $D_{1}$ and $D_{2}$,
 \begin{align}
\frac{\delta}{\delta{}D_{1}}:&\qquad&
(\tilde{U}^{-1})_{11}\bar{D}_{1}+(\tilde{U}^{-1})_{21}\bar{D}_{2}-\tr\mbr{{G_{0}}\cdot\cmtrx{0&1&0\\0&0&0\\0&0&0}}=0\\
\frac{\delta}{\delta{}D_{2}}:&\qquad&
(\tilde{U}^{-1})_{12}\bar{D}_{1}+(\tilde{U}^{-1})_{22}\bar{D}_{2}-\tr\mbr{{G_{0}}\cdot\cmtrx{0&0&1\\0&0&0\\0&0&0}}=0
 \end{align}


 If we take $D$ as real constant\footnote{Actually $D_{2}{_{\vk}}$ cannot be constant at high momentum.  However, for the momentum we are interested, i.e. the momentum lower or in the order of Fermi momentum, it slowly varies.  Therefore  it is reasonable to take it as constant.},     we can find the mean field result.  Usting Eq. (\ref{eq:pathInt2:Gexpand}),
 
 \begin{equation}\label{eq:pathInt2:G0}
 \begin{split}
 G_{0}=&
 \begin{pmatrix}
 {g_{1\,\vk}}&
-u_{k}v_{k}{g_{2\,\vk}}&0\\
-u_{k}v_{k}{g_{2\,\vk}}& {g_{3\,\vk}}&0\\
  0&0&\nth{i\omega_{n}-\xi_{3}{}_{\vk}}
 \end{pmatrix}\\
&+\frac{D_{2}}{\eta}
\begin{pmatrix}
\frac{D_{1}^{2}D_{2}}{4E_{\vk}^{3}}g_{2\,\vk}&\frac{D_{1}D_{2}\xi_{\vk}}{4E_{\vk}^{3}}g_{2\,\vk}&-g_{1\,\vk}+\nth{i\omega_{n}-\xi_{3}{}_{\vk}}\\
\frac{D_{1}D_{2}\xi_{\vk}}{4E_{\vk}^{3}}g_{2\,\vk}&-\frac{D_{1}^{2}D_{2}}{4E_{\vk}^{3}}g_{2\,\vk}&u_{k}v_{k}{g_{2\,\vk}}\\
-g_{1\,\vk}+\nth{i\omega_{n}-\xi_{3}{}_{\vk}}&u_{k}v_{k}{g_{2\,\vk}}&0
\end{pmatrix}
\end{split}
 \end{equation}
\begin{gather}
g_{1}{}_{\vk}=\frac{u_{\vk}^{2}}{i\omega_{n}-\xi_{1}{}_{\vk}}+\frac{v_{\vk}^{2}}{i\omega_{n}-\xi_{2}{}_{\vk}}\\
g_{2}{}_{\vk}=\nth{i\omega_{n}-\xi_{1}{}_{\vk}}-\nth{i\omega_{n}-\xi_{2}{}_{\vk}}\\
g_{3}{}_{\vk}=\frac{v_{\vk}^{2}}{i\omega_{n}-\xi_{1}{}_{\vk}}+\frac{u_{\vk}^{2}}{i\omega_{n}-\xi_{2}{}_{\vk}}
\end{gather}

 \begin{align}
\tr\mbr{G_{0}\cdot\cmtrx{0&1&0\\0&0&0\\0&0&0}}&=
\sum_{\vk}\sum_{\omega_{n}}
	\mbr{(\nth{i\omega_{n}-\xi_{1}{}_{\vk}}-\nth{i\omega_{n}-\xi_{2}{_{\vk}}})
	(-\frac{D_{1}}{2E_{\vk}}+\frac{D_{1}D_{2}^{2}\xi_\vk}{4E_{\vk}^{3}\eta})}\\
	&=\sum_{\vk}(\frac{D_{1}}{2E_{\vk}}-\frac{D_{1}D_{2}^{2}\xi_\vk}{4E_{\vk}^{3}\eta})\\
\tr\mbr{G_{0}\cdot\cmtrx{0&0&1\\0&0&0\\0&0&0}}&=
\sum_{\vk}\sum_{\omega_{n}}
\mbr{\nth{i\omega_{n}-\xi_{3}{}_{\vk}}-
\frac{u_{\vk}^{2}}{i\omega_{n}-\xi_{1}{}_{\vk}}-\frac{v_{\vk}^{2}}{i\omega_{n}-\xi_{2}{}_{\vk}}}\frac{D_{2}}{\eta}\label{eq:pathInt2:F20}\\
&=\sum_{\vk}\frac{D_{2}}{\eta}u_{\vk}^2
  \end{align}
Here we take the interest only in $T=0$, so we only need to consider the negative frequencies ($\xi_{2\,\vk}$, $\xi_{3\,\vk}$) for summation of Matsubara frequency. Note that the second summation diverges badly in high-momentum. %this is because we take $D_2$ as constant.  It decreases at high-energy in the scale of $\eta$ and needs to be regularize carefully.  
We notice that this term is controlled by parameter $D_{2}/\eta$, this actually goes back to the fact that we only keep the first order in $L$ expansion (Eq. \eqref{eq:pathInt2:L1}), which is only valid for energy smaller or in order of Fermi energy.  In a more careful study, this term should be like $D_{2}/(\epsilon_{k}+\eta)$, is approximately $D_{2}/\eta$ when the interesting region  is lower or at the Fermi energy.   We can reestablish the $F_{k}\propto1/\epsilon_{k}$ if we retain all terms in the expansion of $L$, i.e. inverting Green's function $G$ exactly.       Indeed it should be just proportional to simple bounded two-body solution of isolated close-channel, $\phi_{0\,\vk}$ at high-momentum, which is not of interest for the many-body problem. 
 Another interesting thing about this term is the $u_\vk$ factor, which is small below chemical potential $\mu$ in BCS side.  This shows the fact that the low momentum is filled mostly by open-channel and close-channel is crowded out.  However, this does not affect close-channel too much as it is much more extended in momentum space and its occupation over each level is low due to the smaller size of close-channel bound state.  With above result, we can rewrite gap equations as 
\begin{equation*}
\tilde{U}^{-1}D-\mtrx{\sum_{\vk}(\frac{D_{1}}{2E_{\vk}}-\frac{D_{1}D_{2}^{2}\xi_\vk}{4E_{\vk}^{3}\eta})\\\sum_{\vk}\frac{D_{2}}{\eta}u_{\vk}^2}=0
\end{equation*}
 We can multiply it with $\tilde{U}$ and we have 
\begin{equation}\label{eq:pathInt2:meanfield}
\mtrx{D_1\\D_2}=\mtrx{U&Y\\Y^{*}&V}\sum_{\vk}\mtrx{\frac{D_{1}}{2E_{\vk}}-\frac{D_{1}D_{2}^{2}\xi_\vk}{4E_{\vk}^{3}\eta}\\
\frac{D_{2}}{\eta}u_{\vk}^2}
\end{equation}
 This equation can be renormalized in a very similar fashion as variation method. Notice that the second term in the first component describes the Pauli exclusion between two channels.  

\subsection{Renormalizing mean-field equation\label{sec:pathIntRenorm}}
We define two quantities for the summand in the mean-field equation Eq. \ref{eq:pathInt2:meanfield}.
\begin{gather}
F_{1\,\vk}=\frac{D_{1}}{2E_{\vk}}-\frac{D_{1}D_{2}^{2}\xi_\vk}{4E_{\vk}^{3}\eta}\\
F_{2\,\vk}=\frac{D_{2}}{\eta}u_{\vk}^2\label{eq:pathInt2:F2k}
\end{gather}
Considering the argument from last section, we modify equation of $F_2$ to the following,
\begin{equation}
\tilde{F}_{2\,\vk}=\frac{D_{2}}{\eta+2\epsilon_{\vk}}u_{\vk}^2\label{eq:pathInt2:F2kMod}
\end{equation}
And now $F_2$ has the same behavior at high momentum as $F_1$, actually this is the behavior we expected for $\epsilon_k<\eta$, it falls off even faster beyond energy scale of $\eta$, which is determined by the specific shape of close-channel potential.

We can rewrite the mean-field equation as
\begin{gather}
D_{1}=\sum_{\vk}(U F_{1\,\vk}+Y \tilde{F}_{2\,\vk})\label{eq:pathInt2:D2}\\
D_{2}=\sum_{\vk}(Y^{*} F_{1\,\vk}+V \tilde{F}_{2\,\vk})\label{eq:pathInt2:D2}
\end{gather}
Here we see the $F_{1\,\vk}$ and $\tilde{F}_{2}$  both go as $1/\epsilon_{\vk}$  at high-momentum.  %To resolve divergence in the summation of $F_{2\,\vk}$ we restore the momentum dependence on $D_{2\,\vk}$ and therefore $F_{2\,\vk}$.  
  And we can see $\frac{D_{2}}{\eta}$ is actually a good approximation at low-momentum, where kinetic energy is much smaller than $\eta$.  We can rewrite $\tilde{F}_{2}=\alpha\phi_{0\,\vk}u_{\vk}^{2}$. Eq. (\ref{eq:pathInt2:D2}) can be rewritten as
\begin{equation*}
\eta{}F_{2\,\vp}=\sum_{\vk}(Y^{*} F_{1\,\vk}+V F_{2\,\vk})\label{eq:pathInt2:D2}
\end{equation*}
comparing this to a two-body \sch equation
\begin{equation}
-E_{0}\phi_{0\,\vp}=\epsilon_{\vp}\phi_{0\,\vp}-\sum_{\vk}V \phi_{0\,\vk}
\end{equation}
where $E_{0}$ is the binding energy of two-body bound state of isolated close-channel.   




\section{Fermionic excitation mode and Bogoliubov transformation}
In fact, if we limit ourselves in mean field level, we can interpret transform $T_\vk{}L_\vk$ in Eq. \eqref{eq:pathInt2:B} as Bogoliubov canonical transformation and $3\times3$ matrix $B$ in Eq. \eqref{eq:pathInt2:Bapprox} gives us spectrum of fermionic quasi-particle excitation.  At mean field, both $D_{1}$ and $D_{2}$ are taken as constants and real.  From Eqs. (\ref{eq:pathInt2:xiExpand}-\ref{eq:pathInt2:xiExpand3}), we see that the fermionic excitation modes basically follow the pattern in broad resonance.  In broad resonance where close channel only modifies the interaction in open channel,  there are basically three fermionic quasi-particle modes: two (degenerate) Bogoliubov quasi-particle modes  in open channel, $E_{\vk}=\sqrt{\xi^{2}_{\vk}+D_{1}^{2}}$ as in BCS theory; and one high fermionic excitation mode in close-channel, $\xi_{\vk}+\eta$, as in normal gas.  In narrow resonance, first of all, the gap $D_{1}$ is modified  by consideration of extra Pauli exclusion though the correction is in order of $\frac{D_2^2}{\eta}$.  Once that is taken into account, the above conclusion is approximately correct except high-order correction in $\frac{D_{2}^{2}}{\eta}$.   The originally double degenerate excitation now split by $\frac{D_{2}^{2}}{\xi_{\vk}+\eta}$, while the third high excitation has a small correction in the same order as well.  From Eq. \eqref{eq:pathInt2:actionMix} and Eq. \eqref{eq:pathInt2:B}, the action is diagonal for new fermionic (quasi-particle) field $\Phi_\vk\equiv(\phi_{\alpha\,\vk},\bar\phi_{\beta\,-\vk},\bar\phi_{\gamma\,-\vk})=L^{\dg}_{\vk}T^{\dg}_{\vk}\Psi_{\vk}$ ($\Psi_\vk=(\psi_{\alpha\,\vk},\bar\psi_{\beta\,-\vk},\bar\psi_{\gamma\,-\vk}$)).   Translate it into operator language and use $\alpha$, $\beta$, and $\gamma$ to denote the new quasi-particles.  We have the relation
\begin{equation}
\mtrx{\alpha_{\vk}\\\beta_{-\vk}^{\dg}\\\gamma_{-\vk}^{\dg}}
=L^{\dg}_{\vk}T^{\dg}_{\vk}  \mtrx{a_{\vk}\\b_{-\vk}^{\dg}\\c_{-\vk}^{\dg}}
\end{equation}   
and the new Hamiltonian at mean-field level is 
\begin{equation}
\hat{H}=f(D)+E_{1\,\vk}\alpha^{\dg}_{\vk}\alpha^{}_{\vk}
+E_{2\,\vk}\beta^{\dg}_{-\vk}\beta^{}_{-\vk}+E_{3\,\vk}\gamma^{\dg}_{-\vk}\gamma^{}_{-\vk}
\end{equation}
\begin{align}\tag{\ref{eq:pathInt2:xiExpand}}
E_{1\vk}&\approx{}E_{\vk}+\frac{D_{2}^{2}u_{\vk}^{2}}{\xi_{\vk}+\eta}&\equiv{}&E_{\vk}+\gamma_{1\vk}\\
E_{2\vk}&\approx{}E_{\vk}-\frac{D_{2}^{2}v_{\vk}^{2}}{\xi_{\vk}+\eta}&\equiv{}&E_{\vk}+\gamma_{2\vk}
\label{eq:pathInt2:xiExpand2}\\
E_{3\vk}&\approx{}\xi_{\vk}+\eta-\frac{D_{2}^{2}}{2(\xi_{\vk}+\eta)}&\equiv{}&\xi_{\vk}+\eta+\gamma_{3\vk}
\label{eq:pathInt2:xiExpand3}
\end{align}
This clearly shows that $\alpha_{\vk}$, $\beta_{\vk}$, and $\gamma_{\vk}$ is fermionic quasi-particle excitation modes with spectrum $E_{i\,\vk}$.  And $L^{\dg}_{\vk}T^{\dg}_{\vk}$ is  Bogoliubov canonical transformation for them.  Here we see in the excitation, different species of opposite momentum, $(a,b)$ and $(a,c)$, mixed together to form the elementary excitation due to the paring in the ground state.  




\section{Collective excitation mode}
%\subsection{Simple approach for collective mode}
With the above arrangement, it is easy to study the collective mode of the system, which corresponding the second order expansion over $D$. We introduce the deviation over $\nG=G_{0}^{-1}+K_{q}$, where $G_{0}$ is described by real constant $D_{1,2}$
\begin{equation}
K_\vq=\mtrx{0&\theta_{1\,\vq}&\theta_{2\,\vq}\\\theta_{1\,-\vq}^*&0&0\\\theta_{2\,-\vq}^*&0&0}
\end{equation}
Follow the same approach in single-channel (Sec. \ref{sec:collective1}), we need to calculate $\tr(\hat{G_{0}}\hat{K}\hat{G_{0}}\hat{K})$, 
\begin{equation}
\tr({G_{0\,k}}{K_{q}}{G}_{0\,k+q}{K_{-q}})=\Theta_{q}^{\dg}M_{q}\Theta_{q}
\end{equation}
where 
\begin{equation}
\Theta_{q}=\mtrx{\theta^{*}_{1\,\vq}\\\theta^{}_{1\,-\vq}\\\theta^{*}_{2\,\vq}\\\theta^{}_{2\,-\vq}}
\qquad
\Theta_{q}^{\dg}=\mtrx{\theta^{}_{1\,\vq}&\theta^{*}_{1\,-\vq}&\theta^{}_{2\,\vq}&\theta^{*}_{2\,-\vq}}
\end{equation}
At the lowest order, if we take $L_{\vq}=I$,
\begin{equation}
M^{(0)}_{q}=\sum_{\vk}
\begin{pmatrix}
-u_{\vk}^{2}u_{\vk+\vq}^{2}f_{1\,q}+v_{\vk}^{2}v_{\vk+\vq}^{2}f_{2\,q}&u_{\vk}v_{\vk}u_{\vk+\vq}v_{\vk+\vq}(f_{2\,q}-f_{1\,q})&0&0\\
u_{\vk}v_{\vk}u_{\vk+\vq}v_{\vk+\vq}(f_{2\,q}-f_{1\,q})&u_{\vk}^{2}u_{\vk+\vq}^{2}f_{2\,q}-v_{\vk}^{2}v_{\vk+\vq}^{2}f_{1\,q}&0&0\\
0&0&-u_{\vk}^{2}f_{3\,q}&0\\
0&0&0&u_{\vk+\vq}^{2}f_{4\,q}
\end{pmatrix}
\end{equation}
where
\begin{gather}
f_{1\,q}=\nth{i q_{l}+\xi_{1\,\vk}-\xi_{2\,\vk+\vq}}\\
f_{2\,q}=\nth{i q_{l}-\xi_{1\,\vk+\vq}+\xi_{2\,\vk}}\\
f_{3\,q}=\nth{i q_{l}+\xi_{1\,\vk}-\xi_{3\,\vk+\vq}}\\
f_{4\,q}=\nth{i q_{l}-\xi_{1\,\vk+\vq}+\xi_{3\,\vk}}
\end{gather}
The two channels are completely decoupled at this level and the upper quarter is exactly same as single channel case if we ignore the correction over quasi-particle spectrum.   We can also expand each $f_{i\,q}$ to get the first order about $D_{2}/\eta$. (We take $q=0$ for it. See Appendix \ref{sec:expansionM})
\begin{equation}
M^{(1a)}=\sum_{\vk}
\begin{pmatrix}
(1-\frac{D_{1}^{2}}{2E_{\vk}^{2}})\frac{D_{1}^{2}-D_{2}^{2}}{4E_{\vk}^{2}\eta}
&-\frac{D_{1}^{2}}{4E_{\vk}^{2}}\frac{D_{1}^{2}-D_{2}^{2}}{4E_{\vk}^{2}\eta}&0&0\\
-\frac{D_{1}^{2}}{4E_{\vk}^{2}}\frac{D_{1}^{2}-D_{2}^{2}}{4E_{\vk}^{2}\eta}
&(1-\frac{D_{1}^{2}}{2E_{\vk}^{2}})\frac{D_{1}^{2}-D_{2}^{2}}{4E_{\vk}^{2}\eta}&0&0\\
0&0&\nth{2}(1+\frac{\xi_{\vk}}{E_{\vk}})\frac{3D_{1}^{2}+D_{2}^{2}}{2\eta(E_{\vk}+\xi_{\vk}+\eta)}&0\\
0&0&0&\nth{2}(1+\frac{\xi_{\vk}}{E_{\vk}})\frac{3D_{1}^{2}+D_{2}^{2}}{2\eta(E_{\vk}+\xi_{\vk}+\eta)}
\end{pmatrix}
\end{equation}

To be consistent, we also need to expand $L_{q}$ to the first order of  $D_{2}/\eta$ as well.  
\begin{equation}
\begin{split}
M^{(1b)}=&\qquad\frac{D_{2}}{\eta}\sum_{\vk}\\
&\begin{pmatrix}
-\frac{D_{1}^{2}D_{2}\xi_{\vk}}{4E^{5}_{\vk}}&-\frac{D_{1}^{2}D_{2}\xi_{\vk}}{2E^{5}_{\vk}}
&\frac{D_{1}\xi_{\vk}}{4E_{\vk}^{3}}&\frac{D_{1}}{2E_{\vk}}(\nth{E_{\vk}+\xi_{\vk}+\eta}+\nth{2E_{\vk}})\\
-\frac{D_{1}^{2}D_{2}\xi_{\vk}}{4E^{5}_{\vk}}&-\frac{D_{1}^{2}D_{2}\xi_{\vk}}{2E^{5}_{\vk}}
&\frac{D_{1}}{2E_{\vk}}(\nth{E_{\vk}+\xi_{\vk}+\eta}+\nth{2E_{\vk}})&\frac{D_{1}\xi_{\vk}}{4E_{\vk}^{3}}\\
\frac{D_{1}\xi_{\vk}}{4E_{\vk}^{3}}&\frac{D_{1}}{2E_{\vk}}(\nth{E_{\vk}+\xi_{\vk}+\eta}+\nth{2E_{\vk}})
&-\frac{D_{1}^{2}D_{2}}{4E_{\vk}^{3}(E_{\vk}+\xi_{\vk}+\eta)}&0\\
\frac{D_{1}}{2E_{\vk}}(\nth{E_{\vk}+\xi_{\vk}+\eta}+\nth{2E_{\vk}})&\frac{D_{1}\xi_{\vk}}{4E_{\vk}^{3}}
&0&-\frac{D_{1}^{2}D_{2}}{4E_{\vk}^{3}(E_{\vk}+\xi_{\vk}+\eta)}\\
\end{pmatrix}
\end{split}
\end{equation}
The interesting thing here is that if we only takes the first order of $D_{2}/\eta$, the two-channel is still decoupled when we write down the secular equation.  And the correction of the elements at the upper-left corner ($\theta_{1\,\pm{q}}$) is the same and that will gives an small finite value for $\omega_{q}$ at $q=0$. \emph{This conclusion is however problemetic as it violates the f sum rule.  Phase flucturation should be a Goldstone mode without mass.(see Sec. \ref{sec:phaseFluctuation}) }

%\subsubsection{Alternative Approach: Phase fluctuation mode \label{sec:phaseFluctuation}}
 Action of $D$, $S(\bar{D_i},D_i)$ Eq. \eqref{eq:pathInt2:nG} and Eq. \eqref{eq:pathInt2:actionD}, is invariant if the phases of $D_{1\,\vk}$ and $D_{2\,\vk}$ rotate simultaneously. We therefore conclude that there exists a massless (Goldstone) mode.  And particularly, this mode is related to the local phase invariance. 
\begin{equation*}
D_{i}(x)\rightarrow{}D_{i}(x)e^{i2\theta(x)}\qquad{}
\bar{D}_{i}(x)\rightarrow{}\bar{D}_{i}(x)e^{-i2\theta(x)}
\end{equation*}
This corresponding to  phase  rotation of fermionic variable $\psi$
\begin{equation*}
\psi_{i}(x)\rightarrow{}\psi_{i}(x)e^{i\theta(x)}\qquad{}
\bar{\psi}_{i}(x)\rightarrow{}\bar{\psi}_{i}(x)e^{-i\theta(x)}
\end{equation*}
Note that the phase $\theta(x)$ is common for the different components. The rest three degree of freedom of $D_i$, magnitude variation of two $D_i$ and internal phase between two $D_i$, change the action accordingly, while the overall phase $\theta(x)$ leaves action invariant.  We here isolate this particular degree of freedom while leave other three at their mean-field values. With a phase shift, we can rewrite the action (taken mean-field value $D^{(0)}$) (here we closely follow Nagaosa\cite{Nagaosa})
\begin{subequations}
\begin{align}\label{eq:pathInt2:actionPhase}
S[\theta,\bar\psi_{i},\psi_{i}]=&S_0[\bar\psi_{i},\psi_{i}]+S_1[\theta,\bar\psi_{i},\psi_{i}]+S_2[\theta,\bar\psi_{i},\psi_{i}]\\
S_0[\bar\psi_{i},\psi_{i}]=&\int{dx}
\Big\{\sum_{j}\bar\psi_{j}(\partial_\tau-\nth{2m}\nabla^{2}-\mu+\eta_{j})\psi_{j}\nonumber\\
&\quad+[D^{(0)}{}^{\dg}\tilde{U}^{-1}D^{(0)}-(\bar\psi\bar\psi)D^{(0)}-{D^{(0)}}{}^{\dg}(\psi\psi)\Big\}\\
S_1[\theta,\bar\psi_{i},\psi_{i}]=&\int{dx}\sum_{j}\Big\{
   i\,\bar\psi_{j}(\partial_{\tau}\theta)\psi_{j}+\nabla\theta\cdot\nth{2mi}[\bar\psi_{j}\nabla\psi_{j}-(\nabla\bar{\psi}_{j}\psi_{j})]\Big\}\\
S_2[\theta,\bar\psi_{i},\psi_{i}]=&\int{dx}\sum_{j}\nth{2m}(\nabla\theta)^{2}\bar\psi_{j}\psi_{j}
\end{align}
\end{subequations}
Note that here $D^{(0)}$ is now a constant 2-component vector, no longer functional variables.  Here we see that $S_{0}$ has the same form as before except it only takes the mean field value of $D$ and it is described by the same correlation $G_{0}$ (Eq. \ref{eq:pathInt2:nG}).  We can regard $S_{1}$ and $S_{2}$ as perturbation for the so-called gradient expansion on $\nabla\theta$.  It is then obvious that $S_{1}$ is in the first order while $S_{2}$ is in the second order.  Use the same spinor representation as before, $S$ is bilinear of $\psi$ and therefore we can formally integrate out $\psi$. 
\begin{equation}
S[\theta]=const.+\ln\det\nG(\theta)
\end{equation}
 We write out the formal Green function according to above action (with respect to Nambu-like spinor)
\begin{subequations}
\begin{align}
\nG=&G_{0}^{-1}+K_{1}+K_{2},\\
K_{1\, k,k'}=
	&\nth{({\beta{}V)}^{1/2}}(\omega_n-\omega_{n'})\theta(k-k')\sigma_3+
		\nth{{(\beta{}V)}^{1/2}}i\frac{(\vk-\vk')\cdot(\vk+\vk')}{2m}\theta(k-k')\hat{1}\\
K_{2\, k,k'}=
	&\nth{2m}\sum_{q,q'}\nth{{\beta{}V}}(\vq\cdot\vq')\theta(q)\theta(q')\delta(q+q'+k-k')\sigma_3
\end{align}
\end{subequations}
where $G_{0}$ is the same as (Eq. \eqref{eq:pathInt2:nGDelta}, \eqref{eq:pathInt2:nG}).  And like in the single channel, 
\begin{equation}
\sigma_3=\mtrx{1&0&0\\0&-1&0\\0&0&-1}
\end{equation}
and $\hat{1}$ is $3\times3$ identity matrix.  As the expansion in Eq. \ref{eq:pathInt:expand}, we can look for the expansion of $\nG$ over $K_{1,2}$.  
\begin{equation}\tag{\ref{eq:pathInt:expand}}
\tr\ln \hat{G}^{-1}=\tr\ln\hat{G_{0}}^{-1}+\tr(\hat{G_{0}}\hat{K})-\nth{2}\tr(\hat{G_{0}}\hat{K}\hat{G_{0}}\hat{K})+\cdots
\end{equation}
For the first order, $\tr(\hat{G_{0}}\hat{K})$, 
\begin{align}
\tr(\hat{G_{0}}\hat{K_1})=&\sum_{k}{G_{0\,k}K_{1\,k,k}}=0\\
\tr(\hat{G_{0}}\hat{K_2})=&\sum_{k}{G_{0\,k}K_{2\,k,k}}\nonumber\\
	=&-\nth{2m}\nth{{\beta{}V}}\sum_{k}\tr(\hat{G}_{0\,k}\sigma_3)\sum_{q}q^2\theta{q}\theta{-q}\nonumber\\
	=&-\frac{n}{2m}\sum_{q}q^2\theta{(q)}\;\theta{(-q)}
\end{align}
Here we use the fact $\nth{{\beta{}V}}\sum_{k}\tr(\hat{G}_{0\,k}\sigma_3)=n$. $\tr(\hat{G_{0}}\hat{K_2})$ is already in the second order of $\theta$, and we only need to keep the expansion of $K_2$ to this order. On the other hand, we have to go to the second order of $K_1$ for the second order of $\theta$. 
\begin{align}		
\tr(\hat{G_{0}}{K_1}\hat{G_{0}}{K_1})=&\sum_{k,q}\tr(\hat{G}_{0,k+q}K_{1\,k+q,k}\hat{G}_{0\,k}K_{1\,k,k+q})\\
=&\nth{{\beta{}V}}\sum_{k,q}\theta(q)\theta(-q)\Big[(-\omega_m^2)\tr(\hat{G}_{0,k+q}\sigma_3\hat{G}_{0\,k}\sigma_3)\nonumber\\
&\quad+\nth{m^2}\sum_{i,j}q_iq_j(k_i+\frac{q_i}{2})(k_j+\frac{q_j}{2})\tr(\hat{G}_{0,k+q}\hat{G}_{0\,k})\Big]\nonumber\\
\equiv&\sum_{q}\theta(q)\theta(-q)\big[-\pi^{(0)}(q)\omega_m^2+\sum_{ij}\pi^{(\perp)}_{ij}(q)q_iq_j\big]
\end{align}
\emph{Here we only interested in the low-energy behavior of the mode and therefore we only use $\pi^{(0)}(0)$ and $\pi^{(\perp)}(0)$.} Use the previous result of Sec. \ref{sec:diagonalGreen}, we can calculate the Green's function in the lowest and first order of $D_i/\eta$ in $\pi$.  After some long but straigh-forward algebra (see Sec. \ref{sec:calculatePi}), we find
\begin{gather}
\pi^{(0)}(0)\approx\sum_{\vk}\frac{D_{1}^{2}}{E_{\vk}^{3}}-\sum_{\vk}\frac{D_{1}^{2}D_{2}^{2}\xi_{\vk}}{2E_{\vk}^{5}(\xi_{\vk}+\eta)}
\label{eq:pathInt2:pi0}\\
\pi^{(\perp)}(0)=0
\end{gather}
Combine all these together
\begin{equation}
S[\theta]=\int{dx}\sum_{q}\theta(q)\theta(-q)\big[\nth{2}\pi^{(0)}(0)\omega_m^2-\frac{n}{2m}q^2\big]
\end{equation}
And this determines the velocity of the Anderson-Bogoliubov collective mode.  We see that its behavior is qualitatively same as  single-channel with some correction in the order of $D_{i}/\eta$. We can see that  the second term in Eq. \eqref{eq:pathInt2:pi0} is
the only correction in the next order.


\begin{subappendices}
% !TeX root =thesis.tex


\section{Diagonalize Matrix Eq. (\ref{eq:pathInt2:G2})\label{sec:diagonalize}}
We need to find a unitary transformation $L$ to diagonalize matrix 
\begin{equation*}
i\omega_{n}I+\mtrx{-E_k&0&u _kD_2\\0&+E_k&v_kD_2\\u_kD_2&v_kD_2&+\xi_k+\eta}
\end{equation*}
We drops all the $k$ subscripts in this section.  We notice that the first term is proportional to identity matrix and does not change by unitary transformation, we only need to concentrate for the second term.  We rescale all elements with $E$, 
\begin{equation*}
R=
\begin{pmatrix}
-1&0&y_1\\
0&1&y_2\\
y_1&y_2&t
\end{pmatrix}
\end{equation*}
The secular equation is 
\begin{equation}\label{eq:pahtApp:secular}
(x^{2}-1)(x-t)-(y_{1}^{2}+y_{2}^{2})x+y_{1}^{2}-y_{2}^{2}=0
\end{equation}
We will assume at the zeroth order, the three eigenvalues are $-1$, $1$ and $t$.  ($t$ has weak dependency on energy as $(\xi_{k}+\eta)/E_{k}$, however, at the low energy region of interest, we ignore $\xi_{k}$.) Both $y_{1,2}$ and t are larger than 1, however, we will verify that given condition $y_{i}^{2}\ll{t}$, the correction is indeed small and the expansion is legit.(See Sec.\ref{sec:pathApp:consistency})  \emph{Indeed, this approximation is not as bad as it seems to be, close-channel component can still be smaller than open-channel at low-k (in order of $k_{F}$)  due to close-channel bound state is much smaller than inter-particle distance even when total close-channel is more than open-channel.  And here all the quantities are about low-k unless specifically noticed.} 
We expand the system to the first order of $y_{i}^{2}/{t}$, and find
\begin{equation}
\begin{array}{ccc}
x^{(0)}&\quad{}x^{(1)}&\quad{}Eigenvector\nonumber\\
-1&-\frac{y_{1}^{2}}{t}&\mtrx{1&\frac{y_{1}y_{2}}{2t}&-\frac{y_{1}}{t}}\\
1&-\frac{y_{2}^{2}}{t}&\mtrx{-\frac{y_{1}y_{2}}{2t}&1&-\frac{y_{2}}{t}}\\
t&\frac{y_{1}^{2}+y_{2}^{2}}{2t}&\mtrx{\frac{y_{1}}{t}&\frac{y_{2}}{t}&1}
\end{array}
\end{equation}
Now it is easy to write down the corresponding diagonal matrix and unitary transformation
\begin{equation}
B=i\omega_{n}I+E\mtrx{-1-\frac{y_{1}^{2}}{t}&0&0\\0&1-\frac{y_{2}^{2}}{t}&0\\0&0&t+\frac{y_{1}^{2}+y_{2}^{2}}{2t}}
\end{equation}
\begin{equation}
L=\mtrx{1&-\frac{y_{1}y_{2}}{2t}&\frac{y_{1}}{t}\\\frac{y_{1}y_{2}}{2t}&1&\frac{y_{2}}{t}\\-\frac{y_{1}}{t}&-\frac{y_{2}}{t}&1}
\end{equation}
Here $L$ is not exactly unitary transformation, it only unitary in the first order of  $y_{i}^{2}/{t}$ (or $D_{i}^{2}/(E\eta)$). We have 
\[
B=L^{\dg}RL+o(\frac{y_{i}^{2}}{t})
\]
Alternatively, we can write $L$ as 
\begin{equation}
L=I+
\mtrx{0&-\frac{D_{1}D_{2}}{4E^{2}}&u\\
\frac{D_{1}D_{2}}{4E^{2}}&0&v\\
-u&v&0
}\frac{D_{2}}{\eta}
\end{equation}
Here we use $uv=D_{1}/2E$


\section{Derive mean-field equation \ref{eq:pathInt2:mf}\label{sec:pathInt2:deriveMF}}
For a $3\times3$ matrix as in Eq. (\ref{eq:pathInt2:nG}), 
\begin{equation}\tag{\ref{eq:pathInt2:nG}}
\mathcal{G}^{-1}=\begin{pmatrix}
i\omega_{n}-\xi_{k}&D_{1}&D_{2}\\
\bar{D}_{1}&i\omega_{n}+\xi_{k}&0\\
\bar{D}_{2}&0&i\omega_{n}+\xi_{k}+\eta
\end{pmatrix}
\end{equation}
A general matrix inverted as such, 
  \begin{equation}
  \mtrx{A_{11}&A_{12}&A_{13}\\A_{12}^{*}&A_{22}&0\\A_{13}^{*}&0&A_{33}}^{-1}=
  \nth{|A|}
  \mtrx{A_{22}A_{33}&-A_{12}A_{33}&-A_{13}A_{22}\\
  	-A_{12}^{*}A_{33}&A_{11}A_{33}-A_{13}A_{13}^{*}&A_{12}^{*}A_{13}\\
	-A_{13}^{*}A_{22}&A_{12}A_{13}^{*}&A_{11}A_{22}-A_{12}A_{12}^{*}}
  \end{equation}
where $|A|$ is the determined of $A$ matrix.  Here we work in momentum space, in which the system is nicely decoupled at least to the mean-field order.  And we therefore drop all the $k$ subscript in the rest of section.  For Eq. (\ref{eq:pathInt2:nG}),
\begin{equation}
|A|=(i\omega_{n}-\epsilon_{1})(i\omega_{n}-\epsilon_{2})(i\omega_{n}-\epsilon_{3})
\end{equation}
From Eq. \ref{eq:pathInt2:mf01}, 
\begin{equation}
\begin{split}
\tr\mbr{{G_{0}}\cdot\cmtrx{0&1&0\\0&0&0\\0&0&0}}&=\sum_{\vk\omega_{n}}G_{0\,(21)}
=\sum_{\vk}\sum_{\omega_{n}}\frac{-D_{1}^{*}(i\omega_{n}+\xi+\eta)}{(i\omega_{n}-\epsilon_{1})(i\omega_{n}-\epsilon_{2})(i\omega_{n}-\epsilon_{3})}\\
&=\sum_{\vk}D_{1}^{*}\frac{\xi_{1}+\xi+\eta}{(\xi_{1}+\xi_{2})(\xi_{1}+\xi_{3})}\equiv\sum_{\vk}h_{1\,\vk}
\end{split}
\end{equation}
Similarly
\begin{equation}
\begin{split}
\tr\mbr{{G_{0}}\cdot\cmtrx{0&0&1\\0&0&0\\0&0&0}}&=\sum_{\vk\omega_{n}}G_{0\,(31)}
=\sum_{\vk}D_{2}^{2}\frac{\xi_{1}+\xi}{(\xi_{1}+\xi_{2})(\xi_{1}+\xi_{3})}\equiv\sum_{\vk}h_{2\,\vk}
\end{split}
\end{equation}
And we have 
 \begin{align*}
(\tilde{U}^{-1})_{11}\bar{D}_{1}+(\tilde{U}^{-1})_{21}\bar{D}_{2}-\sum_{\vk}h_{1\,\vk}=0\\
(\tilde{U}^{-1})_{12}\bar{D}_{1}+(\tilde{U}^{-1})_{22}\bar{D}_{2}-\sum_{\vk}h_{2\,\vk}=0
 \end{align*}
Invert the interaction matrix $\tilde{U}$ and we have Eq.  \ref{eq:pathInt2:mf}.



\section{Wave function for short-range potential}\label{sec:pathInt2:short-range}
Here we discuss some possible generalization on the wave function for short-range potential.  This topic has been studied by Shizhong Zhang \cite{shizhongUniv}. We will use some similar ideas.  When we have a short-range potential with range $r_{c}$, outside this range, \sch equation is very simple
\begin{equation}
-\frac{\hbar^{2}}{2m}\nabla^{2}\psi+U\psi=E\psi
\end{equation}
And the equation has a simple solution for s-wave $\psi=A{e^{-\kappa{r}}}/{r}$ ($\kappa$ is imaginary for scattering state), here normalization $A$ is determined  by connecting it with the short-range part of the wave function, $\varphi_0$. 

Let us discuss the bound-state first, $\kappa>0$.  In the momentum space, there is also a universal behavior at low-momentum, where $kr_{c}\ll1$.   
\begin{equation*}
\psi_{k}=\nth{(2\pi)^{3/2}}\int{d\vr}(\varphi_{0}+A\frac{e^{-\kappa{r}}}{r})e^{-i\vk\cdot\vr}
\end{equation*}
The first part for $\varphi_{0}$ has very little $k$ dependence and the second terms give
\begin{equation*}
\psi_{k}=\varphi_{0\,k}+\nth{(2\pi)^{3/2}}\int{d\vr}(A\frac{e^{-\kappa{r}}}{r})e^{-i\vk\cdot\vr}=\varphi_{0\,k}-(\frac{2}{\pi})^{1/2}A\nth{k^{2}+\kappa^{2}}
\end{equation*}

Furthermore, if the bound-state is the one close to threshold, the most weight is outside $r_{c}$, we can neglect the first term and we have universal behavior at low-momentum while the normalization determined in two-body problem.   When considering many-body physics, the really low momentum, (at order of Fermi energy or below), is modified, but in the medium momentum, (still much smaller than $1/r_{c}$), this universal behavior is preserved.  This is actually the high-momentum behavior ($C/k^{4}$) described in Tan's work about universality\cite{Tan2008-1,Tan2008-2}. 


\section{$\lambda_{1}$ and $\lambda_{2}$\label{sec:pathInt2:lambda}}
\begin{equation}\tag{\ref{eq:pathInt2:lambda1}}
\lambda_{1}\equiv\avs{\phi}{(E_{\vk}-\xi_{\vk})}{\phi}-\avs{\phi}{v_{\vk}^{2}V}{\phi}
\end{equation}
Use relationship $v_{k}=\frac{E_{\vk}-\xi_{\vk}}{2E_{\vk}}$, we can rewrite the above equation into 
\begin{equation*}
\lambda_{1}=\avs{\phi}{v_{\vk}^{2}(2E_{\vk}-V)}{\phi}
\end{equation*}
$v_{\vk}$ is close to zero when momentum much higher Fermi momentum, on this region, $E_{\vk}$ is much smaller comparing to potential energy $V$.  So $V\ket{\phi}\approx-\eta\ket{\phi}$, and $\phi\approx\frac{A}{\kappa^{2}}$Therefore, we can estimate this term 
\begin{equation}
\lambda_{1}=\frac{A^{2}}{\kappa^{2}}\sum{}v_{\vk}^{2}=n\frac{A^{2}}{\kappa^{2}}\sim{}D_{1}^{2}/\eta
\end{equation}







\begin{equation}\tag{\ref{eq:pathInt2:lambda2}}
\lambda_{2}=(1+GT)\frac{Y\ket{\phi}\bra{\phi}{Y}}{\br{-E_{b}+\eta-2\mu-\lambda_{1}}}v_{\vk}^{2}\ket{{h_{1}}}
\end{equation}
Here the argument is more or less the same as in $\lambda_{1}$, considering the short-range nature of $Y$, and $v_{k}^{2}$ introduces an extra $nA^{2}/\kappa^{2}\sim{}D_{1}^{2}/\eta$ factor.  

We can see, both of them depends on many-body effect through density.  Particularly, they depend on density of open-channel component linearly at the lowest order.  $\lambda_{i}(n_o)=\lambda_{i}^{(0)}n_o$.  When not too close to resonance, $n_o$ is close to total density.  And $\lambda_{i}$ can be measured at different densities,  $\lambda_{i}^{(0)}$ estimated then accordingly. 




\section{Calculate $\pi^{(0)}$ and $\pi^{\perp}$\label{sec:calculatePi}}
Here we will calculate $\pi^{(0)}$ and $\pi^{\perp}$ to the first order of $D_i/\eta$ using the expansion of Green's function in Sec. \ref{sec:diagonalGreen}.

\begin{equation}\label{eq:pathInt2:pi0}
\begin{split}
\pi^{(0)}(0)=&\sum_k\tr(\hat{G}_{0\,k}\sigma_3\hat{G}_{0\,k}\sigma_3)\\
	\approx&\sum_k\tr\big(T_{\vk}B_{k}^{-1}T_{\vk}^{\dg}\sigma_3T_{\vk}B_{k}^{-1}T_{\vk}^{\dg}\sigma_3\big)\\
	&\quad+\tr\Big(T_{\vk}\delta_{\vk}B_{k}^{-1}T_{\vk}^{\dg}\sigma_3T_{\vk}B_{k}^{-1}T_{\vk}^{\dg}\sigma_3
	-T_{\vk}B_{k}^{-1}\delta_{\vk}T_{\vk}^{\dg}\sigma_3T_{\vk}B_{k}^{-1}T_{\vk}^{\dg}\sigma_3\\
	&\qquad+T_{\vk}B_{k}^{-1}T_{\vk}^{\dg}\sigma_3T_{\vk}\delta_{\vk}B_{k}^{-1}T_{\vk}^{\dg}\sigma_3
	-T_{\vk}B_{k}^{-1}T_{\vk}^{\dg}\sigma_3T_{\vk}B_{k}^{-1}\delta_{\vk}T_{\vk}^{\dg}\sigma_3\Big)\\
	=&\sum_k\tr\big(M_{k}\big)+2\tr\Big(\delta_{\vk}M_{k}-\delta_{\vk}M_{k}\Big)
\end{split}
\end{equation}
where 
\begin{equation}
M_{k}=T_{\vk}^{\dg}\sigma_3T_{\vk}B_{k}^{-1}T_{\vk}^{\dg}\sigma_3T_{\vk}B_{k}^{-1}
\end{equation}
Here we use the cyclical  property of the trace $\tr(AB)=\tr(BA)$.  
It is straightforward to calculate
\begin{equation*}
T_{\vk}^{\dg}\sigma_3T_{\vk}B_{k}^{-1}=
\begin{pmatrix}
{\frac{\xi_{\vk}}{E_{\vk}(i\omega_{k}-\xi_{1\,\vk})}}&\frac{D_{1}}{E_{\vk}(i\omega_{k}+\xi_{2\,\vk})}&0\\
{\frac{D_{1}}{E_{\vk}(i\omega_{k}-\xi_{1\,\vk})}}&-\frac{\xi_{\vk}}{E_{\vk}(i\omega_{k}+\xi_{2\,\vk})}&0\\
0&0&-\nth{i\omega_{k}+\xi_{3\,\vk}}\\
\end{pmatrix}
\end{equation*}
Now it is easy to calculate the first term
\begin{equation}
\begin{split}
\sum_k\tr\big(M_{k}\big)=&
\sum_{k}\mbr{
\frac{2D_{1}^{2}}{E_{\vk}^{2}(i\omega_{k}-\xi_{1\,\vk})(i\omega_{k}+\xi_{2\,\vk})}+
\br{\frac{\xi_{\vk}^{2}}{E_{\vk}^{2}(i\omega_{k}-\xi_{1\,\vk})^{2}}+\frac{\xi_{\vk}^{2}}{E_{\vk}^{2}(i\omega_{k}+\xi_{2\,\vk})^{2}}
-\nth{(i\omega_{k}+\xi_{3\,\vk})^{2}}}
}
\end{split}
\end{equation}
Only root $-\xi_{2\,\vk}$ in the first term contributes in Matrubara frequency summation.
\begin{equation}\label{eq:pathInt2:pi0-1}
\sum_k\tr\big(M_{k}\big)=\sum_{\vk}\frac{2D_{1}^{2}}{E_{\vk}^{2}(\xi_{1\,\vk}+\xi_{2\,\vk})}
\approx\sum_{\vk}\frac{D_{1}^{2}}{E_{\vk}^{3}}-\sum_{\vk}\frac{D_{1}^{2}D_{2}^{2}\xi_{\vk}}{2E_{\vk}^{5}(\xi_{\vk}+\eta)}
\end{equation}

For the lowest order of the second term in Eq. \eqref{eq:pathInt2:pi0}, we only need to take the lowest order of $B_{k}$
\begin{equation}
B_{k}=\mtrx{i\omega_{k}-E_{\vk}&0&0\\0&i\omega_{k}+E_{\vk}&0\\0&0&i\omega_{k}+\xi_{\vk}+\eta}
\end{equation}
It is easy to verify at this approximation
\begin{equation}\label{eq:pathInt2:pi0-2}
\tr\Big(\delta_{\vk}M_{k}-\delta_{\vk}M_{k}\Big)=0
\end{equation}
Combine Eq. \eqref{eq:pathInt2:pi0-1} and Eq. \eqref{eq:pathInt2:pi0-2}, we have 
\begin{equation}
\pi^{(0)}(0)\approx\sum_{\vk}\frac{D_{1}^{2}}{E_{\vk}^{3}}-\sum_{\vk}\frac{D_{1}^{2}D_{2}^{2}\xi_{\vk}}{2E_{\vk}^{5}(\xi_{\vk}+\eta)}\end{equation}

and it is actually exact for $\pi^{\perp}(0)$
\begin{equation}
\begin{split}
\pi^{\perp}(0)=&\sum_k\tr(\hat{G}_{0\,k}\hat{G}_{0\,k})\\
	=&\sum_k\tr\big(T_{\vk}L_{\vk}B_{k}^{-1}L_{\vk}^{\dg}T_{\vk}^{\dg}T_{\vk}L_{\vk}B_{k}^{-1}L_{\vk}^{\dg}T_{\vk}^{\dg}\big)\\
	=&\sum_k\tr\big(B_{k}^{-1}B_{k}^{-1}\big)\\
	=&\sum_{\vk,i}(\sum_{\omega_{k}}(i\omega_{k}-\xi_{i})^{-2})\\
	=&0
\end{split}
\end{equation}


\section{Check for consistency of expansion\label{sec:pathApp:consistency}}
In our treatment here, one crucial assumption in expansion is the smallness of $D_{2}/\eta$.  Here we check it.  We have 
\begin{equation}
D_{2}=\sum{}Yh_{1\vk}+\sum{}Vh_{2\vk}\tag{\ref{eq:pathInt2:mfclose}}
\end{equation}
The first term on the right is relatively small comparing to the second term.  In an estimation we just keep the second term.  Furthermore,  we assume $h_{2\,\vk}=\sqrt{N_{c}}\phi_{0\,\vk}$, where $\phi_{0\,\vk}$ is the normalized wave function of isolated close-channel potential satisfying \sch equation
\begin{equation}\tag{\ref{eq:pathInt2:phi}}
-E_{b}^{(0)}\phi_{0\,\vp}=\epsilon_{\vp}\phi_{0\,\vp}-\sum_{\vk}V \phi_{0\,\vk}
\end{equation}
rearrange it we have (especially at low-momentum)
\begin{equation*}
\sum_{\vk}V \phi_{0\,\vk}=(\epsilon_{\vp}+E_{b})\phi_{0\,\vp}\approx{\eta}\phi_{0\,\vp}
\end{equation*}
The second approximation is correct at low-momentum (smaller or in the same order of Fermi momentum) as $\epsilon_{\vp}\ll{}E_{b}\approx\eta$.  Put all these together, we have
\begin{equation*}
D_{2}\approx\alpha{}E_{b}\phi
\end{equation*}
If we assume a simple exponentially decayed wave function
\begin{equation*}
\phi_{\vk}=\sqrt{\frac{8\pi\kappa}{V_{0}}}\frac{1}{k^{2}+\kappa^{2}}\approx\sqrt{\frac{8\pi\kappa}{V_{0}}}\frac{1}{\kappa^{2}}
\end{equation*}
Here  $V_{0}$ is the total volume.  The second approximation is for low-momentum as previous equation.  Collect all these together, we have
\begin{equation}
D_{2}\approx\sqrt{N_{c}}\eta\sqrt{\frac{8\pi\kappa}{V_{0}}}\frac{1}{\kappa^{2}}
\sim\eta\sqrt\frac{n_{c}}{\kappa^{3}}
\sim\eta\br{\frac{k_{Fc}}{\kappa}}^{\frac{3}{2}}
\sim\eta\br{\frac{E_{Fc}}{\eta}}^{\frac{3}{4}}
\end{equation}
%:
$k_{Fc}$ is the Fermi momentum corresponding to the particles in close-channel, which is much smaller than the characteristic momentum for bound-state, $\kappa$.   Therefore we have $D_{2}\ll\eta$, even when $n_{c}$ is close to total density $n$. 

Now we check the correction in Fermionic spectrum (Eq. \ref{eq:pathInt2:xiExpand3}-\ref{eq:pathInt2:xiExpand3}), 
is indeed small comparing to the main term.  
\begin{align}\tag{\ref{eq:pathInt2:xiExpand}}
\xi_{1\vk}&\approx{}E_{\vk}+\frac{D_{2}^{2}u_{\vk}^{2}}{\xi_{\vk}+\eta}&\equiv{}&E_{\vk}+\gamma_{1\vk}\\
\xi_{2\vk}&\approx{}E_{\vk}-\frac{D_{2}^{2}v_{\vk}^{2}}{\xi_{\vk}+\eta}&\equiv{}&E_{\vk}+\gamma_{2\vk}
\tag{\ref{eq:pathInt2:xiExpand2}}\\
\xi_{3\vk}&\approx{}\xi_{\vk}+\eta-\frac{D_{2}^{2}}{2(\xi_{\vk}+\eta)}&\equiv{}&\xi_{\vk}+\eta+\gamma_{3\vk}
\tag{\ref{eq:pathInt2:xiExpand3}}
\end{align}
Here, we mostly only concern of low-momentum ($k\sim{}k_{F}$).  In Eq. \ref{eq:pathInt2:xiExpand3}, 
\begin{equation*}
\frac{\gamma_{3\vk}}{\xi_{3\vk}}\sim{}\frac{D_{2}^{2}}{\eta^{2}}\sim\br{\frac{k_{Fc}}{\kappa}}^{2}\ll1
\end{equation*}
Eqs. \ref{eq:pathInt2:xiExpand} and Eqs. \ref{eq:pathInt2:xiExpand2} is slightly more complicated.  Both of them involve $\frac{D_{2}^{2}}{E_{\vk}\eta}$,  at very BCS side, close-channel density is small, $k_{F\,c}$ is small and that makes this ratio small; when close to (narrow) resonance, where $n_{c}$ is comparable to to total density, at low energy, $D_{1}$ is in the order of Fermi energy, so does $E_{\vk}$.   We have (we no longer distinguish $k_{F\,c}$ with $k_{F}$)
 \begin{equation*}
 \frac{\gamma_{i}}{\xi_{i}}<\frac{D_{2}^{2}}{E_{\vk}\eta}\sim\frac{\eta^{2}\frac{k_{Fc}^{3}}{\kappa^{3}}}{k_{F}^{2}\eta}\sim\frac{k_{F}}{\kappa}\ll1
\end{equation*}

More deeply, in the secular equation that leads to spectrum, Eq. \ref{eq:pahtApp:secular}.  We convert it back to normal scale without $E_{\vk}$,  (ignore subscript $\vk$ in the following equations)
\begin{equation*}
(x^{2}-E^{2})(x-\xi-\eta)-D_{2}^{2}x+D_{2}^{2}E(u^{2}-v^{2})=0
\end{equation*}
It is not hard to use definition of $u$ and $v$ to find $u^{2}-v^{2}=\xi/E$, and express $E^{2}=\xi^{2}+D_{1}^{2}$, therefore we have
\begin{equation*}
(x-\xi)(x+\xi)(x-\xi-\eta)-D_{1}^{2}(x-\xi-\eta)-D_{2}^{2}(x-\xi)=0
\end{equation*}
Here the first term is for free particles, and let us estimate the size of the last two terms.  For low-momentum solution, we simply use $D_{1}\sim{}E_{F}$, we find
\begin{equation*}
\frac{D_{1}^{2}(x-\xi-\eta)}{D_{2}^{2}(x-\xi)}\sim\frac{E_{F}^{2}\eta}{D_{2}^{2}E_{F}}\sim\frac{\kappa}{k_{F}}\gg1
\end{equation*}
This justifies our choice to neglect the last term when finding the lowest-order solution and use the last term for correction.  

 



 In another word, the above estimation is saying that the total occupation number of close-channel at low-momentum is much smaller than 1 in all region of resonance (narrow or broad) because the close-channel bound-state is much smaller than inter-particle distance.  This factor gives us a small factor, $\frac{r_{c}}{a_{0}}\sim\frac{k_{F}}{\kappa}\sim\sqrt\frac{E_{F}}{\eta}$, upon which we can do the expansion.  

\end{subappendices}






