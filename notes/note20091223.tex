
 \subsection{More on Renormalization}
 The single pole approximation fit well in the real space, while T matrix is generally expressed in k-space. 
The renormalization T-matrix is just at one particular point of the 

\subsection{Tony's comment}

He asked me to write down the renomalization with explicitly the renormalization equations and the semi-Cooper problem. He think the low close-channel weight might be the limit for semi-Cooper problem.  

\subsubsection{Richardson equation paper}
He thinks that the community generally agrees that the BCS hamiltonian is an approximation with lots of possible correction.  Before the correction of the BCS wave function to the hamiltonian becomes important (as pointed by Richardson approach), other correction terms of hamiltonian probably becomes more important. That is perhaps the reason people do not take great attention to Richardson's paper.  He feels that the community, at least  those who really think about it, generally agree that BCS wave function and the hamiltonian is some approximation but with the hope that it captures the essence of the problem, particularly the correct two-body density matrix. 

He does not fully agree that Richardson approach is a good model to describe the crossover as the diluteness or dense is not exact the same thing referred in crossover.     

He is very interested in the discrepancy between the Richardson solution and the BCS ansatz and wish to see the continuing exploration along the line.   He also mentions that BCS ansatz is in the sense the mean field level solution for the BCS hamiltonian.  He is wondering whether the higher order expansion/fluctuation around it, (maybe 1/N or some other techniques), might gives something related to  the Richardson solution correction.  

For the experiments of varying pair numbers, he mentioned that in the cuprate, some people proposed that changing of doping might lead to change the the density of the state and change the $T_c$, which might related to the change of pairs.  That might be related. 
