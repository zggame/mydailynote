If we are looking for the many-body effects on the close-channel.  One possible way is that mediated by the open-channel.  In that situation, it seems not much difference between the large close-channel bound-state and the small close-channel bound-state.  The key part in such situation is the coupling between two channel which happens in the small potential range $a_0$ anyway. And in that sense, the open-channel's part that within $a_0$ also can be modified by the many-body part at least due to the normalization.  

If the many-body effects of the close-channel is restricted to the coupling of the open-close channel within $a_0$, the many-body effect probably only affects the normalization without modify the wave function, at least for the short-range. But what we really modified is the long-range behavior. 

In our assumption, the wave-functions  in both channels are fixed within $a_0$.   And there are two possibilities, fixing the ratio between channels as well or letting that change. It seems reasonable to fix the ratio with the same argument as the wave-function is fixed as the two-body solution within $a_0$.  

However, in that case, the only many-body contribution is come from the open-channel large $r$.   If the many-body effects show up only in the range $r\sim{}1/{k_F}$, probably only the open-channel feels it.  And this is probably just the broad resonance.  (?????) The close-channel feels no many-body effects at all.  The only thing is the normalization.  On the other hand, if the many-body effects show on $r>a_0$, both the portion of close-channel bound state outside $a_0$ and the open-channel part is affected.  This seems corresponding to the case where each of wave function can varies independently.  

\emph{What does broad/narrow resonance mean in the real-space?}
In the model of two-body, the close-channel part is fixed and only changes in normalization.  Broad/narrow are still meaningful here.  
