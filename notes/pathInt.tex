\subsection{Path integral approach for single channel}
Here we reviewed the path integral approach for the original single-channel BEC-BCS crossover problem. Randeria and the company \cite{RanderiaBEC, Randeria1997, Randeria2008} has studied this problem with path integral and it is proved to be a rather nice tool for the problem.  

We started with a attractive $\delta$-potential in real space.  This is not equivalent to normal reduced pairing potential as in original BCS work.  However, reduced paring potential only couples only particles of the opposite momentum and does not support a collective mode as the $\delta$-potential. 
\begin{equation}
\hat{H}-\mu\hat{N}=\sum_{\sigma}\int{d^{d}r}c^{\dagger}_{\sigma}\br{\partial_{\tau}-\nth{2m}\nabla^{2}-\mu}c^{}_{\sigma}-g\int{d^{d}r}c^{\dagger}_{\uparrow}c^{\dagger}_{\downarrow}c^{}_{\downarrow}c^{}_{\uparrow}
\end{equation}
where $\xi_{\vk}=\epsilon_{\vk}-\mu$, $\epsilon_{\vk}=\vk^{2}/2m$.  We can write down the action for the quantum partition function $\mathcal{Z}=\int{D(\bar\psi,\psi)\exp\br{-S[\bar\psi,\psi]}}$
\begin{equation}
S[\bar\psi,\psi]=\int^{\beta}_{0}d\tau\int{d^{d}r}\mbr{\sum_{\sigma}\bar\psi_{\sigma}\br{\partial_{\tau}-\nth{2m}\nabla^{2}-\mu}\psi_{\sigma}-g\bar\psi_{\uparrow}\bar\psi_{\downarrow}\psi^{}_{\downarrow}\psi^{}_{\uparrow}}
\end{equation}
We try to solve this system by introduce Hubbard-Stratonovich transformation.   Introduce a bosonic field $\Delta(\vr,\tau)$ coupled with Cooper channel $\psi(\vr,\tau)\psi(\vr,\tau)$. %Here we follow the normal notation from path integral, $r$ is four tempo-space coordinator.  
We write down first the Gaussian integral of $\Delta$
\begin{equation}
1=\int{D(\bar\Delta,\Delta)}\exp\br{-\nth{g}\int{d\tau{d}^{d}r}\bar\Delta\Delta}
\end{equation}
Note that we absorb the extra constant of integration into the measure of $D(\bar\Delta,\Delta)$.
And with a shift of $\Delta(\vr,\tau)\rightarrow\Delta(\vr,\tau)-g\psi(\vr,\tau)\psi(\vr,\tau)$, we have 
\footnote{$\int{D(\bar\Delta,\Delta)}1$ is only a constant factor on partition function $\mathcal{Z}$ and has no effect on real physical quantity, therefore, we can take it as 1, (equivalently divide the $\mathcal{Z}$ by a constant)}
\begin{equation}
\exp\br{g\int{d\tau{}d^{d}r}\psi_{\uparrow}\bar\psi_{\downarrow}\psi_{\downarrow}\psi_{\uparrow}}=
\int{D(\bar\Delta,\Delta)}\exp\bbr{-\int{d\tau{d^{d}r}}\mbr{\nth{g}\abs{\Delta}^{2}-\br{\bar\Delta\psi_{\downarrow}\psi_{\uparrow}+\Delta\bar\psi_{\uparrow}\bar\psi_{\downarrow}}}}
\end{equation}
Now the interaction term can be replaced.
\begin{equation*}
\mathcal{Z}=\int{D(\bar\psi,\psi)\int{D(\bar\Delta,\Delta)}\exp\bbr{-\int{d\tau{d^{d}r}}\mbr{\sum_{\sigma}\bar\psi_{\sigma}\br{\partial_{\tau}-\nth{2m}\nabla^{2}-\mu}\psi_{\sigma}+\nth{g}\abs{\Delta}^{2}-\br{\bar\Delta\psi_{\downarrow}\psi_{\uparrow}+\Delta\bar\psi_{\uparrow}\bar\psi_{\downarrow}}}}}
\end{equation*}
This form is bilinear to $\psi$, and we can rewrite it into a nicer form in Nambu spinor representation
\begin{equation}
\bar\Psi=\begin{pmatrix}\bar{\psi}_{\uparrow}&\psi_{\downarrow}\end{pmatrix}\text{,  }\qquad
\Psi=\begin{pmatrix}{\psi}_{\uparrow}\\\bar\psi_{\downarrow}\end{pmatrix}
\end{equation}
\begin{equation}
\mathcal{Z}=\int{D(\bar\psi,\psi)}\int{D(\bar\Delta,\Delta)}\exp
	\bbr{-\int{d\tau{d^{d}r}}\mbr{\nth{g}\abs{\Delta}^{2}-\bar\Psi \nG\Psi}}
\end{equation}
where 
\begin{equation}
\nG=\begin{pmatrix}
[\hat{G}_{0}^{(p)}]^{-1}&\Delta\\\bar\Delta&[\hat{G}_{0}^{(h)}]^{-1}
\end{pmatrix}
\end{equation}
is known as Gor'kov Green function, and $[\hat{G}_{0}^{(p)}]^{-1}=-\partial_{\tau}+\nth{2m}\nabla^{2}+\mu$, and $[\hat{G}_{0}^{(h)}]^{-1}=-\partial_{\tau}-\nth{2m}\nabla^{2}-\mu$ represent the non-interacting Green functions of the particle and hole respectively. Now $\Psi$ can be integrated out formally and partition function then only depends on bosonic field $\Delta$.
\begin{equation}
\mathcal{Z}=\int{D(\bar\Delta,\Delta)}\exp
	\bbr{-\int{d\tau{d^{d}r}}\mbr{\nth{g}\abs{\Delta}^{2}-\ln\det\nG}}
\end{equation}
Note that $\ln\det\nG$ goes through both the normal space and $2\times2$ Nambu spinor space.  