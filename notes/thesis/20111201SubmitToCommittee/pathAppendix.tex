% !TeX root =thesis.tex


\section{Diagonalization of  the matrix Eq. (\ref{eq:pathInt2:G2})\label{sec:diagonalize}}
We need to find a unitary transformation $L$ to diagonalize the  matrix 
\begin{equation}\tag{\ref{eq:pathInt2:G2}}
T_k^{\dg}G_{\omega_{n},\vk}^{-1}T_k=i\omega_nI+\mtrx{-E_k&0&u_k\Delta_2\\0&+E_k&v_k\Delta_2\\u_k\Delta_2&v_k\Delta_2&+\xi_k+\eta}
\end{equation}
We drops all the $k$ subscripts in this section because matrices in this section are decoupled in momentum and we only deal with one particular momentum $\vk$ a time. We notice that the first term is proportional to an identity matrix and does not change by unitary transformation, we only need to concentrate on the second term.  We rescale all elements with $E_{\vk}$ for simplicity in the following of this section. 
\begin{equation*}
y=\frac{\Delta_2}{E_{\vk}},\qquad
 t=\frac{\xi_k+\eta}{E_{\vk}},\qquad
\end{equation*}
 
\begin{equation*}
R=
\begin{pmatrix}
-1&0&uy\\
0&1&vy\\
uy&vy&t
\end{pmatrix}
\end{equation*}
The secular equation of $R$ is ($\abs{x\,I-R}=0$)
\begin{equation}\label{eq:pahtApp:secular}
(x^{2}-1)(x-t)-y^{2}x+(u^{2}-v^{2})y^{2}=0
\end{equation}
We use $u^{2}+v^{2}=1$ here.  We  assume at the zeroth order, the three eigenvalues are $-1$, $1$ and $t$.  ($t$ has weak dependency on energy as $(\xi_{k}+\eta)/E_{k}$, however, at the low energy region of interest, we ignore $\xi_{k}$.) Both $y$ and t are larger than 1, however, we will verify that given condition $y^{2}\ll{t}$, the correction is indeed small and the expansion is reasonable (See Sec.\ref{sec:pathApp:consistency}).  \emph{Indeed,  close-channel component can still be smaller than the open-channel component at low-k (in the order of $k_{F}$)  due to the close-channel bound state is much smaller than the interparticle distance even when the total close-channel atom number  is more than that of open-channel.  And here all the quantities are about low-k unless specifically noticed.} 
We expand the system to the first order of the dimensionless parameter $\tilde\zeta=y^{2}/{t}$ (\eef{eq:pathInt2:zetaDef})%\footnote{Note that now $\zeta$ has no momentum dependency.}
, and find
\begin{equation}
\begin{array}{ccc}
x^{(0)}&\quad{}x^{(1)}&\quad{}Eigenvector\nonumber\\
-1&-u^{2}\tilde\zeta&\mtrx{1&\frac{uvy^{2}}{2t}&-\frac{uy}{t}}\\
1&-v^{2}\tilde\zeta&\mtrx{-\frac{uvy^{2}}{2t}&1&-\frac{vy}{t}}\\
t&\nth2\tilde\zeta&\mtrx{\frac{uy}{t}&\frac{vy}{t}&1}
\end{array}
\end{equation}
Now it is easy to write down the corresponding diagonal matrix and the unitary transformation
\begin{equation}
B=i\omega_{n}I+E\mtrx{-1-u^{2}\tilde\zeta&0&0\\0&1-v^{2}\tilde\zeta&0\\0&0&t\nth2\tilde\zeta}
\end{equation}
\begin{equation}
L=\mtrx{1&-\frac{uvy^{2}}{2t}&\frac{uy}{t}\\\frac{uvy^{2}}{2t}&1&\frac{vy}{t}\\-\frac{uy}{t}&-\frac{vy}{t}&1}
\end{equation}
Here $L$ is not exactly unitary transformation, it is only unitary in the first order of  $\tilde\zeta$. We have 
\[
B=i\omega_{n}I+E\,L^{\dg}RL+o(\tilde\zeta)
\]
Alternatively, we can write $L$ as 
\begin{equation}
L=I+
\mtrx{0&-\frac{\Delta_{1}\Delta_{2}}{4E^{2}}&u\\
\frac{\Delta_{1}\Delta_{2}}{4E^{2}}&0&v\\
-u&v&0
}\frac{\Delta_{2}}{\eta}
\end{equation}
Here we use $uv=\Delta_{1}/2E$.

In the above treatment, the small parameter $\tilde\zeta$ is momentum dependent.  If we restore the subscript $\vk$ and scale it  back with $E_{\vk}$
\begin{equation}
\tilde\zeta=\frac{\Delta_{2}^{2}}{E_{\vk}(\xi_{\vk}+\eta)}
\end{equation}
A momentum-dependent small parameter is not very convenient to work with, so we take its maximum value in low momentum ($\lesssim{}E_{F}$).  In the BCS-like states ($\mu>0$), $\min{E_{k}}=\Delta_{1}$, $\min{\xi_{\vk}}=0$; in the BEC-like states ($\mu<0$), $\min{E_{k}}=\sqrt{\Delta_{1}^{2}+\mu^{2}}$ and $\min{\xi_{\vk}}=\abs{\mu}$. We take the smaller values and have our expanding small parameter $\zeta$(\eef{eq:pathInt2:zetaDef})
\begin{equation}
\zeta=\max\tilde{\zeta}=\frac{\Delta_{2}^{2}}{\Delta_{1}\eta}
\end{equation}


\section{Derivation of  the mean-field equations \eqref{eq:pathInt2:mf}\label{sec:pathInt2:deriveMF}}
We have fermion correlation as a $3\times3$ matrix (Eq. \ref{eq:pathInt2:nG}), 
\begin{equation}\tag{\ref{eq:pathInt2:nG}}
\mathcal{G}^{-1}=\begin{pmatrix}
i\omega_{n}-\xi_{k}&\Delta_{1}&\Delta_{2}\\
\bar{\Delta}_{1}&i\omega_{n}+\xi_{k}&0\\
\bar{\Delta}_{2}&0&i\omega_{n}+\xi_{k}+\eta
\end{pmatrix}
\end{equation}
Here we work in the momentum space, in which the system is nicely decoupled at least to the mean-field order.  And we therefore drop all the $k$ subscript in the rest of section. A general $3\times3$ matrix inverted as such, 
  \begin{equation}
  \mtrx{A_{11}&A_{12}&A_{13}\\A_{12}^{*}&A_{22}&0\\A_{13}^{*}&0&A_{33}}^{-1}=
  \nth{|A|}
  \mtrx{A_{22}A_{33}&-A_{12}A_{33}&-A_{13}A_{22}\\
  	-A_{12}^{*}A_{33}&A_{11}A_{33}-A_{13}A_{13}^{*}&A_{12}^{*}A_{13}\\
	-A_{13}^{*}A_{22}&A_{12}A_{13}^{*}&A_{11}A_{22}-A_{12}A_{12}^{*}}
  \end{equation}
where $|A|$ is the determent of $A$.   At the mean field level, all $\Delta_{i}$'s are real constants.  We denote the mean field value of $\mathcal{G}$ as $G_{0}$.  The determent of ${G}_{0}^{-1}$ can be expressed as 
\begin{equation}
|G_{0}^{-1}|=(i\omega_{n}-E_{1})(i\omega_{n}+E_{2})(i\omega_{n}+E_{3})
\end{equation}
where $E_{i}$'s are defined in Eqs. (\ref{eq:pathInt2:xiExpand}-\ref{eq:pathInt2:xiExpand3}).
And $G_{0}$ can be obtained according to the above rule. Now we can find the last term in \ref{eq:pathInt2:mf01}, 
\begin{equation}
\begin{split}
\tr\mbr{{G_{0}}\cdot\cmtrx{0&1&0\\0&0&0\\0&0&0}}&=\sum_{\vk\omega_{n}}G_{0\,(21)}\\
&=\sum_{\vk}\sum_{\omega_{n}}\frac{-\Delta_{1}^{*}(i\omega_{n}+\xi+\eta)}{(i\omega_{n}-E_{1})(i\omega_{n}+E_{2})(i\omega_{n}+E_{3})}\\
&=\sum_{\vk}\Delta_{1}^{*}\frac{E_{1}+\xi+\eta}{(E_{1}+E_{2})(E_{1}+E_{3})}\equiv\sum_{\vk}h_{1\,\vk}
\end{split}
\end{equation}
Here we perform the zero-temperature Matsubara summation  in the third equal sign with the normal trick (see sec. 4.2.1 in \cite{Altland}, sec. 25 in \cite{Fetter}, also refer to Footnote \ref{foot:intro:sum} at Page. \pageref{foot:intro:sum}). Because of zero-temperature,  within three roots, $E_{1}$, $-E_{2}$ and $-E_{3}$, we only need to take into account two negative roots $-E_{2}$ and $-E_{3}$, assuming the correction is small. 
Similarly
\begin{equation}
\begin{split}
\tr\mbr{{G_{0}}\cdot\cmtrx{0&0&1\\0&0&0\\0&0&0}}&=\sum_{\vk\omega_{n}}G_{0\,(31)}
=\sum_{\vk}\Delta_{2}\frac{E_{1}+\xi}{(E_{1}+E_{2})(E_{1}+E_{3})}\equiv\sum_{\vk}h_{2\,\vk}
\end{split}
\end{equation}
And we have 
 \begin{align*}
(\tilde{U}^{-1})_{11}\bar{\Delta}_{1}+(\tilde{U}^{-1})_{21}\bar{\Delta}_{2}-\sum_{\vk}h_{1\,\vk}=0\\
(\tilde{U}^{-1})_{12}\bar{\Delta}_{1}+(\tilde{U}^{-1})_{22}\bar{\Delta}_{2}-\sum_{\vk}h_{2\,\vk}=0
 \end{align*}
Invert the interaction matrix $\tilde{U}$ and we have Eq.  \ref{eq:pathInt2:mf}.


\section{The wave function for a short-range potential}\label{sec:pathInt2:short-range}
Here we discuss some possible generalization on the wave function for a short-range potential.  This topic has been studied by Zhang \cite{shizhongUniv}. We will use some similar ideas.  Outside the range $r_{c}$ of a short-range potential,  an atom is free and  the  \sch equation is very simple.
\begin{equation}
-\frac{\hbar^{2}}{2m}\nabla^{2}\psi=E\psi
\end{equation}
The equation has a simple solution for s-wave, $\psi=A'{e^{-\kappa{r}}}/{r}$ ($\kappa$ is imaginary for a scattering state). For a bound state. normalization $A'$ is determined  by connecting it with the short-range part of the wave function, $\varphi_0$, and then requiring the full wave function normalized to $1$. 

Let us discuss the bound-state first, where $\kappa>0$.  In the momentum space, there is also a universal behavior at low momentum, where $kr_{c}\ll1$.   
\begin{equation*}
\psi_{k}=\nth{(2\pi)^{3/2}}\int{d\vr}(\varphi_{0}+A'\frac{e^{-\kappa{r}}}{r})e^{-i\vk\cdot\vr}
\end{equation*}
The first part for $\varphi_{0}$ corresponds to the short-range part of the wave-function. 
\begin{equation*}
\psi_{k}=\varphi_{0\,k}+\nth{(2\pi)^{3/2}}\int{d\vr}(A'\frac{e^{-\kappa{r}}}{r})e^{-i\vk\cdot\vr}=\varphi_{0\,k}-A\nth{k^{2}+\kappa^{2}}
\end{equation*}
The first term has  very little $k$ dependence for low memontum $k\ll1/r_{c}$ and the second terms is more important in this range. 
Furthermore, if the bound-state is  close to threshold, the most weight is outside $r_{c}$, we can neglect the first term and we have universal behavior at low-momentum while the normalization factor $A$ can be easily determined.
\begin{equation}
A=\sqrt{\frac{8\pi\kappa}{\mathcal{V}_{0}}}
\end{equation}
Where $\mathcal{V}_{0}$ is the total volume of the system. And the wave-function is
\begin{equation}\label{eq:pathInt2:phi2body}
\psi_{\vk}\approx\sqrt{\frac{8\pi\kappa}{\mathcal{V}_{0}}}\frac{1}{k^{2}+\kappa^{2}}\approx\sqrt{\frac{8\pi\kappa}{\mathcal{V}_{0}}}\frac{1}{\kappa^{2}}
\end{equation}
The second approximation is when the momentum is low ($\lesssim{}k_{F}$).

   Besides  the bound-state, if the interaction is weak and short-range, the low energy scattering state is well described by the s-wave scattering state $\psi\propto1/r-1/a$ (Eq. \ref{eq:intro:Bethe}), and its Fourier transformation in the momentum space has the similar form $1/k^{2}$.  When considering many-body physics, in the low momentum below or around the Fermi momentum,  wave function  is modified by the many-body effect; but in the medium momentum, (still much smaller than $1/r_{c}$), this $1/k^2$ universal behavior is preserved.  The distribution of particle in such momentum, $k_{F}\ll{k}\ll{1/r_{c}}$, is $1/k^{4}$. This is actually the ``high-momentum'' (medium here) behavior ($C/k^{4}$) described in Tan's work about universality\cite{Tan2008-1,Tan2008-2}. 

On the other hand, at very higher momentum ($k\gg1/r_{c}$), the second term in the above is very small.  This is because the smooth tail part of the wave function ($\phi_0$) cancels out and contributes little in high frequency momentum oscillation.  The high-frequency Fourier component in momentum space is solely determined by the wave function within the potential range ($r_c$).   This can be extend beyond the two-body wave function to the two-body correlation as long as the long-wave-length part is smooth.  In all cases, two-body, or many-body, very high-frequency of two-body correlation follows the two-body wave function.  

Incidentally, the two-body \sch equation in the momentum space reads as 
\begin{equation}
\frac{\hbar^{2}k^{2}}{2m}\psi_{\vk}+\sum_{\vk'}U_{\vk\vk'}\psi_{\vk'}=E\psi_{\vk}
\end{equation}
At very high momentum (determined by the interaction strength and  potential range), the first term dominates, and we have the asymptotic form of the wave-function similar as Eq. \ref{eq:pathInt2:phi2body},
\begin{equation}
\lim_{k\rightarrow\infty}\psi_{\vk}=\tilde{A}\nth{k^{2}+\kappa^{2}}
\end{equation}
where $-\frac{\hbar^{2}\kappa^{2}}{2m}=E$.  Note that this behavior is for a different reason and $\tilde{A}$ is not necessarily equal $A$ at low momentum discussed before.  



%\section{Evaluating $\lambda_{1}$ and $\lambda_{2}$}
%\begin{equation}\tag{\ref{eq:pathInt2:lambda1}}
%\lambda_{1}\equiv\avs{\phi}{(E_{\vk}-\xi_{\vk})}{\phi}-\avs{\phi}{v_{\vk}^{2}V}{\phi}
%\end{equation}
%Use relationship $v_{k}=\frac{E_{\vk}-\xi_{\vk}}{2E_{\vk}}$, we can rewrite the above equation into 
%\begin{equation*}
%\lambda_{1}=\avs{\phi}{v_{\vk}^{2}(2E_{\vk}-V)}{\phi}
%\end{equation*}
%$v_{\vk}$ is close to zero when momentum much higher Fermi momentum, on this region, $E_{\vk}$ is much smaller comparing to potential energy $V$.  So $V\ket{\phi}\approx-\eta\ket{\phi}$, and $\phi\approx\frac{A}{\kappa^{2}}$Therefore, we can estimate this term 
%\begin{equation}
%\lambda_{1}=\frac{A^{2}}{\kappa^{2}}\sum{}v_{\vk}^{2}=n\frac{A^{2}}{\kappa^{2}}\sim{}D_{1}^{2}/\eta
%\end{equation}
%
%
%
%
%
%
%
%\begin{equation}\tag{\ref{eq:pathInt2:lambda2}}
%\lambda_{2}=(1+GT)\frac{Y\ket{\phi}\bra{\phi}{Y}}{\br{-E_{b}+\eta-2\mu-\lambda_{1}}}v_{\vk}^{2}\ket{{h_{1}}}
%\end{equation}
%Here the argument is more or less the same as in $\lambda_{1}$, considering the short-range nature of $Y$, and $v_{k}^{2}$ introduces an extra $nA^{2}/\kappa^{2}\sim{}D_{1}^{2}/\eta$ factor.  
%
%We can see, both of them depends on many-body effect through density.  Particularly, they depend on density of open-channel component linearly at the lowest order.  $\lambda_{i}(n_o)=\lambda_{i}^{(0)}n_o$.  When not too close to resonance, $n_o$ is close to total density.  And $\lambda_{i}$ can be measured at different densities,  $\lambda_{i}^{(0)}$ estimated then accordingly. 
%
%In the replacement of $1/(E_{1\vk}+E_{3\vk})$ by $\phi_{k}$ in Eq. \ref{eq:pathInt2:hphi}, certain error is introduced by directly replacement.  We expect the error is in higher order.  They might lead a non-linear relationship between $\lambda_{i}$ and density $n$.  But we expect the non-linearity is weak and we can still include the Pauli exclusion in these two parameters $\lambda_{i}(n)$.




\section{Evaluation of $\pi^{(0)}(0)$ and $\pi^{\perp}(0)$\label{sec:calculatePi}}
Here we  calculate $\pi^{(0)}$ and $\pi^{\perp}$ (Eqs. \ref{eq:pathInt2:GKGK2}, \ref{eq:pathInt2:pi}) to the first order of $\zeta$ (Eq. \ref{eq:pathInt2:zetaDef}) using the expansion of the Green's function (Eqs. \ref{eq:pathInt2:Gexpand}) described in Sec. \ref{sec:diagonalGreen}.
\begin{equation}\label{eq:pathInt2:pi0long}
\begin{split}
\pi^{(0)}(0)=&\sum_k\tr(\hat{G}_{0\,k}\sigma_3\hat{G}_{0\,k}\sigma_3)\\
	\approx&\sum_k\tr\big(T_{\vk}B_{k}^{-1}T_{\vk}^{\dg}\sigma_3T_{\vk}B_{k}^{-1}T_{\vk}^{\dg}\sigma_3\big)\\
	&\quad+\tr\Big(T_{\vk}\delta_{\vk}B_{k}^{-1}T_{\vk}^{\dg}\sigma_3T_{\vk}B_{k}^{-1}T_{\vk}^{\dg}\sigma_3
	-T_{\vk}B_{k}^{-1}\delta_{\vk}T_{\vk}^{\dg}\sigma_3T_{\vk}B_{k}^{-1}T_{\vk}^{\dg}\sigma_3\\
	&\qquad+T_{\vk}B_{k}^{-1}T_{\vk}^{\dg}\sigma_3T_{\vk}\delta_{\vk}B_{k}^{-1}T_{\vk}^{\dg}\sigma_3
	-T_{\vk}B_{k}^{-1}T_{\vk}^{\dg}\sigma_3T_{\vk}B_{k}^{-1}\delta_{\vk}T_{\vk}^{\dg}\sigma_3\Big)
	\end{split}
\end{equation}
Note that $k$ stands for both the momentum and  the Matsubara frequency, $(\omega_{n},\vk)$. $\delta_{\vk}$ is defined in Eq. \ref{eq:pathInt2:L1}. 
\begin{equation*}
\delta_{c}\equiv\mtrx{0&-\frac{\Delta_{1}{}\Delta_{2}{}}{4E^{2}_{\vk}}&u_{\vk}\\
\frac{\Delta_{1}{}\Delta_{2}{}}{4E^{2}_{\vk}}&0&v_{\vk}\\
-u_{\vk}&-v_{\vk}&0
}\frac{\Delta_{2}{}}{\eta}
\end{equation*}
Introduce two matricies $M_{k}$ and $\widetilde{M}_{k}$, 
\begin{gather}
M_{k}\equiv{}T_{\vk}^{\dg}\sigma_3T_{\vk}B_{k}^{-1}T_{\vk}^{\dg}\sigma_3T_{\vk}B_{k}^{-1}\\
\widetilde{M}_{k}\equiv{}B_{k}^{-1}T_{\vk}^{\dg}\sigma_3T^{}_{\vk}B_{k}^{-1}T_{\vk}^{\dg}\sigma_3T^{}_{\vk}
\end{gather}
And we can rewrite $\pi^{(0)}(0)$ as
\begin{equation}
\pi^{(0)}(0)=\sum_k\tr\big({M}_{k}\big)+2\tr\Big(\delta_{\vk}\widetilde{M}_{k}-\delta_{\vk}M_{k}\Big)
\end{equation}
Here we use the cyclical  property of the trace $\tr(AB)=\tr(BA)$.  
It is straightforward to calculate
\begin{equation*}
T_{\vk}^{\dg}\sigma_3T_{\vk}B_{k}^{-1}=
\begin{pmatrix}
{\frac{\xi_{\vk}}{E_{\vk}(i\omega_{k}-E_{1\,\vk})}}&\frac{\Delta_{1}}{E_{\vk}(i\omega_{k}+E_{2\,\vk})}&0\\
{\frac{\Delta_{1}}{E_{\vk}(i\omega_{k}-E_{1\,\vk})}}&-\frac{\xi_{\vk}}{E_{\vk}(i\omega_{k}+E_{2\,\vk})}&0\\
0&0&-\nth{i\omega_{k}+E_{3\,\vk}}\\
\end{pmatrix}
\end{equation*}
Now it is easy to calculate the first term
\begin{equation}
\begin{split}
\sum_k\tr\big(M_{k}\big)=&
\sum_{k}\mbr{
\frac{2\Delta_{1}^{2}}{E_{\vk}^{2}(i\omega_{k}-E_{1\,\vk})(i\omega_{k}+E_{2\,\vk})}+
\br{\frac{\xi_{\vk}^{2}}{E_{\vk}^{2}(i\omega_{k}-E_{1\,\vk})^{2}}+\frac{\xi_{\vk}^{2}}{E_{\vk}^{2}(i\omega_{k}+E_{1\,\vk})^{2}}
-\nth{(i\omega_{k}+E_{3\,\vk})^{2}}}
}
\end{split}
\end{equation}
Only root $-E_{2\,\vk}$ in the first term contributes in the Matsubara frequency summation at zero temperature.
\begin{equation}\label{eq:pathInt2:pi0-1}
\sum_k\tr\big(M_{k}\big)=\sum_{\vk}\frac{2\Delta_{1}^{2}}{E_{\vk}^{2}(E_{1\,\vk}+E_{2\,\vk})}
\approx\sum_{\vk}\frac{\Delta_{1}^{2}}{E_{\vk}^{3}}-\sum_{\vk}\frac{\Delta_{1}^{2}\Delta_{2}^{2}\xi_{\vk}}{2E_{\vk}^{5}(\xi_{\vk}+\eta)}
\end{equation}

For the lowest order of the second term in Eq. \eqref{eq:pathInt2:pi0long}, we only need to take the lowest order of $B_{k}$
\begin{equation}
B_{k}=\mtrx{i\omega_{k}-E_{\vk}&0&0\\0&i\omega_{k}+E_{\vk}&0\\0&0&i\omega_{k}+\xi_{\vk}+\eta}
\end{equation}
It is easy to verify at this approximation
\begin{equation}\label{eq:pathInt2:pi0-2}
\tr\Big(\delta_{\vk}\widetilde{M}_{k}-\delta_{\vk}M_{k}\Big)=0
\end{equation}
Combine Eq. \eqref{eq:pathInt2:pi0-1} and Eq. \eqref{eq:pathInt2:pi0-2}, we have 
\begin{equation}
\pi^{(0)}(0)\approx\sum_{\vk}\frac{\Delta_{1}^{2}}{E_{\vk}^{3}}-\sum_{\vk}\frac{\Delta_{1}^{2}\Delta_{2}^{2}\xi_{\vk}}{2E_{\vk}^{5}(\xi_{\vk}+\eta)}\end{equation}

$\pi^{\perp}(0)$ can actually be calculated   exactly
\begin{equation}
\pi_{ij}^{\perp}(0)=\sum_k(k_i)(k_j)\tr(\hat{G}_{0\,k}\hat{G}_{0\,k})
\end{equation}
\begin{equation}
\begin{split}
	\tr(\hat{G}_{0\,k}\hat{G}_{0\,k})=&\sum_k\tr\big(T_{\vk}L_{\vk}B_{k}^{-1}L_{\vk}^{\dg}T_{\vk}^{\dg}T_{\vk}L_{\vk}B_{k}^{-1}L_{\vk}^{\dg}T_{\vk}^{\dg}\big)\\
	=&\sum_k\tr\big(B_{k}^{-1}B_{k}^{-1}\big)\\
	=&\sum_{\vk,i}(\sum_{\omega_{n}}(i\omega_{k}-\xi_{i})^{-2})\\
	=&0
\end{split}
\end{equation}


\section{Consistency of the expansion over $\zeta$\label{sec:pathApp:consistency}}
In our treatment here, one crucial assumption in expansion is the smallness of $\Delta_{2}/\eta$ comparing to $1$.  Here we check it.  We have the closed-channel gap equation (\eef{eq:pathInt2:mfclose})
\begin{equation}
\Delta_{2}=\sum{}Yh_{1\vk}+\sum{}Vh_{2\vk}\tag{\ref{eq:pathInt2:mfclose}}
\end{equation}
The first term on the right is relatively small comparing to the second term.  We just keep the second term for estimation.  Furthermore,  we assume $h_{2\,\vk}=\sqrt{N_{c}}\phi_{0\,\vk}$, where $\phi_{0\,\vk}$ is the normalized wave function of the  isolated closed-channel potential satisfying \sch equation (\eef{eq:pathInt2:phi})
\begin{equation}\tag{\ref{eq:pathInt2:phi}}
-E_{b}^{(0)}\phi_{0\,\vp}=2\epsilon_{\vp}\phi_{0\,\vp}-\sum_{\vk}V \phi_{0\,\vk}
\end{equation}
Rearranging it, we have (especially at low momentum)
\begin{equation*}
\sum_{\vk}V \phi_{0\,\vk}=(2\epsilon_{\vp}+E_{b})\phi_{0\,\vp}\approx{\eta}\phi_{0\,\vp}
\end{equation*}
Here $E_{b}$ is the binding energy of the closed-channel bound state and $\eta$ is the Zeeman energy difference. The second approximation is correct at low momentum (smaller or in the same order of the Fermi momentum) as $\epsilon_{\vp}\ll{}E_{b}\approx\eta$ not too far away from the resonance.  Put all these together, we have
\begin{equation*}
\Delta_{2}\approx\alpha{}E_{b}\phi_{k=0}
\end{equation*}
If we assume a simple exponentially decayed wave function as in Eq. \ref{eq:pathInt2:phi2body} from Sec. \ref{sec:pathInt2:short-range} 
\begin{equation}\tag{\ref{eq:pathInt2:phi2body}}
\phi_{\vk}=\sqrt{\frac{8\pi\kappa}{\mathcal{V}_{0}}}\frac{1}{k^{2}+\kappa^{2}}\approx\sqrt{\frac{8\pi\kappa}{\mathcal{V}_{0}}}\frac{1}{\kappa^{2}}
\end{equation}
Here  $\mathcal{V}_{0}$ is the total volume and $\kappa$ is the characteristic momentum of the closed-channel bound state, $\eta\approx{}E_{b}=\hbar^{2}\kappa^{2}/2m$.  The second approximation above is only for  low momentum.  Collect all these together, we have
\begin{equation}
\Delta_{2}\approx\sqrt{N_{c}}\eta\sqrt{\frac{8\pi\kappa}{\mathcal{V}_{0}}}\frac{1}{\kappa^{2}}
\sim\eta\sqrt\frac{n_{c}}{\kappa^{3}}
\sim\eta\br{\frac{k_{Fc}}{\kappa}}^{\frac{3}{2}}
\sim\eta\br{\frac{E_{Fc}}{\eta}}^{\frac{3}{4}}
\end{equation}
%:
$k_{Fc}$ is the Fermi momentum corresponding to density of atoms in the close-channel, which is much smaller than the characteristic momentum for the bound-state, $\kappa$.   Therefore we have $\Delta_{2}\ll\eta$, even when $n_{c}$ is close to the total density $n$. 

Now we check whether the corrections in Fermionic spectrum (Eqs. \ref{eq:pathInt2:xiExpand}-\ref{eq:pathInt2:xiExpand3}), 
are indeed small comparing to the zeroth order terms.  
\begin{align}
E_{1\vk}\approx{}E_{\vk}+u_{\vk}^{2}\Delta_{1}\zeta\tag{\ref{eq:pathInt2:xiExpand}}\\
E_{2\vk}\approx{}E_{\vk}-v_{\vk}^{2}\Delta_{1}\zeta\tag{\ref{eq:pathInt2:xiExpand2}}\\
E_{3\vk}\approx{}\epsilon_{\vk}+\eta+\frac{\zeta}{2}\Delta_{1}
\tag{\ref{eq:pathInt2:xiExpand3}}
\end{align}
Here, we mostly only concern of case of low momentum ($k\sim{}k_{F}$).  In Eq. \ref{eq:pathInt2:xiExpand3}, 
\begin{equation*}
\frac{\zeta\Delta_{1}}{E_{3\vk}}\sim{}\frac{\Delta_{2}^{2}}{\eta^{2}}\sim\br{\frac{k_{Fc}}{\kappa}}^{3}\ll1
\end{equation*}
Eq. \ref{eq:pathInt2:xiExpand} and Eq. \ref{eq:pathInt2:xiExpand2} are slightly more complicated.  Both of them involve $\frac{\Delta_{2}^{2}}{E_{\vk}\eta}$,  at the  BCS limit, the closed-channel density is small, $k_{F\,c}$ is small and that makes this ratio small; when close to the (narrow) resonance, where $n_{c}$ is comparable to to the total density, at low energy, $\Delta_{1}$ is in the order of the Fermi energy, so does $E_{\vk}$.   We have (we no longer distinguish $k_{F\,c}$ with $k_{F}$)
 \begin{equation}\label{eq:pathApp:zetaEs}
 \zeta=\frac{\Delta_{2}^{2}}{\Delta_{1}\eta}\sim\frac{\eta^{2}\frac{k_{Fc}^{3}}{\kappa^{3}}}{k_{F}^{2}\eta}\sim\frac{k_{F}}{\kappa}\ll1
\end{equation}

More deeply, in the secular equation that leads to spectrum, Eq. \ref{eq:pahtApp:secular}.  We rewrite it without scaling to $E_{\vk}$,  (We drop subscript $\vk$ in the following equations for simplicity)
\begin{equation*}
(x^{2}-E^{2})(x-\xi-\eta)-\Delta_{2}^{2}x+\Delta_{2}^{2}E(u^{2}-v^{2})=0
\end{equation*}
It is not hard to use definition of $u$ and $v$ to find $u^{2}-v^{2}=\xi/E$, and express $E^{2}=\xi^{2}+\Delta_{1}^{2}$. Therefore we have
\begin{equation*}
(x-\xi)(x+\xi)(x-\xi-\eta)-\Delta_{1}^{2}(x-\xi-\eta)-\Delta_{2}^{2}(x-\xi)=0
\end{equation*}
Here the first term is for free particles, and let us estimate the relative size of the last two terms.  For low-momentum solution, we simply use $\Delta_{1}\sim{}E_{F}$, we find
\begin{equation*}
\frac{\Delta_{1}^{2}(x-\xi-\eta)}{\Delta_{2}^{2}(x-\xi)}\sim\frac{E_{F}^{2}\eta}{\Delta_{2}^{2}E_{F}}\sim\frac{\kappa}{k_{F}}\gg1
\end{equation*}
This justifies our choice to neglect the last term when finding the lowest-order solution and then use the last term for correction.  

 



 In another word, the above estimation is just saying that the total occupation number of the closed-channel at a low momentum level is much smaller than 1 in all regions of resonance (narrow or broad) because the closed-channel bound state is much smaller than the interparticle distance.  This factor gives us a small factor, $\zeta\sim\frac{r_{c}}{a_{0}}\sim\frac{k_{F}}{\kappa}\sim\sqrt\frac{E_{F}}{\eta}$, upon which we can do the expansion.  
