\subsection{Cooper-like problem}
To accommodate the situation where the background is not a simple filled Fermi sea, a slightly modified approach is needed.  Put into the second-quantization language, basically the change is about the vacuum.  It is the BCS wave-function as the vacuum, $\ket{\bar{v}}=\prod\br{u_\vk+v_\vk{a^\dg_\vk{}c^\dg_{-\vk}}}\ket{0}$.   And we want to solve a sigle-close-channel hamiltonian two-body equation with this special vacuum, 
\begin{equation}\label{eq:wfSemiCooper}\ket{\Psi}=\br{\sum\phi_\vk{a^\dg_\vk{}c^\dg_{-\vk}}}\ket{\bar{v}}\end{equation}. 
A simple Fermi sea is $v_\vk=1, u_\vk=0$ for $k<k_F$ and $v_\vk=0, u_\vk=1$ for $k>k_F$.

\subsection{Tony's comment}
Tony is still troubling by the failure that the original u/v/w scheme cannot be renormalized and compared with 2-body solution by me.

He made a nice comment about the crossover in sigle-channel.  In BCS side, the close-channel bound state is above the threshhold and thus a virtual state, and the gound state is the extended state and therefore BCS state as many-body effect kicks in.  When it evolves to BEC side, there are two branches, the scattering state, or the close-channel bound state, which is below threshhold and a real state now. (It is highly mixed near resonance.) The claim is that by adiabatic switch, many-body BCS state goes to this bound-state branches because at resonance, instead of the relaxation as usually $\tau^{-1}\sim{\hbar^2/ma_r^2}$ and diverges at resonance, it is in the order of ``gap'', $\tau^{-1}\sim\Delta$, which in turn is order of Fermi energy.  And therefore, it is achievalble as the abiabatic switch, instead of falling into both branches as sudden switch.  
