\section{note of 2009.09.16}
\subsection{More Discussion on the Equal-Time Abnormal Green Function}
I continue to think about the Shizhong's approach \cite{ZhangThesis} with equal-time abnormal Green's Function from last section \ref{subsec:zhangEq090915}.  Well, the equal-time function seems better for most of the physical application.  For those non-equal-time function, one usually has to take the limit of $t\rightarrow0$ to get back to the real physical quantities. Well, the four-vector, which includes time as the special dimension, is natural in the high-energy application where Lorentz invariance put time into a quite equal footing as space coordinators.  But in statistical physics, time is always special.  Is the non-equal-time sufficient to handle the problems, why do we need the non-equal time Green's function? 

Normal Green's function has non-dependence on the absolute value of the time, only the time difference $t_2-t_1$.  And all the $\omega\neq0$ quantities describe the dynamics.  By fixing $t_2=t_1$, we only have the static quantities.  First of all, these equal-time quantity is not 0.   Go to the Lehmann representation, we can find out that the equal-time operator pairs have no time dependence at all, at least for the normal case, no matter in zero temperature $\bra{0}...\ket{0}$ or finite temperature $\sum_m{e^{-\beta{E_m}}}\bra{m}...\ket{m}$.  I am not sure whether the speciality of the BCS ground state makes any difference. Especially, for the abnormal Green's function, you have $\bra{N}...\ket{N\pm2}$, which gives you some extra $\exp\mbr{i(2\mu{t}/\hbar)}$.
