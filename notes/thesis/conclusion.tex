% !TeX root =thesis.tex
% 
In this thesis, we studied the narrow resonance, especially the case where two channels share the same species.  In general, for narrow resonance without shared species, the main correction is from the weight in closed-channel.  Two number equations exists, one for open-channel and one for closed-channel.  The number equation resembling the single channel is for the density of open-channel, as in  \cite{GurarieNarrow}.  

However, when there is a common species,  the Pauli exclusion in three-species narrow resonance between two channels calls for careful  consideration.  Overall, our treatment goes along in the spirit of universality.  For the dilute system with short-range potential, such as dilute ultracold alkali gas, the short-range part of  wave function is essentially unchanged from two-body  to many-body.  This particular feature serves formally as the boundary condition.  On the other hand, the rest of the wave function is essentially free where no two particles are close in short-range.  Different from the universality treatment is the involving resonance. A resonance makes the system, even the two body part, extremely sensitive to the change in small relative weight between two channels.  This  requires extra care.  In Feshbach resonance, the closed-channel bound state, $\phi_{0}$,  is actually not sensitive to resonance; therefore, we apply the simple boundary condition on it. On the other hand, the indirect interaction mediated by closed-channel dominates direct interaction within open-channel. Consequently, the small change in the closed-channel bound-state weight \emph{does} affect the short-range wave function in open-channel, which shows in the boundary condition through $a_{s}$. 

When the size of  closed-channel bound state is in the order of   inter-particle distance or even larger, it is a genuine three-species many-body problem and no obvious solution is available.  However, when this bound-state is much smaller than inter-particle distance, ratio ($a_{c}/a_{0}$) serves as the expanding coefficient and we can extract  the effect perturbatively.  In essence, we can ignore the many-body effects between these closed-channel bound-states, while only taking into consideration of the Pauli exclusion between channels.  A few new parameters need to be introduced and can be calibrated from experiments.  System's mean field properties can still be determined by gap equations and number equations as in single-channel problems.  The excitation modes are also close to the original single-channel result with correction in the order of $r_{c}/a_{0}$.

There are three different regimes for the many-body energy scale $E_F$ when comparing with two-body physics.  

First, when $E_F$ is much smaller than the characteristic energy $\delta_c$ (Sec. \ref{sec:intro:twobody}), around resonance, closed-channel weight is negligible. We can ignore closed-channel at all and approximate the open-channel interaction with pseudo-potential characterized by s-wave scattering length $a_s$. 

Second, when $E_F$ is larger than $\delta_c$, but smaller than the closed-channel bound state binding energy $E_b$, or the bare detuning $\eta$, closed-channel weight is significant and cannot be ignored, which means $n_{\text{open}}+n_{\text{close}}=n_{\text{total}}$. Also, we need to take into account that in the resonance formula of $a_s$, shifting should count from Fermi sea instead of zero as in two-body physics.    This is extensively treated in \cite{GurarieNarrow} as well as in this thesis.  Furthermore, Pauli exclusion between two channels needs to be taken into account when they share common species.  Nevertheless, in the low-momentum, closed-channel bound state has very small weight and  open-channel dominates.  So we can still treat the Pauli exclusion between channels perturbatively with the expansion over ratio $E_F/\eta$.  This is the emphasis of the current thesis.  

Third, when $E_F$ is larger than binding energy $E_b$ or bare detuning $\eta$, we have a pure three species many-body problem. This regime can be achieved in two ways.  One way requires a very large $E_F$, which indicates a very dense system.  However, it is hard to image that  the original dilute alkali gas model still applies at all. Various approximation in the model would probably break down beforehand, such as the pseudo-potential approximation with $a_s$. The other way requires a very small $\eta$, which indicates a genuine three-species many-body problem and remains an open-problem.   

This thesis focuses on the second regime.  By careful analysis, we distinguish two related but different concepts, the total weight of closed-channel  and the weight of closed-channel in a particular momentum level.  
       Within our assumption (mainly closed-channel bound state is much smaller than inter-particle distance), very low occupation in low-momentum levels ($k\lesssim{}k_{F}$) persists despite the large total weight of closed-channel.     A suitable small parameter $E_F/\eta$ (or $a_0/a_c$), upon which a perturbation theory can be developed over a non-perturbative zeroth order solution, emerges with this discovery.  Paring in many-body problems is known to be  non-perturbative and has to be handled with proper non-perturbative techniques.  Nonetheless, we can concentrate the non-perturbative part into the zeroth order  (broad resonance) solution, while treating the rest correction perturbatively.   The resulting theory  handles two channels in step self-consistently\footnote{Here by ``self-consistent'', we refer to the self-consistency in treating close-channel according to zeroth order of open-channel pairing.  The BCS-type treatment of paring in open-channel   is known to be not self-consistent. }. 
       
       On the other hand, it is also known that the simple BCS-type pairing treatment is not adequate  to quantitatively describe the whole BEC-BCS crossover region.  Therefore the zeroth order solution used in this thesis (simple BCS type ansatz or saddle point) can be improved with further theoretical development.  Nevertheless, we expect the perturbation established here still valid.  Once the zeroth order solution (for broad resonance or single channel BEC-BCS crossover model) is patched over with whatever advancement, the correction of narrow resonance in such a parameter regime can still be obtained with similar procedure as in this thesis.  