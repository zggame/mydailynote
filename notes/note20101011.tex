\subsection{number equation}
Put \eef{eq:20100915:F} into number equation \eef{eq:20100909:number}, we have\footnote{Here we drop $\sgn_{k}$ as $(\epsilon^{ab}_{\vk}-2\mu+  G_{0}^2\eta)$ takes care of the sign}
\begin{equation}\label{eq:20101011:number}
N=\sum{1-\sqrt{1-4G_{k}^{2}}\frac{(\epsilon^{ab}_{\vk}-2\mu+  G_{\vk}^2\eta)}{{\sqrt{{(\epsilon^{ab}_{\vk}-2\mu+  G_{\vk}^2\eta)^{2}+\Delta^{2}}}}}}
\end{equation}
This equation also requires to take into account that $G_{k}$ approaches 0 at high momentum to be convergent.  
\subsection{typical energy scale}
I calculated $^{6}\text{Li}$'s Fermi energy for density $10^{15}\text{cm}^{-3}$, it is $5\times10^{-28}J$, for Bohr magneton $\mu_{B}=9.27\times10^{-24}\text{J T}^{-1}$, it corresponds magnetic field $B=0.5\times10^{-4}\text{T}=0.5\text{G}$.

\subsection{more on $G_{k}$ and $\eta$}
$\phi^{0}$ the w.f. of two-body close-channel bound state varies in different Feshbach resonance and is a case-specific quantity.  Its property can be lumped into the general two-body Feshbach resonance formula and usually does not matter.  It does matter in narrow resonance problem.  However, this can be simplified if it is a high-energy bound-state, that mostly is outside the potential.  Therefore, most weight of $\phi^{0}$ follows the simple form 
\[
\phi^{0}(r)=C\frac{e^{-\kappa{r}}}{r}
\]
 where $\frac{\hbar^{2}\kappa^{2}}{2m}=E^{0}$ and $\kappa>0$.  $C^{2}=\frac{\kappa}{2\pi}$.  Convert it into k-space
 \begin{equation}\label{eq:20101011:phi}
\phi^{0}_k=\nth{(2\pi)^{\frac32}}\int{d^{3}\vr\phi^{0}(r)e^{-i\vk\cdot\vr}}=\nth{\pi}\frac{\kappa}{k^{2}+\kappa^{2}}
\end{equation}
Now we can estimated $G_{k}^{2}\eta$. $G_{k}=\alpha\phi^{0}_{k}$ where 
\[|\alpha|^{2}<n\sim{E_{F}^{3/2}}\]
\[
\abs{\phi^{0}_{k}}^{2}=\nth{\pi^{2}\kappa^{3}}\sim\nth{(E^{0})^{3/2}}
\]
And $\eta\sim{E^{0}}$, So 
\[
G_{k}^{2}\eta<G_{k=0}^{2}\eta<E_{F}(\frac{E_{F}}{E^{0}})^{1/2}
\]
And we have $E_{F}\ll{E^{0}}$, so this term is much smaller than $\mu\sim{E_{F}}$ in BCS side. For now, we just omit this term.  Considering $G_{k}^{2}\ll1$, we can expand $\sqrt{1-4G_{k}^{2}}$ in \eef{eq:20101011:number} and \eef{eq:20101004:gapS} 
\begin{gather}
\nth{{t_{0}}(\mu)}=\sum_{\vk}
\br{\nth{\epsilon_{\vk}}-\frac{1}{\sqrt{{(\epsilon_{\vk}-2\mu)^{2}+\Delta^{2}}}}
+\frac{2\alpha^{2}|\phi^{0}_{k}|^{2}}{\sqrt{{(\epsilon_{\vk}-2\mu)^{2}+\Delta^{2}}}}}\label{eq:20101011:gapa}
\\
N=\sum{1-\frac{(\epsilon_{\vk}-2\mu)}{{\sqrt{{(\epsilon_{\vk}-2\mu)^{2}+\Delta^{2}}}}}
+2\alpha^{2}|\phi^{0}_{k}|^{2}\frac{(\epsilon_{\vk}-2\mu)}{{\sqrt{{(\epsilon_{\vk}-2\mu)^{2}+\Delta^{2}}}}}}
\label{eq:20101011:gapb}
\end{gather}
Here $\phi^{0}_{k}$ takes the form in \eef{eq:20101011:phi} and we have an extra quantity $\kappa$ ($E^{0}$), also, we have 
\begin{equation}
\Delta_{\vk}=U_{\vk\vk'}F_{\vk'}+Y_{\vk\vk'}G_{\vk'}\approx\alpha{Y_{\vk\vk'}\phi^{0}_{\vk'}}
\end{equation}
\[\Delta^{2}\approx\alpha^{2}(\Delta^{2})|_{\text{very BEC end}}
\]
Here we can see that $\Delta$ saturates in BEC end where close-channel dominates.  
\subsection{}
A more convenient choice for $\phi^{0}$ is to normalized to total number N, i.e. $\sum|\phi^{0}_{k}|^{2}=N$.  So $0<|\alpha|^{2}<1$ is the ratio of atoms in close-channel to total number.  For this normalization, we find 
\begin{equation}
\phi^{0}_{k}=\sqrt{8\pi\kappa\rho}\nth{k^{2}+\kappa^{2}}
\end{equation}
In \eef{eq:20101011:gapa}, 
\[
t_{0}=\frac{4\pi\hbar^{2}a_{s}(\mu)}{m}
\]
and $a_{s}$ follow the normal Feshbach resonance formula
\begin{equation}
a_{s}(\mu)=a_{bg}(1+\frac{\Delta{B}}{B-B_{0}-(\mu/\delta\mu_{B})})
\end{equation}
Here $\delta\mu_{B}$ is roughly the magnetic moment of close-channel bound-state.   Put this back to \eef{eq:20101011:gapa}  and we have self-consistent equations for $\mu$ and $\Delta$ ($\alpha$?)
