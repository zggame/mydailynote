\section{note of 2009.09.17}
\subsection{Thermodynamics of BCS Wave Function}
The more I think about the equal-time Green's function, the weird I felt. Golden Baym comments on the problem.  In the order parameter setting, $\psi\psi$ in BCS or $\psi$ in BEC, they do have dependence on absolute time related to chemical potential, $2\mu$ in BCS and $\mu$ in BEC.  In BCS, $F(t_1,t_2)=e^{-i2\mu{t_1}/\hbar}F(0,t_2-t_1)$. So equation \eqref{eq:zhang536} equals $2\mu$.  


\subsection{Baym's Comment on 3-Species Problem}
He comment on the 3-species problem.  In Landau Fermi liquid, short-range correlation becomes the equivalent repulsion, so maybe it is the same here.  Maybe it can be dedued to some equivalent repulsion.

\subsection{Decoupling Four-Operator Terms in Equal-Time Equation of Motion }
It is not clear how to decouple four-operator terms in two-channel problems, such as Shizhong's equation\eqref{eq:zhang536}.  It is not obvious how to decouple these two terms.  There are several possibilities.  We may follow the similar way as in normal BCS/AGD methods.  For example, for the third term in equation \eqref{eq:zhang536}, with $U_{2113}(\vr_1-\vr')$ and $F_{21}(\vr_1-\vr_2)=\av{\psi_2(\vr_1)\psi_1(\vr_2)}$, there are three different possible decoupling.  
\begin{equation}\begin{split}\label{eq:decouple}
&\int{d}\vr'U_{2113}\br{\vr_1-\vr'}\av{\psi^\dagger_{1}\br{\vr'}\psi_1\br{\vr'}\psi_3\br{\vr_1}\psi_1\br{\vr_2}}\\
=&+\int{d}\vr'U_{2113}\br{\vr_1-\vr'}\av{\psi^\dagger_{1}\br{\vr'}\psi_1\br{\vr'}}\av{\psi_3\br{\vr_1}\psi_1\br{\vr_2}}\\
&-\int{d}\vr'U_{2113}\br{\vr_1-\vr'}\av{\psi^\dagger_{1}\br{\vr'}\psi_3\br{\vr_1}}\av{\psi_1\br{\vr'}\psi_1\br{\vr_2}}\\
&-\int{d}\vr'U_{2113}\br{\vr_1-\vr'}\av{\psi^\dagger_{1}\br{\vr'}\psi_1\br{\vr_2}}\av{\psi_3\br{\vr_1}\psi_1\br{\vr'}}
\end{split}\end{equation}
\begin{itemize}
	\item The second term is ignored because $\av{\psi_1\br{\vr'}\psi_1\br{\vr_2}}=0$ for the BCS type wave function. 
	\item The first term is a Hatree term where the cloud of species 1 around species 3 to make the transition from (1,3) to (1,2).  And $\psi^\dagger_{1}\br{\vr'}\psi_1\br{\vr'}$ is nothing but the number operator $n_1$.  And this term gives only in order of $n\cdot{a^3}\ll1$ due to the short-range nature of the interaction, only $\abs{\vr_1-\vr}\lesssim{a}$ is useful part of integration, where $a$ is the range of the interaction.  This term is usally ignored by the BCS theory because it gives the same boost for the normal state as the BCS state.
\item  The third term is the term as $G\cdot{F}$ in the AGD approach. However, here it involves both $\vr_1-\vr'$ and $\vr_2-\vr'$, and at least one of them is larger than the interaction range $a$.  In simple BCS, $\av{\psi^\dagger_{1}\br{\vr'}\psi_1\br{\vr_2}}=\sum{\exp\mbr{-i\vk\cdot(\vr_1-\vr_2)}}\abs{v_k}^2$, for the short-range in order of $a$, it is close to n, but \textbf{for larger than $a$, it is not clear}. 
 \end{itemize} 
