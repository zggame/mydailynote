% !TeX root =thesis.tex


\section{Diagonalize Matrix Eq. (\ref{eq:pathInt2:G2})\label{sec:diagonalize}}
We need to find a unitary transformation $L$ to diagonalize matrix 
\begin{equation*}
i\omega_{n}I+\mtrx{-E_k&0&u _kD_2\\0&+E_k&v_kD_2\\u_kD_2&v_kD_2&+\xi_k+\eta}
\end{equation*}
We drops all the $k$ subscripts in this section.  We notice that the first term is proportional to identity matrix and does not change by unitary transformation, we only need to concentrate for the second term.  We rescale all elements with $E$, 
\begin{equation*}
R=
\begin{pmatrix}
-1&0&y_1\\
0&1&y_2\\
y_1&y_2&t
\end{pmatrix}
\end{equation*}
The secular equation is 
\begin{equation}\label{eq:pahtApp:secular}
(x^{2}-1)(x-t)-(y_{1}^{2}+y_{2}^{2})x+y_{1}^{2}-y_{2}^{2}=0
\end{equation}
We will assume at the zeroth order, the three eigenvalues are $-1$, $1$ and $t$.  ($t$ has weak dependency on energy as $(\xi_{k}+\eta)/E_{k}$, however, at the low energy region of interest, we ignore $\xi_{k}$.) Both $y_{1,2}$ and t are larger than 1, however, we will verify that given condition $y_{i}^{2}\ll{t}$, the correction is indeed small and the expansion is legit.(See Sec.\ref{sec:pathApp:consistency})  \emph{Indeed, this approximation is not as bad as it seems to be, close-channel component can still be smaller than open-channel at low-k (in order of $k_{F}$)  due to close-channel bound state is much smaller than inter-particle distance even when total close-channel is more than open-channel.  And here all the quantities are about low-k unless specifically noticed.} 
We expand the system to the first order of $y_{i}^{2}/{t}$, and find
\begin{equation}
\begin{array}{ccc}
x^{(0)}&\quad{}x^{(1)}&\quad{}Eigenvector\nonumber\\
-1&-\frac{y_{1}^{2}}{t}&\mtrx{1&\frac{y_{1}y_{2}}{2t}&-\frac{y_{1}}{t}}\\
1&-\frac{y_{2}^{2}}{t}&\mtrx{-\frac{y_{1}y_{2}}{2t}&1&-\frac{y_{2}}{t}}\\
t&\frac{y_{1}^{2}+y_{2}^{2}}{2t}&\mtrx{\frac{y_{1}}{t}&\frac{y_{2}}{t}&1}
\end{array}
\end{equation}
Now it is easy to write down the corresponding diagonal matrix and unitary transformation
\begin{equation}
B=i\omega_{n}I+E\mtrx{-1-\frac{y_{1}^{2}}{t}&0&0\\0&1-\frac{y_{2}^{2}}{t}&0\\0&0&t+\frac{y_{1}^{2}+y_{2}^{2}}{2t}}
\end{equation}
\begin{equation}
L=\mtrx{1&-\frac{y_{1}y_{2}}{2t}&\frac{y_{1}}{t}\\\frac{y_{1}y_{2}}{2t}&1&\frac{y_{2}}{t}\\-\frac{y_{1}}{t}&-\frac{y_{2}}{t}&1}
\end{equation}
Here $L$ is not exactly unitary transformation, it only unitary in the first order of  $y_{i}^{2}/{t}$ (or $D_{i}^{2}/(E\eta)$). We have 
\[
B=L^{\dg}RL+o(\frac{y_{i}^{2}}{t})
\]
Alternatively, we can write $L$ as 
\begin{equation}
L=I+
\mtrx{0&-\frac{D_{1}D_{2}}{4E^{2}}&u\\
\frac{D_{1}D_{2}}{4E^{2}}&0&v\\
-u&v&0
}\frac{D_{2}}{\eta}
\end{equation}
Here we use $uv=D_{1}/2E$


\section{Derive mean-field equation \eqref{eq:pathInt2:mf}\label{sec:pathInt2:deriveMF}}
For a $3\times3$ matrix as in Eq. (\ref{eq:pathInt2:nG}), 
\begin{equation}\tag{\ref{eq:pathInt2:nG}}
\mathcal{G}^{-1}=\begin{pmatrix}
i\omega_{n}-\xi_{k}&D_{1}&D_{2}\\
\bar{D}_{1}&i\omega_{n}+\xi_{k}&0\\
\bar{D}_{2}&0&i\omega_{n}+\xi_{k}+\eta
\end{pmatrix}
\end{equation}
A general $3\times3$ matrix inverted as such, 
  \begin{equation}
  \mtrx{A_{11}&A_{12}&A_{13}\\A_{12}^{*}&A_{22}&0\\A_{13}^{*}&0&A_{33}}^{-1}=
  \nth{|A|}
  \mtrx{A_{22}A_{33}&-A_{12}A_{33}&-A_{13}A_{22}\\
  	-A_{12}^{*}A_{33}&A_{11}A_{33}-A_{13}A_{13}^{*}&A_{12}^{*}A_{13}\\
	-A_{13}^{*}A_{22}&A_{12}A_{13}^{*}&A_{11}A_{22}-A_{12}A_{12}^{*}}
  \end{equation}
where $|A|$ is the determined of $A$ matrix.  Here we work in momentum space, in which the system is nicely decoupled at least to the mean-field order.  And we therefore drop all the $k$ subscript in the rest of section.  For Eq. (\ref{eq:pathInt2:nG}),
\begin{equation}
|A|=(i\omega_{n}-E_{1})(i\omega_{n}+E_{2})(i\omega_{n}+E_{3})
\end{equation}
Now we have $G_{0}$, and we can find the last term in \ref{eq:pathInt2:mf01}, 
\begin{equation}
\begin{split}
\tr\mbr{{G_{0}}\cdot\cmtrx{0&1&0\\0&0&0\\0&0&0}}&=\sum_{\vk\omega_{n}}G_{0\,(21)}\\
&=\sum_{\vk}\sum_{\omega_{n}}\frac{-D_{1}^{*}(i\omega_{n}+\xi+\eta)}{(i\omega_{n}-E_{1})(i\omega_{n}+E_{2})(i\omega_{n}+E_{3})}\\
&=\sum_{\vk}D_{1}^{*}\frac{E_{1}+\xi+\eta}{(E_{1}+E_{2})(E_{1}+E_{3})}\equiv\sum_{\vk}h_{1\,\vk}
\end{split}
\end{equation}
Here we perform Matsubara summation at zero temperature in the third equal sign with the normal trick (see sec. 4.2.1 in \cite{Altland}, sec. 25 in \cite{Fetter}, also refer to Footnote \ref{foot:intro:sum} at Page. \pageref{foot:intro:sum}).  We notice that within three roots, $E_{1}$, $-E_{2}$ and $-E_{3}$, we only need to take into account two negative roots $-E_{2}$ and $-E_{3}$, assuming the correction is small. 
Similarly
\begin{equation}
\begin{split}
\tr\mbr{{G_{0}}\cdot\cmtrx{0&0&1\\0&0&0\\0&0&0}}&=\sum_{\vk\omega_{n}}G_{0\,(31)}
=\sum_{\vk}D_{2}\frac{E_{1}+\xi}{(E_{1}+E_{2})(E_{1}+E_{3})}\equiv\sum_{\vk}h_{2\,\vk}
\end{split}
\end{equation}
And we have 
 \begin{align*}
(\tilde{U}^{-1})_{11}\bar{D}_{1}+(\tilde{U}^{-1})_{21}\bar{D}_{2}-\sum_{\vk}h_{1\,\vk}=0\\
(\tilde{U}^{-1})_{12}\bar{D}_{1}+(\tilde{U}^{-1})_{22}\bar{D}_{2}-\sum_{\vk}h_{2\,\vk}=0
 \end{align*}
Invert the interaction matrix $\tilde{U}$ and we have Eq.  \ref{eq:pathInt2:mf}.


\section{Wave function for short-range potential}\label{sec:pathInt2:short-range}
Here we discuss some possible generalization on the wave function for short-range potential.  This topic has been studied by Shizhong Zhang \cite{shizhongUniv}. We will use some similar ideas.  Outside the range $r_{c}$ of a short-range potential,  atom is free and  \sch equation is very simple.
\begin{equation}
-\frac{\hbar^{2}}{2m}\nabla^{2}\psi=E\psi
\end{equation}
The equation has a simple solution for s-wave, $\psi=A{e^{-\kappa{r}}}/{r}$ ($\kappa$ is imaginary for scattering state), here normalization $A$ is determined  by connecting it with the short-range part of the wave function, $\varphi_0$. 

Let us discuss the bound-state first, $\kappa>0$.  In the momentum space, there is also a universal behavior at low-momentum, where $kr_{c}\ll1$.   
\begin{equation*}
\psi_{k}=\nth{(2\pi)^{3/2}}\int{d\vr}(\varphi_{0}+A\frac{e^{-\kappa{r}}}{r})e^{-i\vk\cdot\vr}
\end{equation*}
The first part for $\varphi_{0}$ has very little $k$ dependence and the second terms give
\begin{equation*}
\psi_{k}=\varphi_{0\,k}+\nth{(2\pi)^{3/2}}\int{d\vr}(A\frac{e^{-\kappa{r}}}{r})e^{-i\vk\cdot\vr}=\varphi_{0\,k}-(\frac{2}{\pi})^{1/2}A\nth{k^{2}+\kappa^{2}}
\end{equation*}

Furthermore, if the bound-state is the one close to threshold, the most weight is outside $r_{c}$, we can neglect the first term and we have universal behavior at low-momentum while the normalization is determined in two-body problem.   Besides  bound-state, if interaction is weak and short-range, the low energy scattering state is well described by s-wave scattering state $\psi\propto1/r-1/a$ (Eq. \ref{eq:intro:Bethe}), and its Fourier transform in momentum space has the similar form $1/k^{2}$.  When considering many-body physics, in the low momentum below or around the scale of Fermi momentum,  wave function  is modified by many-body effect; but in the medium momentum, (still much smaller than $1/r_{c}$), this universal behavior is preserved.  The distribution of particle in such momentum, $k_{F}\ll{k}\ll{1/r_{c}}$, is $1/k^{4}$. This is actually the ``high-momentum'' (medium here) behavior ($C/k^{4}$) described in Tan's work about universality\cite{Tan2008-1,Tan2008-2}. 

On the other hand, at very high momentum ($k\gg1/r_{c}$), the second term in the above is very small.  This is because the smooth tail part of the wave function contributes little in high-oscillation.  The high-frequency Fourier component is solely determined by the wave function within the potential range.   This can be extend beyond two-body wave function to two-body correlation as long as the long-wave-length part is smooth.  In all cases, two-body, or many-body, very high-frequency of two-body correlation follows the two-body wave function.  





\section{$\lambda_{1}$ and $\lambda_{2}$\label{sec:pathInt2:lambda}}
\begin{equation}\tag{\ref{eq:pathInt2:lambda1}}
\lambda_{1}\equiv\avs{\phi}{(E_{\vk}-\xi_{\vk})}{\phi}-\avs{\phi}{v_{\vk}^{2}V}{\phi}
\end{equation}
Use relationship $v_{k}=\frac{E_{\vk}-\xi_{\vk}}{2E_{\vk}}$, we can rewrite the above equation into 
\begin{equation*}
\lambda_{1}=\avs{\phi}{v_{\vk}^{2}(2E_{\vk}-V)}{\phi}
\end{equation*}
$v_{\vk}$ is close to zero when momentum much higher Fermi momentum, on this region, $E_{\vk}$ is much smaller comparing to potential energy $V$.  So $V\ket{\phi}\approx-\eta\ket{\phi}$, and $\phi\approx\frac{A}{\kappa^{2}}$Therefore, we can estimate this term 
\begin{equation}
\lambda_{1}=\frac{A^{2}}{\kappa^{2}}\sum{}v_{\vk}^{2}=n\frac{A^{2}}{\kappa^{2}}\sim{}D_{1}^{2}/\eta
\end{equation}







\begin{equation}\tag{\ref{eq:pathInt2:lambda2}}
\lambda_{2}=(1+GT)\frac{Y\ket{\phi}\bra{\phi}{Y}}{\br{-E_{b}+\eta-2\mu-\lambda_{1}}}v_{\vk}^{2}\ket{{h_{1}}}
\end{equation}
Here the argument is more or less the same as in $\lambda_{1}$, considering the short-range nature of $Y$, and $v_{k}^{2}$ introduces an extra $nA^{2}/\kappa^{2}\sim{}D_{1}^{2}/\eta$ factor.  

We can see, both of them depends on many-body effect through density.  Particularly, they depend on density of open-channel component linearly at the lowest order.  $\lambda_{i}(n_o)=\lambda_{i}^{(0)}n_o$.  When not too close to resonance, $n_o$ is close to total density.  And $\lambda_{i}$ can be measured at different densities,  $\lambda_{i}^{(0)}$ estimated then accordingly. 

In the replacement of $1/(E_{1\vk}+E_{3\vk})$ by $\phi_{k}$ in Eq. \ref{eq:pathInt2:hphi}, certain error is introduced by directly replacement.  We expect the error is in higher order.  They might lead a non-linear relationship between $\lambda_{i}$ and density $n$.  But we expect the non-linearity is weak and we can still include the Pauli exclusion in these two parameters $\lambda_{i}(n)$.




\section{Calculate $\pi^{(0)}$ and $\pi^{\perp}$\label{sec:calculatePi}}
Here we will calculate $\pi^{(0)}$ and $\pi^{\perp}$ to the first order of $D_i/\eta$ using the expansion of Green's function in Sec. \ref{sec:diagonalGreen}.

\begin{equation}\label{eq:pathInt2:pi0}
\begin{split}
\pi^{(0)}(0)=&\sum_k\tr(\hat{G}_{0\,k}\sigma_3\hat{G}_{0\,k}\sigma_3)\\
	\approx&\sum_k\tr\big(T_{\vk}B_{k}^{-1}T_{\vk}^{\dg}\sigma_3T_{\vk}B_{k}^{-1}T_{\vk}^{\dg}\sigma_3\big)\\
	&\quad+\tr\Big(T_{\vk}\delta_{\vk}B_{k}^{-1}T_{\vk}^{\dg}\sigma_3T_{\vk}B_{k}^{-1}T_{\vk}^{\dg}\sigma_3
	-T_{\vk}B_{k}^{-1}\delta_{\vk}T_{\vk}^{\dg}\sigma_3T_{\vk}B_{k}^{-1}T_{\vk}^{\dg}\sigma_3\\
	&\qquad+T_{\vk}B_{k}^{-1}T_{\vk}^{\dg}\sigma_3T_{\vk}\delta_{\vk}B_{k}^{-1}T_{\vk}^{\dg}\sigma_3
	-T_{\vk}B_{k}^{-1}T_{\vk}^{\dg}\sigma_3T_{\vk}B_{k}^{-1}\delta_{\vk}T_{\vk}^{\dg}\sigma_3\Big)\\
	=&\sum_k\tr\big(M_{k}\big)+2\tr\Big(\delta_{\vk}M_{k}-\delta_{\vk}M_{k}\Big)
\end{split}
\end{equation}
where 
\begin{equation}
M_{k}=T_{\vk}^{\dg}\sigma_3T_{\vk}B_{k}^{-1}T_{\vk}^{\dg}\sigma_3T_{\vk}B_{k}^{-1}
\end{equation}
Here we use the cyclical  property of the trace $\tr(AB)=\tr(BA)$.  
It is straightforward to calculate
\begin{equation*}
T_{\vk}^{\dg}\sigma_3T_{\vk}B_{k}^{-1}=
\begin{pmatrix}
{\frac{\xi_{\vk}}{E_{\vk}(i\omega_{k}-E_{1\,\vk})}}&\frac{D_{1}}{E_{\vk}(i\omega_{k}+E_{2\,\vk})}&0\\
{\frac{D_{1}}{E_{\vk}(i\omega_{k}-E_{1\,\vk})}}&-\frac{\xi_{\vk}}{E_{\vk}(i\omega_{k}+E_{2\,\vk})}&0\\
0&0&-\nth{i\omega_{k}+E_{3\,\vk}}\\
\end{pmatrix}
\end{equation*}
Now it is easy to calculate the first term
\begin{equation}
\begin{split}
\sum_k\tr\big(M_{k}\big)=&
\sum_{k}\mbr{
\frac{2D_{1}^{2}}{E_{\vk}^{2}(i\omega_{k}-E_{1\,\vk})(i\omega_{k}+\xi_{2\,\vk})}+
\br{\frac{\xi_{\vk}^{2}}{E_{\vk}^{2}(i\omega_{k}-E_{1\,\vk})^{2}}+\frac{\xi_{\vk}^{2}}{E_{\vk}^{2}(i\omega_{k}+E_{1\,\vk})^{2}}
-\nth{(i\omega_{k}+E_{3\,\vk})^{2}}}
}
\end{split}
\end{equation}
Only root $-\xi_{2\,\vk}$ in the first term contributes in Matrubara frequency summation.
\begin{equation}\label{eq:pathInt2:pi0-1}
\sum_k\tr\big(M_{k}\big)=\sum_{\vk}\frac{2D_{1}^{2}}{E_{\vk}^{2}(E_{1\,\vk}+\xi_{2\,\vk})}
\approx\sum_{\vk}\frac{D_{1}^{2}}{E_{\vk}^{3}}-\sum_{\vk}\frac{D_{1}^{2}D_{2}^{2}\xi_{\vk}}{2E_{\vk}^{5}(\xi_{\vk}+\eta)}
\end{equation}

For the lowest order of the second term in Eq. \eqref{eq:pathInt2:pi0}, we only need to take the lowest order of $B_{k}$
\begin{equation}
B_{k}=\mtrx{i\omega_{k}-E_{\vk}&0&0\\0&i\omega_{k}+E_{\vk}&0\\0&0&i\omega_{k}+\xi_{\vk}+\eta}
\end{equation}
It is easy to verify at this approximation
\begin{equation}\label{eq:pathInt2:pi0-2}
\tr\Big(\delta_{\vk}M_{k}-\delta_{\vk}M_{k}\Big)=0
\end{equation}
Combine Eq. \eqref{eq:pathInt2:pi0-1} and Eq. \eqref{eq:pathInt2:pi0-2}, we have 
\begin{equation}
\pi^{(0)}(0)\approx\sum_{\vk}\frac{D_{1}^{2}}{E_{\vk}^{3}}-\sum_{\vk}\frac{D_{1}^{2}D_{2}^{2}\xi_{\vk}}{2E_{\vk}^{5}(\xi_{\vk}+\eta)}\end{equation}

and it is actually exact for $\pi^{\perp}(0)$
\begin{equation}
\begin{split}
\pi^{\perp}(0)=&\sum_k\tr(\hat{G}_{0\,k}\hat{G}_{0\,k})\\
	=&\sum_k\tr\big(T_{\vk}L_{\vk}B_{k}^{-1}L_{\vk}^{\dg}T_{\vk}^{\dg}T_{\vk}L_{\vk}B_{k}^{-1}L_{\vk}^{\dg}T_{\vk}^{\dg}\big)\\
	=&\sum_k\tr\big(B_{k}^{-1}B_{k}^{-1}\big)\\
	=&\sum_{\vk,i}(\sum_{\omega_{k}}(i\omega_{k}-\xi_{i})^{-2})\\
	=&0
\end{split}
\end{equation}


\section{Check for consistency of expansion\label{sec:pathApp:consistency}}
In our treatment here, one crucial assumption in expansion is the smallness of $D_{2}/\eta$.  Here we check it.  We have 
\begin{equation}
D_{2}=\sum{}Yh_{1\vk}+\sum{}Vh_{2\vk}\tag{\ref{eq:pathInt2:mfclose}}
\end{equation}
The first term on the right is relatively small comparing to the second term.  In an estimation we just keep the second term.  Furthermore,  we assume $h_{2\,\vk}=\sqrt{N_{c}}\phi_{0\,\vk}$, where $\phi_{0\,\vk}$ is the normalized wave function of isolated close-channel potential satisfying \sch equation
\begin{equation}\tag{\ref{eq:pathInt2:phi}}
-E_{b}^{(0)}\phi_{0\,\vp}=\epsilon_{\vp}\phi_{0\,\vp}-\sum_{\vk}V \phi_{0\,\vk}
\end{equation}
rearrange it we have (especially at low-momentum)
\begin{equation*}
\sum_{\vk}V \phi_{0\,\vk}=(\epsilon_{\vp}+E_{b})\phi_{0\,\vp}\approx{\eta}\phi_{0\,\vp}
\end{equation*}
The second approximation is correct at low-momentum (smaller or in the same order of Fermi momentum) as $\epsilon_{\vp}\ll{}E_{b}\approx\eta$.  Put all these together, we have
\begin{equation*}
D_{2}\approx\alpha{}E_{b}\phi
\end{equation*}
If we assume a simple exponentially decayed wave function
\begin{equation*}
\phi_{\vk}=\sqrt{\frac{8\pi\kappa}{V_{0}}}\frac{1}{k^{2}+\kappa^{2}}\approx\sqrt{\frac{8\pi\kappa}{V_{0}}}\frac{1}{\kappa^{2}}
\end{equation*}
Here  $V_{0}$ is the total volume.  The second approximation is for low-momentum as previous equation.  Collect all these together, we have
\begin{equation}
D_{2}\approx\sqrt{N_{c}}\eta\sqrt{\frac{8\pi\kappa}{V_{0}}}\frac{1}{\kappa^{2}}
\sim\eta\sqrt\frac{n_{c}}{\kappa^{3}}
\sim\eta\br{\frac{k_{Fc}}{\kappa}}^{\frac{3}{2}}
\sim\eta\br{\frac{E_{Fc}}{\eta}}^{\frac{3}{4}}
\end{equation}
%:
$k_{Fc}$ is the Fermi momentum corresponding to the particles in close-channel, which is much smaller than the characteristic momentum for bound-state, $\kappa$.   Therefore we have $D_{2}\ll\eta$, even when $n_{c}$ is close to total density $n$. 

Now we check the correction in Fermionic spectrum (Eq. \ref{eq:pathInt2:xiExpand3}-\ref{eq:pathInt2:xiExpand3}), 
is indeed small comparing to the main term.  
\begin{align}\tag{\ref{eq:pathInt2:xiExpand}}
E_{1\vk}&\approx{}E_{\vk}+\frac{D_{2}^{2}u_{\vk}^{2}}{\xi_{\vk}+\eta}&\equiv{}&E_{\vk}+\gamma_{1\vk}\\
\xi_{2\vk}&\approx{}E_{\vk}-\frac{D_{2}^{2}v_{\vk}^{2}}{\xi_{\vk}+\eta}&\equiv{}&E_{\vk}+\gamma_{2\vk}
\tag{\ref{eq:pathInt2:xiExpand2}}\\
E_{3\vk}&\approx{}\xi_{\vk}+\eta-\frac{D_{2}^{2}}{2(\xi_{\vk}+\eta)}&\equiv{}&\xi_{\vk}+\eta+\gamma_{3\vk}
\tag{\ref{eq:pathInt2:xiExpand3}}
\end{align}
Here, we mostly only concern of low-momentum ($k\sim{}k_{F}$).  In Eq. \ref{eq:pathInt2:xiExpand3}, 
\begin{equation*}
\frac{\gamma_{3\vk}}{E_{3\vk}}\sim{}\frac{D_{2}^{2}}{\eta^{2}}\sim\br{\frac{k_{Fc}}{\kappa}}^{2}\ll1
\end{equation*}
Eqs. \ref{eq:pathInt2:xiExpand} and Eqs. \ref{eq:pathInt2:xiExpand2} is slightly more complicated.  Both of them involve $\frac{D_{2}^{2}}{E_{\vk}\eta}$,  at very BCS side, close-channel density is small, $k_{F\,c}$ is small and that makes this ratio small; when close to (narrow) resonance, where $n_{c}$ is comparable to to total density, at low energy, $D_{1}$ is in the order of Fermi energy, so does $E_{\vk}$.   We have (we no longer distinguish $k_{F\,c}$ with $k_{F}$)
 \begin{equation*}
 \frac{\gamma_{i}}{\xi_{i}}<\frac{D_{2}^{2}}{E_{\vk}\eta}\sim\frac{\eta^{2}\frac{k_{Fc}^{3}}{\kappa^{3}}}{k_{F}^{2}\eta}\sim\frac{k_{F}}{\kappa}\ll1
\end{equation*}

More deeply, in the secular equation that leads to spectrum, Eq. \ref{eq:pahtApp:secular}.  We convert it back to normal scale without $E_{\vk}$,  (We drop subscript $\vk$ in the following equations for simplicity)
\begin{equation*}
(x^{2}-E^{2})(x-\xi-\eta)-D_{2}^{2}x+D_{2}^{2}E(u^{2}-v^{2})=0
\end{equation*}
It is not hard to use definition of $u$ and $v$ to find $u^{2}-v^{2}=\xi/E$, and express $E^{2}=\xi^{2}+D_{1}^{2}$, therefore we have
\begin{equation*}
(x-\xi)(x+\xi)(x-\xi-\eta)-D_{1}^{2}(x-\xi-\eta)-D_{2}^{2}(x-\xi)=0
\end{equation*}
Here the first term is for free particles, and let us estimate the size of the last two terms.  For low-momentum solution, we simply use $D_{1}\sim{}E_{F}$, we find
\begin{equation*}
\frac{D_{1}^{2}(x-\xi-\eta)}{D_{2}^{2}(x-\xi)}\sim\frac{E_{F}^{2}\eta}{D_{2}^{2}E_{F}}\sim\frac{\kappa}{k_{F}}\gg1
\end{equation*}
This justifies our choice to neglect the last term when finding the lowest-order solution and use the last term for correction.  

 



 In another word, the above estimation is saying that the total occupation number of close-channel at low-momentum is much smaller than 1 in all region of resonance (narrow or broad) because the close-channel bound-state is much smaller than inter-particle distance.  This factor gives us a small factor, $\frac{r_{c}}{a_{0}}\sim\frac{k_{F}}{\kappa}\sim\sqrt\frac{E_{F}}{\eta}$, upon which we can do the expansion.  
