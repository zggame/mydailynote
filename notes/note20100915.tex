\subsection{Single channel problem}
From \eef{eq:20100909:fullgap}, the single-channel equation is 
\begin{equation}
\frac{2F_{\vk}}{\sqrt{1-4 F_{\vk}^2}} \epsilon^{ab}_{\vk}+U_{\vk\vk'}F_{\vk'}=0
\end{equation}
The factor $\nth{\sqrt{1-4 F_{\vk}^2}}$ gives the many-body correction comparing to two-body \sch equation.  This factor is significant when $F_{\vk}$ is large.  In BCS, this is near Fermi surface;  in BEC, all $F_{\vk}$ is small, and therefore, the equation is closer to \sch  equation.  This is understandable because that there are much less overlapping for BEC, where molecule (overlapping) is small.  

In each equation of \eef{eq:20100909:fullgap},  the first term clearly involves the pauli exclusion of the channel itself as well as from the other channel;  the second terms seems to be in higher order due to statistics;  the third term is the normal inner-channel interaction; and the last term is the inter-channel interaction.  In the first term of the second equation,  $F_{\vk}$ should be larger than $G_{\vk}$ in low-k ($k_{F}$)and indicates that the inter-channel exclusion is more important than inner-channel effect.  This makes sense as open channel weight is larger in low k.  