\section{note of 2009.10.12}
The key problem of the momentum space is that the condition that many-body w.f is like the two-body one in short-distance is not readily expressed in momentum space. 

In another word, we can ask the question from the other side, what happens for those system that the two-body physics is clear to us when we pack them more and more densely.  

\subsection{Tony's Comment}
\begin{itemize}
\item Large close-channel bound state.  What I am trying to look at is the close-channel bound state affected by Pauli principle.  But the open-channel is not affected at least within the potential range.  If the close-channel bound state is small and comparable to the potential range, we cannot safely assume the open-channel part within this range is not affected as well.   So there is the Pauli-principle for $r>a_c>>r_0$, no Pauli-principle consideration for short range $r\sim{r_0}$.  \\ Beside, the problem is simplified by the fact no need to think about the oscillation for the bound-state within the potential range. 
\item another correction (not the one I am most interested) is the fact the non-zero close-channel state deplete the open channel and affect other stuff.  This is simply the result of number equation. 
\item far away from the resonance, the close-channel can be treated perturbatively. 
\end{itemize}
