\subsection{}
What are we exactly looking for in the problem? Things like how the chemical potential changes, how $\Delta$ changes due to the narrowness nature.  On the other hand, it should be governed by the density of the state.  Still, the 0-density should correspond to the two-body solution and the higher the density, the higher Fermi energy and the narrower.  I should have an extra coordinate for most pictures, the density.  

\subsection{Green function for the close-channel}
Looked from the Lehmann representation's definition,  the denominator of the Green's function has chemical potential $\mu$. Maybe the Zeeman energy should included in the denominator of the close-channel if take the open-channel as the ground state $\Psi_{0}$.  

\subsection{}
How to describe the fact that the kernel of the state in real space (within the potential range) 

\subsection{}
Let us think about hydrogen atoms, one proton plus one electron.  In the momentum space, an individual proton or electron has momentum $\vk$ in the central of mass framework plus momentum of the central of mass  $\vK$.  The real equilibrium at $T=0K$ is the solid.  But at temperature of  gas phase, the phase space is much larger and momentum extends to much higher region, little or no overlap in momentum space fore two electrons (protons).  In another word, they are in the opposite of the degenerate region we often think about the fermi gas.  Within the atom size, (real space), a proton has very high probability to have an electron around as the molecule wave function.  $\av{n_{e}(x)n_{p}(x+y)}$ decays quickly with increasing $y$.  Similarly for the electron.  This is the meaning of molecule gas in real-space.  How does this translate into momentum space? \be