\section{note of 2009.09.29}
\subsection{}
If the abnormal Greens function, or simply the order parameter $\psi\psi$ is literally taken as the macroscopic eigenstate of the two-body density matrix, (\cite{Leggett}), then its module should not depend on the time as the quantities relating to the two-body density matrix.  It probably can allow some imaginary phase, which should cancel out each other.   The time-derivative of that gives a semi-energy quantity.  

How to bind the short-range part $1-r/a_s$ with the long-range part (many-body)?

\subsection{}
How does the two-body scattering solution $\chi=1-r/a_s$ normalize ($\chi=r\psi_r$)? 

It is normalized against constant flowing of the particles in the scattering problem, not the usual integrating to 1 over the space.  But in our case, as it does not extent to $\infty$, there is no such normalization problem.  Its normalization is determined with the many-body part in the long-range. 

This is just the solution for a Shr\"{o}‌dinger equation with no external potential.  
\[-\hm\dpdiff{\chi}{r}=0\]
It is OK for the $\chi\neq0$ when $r\rightarrow0$ as this is only for the long-range (long compare to potential range, but short compare to the many-body particle distance) behaviour where real potential is negligible.  
