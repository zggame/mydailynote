\section{Conclusion}
We have rederived the Richardson equations using a commutation technique for free electron pairs with zero total momentum similar to the one we have developed for composite boson excitons.  Almost half a century ago, Richardson has shown that the \emph{exact} wave-function and energy for an arbitrary number $N$ of pairs can be written in a compact form in terms of $N$ energy-like quantities $R_1,..., R_N$, which are solution of $N$ coupled non-linear equations.  This $2N$ many-body problem is exactly solvable provided that the interaction potential is taken as a BCS-like $\abs{x}$ potential having a separable scattering $v_{\vk'\vk}=-V\,w_{\vk'}w_{\vk}$ with $w_{\vk}$ ?????? such that $w_\vk^2=1$.  Note that these assumptions are already those necessary to get the energy of a single pair in the compact form obtained by Cooper.  Richardson manged to extend this exact solution to $N$ pairs by decoupling them through rewriting their energy $E_N$ as $R_1+\cdots+R_N$.

The new composite boson derivation we have proposed allows to trace back the physical origin of the various terms of these equations.  It in particular clearly shows that $N$ pairs differ from $N$ independent single pairs, due to Pauli exclusion principle only.  This Pauli blocking also enforces the $R_i$ energy-like parameters to be different, namely the exact $N$-pair eigenstate different from the BCS ansatz.  ?????? the diagrammatic representation of this derivation evidences that, due to the fact that pairs with zero total momentum, do have one degree of freedom only, they only have $2\times2$ scatterings within the $1\times1$ BCS potential. This explains why the $N$ pair energy has terms in $N$ and $N(N-1)$ but not in $N(N-1)(N-2)$ and so on.

One of us (M.C.) wishes to thank the Institute of Condensed Matter Physics of the University of Illinois, Urbana-Champaign, and Tony Leggett in particular, for her one-month invitation.  
