\subsection{One more exogenous parameter}
We can introduce one more exogenous parameter $\Delta_{0}=\Delta|_\text{very BEC end}$, then we are good to go numerically.  However, it is not clear how to pin $\Delta_{0}$ down.  It depends on the strength of inter-channel coupling $Y_{\vk\vk'}$ as well as the nature of the close-channel bound-state.  It seems that we should be able to extract it from two-body quantities.  This seems make sense, when the coupling is small, $\Delta_{0}$ is always small and there is little effect of BCS, it is almost like the extreme narrow case, simply by filling Fermi sea according to energy; when the coupling is strong, we can obtain a large $\Delta_{0}$ and lots of interesting crossover physics.  

\subsection{Tony's comment}
Tony comment on the relation of $\Delta$ and $\alpha$.  If we use the \eef{eq:20100915:GF} and integrate it over $\phi^{0}_{\vk}$, we get $\alpha$.  This is very similar to (4.A.29) in \cite{Leggett} except the normalization of $\phi^{0}$ and $F_{\vk}$.  $\phi^{0}$ is normalized to $N$, which is easy.  The difficulty is $F_{\vk}$.  In \cite{Leggett}, open-channel wave function $\chi(r)$ is normalized to $1-r/a_{s}$, so $\chi(0)\rightarrow1$ to fix the short-range normalization.  \cite{shizhongSumRule} derived  (eq. (25))
\begin{equation}\label{eq:20101025:fr}
F(\vr)=\frac{m\Delta}{4\pi\hbar^{2}}\frac{1-r/a_{s}}{r}
\end{equation}
we have the pre-factor $\frac{m\Delta}{4\pi\hbar^{2}}$ for the normalization and with \cite{Leggett} (Eq. (4.A.17)), we can express the relation between $\Delta$ and $\alpha$.  (asuume $\kappa$ varies slowly with detuning)hj