\section{note of 2009.10.01}
\subsection{}
A simple decay of $e^{-r/a}$'s Fourier transformation (in 3D) is $\frac{8\pi{} a^3 }{\left(1+a^2 k^2\right)^2}$.  
\footnote{
\begin{equation}\label{eq:3dFourier}
\iiint{f(r)e^{i\vk\cdot\vr}d^3\vr}=\iiint{f(r)e^{i\vk\cdot\vr}\sin\theta{r^2}drd\phi{d}\theta}=\frac{4\pi}{k}\int_0^{\infty}dr\;r\sin{kr}f(r)
\end{equation} 
}%end of footnote
But this may not be the case in the problem.  The short-range part, ($r$ within potential range $r_0$), does not affect the high-$k$, as $kr\ll1$ and it onlys contribute the same constant for Fourier component unless for the really large $k\sim{1/r_0}$ .  But the really-long-range part, ($r>1/k_F$), many-body effects set in and it may not be the decay as such, and it might set a new scale.  But due to the fast oscilation, it might be small.

Length scale in the problem
\begin{itemize}
	\item $r_0$, potential range, it does not change, and usually is the shortest length-scale. 
	\item $a_c$, the close-channel bound-state size, it is also small.  It is rigid with fix size when not considering the channel-coupling.  It is probably in the same order as $r_0$
	\item $a_s$, open-channel s-wave scattering length. It varies from $-\infty$ to $+\infty$.
	\item $1/k_F$, the average atome distance, very large for the dilute system.  
\end{itemize}
\subsection{Tony's comment}
\begin{itemize}
\item Tony agrees the problem to bridge between the momentum space approach in many-body and the real space approach in two-body.  He feels the integration is probably in the momentum space.  He suggest to think about how the two-body in momentum space.  
\item He also suggest to think about the problem that how w.f changes when we ramp up density of diatomic molucules.  This is more like the single-channel problem, but just on one side (BEC side).  But we know that it might cross over the $\mu=0$ point.  
\item He suggest to take the close-channel molecule also loosely bound, so $1/k_F\gg{a}\gg{r_0}$.  He feels that may simplify things in momentum space. 
\item To take the narrow resonance, but far away from the resonance, maybe we can apply the perturbation somehow.  
\end{itemize}

