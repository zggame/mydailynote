\subsection{BCS}
\subsubsection{Old BCS Asmpototic Solution}
\begin{itemize}
 \item \cite{BogoliubovColl}: First one get Anderson-Bogoliubov mode in BCS.  (Eq. (4.16))  It also states that non-zero central momentum components are necessary for collective mode(sec. 4.2).  It linearized the Fr\"{o}hlich Hamiltonian with non-zero expectation of the normal expectation as well as abnormal ones (keep the non-zero central momentum quantities). 
  \item \cite{AndersonBCS}: Get bosonic mode, Anderson-Bogoliubov mode, in BCS with  equation of motion method.  One has to introduce none-zero central momentum interaction to reduced BCS hamiltonian to get it.  The linearization is specific for BCS ansatz, i.e., non-zero mean-field value for anonymous pair expectation. In the final result for excitation, the excitation also expanded around only zero-central-momentum and others are approximated by zero-central-momentum quantities, i.e., BCS ground state value. 
 
	\item \cite{BcsExact}: with useful citations to others.         
 \end{itemize}

\subsection{Feshbach Resonance}
\subsubsection{Basics}
\subsubsection{Narrow}
\begin{itemize}
  \item \cite{JacksonNarrow}:Nice study with two-channel model and how to renomalize it. 
	
 \end{itemize}
