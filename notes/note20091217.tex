\subsection{Tony's Comment}
Tony is pleased that I can express the thing in (2-channel) T matrix (although I am still not very clear how exactly it relates to the $a_s$ in both channels, $a_s^{background}$ or $a_s^{effective}$? I think it should be the background one. )    He asked me to think it over.  He suggested me to think about how to get the limit where the Pauli exclusion between channels are not important.  

He talked about another thing that in conventional electron-BCS, we normally have $F(r)$ falls off exponentially at coherence length in large scale while oscillate like two-body free electrons in short-range.  Yet this must be wrong in the very short-range where Coulomb repulsion kicks in.  It will be interesting that we can find some phenomenon that manifest this effect.  He mentioned something like anapole moments(in high energy related to two opposite-spin fermions)? 


He suspected that the fact that I cannot related the correction of binding energy at BEC limit in the semi-Cooper problem (\ref{eq:BecShiftE}) relates to the fact that BCS fall off at large k.  (I just realize that the open-channel molecule is much larger than close-channel one and therefore it is incorrect to assume that $(0,  \bar{k})$ has uniform deduction of $n_s$.  It is instead only deduction of k-space at $(0,\nth{a_s})$.  And it is probably not uniform either. See sec \ref{subsec:additionalSemiCooper}).
