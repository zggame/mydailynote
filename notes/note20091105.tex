\subsection{relation between one-molecule to many-body}
I gave some thought to the relation between one-molecule solution I am looking for in section \ref{subsec:oneMolInFermiSea} and the true many-body solution.  As the bound-state there is much more spread in k-space and the many-body is expected to be much smaller.  Actually, it seems to fit the dilute-limit all the time as the \emph{"moth-eaten"} effect in coboson approach \cite{combescotBCS}.  The effect should be similar to the many-body effect in the Deuterium-molecule gas.   So I should get a very useful and close-to-end result if I can crack the one-molecule problem I had.  

\subsection{comment from Monique Combescot\label{sec:MCombescot}}

She suggest me to put the bound-state nature of close-channel by a ansatz
\begin{equation}
 \ket{\Psi}=\prod_\vk\br{u_\vk+v_\vk{a^+_\vk}b^+_{-\vk}+w_\vk{B^+}}\ket{v}
\end{equation}
where $B^+\ket{0}$ is the two-body solution for the decoupled close-channel hamiltonian.
\begin{equation}
 B^+\equiv\sum_\vp\psi_\vp{a^+_\vk}c^+_{-\vk}
\end{equation}
\begin{equation}
 (H^c-E^c)B^+\ket0=0
\end{equation}
where ($a$,$b$) is the open-channel and ($a$,$c$) is the close-channel.  

Tony raised some concerning on the normalization of it but then agree that should take into account at least some of the Pauli exclusion between the channel and should serve as the first approximation.  
