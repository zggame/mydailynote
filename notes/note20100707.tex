\subsection{Varying parameters}
In order to see the difference between the narrow and broad resonance,  we can increase the density of the gas.  So the two-body case (zero-density) would be the extreme of broad-resonance, then gets into narrow as density increases and Fermi energy increases.  

\subsection{Renormalized equation}
Come back to \ref{eq:renormal2}, it probably can be reduced to 1-d if $F,G$ ($\sin2\theta_{\vk})\cos\phi_{\vk}$,$\sin2\theta_{\vk})\sin\phi_{\vk}$ ) are linear combination of $\widetilde\Delta^{U,V}$.  This is not generally true as we know from single-channel BCS.  However,   in single-channel BEC side,   $\abs{\mu}\gg\Delta$ and $F$ can be approximated as linear combination of $\Delta$.  
\begin{equation}\tag{\ref{eq:renormal2}}
\begin{pmatrix}\widetilde\Delta^U\\\widetilde\Delta^V\end{pmatrix}=
8\pi({a}{k_F})\begin{pmatrix}1&T_{oc}/T_{oo}\\T_{co}/T_{oo}&T_{cc}/T_{oo}\end{pmatrix}
\int\widetilde{k}^2d\widetilde{k}
\begin{pmatrix}
\nth{2}\sin2\theta_{k}\cos\phi_k-\frac{\widetilde\Delta^U}{2\widetilde\epsilon_k}\\
\nth{2}\sin2\theta_{k}\sin\phi_k-\frac{\widetilde\Delta^V}{2\widetilde\epsilon_k+\widetilde{\eta}}
\end{pmatrix}
\end{equation}

Another note about this equation might be that the matrix $\begin{pmatrix}1&T_{oc}/T_{oo}\\T_{co}/T_{oo}&T_{cc}/T_{oo}\end{pmatrix}$ does not have singularity as the $(ak_{F})$.  This matrix is always finite and $T_{cc}/T_{oo}$ is large, but finite and this large number characterize the Feshbach resonance in some sense.  
