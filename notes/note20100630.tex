\subsection{Tony's comment}
Tony suggested to me that if I cannot get a more quantitative result, I should look in the trend such as how the chemical potential or gap change at universality.  

\subsection{A point in renormalization}
A point unclear for all the time is whether  in \eqref{eq:renomal1}, there should be $\eta$ or not.  If there is, what $\eta$ should be, the raw detuning between two channels, the detuning for the close-channel bound level, the detuning from the resonance or 0 (no-detuning)?  

First of all, this does not matter at the high energy end, so either choice, the equation is well normalized.  However, the T matrix changes accordingly.  

\subsection{Simplification over BEC molecule}
Considering that close-channel bound state is much smaller than the  BEC molecule mostly in open-channel when not far away the universality, it seems justified to take the $\phi_\vk$ in (\ref{eq:tan1},\ref{eq:tan2}) as slow varying in the region from 0 to not too far beyond $k_F$.    

This is contradictory to \eqref{eq:tan2}.  Actually it is not simply $\phi_k$ as constant, there is something constant, but not simply $\phi_k$. $w_k$, $u_kw_k$ or $w_k/u_k$?