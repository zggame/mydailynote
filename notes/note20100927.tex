\subsection{}
There are two possibilities for narrow resonance:  
\begin{enumerate}
\item Pauli exclusion for three species, where the close-channel weight is large.  
\item Fermi energy is larger than the width that scattering amplitude (NOT $a_{s}$) changes the sign.  Therefore $a_{s}$ is no longer a good indicator for the interaction.  However, it might still be OK to take $a_{s}$ simply as the quantity introduced by renormalization.   This has been explored by several papers already \cite{NarrowJensen1,NarrowJensen,GurarieNarrow}.  
\end{enumerate}
Four species case can fall into narrow resonance in  the second sense.  \cite{NarrowJensen1,NarrowJensen} uses the single-channel approach but a T-matrix with effective range $r_{0}$, the narrow resonance is $r_{0}k_{F}\ll1$.  In \cite{NarrowJensen}, a single-channel bare interaction is assumed and (unnormalized) gap, number equations  are  solved numerically with different interaction parameters corresponding to different effective ranges. The result is some small corrections in gap, chemical potential...   It is not a two-channel treatment.  

\cite{GurarieNarrow} treats the problem in two-channel (molecule+fermi).  The criteria for narrowness is similar.  There are couple of points not clear to me.   
\begin{itemize}
\item Two channels are treated separately, more in He3-A fashion.  The ansatz I used is more in He3-B fashion where two channels are mixed at the very beginning.   According to \cite{He3B}, in He3 He3-B wave-function has lower energy.  What is the case here?
\item The hamilton has no direct interaction in open-channel, which contributes little in resonance.  But it does play some role.  How does that work for my model?
\end{itemize}
They use path integral, which I do not quite understand.  But I feel I do need to crack it.  