\subsection{}
I have done a very crude estimation on Pauli exclusion on two-channels.  

\begin{itemize}
	\item Assume we have a simple BEC of close-channel molecules as the solution in isolation, and open-channel as simple free fermions.  Open-channel fermions have to occupy some levels over original $E_F$ because close-channel molecules occupy some within old Fermi-sea. This cost some energy. 
	\item Assume opne-channel is simple free fermions and simply occupy states up to $E_F$. Close-channel has to form molecules from $E_F$, which costs a bit more energy and reduce the binding energy.  
\end{itemize}
In both 2D and 3D, the previous case costs less energy then the later one.  In general, the later is in order of $E_b/E_F$ ($E_b$ is the bound energy of close-channel molecules and much larger than $E_F$, the open-channel fermi energy).  This supports my previous treatment.  My treatment is mostly according to the senario of the first one, which costs less energy. 