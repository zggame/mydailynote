\section{note of 2009.09.23}
\subsection{}
By retrospection, this dividing over long-range and short-range is not necessary to be limited for Shizhong's equal-time equation. Maybe it can also be applied to the original AGD approach.  however, there is no two-operator terms there.  	

\subsection{}
Back to simple BCS, $F$, $G$ takes simple forms for the contact interaction.  How do they varies as the non-contact feature of potential is turned on?  The guess is that the short-range behaviour is changed to be more like real two-body bound state while long-range behaviour does not change much.  And how does other properties change?  For example the energy gap?  

In AGD of single channel simple BCS, $i\left.\pdiff{F_k}{t}\right\vert_{t=0}$ is equal to $E_kF_k\vert_{t=0}$. (Note that $F$ is not eigenstate of the hamiltonian and it happens to be $E_k$ only at $t=0$.)

In the original contact potential, or maybe more realistic short-range potential, is it possible to write down equations for short and long-range and in short range use $a_s$, then derive the relation between $a_s$ and potential strength and the many-body quantities, such as gap.  Is it possible to derive everything in real space?  That would be good parallel to Feshbach case with two channel, whose two-body physics is more easily described in real-space.  

If equations \eqref{eq:short} \eqref{eq:long} are like the single-channel case in AGD, the time derivative is not zero and not $\mu$ either.  So the equation \eqref{eq:short} is not exactly for 0 energy where $a_s$ is defined. The time-derivative is probably in the order of the Fermi energy or chemical potential, (not sure whether they are the same in our case, a problem to be determined).   Actually in the narrow resonance, the order parameter can be significant different from that of 0 energy solution where two-body solution is derived.  While in the broad resonance,  the many body energy is smaller than this level difference, and maybe indeed can take $\omega\approx0$. Another possible complication is what should be compared with two-body energy is probably the difference of chemical potential $\mu$ and the time derivative $\omega$, which might be small in both cases.  
