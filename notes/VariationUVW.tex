\subsection{Variation method with the u,v,w\label{sec:variation}}
We start with three hyperfine species a, b, c, where a is the common species in two channel. (a,b) is the open channel while (a,c) is the close channel. And the Hamiltonian is written in the form 
\begin{equation}\label{eq:uvw:hamiltonian}
\begin{split}
 H=&\sum_\vk\epsilon^a_\vk{}a^+_\vk{}a^{}_\vk+\sum_\vk\epsilon^b_\vk{}b^+_\vk{}b^{}_\vk+\sum_\vk\epsilon^c_\vk{}c^+_\vk{}c^{}_\vk\\
  &+\nth{2}\sum_{\vk\vk'}U_{\vk\vk'}a^+_\vk{}b^+_{-\vk}{}b^{}_{-\vk'}a^{}_{\vk'}
	+\nth{2}\sum_{\vk\vk'}V_{\vk\vk'}a^+_\vk{}c^+_{-\vk}{}c^{}_{-\vk'}a^{}_{\vk'}\\
 &+\nth{2}\sum_{\vk\vk'}Y_{\vk\vk'}a^+_\vk{}b^+_{-\vk}{}c^{}_{-\vk'}a^{}_{\vk'}
	+\nth{2}\sum_{\vk\vk'}Y^*_{\vk\vk'}a^+_{\vk'}{}c^+_{-\vk'}{}b^{}_{-\vk}a^{}_{\vk}
\end{split} 
\end{equation}
By the Hermition condition we have 
\begin{equation}
 U_{\vk'\vk}=U^*_{\vk\vk'},\qquad{} V_{\vk'\vk}=V^*_{\vk\vk'}
\end{equation}
  We start from the ansatz as 
\begin{equation}\label{eq:ansatz}
 \ket{\Psi}=\prod_\vk\br{u_\vk+v_\vk{}a^\dg_\vk{}b^\dg_{-\vk}+w_\vk{}a^\dg_\vk{}c^\dg_{-\vk}}\ket{0}
\end{equation}
Here we require $\abs{u_\vk}^2+\abs{v_\vk}^2+\abs{w_\vk}^2=1$ for normalization.  For all the interaction term, there are two types of contribution,
for example, 
\begin{equation*}
\av{U_{\vk\vk'}a^\dg_\vk{}b^\dg_{-\vk}{}b^{}_{-\vk'}a^{}_{\vk'}}
=\sum_{\vk}U_{\vk\vk}\abs{v_\vk}^2+\sum_{\vk\neq\vk'}U_{\vk\vk'}v^{}_{\vk'}u^*_{\vk'}u^{}_\vk{}v^*_\vk
\end{equation*}
The first term is the Hatree term and the second term is more interesting pairing term.



And the free energy is 
\begin{equation}
 \begin{split}
  &F\equiv\av{H-\mu{}N}\\
    =&\sum(\xi^a_\vk+\xi^b_\vk)\abs{v_\vk}^2+\sum(\xi^a_\vk+\xi^c_\vk)\abs{w_\vk}^2\\
    &+\nth2\sum_{\vk}U_{\vk\vk}\abs{v_\vk}^2+\nth2\sum_{\vk\neq\vk'}U_{\vk\vk'}v^{}_{\vk'}u^*_{\vk'}u^{}_\vk{}v^*_\vk\\
    &+\nth2\sum_{\vk}V_{\vk\vk}\abs{w_\vk}^2
      +\nth2\sum_{\vk\neq\vk'}V_{\vk\vk'}w^{}_{\vk'}u^*_{\vk'}u^{}_\vk{}w^*_\vk\\
    &+\nth2\sum_{\vk}Y_{\vk\vk}w^{}_{\vk}v^*_\vk{}
      +\nth2\sum_{\vk\neq\vk'}Y_{\vk\vk'}w^{}_{\vk'}{u^{*}_{\vk'}}v^*_\vk{}u^{}_\vk\\
    &+\nth2\sum_{\vk}Y^*_{\vk\vk}w^*_{\vk}v^{}_{\vk}{}
      +\nth2\sum_{\vk\neq\vk'}Y^*_{\vk\vk'}w^*_{\vk}{u^{}_{\vk}}v^{}_{\vk'}{}u^{*}_{\vk'}
 \end{split}
\end{equation}
Where 
\begin{equation*}
 \xi^a_\vk=\epsilon^a_\vk-\mu^a,\qquad\xi^b_\vk=\epsilon^b_\vk-\mu^b,\qquad\xi^c_\vk=\epsilon^c_\vk-\mu^b
\end{equation*}
The chemical potential is added to make sure the $n_a=n_b+n_c=\nth{2}n$.
 I drop the Hatree term as this in some sense just shift the chemical potentials as it only relates to the density.  (\emph{Not sure still valid in the two-channel problems, especially for the close-channel.})  Also, I ignore the summation on the second only goes through $\vk\neq\vk'$ as the correction is in the higher order. 
 
Now introduce the parameters $\theta_\vk$, $\phi_\vk$ to include the normalization condition.  
\begin{equation}
 u_{\vk}=\cos\theta_\vk,\qquad{} v_{\vk}=\sin\theta_\vk\cos\phi_\vk,\qquad{}
 w_{\vk}=\sin\theta_\vk\sin\phi_\vk
\end{equation}
We take them as real quantities as those that minimize the free energy are real.(?) Now the free energy can be written as 

\begin{equation}
 \begin{split}
  F=&\sum\xi^{ab}_\vk\sin^2{\theta_\vk}\cos^2\phi_\vk+\sum\xi^{ac}_\vk\sin^2{\theta_\vk}\sin^2\phi_vk\\
    &+\nth2\sum_{\vk\vk'}U_{\vk\vk'}\cos\theta_{\vk'}\sin\theta_{\vk'}\cos\phi_{\vk'}\cos\theta_\vk{}\sin\theta_\vk\cos\phi_\vk\\
    &+\nth2\sum_{\vk\vk'}V_{\vk\vk'}\cos\theta_{\vk'}\sin\theta_{\vk'}\sin\phi_{\vk'}\cos\theta_\vk{}\sin\theta_\vk\sin\phi_\vk\\
    &+\nth2\sum_{\vk\vk'}Y_{\vk\vk'}\cos\theta_{\vk'}\sin\theta_{\vk'}\cos\phi_{\vk'}\cos\theta_\vk{}\sin\theta_\vk\sin\phi_\vk\\
    &+\nth2\sum_{\vk\vk'}Y^*_{\vk\vk'}\cos\theta_{\vk'}\sin\theta_{\vk'}\sin\phi_{\vk'}\cos\theta_\vk{}\sin\theta_\vk\cos\phi_\vk\\
    =&\nth4\sum_\vk\xi^{ab}_\vk(1-\cos2\theta_\vk)(1+\cos2\phi_\vk)+\nth4\sum_\vk\xi^{ac}_\vk(1-\cos2\theta_\vk)(1-\cos2\phi_\vk)\\
    &+\nth{8}\sum_{\vk\vk'}U\sin2\theta_\vk\cos\phi_\vk\sin2\theta_{\vk'}\cos\phi_{\vk'}+\nth{8}\sum_{\vk\vk'}V\sin2\theta_\vk\sin\phi_\vk\sin2\theta_{\vk'}\sin\phi_{\vk'}    \\
    &+\nth{4}\sum_{\vk\vk'}Y\sin2\theta_\vk\cos\phi_\vk\sin2\theta_{\vk'}\sin\phi_{\vk'}    
 \end{split}
\end{equation}
We assume that $Y=Y^*$. From the above equation, we can differentiate it with respect to $\theta_\vk$ and $\phi_\vk$ and set the derivative as 0, therefore minimize free energy. 
\begin{align}
0=&\pdiff{F}{\theta_\vk}\notag\\
 =&\nth{2}\sin2\theta_\vk\mbr{\xi^{ab}_\vk(1+\cos2\phi_\vk)+\xi^{ac}_\vk(1-\cos2\phi_\vk)}\notag\\
 &+\nth2\sum_{\vk'}\cos2\theta_\vk\sin2\theta_{\vk'}\mbr{U\cos\phi_\vk\cos\phi_{\vk'}+V\sin\phi_\vk\sin\phi_{\vk'}+Y\sin(\phi_{\vk'}+\phi_\vk)}\\
 0=&\pdiff{F}{\phi_\vk}\notag\\
 =&-\nth{2}(\xi^{ab}_\vk-\xi^{ac}_\vk)\sin2\phi_\vk(1-\cos2\theta_\vk)\notag\\
 &-\nth4\sum_{\vk'}\sin2\theta_\vk\sin2\theta_{\vk'}\mbr{U\sin\phi_\vk\cos\phi_{\vk'}-V\cos\phi_\vk\sin\phi_{\vk'}-Y\cos(\phi_{\vk'}+\phi_\vk)}
\end{align}
  These two set of equations (for each $\vk$, but decoupled) fully determine the wave-function. 
We introduce two quantities:
\begin{align}
\Delta_\vk^U&=\sum_{\vk'}\sin2\theta_{\vk'}(U_{\vk\vk'}\cos\phi_{\vk'}+Y_{\vk\vk'}\sin\phi_{\vk'})\label{eq:gap1}\\
\Delta_\vk^V&=\sum_{\vk'}\sin2\theta_{\vk'}(V_{\vk\vk'}\sin\phi_{\vk'}+Y_{\vk\vk'}\cos\phi_{\vk'})\label{eq:gap2}
\end{align} 
As we can see eqs.(\ref{eq:gap1},\ref{eq:gap2}) is indeed very similar to the structure of two-body Schr\"{o}‌dinger equation.
If we ‌introduce the Zeeman energy detuning between two channels
\begin{equation}
\eta=\xi^{ab}-\xi^{ac}
\end{equation}
These equations can be written into a more compact form
\begin{align}
\tan2\theta_\vk&=-\frac{\cos\phi_\vk\Delta^U_\vk+\sin\phi_\vk\Delta^V_\vk}{2\xi^{ab}_\vk+2\eta\cos^2\phi_\vk}\label{eq:tan1}\\
\tan\theta_\vk&=-\frac{\sin\phi_\vk\Delta^U_\vk-\cos\phi_\vk\Delta^V_\vk}{2\eta\sin2\phi_\vk}\label{eq:tan2}
\end{align} 


Furthermore, when $k\rightarrow\infty$, from eq. (\ref{eq:tan1}), we see that $\theta\rightarrow0$ as $1/\xi$;  and from eq. (\ref{eq:tan2}), we see that $\sin\phi_\vk\Delta^U=\cos\phi_\vk\Delta^V$ as we assume $\Delta$ varies slowly over $\vk$.  We have 
\begin{align*}
\cos^2\phi_\vk&=\frac{{\Delta^U}^2}{{\Delta^V}^2+{\Delta^U}^2}\\
\sin^2\phi_\vk&=\frac{{\Delta^V}^2}{{\Delta^V}^2+{\Delta^U}^2}
\end{align*}
And 
\[\tan2\theta_\vk=-\frac{({{\Delta^V}^2+{\Delta^U}^2})^{1/2}}{2\xi^{ab}_\vk}\]

Eqs. (\ref{eq:gap1},\ref{eq:gap2}) are similar as those derived from the Green's function method in the eq. (\ref{eq:gapMatrix}), and should be able to renormalized in the similar fashion $\Delta=T(F-G\Delta)$, where in our notation $({F^o,F^c})=(\cos\theta_\vk\sin\theta_\vk\cos\phi_\vk,\cos\theta_\vk\sin\theta_\vk\sin\phi_\vk)$.  The problem, however, is that the eqs. (\ref{eq:tan1},\ref{eq:tan2}) does not present a simple analytic solution about $\theta$ and $\phi$ in terms of $\Delta$ and therefore the gap equations cannot be written into a simple renormalized  equation although they are implicit functions of $\Delta$.
 It is a rather cumbersome process to calculate the integration in renormalized gap equations.  Furthermore, information of the close-channel is close to a simple one-level bound-state is not fully incorporated and the equation might be simpler and has a nicer form if I can manage to do that.  

\subsection{more thoughts}
From eqs. (\ref{eq:gap1},\ref{eq:gap2}), $\Delta^{U,V}_\vk$ varies slowly with $k$ if interaction depends on momentum weakly, at least for low momentum.  Therefore, in the renomalization, they can be taken as constants.  In that case, eqs. (\ref{eq:tan1},\ref{eq:tan2}) determine $\theta_k(\xi_k,\Delta^U,\Delta^V)$ and $\phi_k(\xi_k,\Delta^U,\Delta^V)$, which in turn give two-body wave-function $F^{o,c}_k(\xi_k,\Delta^U,\Delta^V)$.  
With the renomalized gap equation,.  
\begin{equation}\label{eq:renomal1}
\begin{pmatrix}\Delta^U\\\Delta^V\end{pmatrix}=\begin{pmatrix}T_{oo}&T_{oc}\\T_{co}&T_{cc}\end{pmatrix}
\begin{pmatrix}\sum{F^o-\frac{\Delta^U}{2\epsilon}}\\\sum{F^c-\frac{\Delta^V}{2\epsilon+\eta}}\end{pmatrix}
\end{equation}
and the number equation, should give us three parameters in the problems: $\Delta^{U,V}$ and $\mu$.  The close-channel infomation is incorporated into the $T$-matrix.  The problem is more about the computation difficulty as there is no simple analytic solution for $F^{o,c}$, therefore, eq. (\ref{eq:renomal1}) cannot be done easily.

What happened in the broad-resonance limit? Or very BEC/BCS end? 

\subsection{T-matrix}
Follow \cite{JacksonNarrow}, the bare interaction $V$ should have the form \footnote{notice the some change of notation}
\begin{equation}
V(r)=\frac{V_s(r)+3V_t(r)}{4}+\mbr{V_t(r)-V_s(r)}\mathbf{S_1}\cdot\mathbf{S_2}
\end{equation}
we can establish the T-matrix from $V$.
\begin{equation}
 {\fmtrx{T_{cc}}{T_{co}}{T_{oc}}{T_{oo}}}^{-1}={\fmtrx{V_{cc}}{V_{co}}{V_{oc}}{V_{oo}}}^{-1}-{\fmtrx{G_{c}}{0}{0}{G_{o}}}
\end{equation}
For 0-temperature, pair propagators $G_o$ and $G_c$ are
\begin{equation}
 G_o(\omega,\vK,\vq)=\nth{\omega+i\delta-\frac{K^2}{4m}-\frac{q^2}{m}}
\end{equation}
\begin{equation}
 G_c(\omega,\vK,\vq,B)=\nth{\omega+i\delta-\eta(B)-\frac{K^2}{4m}-\frac{q^2}{m}}
\end{equation}
where $E_{th}(B)$ is the detuning, depending on magnatic field $B$. Here we can break it down into two steps. First, introduce 
 \begin{equation}
 {\fmtrx{U_{cc}}{U_{co}}{U_{oc}}{U_{oo}}}^{-1}={\fmtrx{V_{cc}}{V_{co}}{V_{oc}}{V_{oo}}}^{-1}-{\fmtrx{G^{vac}}{0}{0}{G^{vac}}}
\end{equation}
where $G^{vac}(q)=(i\delta-q^2/m)^{-1}$ is the vacuum pair propagator ignoring the hyperfine splitting at $\omega=0$ and no center momentum $\vK$.  Furthermore, when the close-channel resonnant state is close to threshold and well-seperated from other discrete levels, the effective interaction is separable.  So 
\begin{equation}
 U(\vq',\vq)=\frac{4\pi}{m}\mbr{\frac{a_s+3a_t}{4}+(a_t-a_s)\mathbf{S_1}\cdot\mathbf{S_2}}g(q')g(q)
\end{equation}
$g(q)\rightarrow0$ for $qr_C\rightarrow\infty$ where $r_C$ is the interaction range.  But This can be projected to the hyperfine-spin space and therefore two-channel.  
\begin{equation}
 {\fmtrx{U_{cc}}{U_{co}}{U_{oc}}{U_{oo}}}=
\frac{4\pi}{m}\mbr{\fmtrx{\frac{c_7a_s+a_t}{1+c_7}}{\frac{a_t-a_s}{\sqrt{1+c_7}\sqrt{1+c_9}}}
{\frac{a_t-a_s}{\sqrt{1+c_7}\sqrt{1+c_9}}}{\frac{c_9a_s+a_t}{1+c_9}}}
\end{equation}
For example, $\ket{9/2,-9/2}$, $\ket{9/2,-7/2}$, $\ket{7/2,-7/2}$ of ${}^{40}K$ at resonance $B=201.6$, $c_7=14.9$ and $c_9=0.059$ Note that both $c_7$ and $c_9$ depends on B.  

The second step deals with the influence of $B$ in the theshold $E_{th}$ as well as the energy dependence of $G$. 
 \begin{equation}
  {\fmtrx{T_{cc}}{T_{co}}{T_{oc}}{T_{oo}}}^{-1}={\fmtrx{U_{cc}}{U_{co}}{U_{oc}}{U_{oo}}}^{-1}-{\fmtrx{\Delta{}G_c}{0}{0}{\Delta{}G_o}}
\end{equation}
where $\Delta{}G_{o/c}=G_{o,c}(\omega,\vK,\vq,B)-G^{vac}(q)$.  Using the seperable properties of $U$, we can concentrate the $\vq$ dependence on two quantities and this leaves us a simple two-by-two algebra matrix equation.  
\begin{equation}
\Pi_{o,c}(\omega,\vK,B)=\int\frac{d^3q}{(2\pi)^3}\Delta{}G_{o,c}g^2(q)                  
\end{equation}
where for the contact interaction with $r_C=0$, we find $\Pi_o(\omega,\vK)=-im^{3/2}\sqrt{\omega-K^2/(4m)}$ and $\Pi_c(\omega,\vK,B)=\Pi_o(\omega-E_{th}(B),\vK)$
Especially, for $\omega=0$ and no central mometum $\vK$, $\Pi_o=0$ and $\Pi_c=\sqrt{-\eta(B)}$
\begin{gather}\label{eq:T}
 T_{cc}=\frac{U_{cc}}{1-U_{cc}\Pi_c}\\
T_{oc}=T_{co}=\frac{U_{oc}}{1-U_{cc}\Pi_c}\\
T_{oo}=U_{oo}+\frac{U_{oc}^2\Pi_c}{1-U_{cc}\Pi_c}
\end{gather}
comparing this with the commonly used formula 
\begin{equation}
 T_{oo}=\frac{4\pi{}a_{bg}}{m}\br{1-\frac{\Delta{}B}{B-B_0}}
\end{equation}
We find the resonance position $B_0$ satisfies
\begin{equation}
 1-U_{cc}(B_0)\Pi_c(B_0)=0
\end{equation}
And other quantities, such as $a_s$, $a_t$ can be expressed with the experimental quantities. 

However, it is still not clear to me how to really map broad /narrow resonance into the many-body eq.(\ref{eq:renomal1}) in a calculable sense.  The criteria of broad resonance as in \cite{JacksonNarrow} is   
\begin{equation}\label{eq:broadCriteria}
\Delta\mu\Delta{}B\gg\frac{k_F}{ma_{bg}}
\end{equation}


\subsection{rescale with Fermi momemtum/energy}
If we rescale renormalized gap equation eq. \ref{eq:renomal1} and the number equation with the Fermi momentum/energy, we have 
\begin{equation}\label{eq:renormal2}
\begin{pmatrix}\widetilde\Delta^U\\\widetilde\Delta^V\end{pmatrix}=
8\pi({a}{k_F})\begin{pmatrix}1&T_{oc}/T_{oo}\\T_{co}/T_{oo}&T_{cc}/T_{oo}\end{pmatrix}
\int\widetilde{k}^2d\widetilde{k}
\begin{pmatrix}
\nth{2}\sin2\theta_{k}\cos\phi_k-\frac{\widetilde\Delta^U}{2\widetilde\epsilon_k}\\
\nth{2}\sin2\theta_{k}\sin\phi_k-\frac{\widetilde\Delta^V}{2\widetilde\epsilon_k+\widetilde{\eta}}
\end{pmatrix}
\end{equation}
and rescaled number equation
\begin{equation}
 \int\widetilde{k}^2\sin^2\theta_kd\widetilde{k}=\nth{3}
\end{equation}
We can find the criteria of broad resonance (eq. \ref{eq:broadCriteria}) becomes 
\begin{equation}
 ak_F\widetilde{\eta}\gg1 \;\text{or}\; (ak_F)\eta\gg\epsilon_F
\end{equation}
Here we use $a=a_{bg}(1+\Delta_B/(B-B_0))$,  ignoring constant $1$ around the resonance, and also use the fact that rougly $\eta=\Delta\mu(B-B_0)$. (Note that the combination $a\eta$ does not depend on B, although they depend on B separately.)   At the broad limit, term $\frac{\widetilde\Delta^U}{2\widetilde\epsilon+\widetilde{\eta}}$ of the second integration in eq. \ref{eq:renormal2} gives a negligible contribution when $\sin2\theta_k$ is substantial.  So a simple solution is to put $\cos\phi_k\approx1$ while $\sin\phi_k\approx0$ for where $\sin2\theta_k$ is substantial (i.e. the particle is concentrated in open-channel).  Therefore, we reduce the two-channel problems into one-channel problem, omitting the close-channel. Here, we can see the interplaying between fermi energy and the two-body quantities, whether eq. \ref{eq:renormal2} can be reduced to a single-channel problem is due to the size of $\widetilde{\eta}$, which is determined by the ratio of the deturning and Fermi energy.  Every other quantities are two-body quantities. 

Looking closely, there are two possibilities to have large $\tilde\eta$, by have a broad resonance, this works in any detuning, $\eta$; or by a large detuning, no matter broad/narrow.    The second case gives us the extremes as in BEC/BCS.  In the BCS side, it is like no resonance; in the BEC side, it would be mostly close-channel molecules.  This is prbably not the most interesting situation however.  More interesting case would be not the extreme large detuning, still close to resonance enough, where, the molecule is not completely dominated by close-channel.  Here in some sense, the corresponding two-body $T$-matrix is for $T(\omega,\eta)=\frac{4\pi}{m}\nth{\nth{a}+r_{eff}k^2}$ where $\omega\sim{}\epsilon_F\gg\delta_c$. The normal contact interaction assumption with $T(0,\eta)$ breaks down here.  In the other sense, a narrow resonance is intrinsically not at extremes. 


\subsection{BEC/BCS Extreme}
Now it is nice to reduce the broad resonance of two-channel into one-channel.  It is nice to find out some result for the BCS/BEC extreme in the narrow limit. 
\subsubsection{BEC side}
At this side, we presume that $\mu\ll-\epsilon_F<0$ (this can be checked in the end). From eq. (\ref{eq:tan1}), $\tan2\theta_\vk$ is always small because $\xi_\vk=\epsilon_\vk+|\mu|\gg\epsilon_F,\Delta^{U,V}$.  So does the $\sin2\theta_\vk\approx\tan2\theta_\vk$ in the gap eq. (\ref{eq:renormal2}) and the integrand is dominated by the second term $-\Delta/(2\epsilon_\vk)$ when $\epsilon_k\leq|\mu|$.  The two equations reduce to the same   
