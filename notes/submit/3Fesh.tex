
\documentclass[reprint,pra]{revtex4-1}
\usepackage{amsfonts}
\usepackage{amssymb}
\usepackage{amsmath}
\usepackage{hyperref}
\usepackage{graphicx}% Include figure files
\usepackage{epstopdf}
\usepackage{subfig}
\usepackage{makeidx}
\usepackage[page,toc,title,titletoc]{appendix}


%\usepackage{comment}
%%\includecomment{mycomment}
%\specialcomment{mycommentzhu} {\begingroup\ttfamily\footnotesize}{\endgroup}
%%\excludecomment{mycomment}

%%adding comment, use the next line for disable comment
\newcommand{\mycomment}[1]{\textit{#1}}
%\newcommand{\mycomment}[1]{}

\newcommand{\vk}{\ensuremath{\mathbf{k}}}
\newcommand{\vK}{\ensuremath{\mathbf{K}}}
\providecommand{\vr}{\ensuremath{\mathbf{r}}}
%\newcommand{\vec}[1]{\ensuremath{\mathbf{#1}}}

\newcommand{\gk}{\ensuremath{{g}(\mathbf{k})}}

\newcommand{\vp}{\ensuremath{\mathbf{p}}}
\newcommand{\gp}{\ensuremath{{g}(\mathbf{p})}}

\newcommand{\vq}{\ensuremath{\mathbf{q}}}

\newcommand{\Fo}{\ensuremath{\mathbf{F_0}}}


\newcommand{\E}{\ensuremath{\mathbf{E}}}
\newcommand{\A}{\ensuremath{\mathbf{A}}}
\newcommand{\J}{\ensuremath{\mathcal{J}}}

\newcommand{\ket}[1]{\ensuremath{\left|#1\right>}}
\newcommand{\bra}[1]{\ensuremath{\left<#1\right|}}

\newcommand{\twoe}{\ensuremath{2\epsilon_\vk-\E_1}}

\newcommand{\nth}[1]{\ensuremath{\frac{1}{#1}}}

\newcommand{\br}[1]{\ensuremath{\left(#1\right)}}
\newcommand{\mbr}[1]{\ensuremath{\left[#1\right]}}
\newcommand{\bbr}[1]{\ensuremath{\left\{#1\right\}}}


\newcommand{\tk}{\ensuremath{\tilde{k}}}

\newcommand{\kp}{\ensuremath{\ket{\Psi}}}

\newcommand{\av}[1]{\ensuremath{\bigl<{#1}\bigr>}}
\newcommand{\avs}[3] {\av{#1{\lvert{#2}\rvert}#3}}
\newcommand{\avv}[2][\nu] {\avs{#1}{#2}{#1}}
\newcommand{\avt}[2]{\av{{#1}|{#2}}}
\newcommand{\avtu}[1]{\av{T_\tau#1}}

\newcommand{\Bop}{\ensuremath{\mathbf{B_0^+}}}
\newcommand{\Bmp}{\ensuremath{\mathbf{B_m^+}}}
\newcommand{\Bnp}{\ensuremath{\mathbf{B_n^+}}}
\newcommand{\Bo}{\ensuremath{\mathbf{B_0}}}
\newcommand{\Bopn}{\ensuremath{\mathbf{{B_0^+}^n}}}
\newcommand{\Bon}{\ensuremath{\mathbf{{B_0}^n}}}


\newcommand{\zmatrix}{\ensuremath{\br{\begin{smallmatrix}0&0\\0&0\end{smallmatrix}}}}
\newcommand{\fmtrx}[4]{\ensuremath{\br{\begin{smallmatrix}#1&#2\\#3&#4\end{smallmatrix}}}}
\newcommand{\smtrx}[6]{\ensuremath{\br{\begin{smallmatrix}#1&#2\\#3&#4\\#5&#6\end{smallmatrix}}}}

\newcommand{\vz}{\ensuremath{v^{\beta\alpha}_{\vk,\vk}}}


\providecommand{\abs}[1]{\ensuremath{\lvert{#1}\rvert}}

\newcommand{\sg}[1][1]{\ensuremath{\sigma_\frac{#1}{2}}}

\newcommand{\rhof}{\ensuremath{\rho(\ef)}}
\newcommand{\omt}{\ensuremath{\tilde{\Omega}}}
\newcommand{\cht}{\ensuremath{\tilde{\chi_0}}}
\newcommand{\Atl}{\ensuremath{\abs{A}^{2l}}}
\newcommand{\ef}{\ensuremath{\epsilon_F}}

\newcommand{\lca}{\ensuremath{\ln\br{1+\frac{\cht}{\alpha}}}}

\newcommand{\com}[2]{\ensuremath{\mbr{#1,#2}}}
\newcommand{\D}{\ensuremath{\mathit{D}}}
\newcommand{\dg}{\ensuremath{\dagger}}
\newcommand{\nG}{\ensuremath{\hat{\mathcal{G}}^{-1}}}

\providecommand{\lvk}{\ensuremath{1/\vk_F}}
\providecommand{\hm}{\ensuremath{\frac{\hbar^2}}{2m}}
\providecommand{\pdiff}[2]{\ensuremath{\frac{\partial{#1}}{\partial{#2}}}}
\providecommand{\dpdiff}[2]{\ensuremath{\frac{\partial^2{#1}}{\partial{{#2}^2}}}}

\providecommand{\H}{\ensuremath{\mathcal{H}}}
\providecommand{\wt}[1]{\widetilde{#1}}

\providecommand{\eef}[1]{Eq. (\ref{#1})}

\providecommand{\sch}{{Schr\"{o}dinger }}

\providecommand{\sgn}{\ensuremath{\text{sgn}}}
\newcommand{\Arctg}{\ensuremath{\text{Arctg}}}

\providecommand{\comm}[1]{\textit{\scriptsize \uwave{(#1)}}}

%\newenvironment{unsure}
%	{\begin{itshape}}% begin code
%	{\end{itshape}}%                    end code
\newenvironment{unsure}{}{}



\begin{document}


\title{BEC-BCS Crossover with Feshbach Resonance for a Three-Hyperfine-Species Model}
\author{Guojun Zhu}
\email{gzhu1@illinois.edu}
\affiliation{Department of physics, University of Illinois at Urbana-Champaign}
\author{Anthony J. Leggett}
\affiliation{Department of physics, University of Illinois at Urbana-Champaign}

%
\begin{abstract}
The BEC-BCS crossover problem has been intensively studied both theoretically and experimentally largely thanks to  Feshbach resonances which allow us to tune the effective interaction between alkali atoms.  In a Feshbach resonance, the effective s-wave scattering length grows when one moves toward the resonance point, and eventually diverges at this point.  There is one characteristic energy scale, $\delta_c$, defined as, in the negative side of the resonance point, the detuning energy at which the weight of the bound state shifts from predominately in the open-channel to predominated in the closed-channel.  When the many-body energy scale (e.g. the Fermi energy, $E_{F}$) is larger than $\delta_c$, the closed-channel weight is significant and has to be included in the many-body theory.  Furthermore, when two channels share a hyperfine species, the Pauli exclusion between fermions from two channels also needs to be taken into consideration in the many-body theory.  

The current  paper addresses the above problem in detail. A set of gap equations and number equations  are derived at the mean-field level.  The fermionic and bosonic excitation spectra are then derived. Assuming that the uncoupled bound-state of the closed-channel in resonance is much smaller than the inter-particle distance, as well as the s-wave scattering length, $a_s$, we find that  the basic equations in the single-channel crossover model are still valid. The correction first comes from the existing of the finite chemical potential and additional counting complication due to the closed-channel.  These two corrections need to be included into the mean-field equations, i.e. the gap equations and the number equations, and be solved self-consistently.  Then the correction due to the inter-channel Pauli exclusion is in the  order of the ratio of the Fermi energy and the Zeeman energy difference between two channels, $E_F/\eta$, which can be analyzed perturbatively over the previous corrections.  
\end{abstract}
\pacs{67.85.Lm}
\maketitle
%% Create a dedication in italics with no heading, centered vertically
%% on the page.


%% Create an Acknowledgements page, many departments require you to
%% include funding support in this.
\section{Introduction}

\section{Model and order parameters}
\section{Mean-field result and renormalization}
\section{Excitation modes}
\section{Conclusion}

\begin{acknowledgements}
We thank  Professor Monique Combescot, Dr. Shizhong Zhang and  Dr. Wei-Cheng Lee for many inspiring discussions. Part of this research  is supported  by NSF under grant No. DMR 09-06921. 
\end{acknowledgements}
\bibliography{../citation}

\end{document}
\endinput
%%
%% End of file `thesis-ex.tex'.
