% !TeX root =thesis.tex
%\subsection{Path integral approach for single channel\label{sec:pathInt}}
\label{sec:pathInt}
Randeria and the company has studied this problem with path integral and it is proved to be a rather nice tool for the problem due to its flexibility and readiness for extension to higher order fluctuation.  

We start with an attractive $\delta$-potential in real space.  This is not equivalent to the  reduced pairing potential as in original BCS work.  However, reduced paring potential only couples  particles of the opposite momentum and does not support simple form of Hubbard-Stratonovich transformation, which is essential to solve the problem in path integral formulation.  
\begin{equation}
\hat{H}-\mu\hat{N}=\sum_{\sigma}\int{d^{d}r}c^{\dagger}_{\sigma}(\vr)\br{-\nth{2m}\nabla^{2}-\mu}c^{}_{\sigma}(\vr)-g\int{d^{d}r}c^{\dagger}_{\uparrow}(\vr)c^{\dagger}_{\downarrow}(\vr)c^{}_{\downarrow}(\vr)c^{}_{\uparrow}(\vr)
\end{equation}
 We can write down the action for the quantum partition function $\mathcal{Z}=\int{\bigD(\bar\psi,\psi)\exp\br{-S[\bar\psi,\psi]}}$
\begin{equation}\label{eq:pathInt2:actionPsi}
S[\bar\psi,\psi]=\int^{\beta}_{0}d\tau\int{d^{d}r}\mbr{\sum_{\sigma}\bar\psi_{\sigma}(\vr,\tau)\br{\partial_{\tau}-\nth{2m}\nabla^{2}-\mu}\psi_{\sigma}(\vr,\tau)-g\bar\psi_{\uparrow}(\vr,\tau)\bar\psi_{\downarrow}(\vr,\tau)\psi^{}_{\downarrow}(\vr,\tau)\psi^{}_{\uparrow}(\vr,\tau)}
\end{equation}
Notice that here fermion field $\psi_{s}$ and $\bar\psi_{s}$ are Grassmann variables. Hence  they are not complex conjugate to each other as in operator language because  complex conjugate is not a well-defined concept for Grassmann variable. 
We try to solve this system by introduce Hubbard-Stratonovich transformation.   Introduce an auxiliary field (functional variable) $\Delta(\vr,\tau)$ coupled with Cooper channel $\psi_{\uparrow}(\vr,\tau)\psi_{\downarrow}(\vr,\tau)$. %Here we follow the normal notation from path integral, $r$ is four tempo-space coordinator.  
We write down first the Gaussian integral of $\Delta$
\begin{equation}
1=\int{\bigD(\bar\Delta,\Delta)}\exp\br{-\nth{g}\int{d\tau{d}^{d}r}\bar\Delta\Delta}
\end{equation}
Note that we absorb the extra constant of integration into the measure of $\bigD(\bar\Delta,\Delta)$.
And with a shift of $\Delta(\vr,\tau)\rightarrow\Delta(\vr,\tau)-g\psi_{\uparrow}(\vr,\tau)\psi_{\downarrow}(\vr,\tau))$, we have 
\footnote{$\int{\bigD(\bar\Delta,\Delta)}\cdot1$ is only a constant factor on partition function $\mathcal{Z}$ and has no effect on real physical quantity, therefore, we can take it as 1, (equivalently divide the $\mathcal{Z}$ by a constant)}
\begin{equation}
\exp\br{g\int{d\tau{}d^{d}r}\bar{\psi}_{\uparrow}\bar\psi_{\downarrow}\psi_{\downarrow}\psi_{\uparrow}}=
\int{\bigD(\bar\Delta,\Delta)}\exp\bbr{-\int{d\tau{d^{d}r}}\mbr{\nth{g}{\bar\Delta}{\Delta}-\br{\bar\Delta\psi_{\downarrow}\psi_{\uparrow}+\Delta\bar\psi_{\uparrow}\bar\psi_{\downarrow}}}}
\end{equation}
Note that  $\Delta(\vr,\tau)$ (or $\bar\Delta(\vr,\tau)$) comes from  Grassmann fields $\psi(\vr,\tau)$ (or $\bar\psi(\vr,\tau)$). Therefore, they are not related to each other as complex conjugate because no such concept exists in Grassmann algebra.  Nevertheless, at mean field level or only at the phase fluctuation around the mean field, $\Delta$  and $\bar\Delta$ are indeed complex conjugate.  Consequently, we will just take $\Delta$  as normal bosonic field in the following and often simply treat $\bar\Delta$ as $\Delta$'s complex conjugate in the following. Now the interaction term can be replaced.
\begin{align*}
\mathcal{Z}=&\int{}\bigD(\bar\psi,\psi)\int{\bigD(\bar\Delta,\Delta)}\\
&\;\exp\bbr{-\int{d\tau{d^{d}r}}\mbr{\sum_{\sigma}\bar\psi_{\sigma}\br{\partial_{\tau}-\nth{2m}\nabla^{2}-\mu}\psi_{\sigma}+\nth{g}{\bar\Delta}{\Delta}-\br{\bar\Delta\psi_{\downarrow}\psi_{\uparrow}+\Delta\bar\psi_{\uparrow}\bar\psi_{\downarrow}}}}
\end{align*}
At the expense of introducing an auxiliary field ($\Delta$) with contact-type coupling to the original field $\psi$, we eliminate the four-field interaction formally.  $\Delta$ field is like a \emph{local potential} for $\psi$, although this \emph{local potential} has to be generated from the original field self-consistently.  Nevertheless, $\Delta$ couples to pair of fermionic field $\psi$, and thus it extracts some special degree of freedom from $\psi$ field.  When properly selected, this degree of freedom is highly non-trivial and has macroscopic importance, which serves as ``order parameter'' for the system.  The above formula is bilinear to $\psi$, and we can rewrite it into a nicer form in Nambu spinor representation
\begin{equation}
\bar\Psi=\begin{pmatrix}\bar{\psi}_{\uparrow}&\psi_{\downarrow}\end{pmatrix}\text{,  }\qquad
\Psi=\begin{pmatrix}{\psi}_{\uparrow}\\\bar\psi_{\downarrow}\end{pmatrix}
\end{equation}
\begin{equation}
\mathcal{Z}=\int{\bigD(\bar\psi,\psi)}\int{\bigD(\bar\Delta,\Delta)}\exp
	\bbr{-\int{d\tau{d^{d}r}}\mbr{\nth{g}{\bar\Delta}{\Delta}-\bar\Psi \nG\Psi}}
\end{equation}
where 
\begin{equation}\label{eq:pathInt:nG}
\nG=\begin{pmatrix}
[\hat{G}_{0}^{(p)}]^{-1}&\Delta\\\bar\Delta&[\hat{G}_{0}^{(h)}]^{-1}
\end{pmatrix}
\end{equation}
is known as Gor'kov Green function, and $[\hat{G}_{0}^{(p)}]^{-1}=-\partial_{\tau}+\nth{2m}\nabla^{2}+\mu$, and $[\hat{G}_{0}^{(h)}]^{-1}=-\partial_{\tau}-\nth{2m}\nabla^{2}-\mu$ represent the non-interacting Green functions of the particle and hole respectively. Now $\Psi$ can be integrated out formally and partition function then only depends on bosonic field $\Delta$.  
\begin{equation}\label{eq:pathInt:DeltaPF}
\mathcal{Z}=\int{\bigD(\bar\Delta,\Delta)}\exp
	\bbr{-\mbr{\br{\int{d\tau{d^{d}r}}\nth{g}{\bar\Delta\Delta}}-\ln\det\nG}}
\end{equation}
And action is
\begin{equation}\label{eq:pathInt:DeltaAction}
S[\bar\Delta,\Delta]=
	{\mbr{\br{\int{d\tau{d^{d}r}}\nth{g}{\bar\Delta\Delta}}-\ln\det\nG}}
\end{equation}
Note that the determinant in $\ln\det\nG$ runs through both the normal space and $2\times2$ Nambu spinor space.  The above formulas are exactly equivalent to the original partition function (action) in fermion field $\psi$ (Eqs. \ref{eq:pathInt2:actionPsi}). It looks nice and compact. Nevertheless, $\ln\det\nG$ term is highly non-trivial and has all the many-body physics.

\section{Mean field result\label{sec:pathInt:meanfield}}
The saddle point equation of Eq. (\ref{eq:pathInt:DeltaPF}) gives the mean-field result of the system.  First we need to find the derivative of $\ln\det\nG$.  We notice the identity
\begin{equation}
\ln\det\hat{A}=\tr\ln\hat{A}
\end{equation}
and differential rule of a function like $\tr\ln$
\begin{equation}\label{eq:pathInt:diffTr}
\frac{\delta}{\delta\phi_q}\tr\ln(\nG)=\tr(\hat{\mathcal{G}}\frac{\delta}{\delta\phi_q}\nG)
\end{equation}
The saddle equation of Eq. (\ref{eq:pathInt:DeltaPF}) (differential with respect to $\Delta$) is
\begin{equation}
\nth{g}\bar{\Delta}(\vr,\tau)-\tr\mbr{\hat{\mathcal{G}}(\vr,\tau,\vr,\tau)\begin{pmatrix}0&1\\0&0\end{pmatrix}}=0
\end{equation}
Here this matrix is in the Nambu Spinor space.  If we seek a tempo-spacial homogeneous solution of $\Delta_0$, we can find the Nambu Green function from Eq. (\ref{eq:pathInt:nG}) in momentum space
\begin{equation}\label{eq:pathInt:G0}
G_0(p)=\nth{(i\omega_n)^2-E_\vp^2}
\begin{pmatrix}
	i\omega_n+\xi_\vp&\;&-\Delta_0\\
	-\bar{\Delta}_0&\;&i\omega_n-\xi_\vp
\end{pmatrix}
\end{equation}
Here $\omega_n$ is Matsubara frequency of Fermions.  $\xi_{\vk}=\epsilon_{\vk}-\mu$, $\epsilon_{\vk}=\vk^{2}/2m$,  $E_\vp=\sqrt{\xi_\vp^2+\abs{\Delta_0}^2}$.  And the saddle point equation can be rewritten as 
\begin{equation}
\nth{g}\bar{\Delta}_0=\frac{T}{L^d}\sum_{\vp,n}\frac{\bar\Delta_0}{\omega_n^2+E_\vp^2}
\end{equation}
The summation of Matsubara frequency can be evaluated and we find 
\begin{equation}
\nth{g}=\nth{L^d}\sum_{\vp}\frac{1-2n_f(E_p)}{2E_p}=\nth{L^d}\sum_{\vp}\frac{\tanh{(E_p/2T)}}{2E_p}
\label{eq:pathInt:gap}
\end{equation}
where $n_f(\epsilon)$ is the fermi distribution function.  This is exactly the gap equation obtained from other methods as well.  On the other hand, $\nG$ in Eq. (\ref{eq:pathInt:nG})  is the inverse of fermion-fermion correlation of $\Psi$.  In mean field, $G_{0}$ as Eq. (\ref{eq:pathInt:G0}) can be diagnosed in momentum space with a canonical (Bogoliubov) transformation.  Nevertheless, poles is where  $\omega^2-E_\vp^2=0$ (with a analytic continue of $i\omega_{n}\rightarrow\omega+0^{+}$) as we can see from  Eq. (\ref{eq:pathInt:G0}) and therefore the spectrum of fermionic excitation is $\pm{}E_{p}$.  \

Summand in Eq. \ref{eq:pathInt:gap} does not decreases fast enough in 3D and the summation does not converges.  This is because our assumption of contact interaction breaks down when reaching real potential range $a_{c}$, i.e., the summation of momentum is capped at some high momentum $\Lambda$ related to $1/a_{c}$.  Notice that in 3-D, we have a relation that connect the bare potential $g$ to more physically observable s-wave scattering length $a_{s}$
\begin{equation}\label{eq:pathInt:as}
\frac{m}{4\pi{}a_{s}}=-\nth{g}+\sum_{k<\Lambda}\nth{2\epsilon_{\vk}}
\end{equation}
We can renormalize Eq. \ref{eq:pathInt:gap} with this relation
\begin{equation}\label{eq:pathInt:gapRenormalized}
-\frac{m}{4\pi{}a_{s}}=\sum_{\vk}\mbr{\frac{\tanh{(E_k/2T)}}{2E_k}-\nth{2\epsilon_{\vk}}}
\end{equation}
Now the gap equation has proper decay in high momentum and no artificial cutoff is necessary.  There are two unknown parameters, $\mu$ and $\Delta$,  in the equation.  We need another equation in order to pin them down. To compliment the gap equation, we can introduce the number equation, $n=-\partial\Omega/\partial\mu$. At the saddle point, the thermodynamic potential is $\Omega_{0}=S[\Delta_{0}]/\beta$, and we have number equation
\begin{equation*}
n=-\nth{\beta}\tr\br{{G_{0}\pdiff{G_{0}^{-1}}{\mu}}}
\end{equation*}
Similarly the summation (due to the trace) over the Mastubara frequency can be evaluated and we have equation
\begin{equation}
n=\nth{L^{d}}\sum_{\vk}\mbr{1-\frac{\epsilon_{\vk}}{E_{\vk}}\tanh{(\frac{E_{\vk}}{2T})}}
\end{equation}
This equation and renormalized gap equation Eq. \ref{eq:pathInt:gapRenormalized} compose the implicit equations for two unknown parameters for mean-field gap $\Delta$ and chemical potential $\mu$.  It not hard to derive the zero temperature analytic result at both ends.  At BCS end ($1/k_{F}a_{s}\rightarrow-\infty$), we obtain $\mu\approx{}E_{F}$ and $\Delta\propto\exp(-\pi/2k_{F}\abs{a_{s}})$; at BEC end ($1/k_{F}a_{s}\rightarrow+\infty$),  $\mu=-\hbar^{2}/2ma_{s}^{2}$, i.e. half of the binding energy of the molecule, while $\Delta\propto{}n^{1/2}a_{s}^{-1/2}$ no longer has  much physical significance.  In the more general crossover region, these two  can be calculated numerically.  They have no singularity in the whole region, which indicates it is a crossover instead of any simple phase transition.  



\section{Gaussian fluctuation and collective mode}\label{sec:collective1}
Once the mean field value is obtained, we can expand partition function Eq. (\ref{eq:pathInt:DeltaPF}) around it ($\Delta(\vr,\tau)=\Delta_{0}+\theta(\vr,\tau)$) The linear order of  expansion is zero because $\Delta_{0}$ is the saddle point.  The next order gives us the bilinear terms on $\theta$, i.e., correlation of bosonic field $\Delta$ (four-fermion correlation).  Note that here the hamiltonian only has an extreme-short-range (contact) potential, therefore it cannot cover the situation of charged system where long-range Columnb interaction cannot be neglected.  We will discuss this later.  Nevertheless, it is conceivable that a more realistic short-range potential only renormalizes some parameters in the following calculation while leaves the qualitative result unmodified.  

Notice that we can expand the second term in Eq. \ref{eq:pathInt:DeltaPF} for $\hat{G}{}^{-1}=\hat{G}_{0}^{-1}+\hat{K}$
\begin{equation}\label{eq:pathInt:expand}
\tr\ln \hat{G}^{-1}=\tr\ln\hat{G_{0}}^{-1}+\tr(\hat{G_{0}}\hat{K})-\nth{2}\tr(\hat{G_{0}}\hat{K}\hat{G_{0}}\hat{K})+\cdots
\end{equation}
In our case,
\begin{equation}
\hat{K}=\begin{pmatrix}
0&\theta\\
\theta^{*}&0
\end{pmatrix}
\end{equation}
Here the linear terms of $\hat{K}$ or $\theta$ ($\theta^{*}$) are zero as the saddle point condition.  So to the second order, the action is 
\begin{equation}\label{eq:pathInt:DeltaActionGaussian}
S[\Delta_{0},\theta,\theta^{*}]=S[\Delta_{0}]+
	\nth{2g}\tr(\hat{K}\hat{K})+\nth{2}\tr(\hat{G_{0}}\hat{K}\hat{G_{0}}\hat{K})
\end{equation}
Write the last term into the momentum representation
\begin{equation}
\tr(\hat{G_{0}}\hat{K}\hat{G_{0}}\hat{K})=\sum_{q,p}\Tr\br{G_{0}({p})K_{q}G_{0}{}({p-q})K_{-q}}
\end{equation}
Notice that the second ``$\Tr$'' and following ``$\Tr$'' in this section only runs in Nambu spinor space and $q={(\vq,q_{l})}$, $p=(\vp,p_{n})$ are all four momentum, where $q_{l}$ is bosonic Matsubara frequency while $p_{n}$ is fermionic Matsubara frequency.
\begin{equation}
K_{q}=\begin{pmatrix}
0&\theta_{q}\\
\theta^{*}_{-q}&0
\end{pmatrix}
\end{equation}
If we introduce the a new vector 
\begin{equation}
\theta{(q)}=\begin{pmatrix}\theta_{q}\\\theta^{*}_{-q}\end{pmatrix}\qquad
\theta^{\dg}{(q)}=\begin{pmatrix}\theta^{*}_{q}&\theta_{-q}\end{pmatrix}
\end{equation}
the action can be rewritten into a more compact form
\begin{equation}
S[\Delta_{0},\theta,\theta^{*}]=S[\Delta_{0}]+\nth{2}\sum_{q}\Tr\mbr{\theta^{\dg}(q)\mathbf{M(q)}\theta(q)}
\end{equation}
Notice that we can always choose a real $\Delta_{0}$ and therefore $G_{0}{\ _{12}}(p)=G_{0}{\ _{21}}(p)$, we have 
\begin{equation}
\mathbf{M(q)}=
\begin{pmatrix}
\nth{g}+\sum_{p}G_{0}{\ }_{11}(p)G_{0}{\ }_{22}(p-q)&\sum_{p}G_{0}{\ }_{12}(p)G_{0}{\ }_{12}(p-q)\\
\sum_{p}G_{0}{\ }_{12}(p)G_{0}{\ }_{12}(p-q)&\nth{g}+\sum_{p}G_{0}{\ }_{11}(p-q)G_{0}{\ }_{22}(p)
\end{pmatrix}
\end{equation}
The summation over (fermionic) Matsubara frequency of $p_{n}$ can be carried out at zero temperature
\footnote{\label{foot:intro:sum}The summation of Matsubara frequency of function $h(i\omega_{n})$ is carried out by the normal trick.  We  multiply $h(z)$ with Fermi distribution function $n_{F}(z)$,  the summation is the sum of residuals at imaginary axis of $n_{F}(z)$.  The contour can be deform into a contour over the rest of singular points of $h(z)$. We just need to find the residuals of the total function $h(z)n_{F}(z)$ over those singular points to find the Matsubara summation.   However, due to zero temperature, the  $n_{F}(z)$ only nonzero at the negative singular points of $h(z)$, $-E_{\vk}$ in our case.  (The other singular point $E_{\vk}$ gives $n_{F}(E_{\vk})=0$ for zero temperature.)}
\begin{equation}
\begin{split}
M_{11}(q)&=M_{22}(-q)\\
	&=\nth{g}+\sum_{\vp{,}p_{n}}G_{0}{\ }_{11}(p)G_{0}{\ }_{22}(p-q)\\
	&=\nth{g}+\sum_{\vp}\br{\frac{u^{2}u'^{2}}{iq_{l}-E-E'}-\frac{v^{2}v'^{2}}{iq_{l}+E+E'}}
\end{split}
\end{equation}
\begin{equation}
\begin{split}
M_{12}(q)&=M_{21}(q)\\
	&=\sum_{\vp{,}p_{n}}G_{0}{\ }_{12}(p)G_{0}{\ }_{12}(p-q)\\
	&=\sum_{\vp}uvu'v'\br{\nth{iq_{l}+E+E'}-\nth{iq_{l}-E-E'}}
\end{split}
\end{equation}
where $u=u_{\vp}$, $v=v_{\vp}$, $E=E_{\vp}$ and $u'=u_{\vp-\vq}$, $v'=v_{\vk-\vq}$, $E'=E_{\vk-\vq}$.  $u$, $v$, $E$ are as defined usually in BCS literature. 
\begin{equation}
v_{\vk}^{2}=1-u_{\vk}^{2}=\nth{2}\br{1-\frac{\xi_{\vk}}{E_{\vk}}}
\end{equation}
 The $G^{(M)}=\mathbf{M}^{-1}$ is the correlation function of $\theta$ (or $\Delta$) and its poles give the spectrum of collective mode as every  $\theta_{q}$ (or $\Delta_{q}$) involves many fermions moving in a coherent manner.  So the spectrum of collective modes can be determined by $\det{M(\omega,\vq)}=0$ after we analytically continue for the frequency $iq_{l}\rightarrow\omega+i0^{+}$.  
 
For low energy modes, where $\omega,\,\abs{\vq}^{2}\ll\min\bbr{E_{\vk}}$ both are much smaller than $\Delta_{0}$, we can expand $M$ with $\omega$ and $\vq$.  The lowest order has the form $\omega\approx{}c\,q$, which suggests a sound wave as expected for any Goldstone mode.  At BCS side, $c=v_{F}/\sqrt{3}$, where $v_{F}$ is Fermi velocity.  This coincides with the famous Anderson-Bogoliubov mode.  At BEC side, we get $c^{2}=\Delta^{2}/8m\abs{\mu}=v_{F}^{2}(k_{F}a_{s})/3\pi=4\pi{}n_{B}a_{B}/m_{B}$, which fits the low momentum part of Bogoliubov spectrum.  Here $m_{B}=2m$ is the molecule mass, $n_{B}=n/2$ is the molecule density and $a_{B}=2a_{s}$ is the inferred the interaction between molecules.  This value differs from the result of more accurate calculation from few-body, $a_{B}=0.6a_{s}$, which indicates the possible deficiency of the theory. 


\section{Alternative method to invert Green's function\label{sec:diagonalizeGreen1}}
In the above section, we have inversion of Gorkov green function Eq. (\ref{eq:pathInt:nG}) and it can be invert easily as Eq. (\ref{eq:pathInt:G0}).   Alternatively, we can use a different approach which proves to be more convenient in two-channel problem.  First, we diagonalize $\nG$ with unitary transformation $T$, in momentum space
\begin{equation}
\nG=\mtrx{i\omega_{n}-\xi_{k}&\Delta\\\bar\Delta&i\omega_{n}+\xi_{k}}=T^{\dg}BT
\end{equation}
It is easy to show that such $T$ and $B$ satisfing above equation are
\begin{equation}
T=\mtrx{u_{k}&v_{k}\\-v_{k}^{*}&u_{k}}\qquad{}B=\mtrx{i\omega_{n}+E_{k}&0\\0&i\omega_{n}-E_{k}}
\end{equation}
where $u_{k}^{2}(v_{k}^{2})=\nth{2}(1\pm\xi_{k}/E_{k})$ and $E_{k}$ are conventionally defined quantities in BCS theory.   Actually, this transformation is nothing but Bogoliubov canonical transformation, and $B$ matrix simply describes spectrum of fermionic quasi-particles.  Now it is easy to invert $\nG$
\begin{equation}
G=T^{\dg}B^{-1}T
\end{equation}
Green's function $G$ takes a more conventional form $A/(i\omega_{n}\pm{}E_{k})$ without any dependency on frequency in nominator as Eq. (\ref{eq:pathInt:G0}). Matsubara frequency summation over $G_{0}(k)$ in mean-field and $G_{0}(k)G_{0}(k+q)$ in Gaussian order are then fairly straight-forward as in text-book.  
