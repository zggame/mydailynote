\subsection{Bogliubov quasiparticle for Fermionic excitation}
I can work out the Fermionic excitation by first linearize the (reduced pairing) hamiltonian and then use the canonical transformation.  The coefficients of the transformation is messy, but the linearized hamiltonian provides the $3\times3$ matrix of the $(a^{\dagger}_{\vk},\; b^{}_{-\vk},\;c^{}_{-\vk})$  and their hermitian conjugate.  The three eigenvalues give the fermionic single pair excitation and eigenvectors give the exact coefficient of transformation.  The secular equation is a cubic equation.  In principle, it has analytical roots, but they are very nasty and do not generate much intuition.  Instead a zeroth order approximation where no Pauli exclusion of two-channel is used and then correction is sought.  

One crucial approximation is made that the close-channel bound state is small, it is small comparing to average inter-particle distance.  This assumption is justified for the category of the problem.  Otherwise, it is no longer a clear Feshbach resonance problem but a problem with two interacted degenerate fermi gas (both in superfluid state possibly? ).  