\subsection{Problem with Ansatz}
M. Combescot comment on the ansatz \eqref{eq:ansatz}, she conjecture that the turning from a canonical ensemble ansatz into the grand canonical ansatz by coherent state blur the energy of the bound-state in close-channel when calculating the energy expectation, therefore the detuning, which probably should appear in the final answer.  And it might wash away the very nature of the bound state, because those extra approximation involved.  We might have to dial back to the canonical ensemble ansatz.  Tony suggested us to think about the simple bound-state without the ope-channel to check this idea.  

BCS wave-function's grand canonical ensemble ansatz can be converted from the particle-conserved state by writing it as coherent state, 
\begin{equation}
\ket{\Psi^{(N)}}=\br{\psi^\dg}^N\ket{0}
\end{equation} 
\begin{equation}
 \psi^\dg=\sum\phi_\vk{a^\dg_\vk{}c^\dg_{-\vk}}
\end{equation}
\begin{equation}\label{eq:coherentSt}
\ket{\Psi}=\sum_N\nth{N!}\ket{\Psi^{(N)}}=exp\br{\psi^\dg}\ket{0}=\prod\br{1+\phi_\vk{a^\dg_\vk{}c^\dg_{-\vk}}}\ket{0}
\end{equation}
And the Eq.\eqref{eq:coherentSt} can be rewritten into the familiar $\prod\br{u_\vk+v_\vk{a^\dg_\vk{}c^\dg_{-\vk}}}\ket{0}$ form by normalizing each $\vk$ component. 

However, in such a state, there is no peak in one particular N-particle state, which is usually the argument for the equivalence between canonical and grand canonical ensemble.  This brings the questions about the validity of the grand canonical ensemble, especially for bound-state.  \emph{This argument are unlikely correct as we are talking about the zero-temperature ground state, nothing about the statistics, which is about finite-temperature.  }   

On the other hand, this N-conserved wave-function is definitely not the real eigenstate.  \emph{It is hard to say whether this one or the normal grand-canonical ensemble one is closer to the truth.}

