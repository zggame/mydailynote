What we are more interested is how the internal structure of the coboson changes in the many-body situation.  In some sense, given the fact that w.f. shape within the potential range does not change, but its normalization changes and so does the open-channel w.f. normalization in the region.  

\textbf{A related question is that when this effect is large?}

Simple answer is that in the ultra-narrow part where close-channel is important.  How about the center of mass movement of close-channel. 

\subsection{Shizhong's Suggestion}
\begin{quote}
Shizhong: I have a very naive suggestion you might try\\
Shizhong: it looks like the first step you might try is to first solve the open channel problem and obtain momentum distribution for both spin state, this step is not hard and in fact is done in some papers\\
 me: what do you mean open channel problem?
  you mean single-channel problem?\\
Shizhong: then in the second step, you freeze the open channel momentum distribution and now consider the molecular formation in the closed channel with those states blocked by the open channel state\\
  yes. I mean single channel problem.
  this is exact the cooper problem\\
 instead of the Fermi sea, you now have something a bit more complicated but the general philosophy is the same!
\end{quote}



I am not sure how to handle the occupation of some fraction of the state as the case in the smearing around the Fermi sea.  We can write down the euqation in terms of the operator.  The valcum is the single-channel BCS instead of the normal valcum, or in the 0th order, maybe the simple Fermi sea.  Then we can look into what happens in the close-channel.  

\subsection{Tony's Comment}
He suggested me to stick with the naive ansatz and ignore the non-zero central momentum molecules.  He suggested me to try maybe variation method to look for how internal structure be modified.  He still think the diluteness plays important role.

So it is probably still the same argument that to find out the momentum distribution in the two-body problem and compare that with the many-body ansatz.   
