\subsection{}
Some arguments in last section \ref{sec:20100908:idea} seems incorrect, at least the argument that in close-channel zero-blocked  or one-pair-blocked within $k_{c}$ dominated multi-pair-blocked situation for open-channel BCS ansatz seems incorrect.  Therefore the argument in the following section \ref{sec:20100908:cal} becomes more or less irrelevant.  Nevertheless, the observation that close-channel bound state is relatively small in real space and therefore much extended in k-space is still useful.  Looked back \eef{eq:uvw:F}, (ignored Hatree terms)
\begin{equation}\tag{\ref{eq:uvw:F}}
 \begin{split}
  &F\equiv\av{H-\mu{}N}\\
    =&\sum(\xi^{ab}_\vk)\abs{v_\vk}^2+\nth2\sum_{\vk\neq\vk'}U_{\vk\vk'}v^{}_{\vk'}u^*_{\vk'}u^{}_\vk{}v^*_\vk\\
    &+\sum(\xi^{ac}_\vk)\abs{w_\vk}^2
      +\nth2\sum_{\vk\neq\vk'}V_{\vk\vk'}w^{}_{\vk'}u^*_{\vk'}u^{}_\vk{}w^*_\vk\\
    &  +\nth2\sum_{\vk\neq\vk'}Y_{\vk\vk'}w^{}_{\vk'}{u^{*}_{\vk'}}v^*_\vk{}u^{}_\vk
        +\nth2\sum_{\vk\neq\vk'}Y^*_{\vk\vk'}w^*_{\vk}{u^{}_{\vk}}v^{}_{\vk'}{}u^{*}_{\vk'}
 \end{split}
\end{equation}
It is useful to recognize that $w_{\vk}$ (or maybe $G_{\vk}=u_{\vk}w_{\vk}$) is always small. Changing  $w_{\vk}$ into $G_{\vk}$ in term $\sum(\xi^{ac}_\vk)\abs{w_\vk}^2$ introduces two problems:  First, the $w_{\vk}$ is finite, but it is small and error is probably small too;  second, within $k_{c}$, $v_{\vk}$ is substantial and therefore reduce the $u_{\vk}$ and makes $G_{\vk}$ much smaller than $w_{\vk}$, the error introduced here is probably also small because $w_{\vk}$ is small within $k_{c}$ and therefore the summation is small comparing to the term itself.  
So we have the free energy as 
\begin{equation}\label{eq:20100909:F}
 \begin{split}
    F=&\sum(\xi^{ab}_\vk)\abs{v_\vk}^2+\nth2\sum_{\vk\neq\vk'}U_{\vk\vk'}F^{}_{\vk'}F^*_\vk\\
    &+\sum(\xi^{ac}_\vk)\abs{G_\vk}^2
      +\nth2\sum_{\vk\neq\vk'}V_{\vk\vk'}G^{}_{\vk'}G^*_\vk\\
    &  +\nth2\sum_{\vk\neq\vk'}Y_{\vk\vk'}G^{}_{\vk'}F^*_\vk{}
        +\nth2\sum_{\vk\neq\vk'}Y^*_{\vk\vk'}G^*_{\vk}F^{}_{\vk'}
 \end{split}
\end{equation}
Take $F_{\vk}$ and $G_{\vk}$ as variables, first we express $v_{\vk}$ in term of them.  We have  $\abs{u_\vk}^2+\abs{v_\vk}^2+\abs{w_\vk}^2=1$, here we replace  $w_{\vk}$  with  $G_{\vk}$  again.  We have 
\[
\begin{split}
v_{\vk}^2=&\nth{2}(1-G_{\vk}^{2}+\sqrt{(1-G_{\vk}^{2})^{2}-4F_{\vk}^{2}})\qquad (\epsilon_{\vk}<\mu)\\
	=&\nth{2}(1-G_{\vk}^{2}-\sqrt{(1-G_{\vk}^{2})^{2}-4F_{\vk}^{2}})\qquad (\epsilon_{\vk}\geq\mu)
\end{split}
\]
In BEC side ($\mu<0$), only the second form exists. (This sign problem is not important anyway as we can freely add a overall phase to $F_{\vk}$ or $u,v$, as mostly we just deal with the square of these quantities. ).   Just put the second case into the free energy \eef{eq:20100909:F}, take derivative over $F_{\vk}$, (take it as real, not complex) we find 
\begin{equation}\label{eq:20100909:fEq}
\xi^{ab}_{\vk}(\frac{2F_{\vk}}{\sqrt{(1-G_{\vk}^{2})^{2}-4F_{\vk}^{2}}})+\sum_{\vk'}(U_{\vk\vk'}F_{\vk'}+Y_{\vk\vk'}G_{\vk'})=0
\end{equation}
Define the normal gap parameters 
\begin{equation}\label{eq:20100909:gap}
\Delta^{F_\vk}_{\vk}=-\sum_{\vk'}(U_{\vk\vk'}F_{\vk'}+Y_{\vk\vk'}G_{\vk'})
\end{equation}
And we find
\begin{equation}
F_{\vk}=\frac{(1-G_{\vk}^{2})\Delta^{F_\vk}_{\vk}}{2\sqrt{(\epsilon^{ab}_{\vk}-\mu)^{2}+\Delta^{F_\vk}_{\vk}{}^{2}}}
\end{equation}
For low-k ($k<k_{c}$), both $U_{\vk\vk'}$ and $Y_{\vk\vk'}$ varies slowly, so $\Delta^{F_\vk}_{\vk}$ varies slowly and we can drop subscript $\vk$ and treat it as constant.  \eef{eq:20100909:fEq} describe the many-body effects when comparing with the two-body Schr\"{o}dinger equation.  $(1-G^{2})$ gives the pauli exclusion from the close-channel, the square root takes into account the pauli exclusion within open-channel.  

\eef{eq:20100909:gap} cannot be renormalized due to the extra factor $(1-G_{\vk}^{2})$.  A close look reveal that it does not reduce to normal \sch equation due to this extra factor.  This is puzzling.  It might be caused by replace $w_{\vk}$ with $G_{\vk}$.  Another possibility is that the coupled equation cannot be renormalized in this way, it has to be renormalized in the two-channel fashion, for the diagonalized  basis.  

\subsection{Tony's comment}
Tony takes \eef{eq:20100909:gap} as renormalized density of states.  From there, the two-body problem is also modified for this renormalized DoS. 

He suggested me to take $u_{\vk}$ as single-channel result for $G_{\vk}$.   

He encourage me to think about whether close-channel is significantly modified within $k_{F_\vk}$, he does not give me a definitive answer.  I feel this might relate to the real close-channel potential that support the bound state.  Not clear idea anyway.  
------ This is probably not modified as close-channel is much higher due to Zeeman energy.  This bare detuning corresponding the size of the close-channel bound state and therefore larger than any other energy scale.   It is the bound state level that close to 0 of open-channel.  By simple energy consideration, it is very disadvantageous to populate any close-channel low k state at all due to the high kinetic energy (with Zeeman energy).  It is only energetic feasible to populate it in the bound-state way where the potential energy gaining is sufficient to compensate the large kinetic energy.  

He also stress that I need to keep an eye on things thrown away in approximation,  whether they are in the order of kept terms.  

\subsection{\label{sec:20100915:deltaN}}\
The difference between $w_{\vk}$ and $G_{\vk}$ are introduced because of the grand-canonical nature of the ansatz?  In BCS side, this introduce really small error as $\delta{N}$ can be shown are small.  Does that hold in BEC end?
\begin{equation}
\delta{N}^{2}=\av{N^{2}}-\bar{N}^{2}=\sum{\abs{v_{\vk}}^{2}-\abs{v_{\vk}}^{4}}=\sum{\abs{F_{\vk}}^{2}}
	=\sum{\abs{\frac{\Delta}{2E}}^{2}}
\end{equation}
This is extremely small for the BCS-side, but not the case in BEC-end.  (\emph{Interesting problem!})  \footnote{See Tony's comment on Sec. \ref{sec:20100915:tony}.}


One thing potentially useful is that to distinguish the region within $k_{c}$ and without where $G_{\vk}$ ($w_{\vk}$) is small comparing to open-channel or not.


\subsection{Exact Gap Equations in Variation Method}
Solve $u_{\vk}$, $v_{\vk}$, $w_{\vk}$ with $F_{\vk}$ and $G_{\vk}$ (treat everything as real)
\begin{gather}
u_{\vk}^2+v^{2}_{\vk}+w^{2}_{\vk}=1\\
u_{\vk}v_{\vk}=F_{\vk}\\
u_{\vk}w_{\vk}=G_{\vk}
\end{gather}
Take the root that $u_{\vk}$ goes in the same direction as $F_{\vk}$ \footnote{\label{foot:20100909:sgn} This is true for the whole region of BEC case, or for $k$ is large in BCS case, $\sgn_{k}=1$. In BCS case, when $k$ is small, $\sgn_{k}=-1$.  In single-channel case $\sgn_{k}=\sgn(\epsilon_{k}-\mu)$, however, in two-channel, this is more delicate.  This is very important in \eef {eq:20100909:number}}
\begin{equation}
\begin{split}
u_{\vk}^2=&\frac{1}{2} \left(1+\sgn_{k}\sqrt{1-4 F_{\vk}^2-4 G_{\vk}^2}\right)\\
v^{2}_{\vk}=&\frac{2 F_{\vk}^2}{1+\sgn_{k}\sqrt{1-4 F_{\vk}^2-4 G_{\vk}^2}}
=\frac{ F_{\vk}^2}{2( F_{\vk}^2+ G_{\vk}^2)} \left(1-\sgn_{k}\sqrt{1-4 F_{\vk}^2-4 G_{\vk}^2}\right)\\\
w^{2}_{\vk}=&\frac{2 G_{\vk}^2}{1-\sgn_{k}\sqrt{1-4 F_{\vk}^2-4 G_{\vk}^2}}
=\frac{G_{\vk}^2}{2( F_{\vk}^2+ G_{\vk}^2)} \left(1-\sgn_{k}\sqrt{1-4 F_{\vk}^2-4 G_{\vk}^2}\right)
\end{split}
\end{equation}
Below, we adopt $\sgn_k=1$. (This works for BEC and most part of BCS.) Derivatives over $F_{\vk}$ are
\begin{equation}
\begin{split}
\pdiff{u_{\vk}^2}{F_\vk}=&-\frac{2 F_\vk}{\sqrt{1-4 F_{\vk}^2-4 G_{\vk}^2}}\\
\pdiff{v_{\vk}^2}{F_\vk}=&\frac{2 F_\vk}{\sqrt{1-4 F_{\vk}^2-4 G_{\vk}^2}}-\frac{8 F_{\vk} G_{\vk}^2}{\sqrt{1-4 F_{\vk}^2-4 G_{\vk}^2} \left(1+\sqrt{1-4 F_{\vk}^2-4 G_{\vk}^2}\right)^2}\\
\pdiff{w_{\vk}^2}{F_\vk}=&\frac{8 F_{\vk} G_{\vk}^2}{\sqrt{1-4 F_{\vk}^2-4 G_{\vk}^2} \left(1+\sqrt{1-4 F_{\vk}^2-4 G_{\vk}^2}\right)^2}
=\frac{F_{\vk} G_{\vk}^2 \left(1-\sqrt{1-4 F_{\vk}^2-4 G_{\vk}^2}\right)^2}{2\sqrt{1-4 F_{\vk}^2-4 G_{\vk}^2} (F_{\vk}^2 + G_{\vk}^2)^2}
\end{split}
\end{equation}
Similarly, the derivative over $G_{\vk}$ can be obtained by exchange $F_{\vk}$ ($v_{\vk}$) and $G_{\vk}$ ($w_{\vk}$).
\begin{equation}
\begin{split}
\pdiff{u_{\vk}^2}{G_\vk}=&-\frac{2 G_\vk}{\sqrt{1-4 F_{\vk}^2-4 G_{\vk}^2}}\\
\pdiff{v_{\vk}^2}{G_\vk}=&\frac{8 F_{\vk} ^{2}G_\vk}{\sqrt{1-4 F_{\vk}^2-4 G_{\vk}^2} \left(1+\sqrt{1-4 F_{\vk}^2-4 G_{\vk}^2}\right)^2}
=\frac{F_{\vk}^{2}G_{\vk} \left(1-\sqrt{1-4 F_{\vk}^2-4 G_{\vk}^2}\right)^2}{2\sqrt{1-4 F_{\vk}^2-4 G_{\vk}^2} (F_{\vk}^2 + G_{\vk}^2)^2}\\
\pdiff{w_{\vk}^2}{G_\vk}=&\frac{2 G_\vk}{\sqrt{1-4 F_{\vk}^2-4 G_{\vk}^2}}-\frac{8 F_{\vk} ^{2}G_\vk}{\sqrt{1-4 F_{\vk}^2-4 G_{\vk}^2} \left(1+\sqrt{1-4 F_{\vk}^2-4 G_{\vk}^2}\right)^2}
\end{split}
\end{equation}
The gap equations are 
\begin{subequations}\label{eq:20100909:fullgap}
\begin{gather}
\frac{2F_{\vk}}{\sqrt{1-4 F_{\vk}^2-4 G_{\vk}^2}} \xi^{ab}_{\vk}+\frac{8 F_{\vk} G_{\vk}^2}{\sqrt{1-4 F_{\vk}^2-4 G_{\vk}^2} \left(1+\sqrt{1-4 F_{\vk}^2-4 G_{\vk}^2}\right)^2}\eta+U_{\vk\vk'}F_{\vk'}+Y_{\vk\vk'}G_{\vk'}=0
\label{eq:20100909:fullgapa}\\
\frac{2G_{\vk}}{\sqrt{1-4 F_{\vk}^2-4 G_{\vk}^2}} \xi^{ac}_{\vk}-\frac{8 F_{\vk}^{2} G_{\vk}}{\sqrt{1-4 F_{\vk}^2-4 G_{\vk}^2} \left(1+\sqrt{1-4 F_{\vk}^2-4 G_{\vk}^2}\right)^2}\eta+V_{\vk\vk'}G_{\vk'}+Y_{\vk\vk'}F_{\vk'}=0
\label{eq:20100909:fullgapb}
\end{gather}
\end{subequations}
where $\eta=\epsilon^{ac}_{\vk}-\epsilon^{ab}_{\vk}$ is the bare Zeeman energy difference and is large than most energy scale, $E_{F}$.  It should be in the order of binding energy of the close-channel bound state.   

The number equation is (see footnote (\ref{foot:20100909:sgn}) for $\sgn_{k}$)
\begin{equation}\label{eq:20100909:number}
N=\sum_{\vk}(v_{\vk}^{2}+w_{\vk}^{2})=\sum_{\vk}\frac{1}{2} \left(1-\sgn_{k}\sqrt{1-4 F_{\vk}^2-4 G_{\vk}^2}\right)
\end{equation} 
