\subsection{Path integral approach for single channel}
Here we reviewed the path integral approach for the original single-channel BEC-BCS crossover problem. Randeria and the company \cite{RanderiaBEC, Randeria1997, Randeria2008} has studied this problem with path integral and it is proved to be a rather nice tool for the problem.  

We started with an attractive $\delta$-potential in real space.  This is not equivalent to normal reduced pairing potential as in original BCS work.  However, reduced paring potential only couples  particles of the opposite momentum and does not support simple form of Habbard-Stratonovich transformation, which is essential to solve the problem in path integral formulation.   More physically,
\begin{equation}
\hat{H}-\mu\hat{N}=\sum_{\sigma}\int{d^{d}r}c^{\dagger}_{\sigma}(\vr)\br{\partial_{\tau}-\nth{2m}\nabla^{2}-\mu}c^{}_{\sigma}(\vr)-g\int{d^{d}r}c^{\dagger}_{\uparrow}(\vr)c^{\dagger}_{\downarrow}(\vr)c^{}_{\downarrow}(\vr)c^{}_{\uparrow}(\vr)
\end{equation}
where $\xi_{\vk}=\epsilon_{\vk}-\mu$, $\epsilon_{\vk}=\vk^{2}/2m$.  We can write down the action for the quantum partition function $\mathcal{Z}=\int{D(\bar\psi,\psi)\exp\br{-S[\bar\psi,\psi]}}$
\begin{equation}
S[\bar\psi,\psi]=\int^{\beta}_{0}d\tau\int{d^{d}r}\mbr{\sum_{\sigma}\bar\psi_{\sigma}(\vr,\tau)\br{\partial_{\tau}-\nth{2m}\nabla^{2}-\mu}\psi_{\sigma}(\vr,\tau)-g\bar\psi_{\uparrow}(\vr,\tau)\bar\psi_{\downarrow}(\vr,\tau)\psi^{}_{\downarrow}(\vr,\tau)\psi^{}_{\uparrow}(\vr,\tau)}
\end{equation}
We try to solve this system by introduce Hubbard-Stratonovich transformation.   Introduce a bosonic field $\Delta(\vr,\tau)$ coupled with Cooper channel $\psi(\vr,\tau)\psi(\vr,\tau)$. %Here we follow the normal notation from path integral, $r$ is four tempo-space coordinator.  
We write down first the Gaussian integral of $\Delta$
\begin{equation}
1=\int{D(\bar\Delta,\Delta)}\exp\br{-\nth{g}\int{d\tau{d}^{d}r}\bar\Delta\Delta}
\end{equation}
Note that we absorb the extra constant of integration into the measure of $D(\bar\Delta,\Delta)$.
And with a shift of $\Delta(\vr,\tau)\rightarrow\Delta(\vr,\tau)-g\psi(\vr,\tau)\psi(\vr,\tau)$, we have 
\footnote{$\int{D(\bar\Delta,\Delta)}1$ is only a constant factor on partition function $\mathcal{Z}$ and has no effect on real physical quantity, therefore, we can take it as 1, (equivalently divide the $\mathcal{Z}$ by a constant)}
\begin{equation}
\exp\br{g\int{d\tau{}d^{d}r}\psi_{\uparrow}\bar\psi_{\downarrow}\psi_{\downarrow}\psi_{\uparrow}}=
\int{D(\bar\Delta,\Delta)}\exp\bbr{-\int{d\tau{d^{d}r}}\mbr{\nth{g}\abs{\Delta}^{2}-\br{\bar\Delta\psi_{\downarrow}\psi_{\uparrow}+\Delta\bar\psi_{\uparrow}\bar\psi_{\downarrow}}}}
\end{equation}
Now the interaction term can be replaced.
\begin{equation*}
\mathcal{Z}=\int{D(\bar\psi,\psi)\int{D(\bar\Delta,\Delta)}\exp\bbr{-\int{d\tau{d^{d}r}}\mbr{\sum_{\sigma}\bar\psi_{\sigma}\br{\partial_{\tau}-\nth{2m}\nabla^{2}-\mu}\psi_{\sigma}+\nth{g}\abs{\Delta}^{2}-\br{\bar\Delta\psi_{\downarrow}\psi_{\uparrow}+\Delta\bar\psi_{\uparrow}\bar\psi_{\downarrow}}}}}
\end{equation*}
This form is bilinear to $\psi$, and we can rewrite it into a nicer form in Nambu spinor representation
\begin{equation}
\bar\Psi=\begin{pmatrix}\bar{\psi}_{\uparrow}&\psi_{\downarrow}\end{pmatrix}\text{,  }\qquad
\Psi=\begin{pmatrix}{\psi}_{\uparrow}\\\bar\psi_{\downarrow}\end{pmatrix}
\end{equation}
\begin{equation}
\mathcal{Z}=\int{D(\bar\psi,\psi)}\int{D(\bar\Delta,\Delta)}\exp
	\bbr{-\int{d\tau{d^{d}r}}\mbr{\nth{g}\abs{\Delta}^{2}-\bar\Psi \nG\Psi}}
\end{equation}
where 
\begin{equation}\label{eq:pathInt:nG}
\nG=\begin{pmatrix}
[\hat{G}_{0}^{(p)}]^{-1}&\Delta\\\bar\Delta&[\hat{G}_{0}^{(h)}]^{-1}
\end{pmatrix}
\end{equation}
is known as Gor'kov Green function, and $[\hat{G}_{0}^{(p)}]^{-1}=-\partial_{\tau}+\nth{2m}\nabla^{2}+\mu$, and $[\hat{G}_{0}^{(h)}]^{-1}=-\partial_{\tau}-\nth{2m}\nabla^{2}-\mu$ represent the non-interacting Green functions of the particle and hole respectively. Now $\Psi$ can be integrated out formally and partition function then only depends on bosonic field $\Delta$.
\begin{equation}\label{eq:pathInt:DeltaPF}
\mathcal{Z}=\int{D(\bar\Delta,\Delta)}\exp
	\bbr{-\mbr{\br{\int{d\tau{d^{d}r}}\nth{g}{\bar\Delta\Delta}}-\ln\det\nG}}
\end{equation}
And action is
\begin{equation}\label{eq:pathInt:DeltaAction}
S[\bar\Delta,\Delta]=
	{\mbr{\br{\int{d\tau{d^{d}r}}\nth{g}{\bar\Delta\Delta}}-\ln\det\nG}}
\end{equation}
Note that $\ln\det\nG$ goes through both the normal space and $2\times2$ Nambu spinor space.  

\subsubsection{Mean Field Result}
The saddle point equation of Eq. (\ref{eq:pathInt:DeltaPF}) gives the mean-field result of the system.  First we need to find the derivative of $\ln\det\nG$.  We notice the identity
\begin{equation}
\ln\det\hat{A}=\tr\ln\hat{A}
\end{equation}
and differential rule of a function like $\tr\ln$
\begin{equation}
\frac{\delta}{\delta\phi_q}\tr\ln(\nG)=\tr(\hat{\mathcal{G}}\frac{\delta}{\delta\phi_q}\nG)
\end{equation}
The saddle equation of Eq. (\ref{eq:pathInt:DeltaPF}) (differential with respect to $\Delta$) is
\begin{equation}
\nth{g}\bar{\Delta}(\vr,\tau)-\tr\mbr{\hat{\mathcal{G}}(\vr,\tau,\vr,\tau)\begin{pmatrix}0&1\\0&0\end{pmatrix}}=0
\end{equation}
Here this matrix is in the Nambu Spinor space.  If we seek a tempo-spacial homogeneous solution of $\Delta_0$, we can find the Nambu Green function from Eq. (\ref{eq:pathInt:nG}) in momentum space
\begin{equation}\label{eq:pathInt:G0}
G_0(p)=\nth{(i\omega_n)^2-E_\vp^2}
\begin{pmatrix}
	i\omega_n+\xi_\vp&-\Delta_0\\
	-\bar{\Delta}_0&i\omega_n-\xi_\vp
\end{pmatrix}
\end{equation}
Here $\omega_n$ is Matsubara frequency of Fermions.   And $E_\vp=\sqrt{\xi_\vp^2+\abs{\Delta_0}^2}$ And the saddle point equation is 
\begin{equation}
\nth{g}\bar{\Delta}_0=\frac{T}{L^d}\sum_{\vp,n}\frac{\bar\Delta_0}{\omega_n^2+E_\vp^2}
\end{equation}
The summation of Matsubara frequency can be evaluated and we find 
\begin{equation}
\nth{g}=\nth{L^d}\sum_{\vp}\frac{1-2n_f(E_p)}{2E_p}=\nth{L^d}\sum_{\vp}\frac{\tanh{(E_p/2T)}}{2E_p}
\label{eq:pathInt:gap}
\end{equation}
where $n_F$ is the fermi distribution function.  This is exactly the gap equation obtained from other methods as well.  On the other hand, $\nG$ in Eq. (\ref{eq:pathInt:nG})  is the inverse of fermion-fermion correlation of $\Psi$.  In mean field, $G_{0}$ as Eq. (\ref{eq:pathInt:G0}) can be diagnosed in momentum space with a canonical (Bogoliubov) transformation.  Nevertheless, poles is where  $\omega^2-E_\vp^2=0$ (with a analytic continue of $i\omega_{n}\rightarrow\omega+0^{+}$) as we can see from  Eq. (\ref{eq:pathInt:G0}) and therefore the spectrum of fermionic excitation is $\pm{}E_{p}$.  

And the number equation is $n=-\partial\Omega/\partial\mu$, at the saddle point the thermodynamic potential $\Omega_{0}=S[\Delta_{0}]/\beta$, and we have number equation
\begin{equation*}
n=-\nth{\beta}\tr\br{{G_{0}\pdiff{G_{0}^{-1}}{\mu}}}
\end{equation*}
similarly the summation over the Mastubara frequency can be evaluated and we have equation
\begin{equation}
n=\nth{L^{d}}\sum_{\vk}\mbr{1-\frac{\epsilon_{\vk}}{E_{\vk}}\tanh{(\frac{E_{\vk}}{2T})}}
\end{equation}

\subsubsection{Gaussian fluctuation and collective mode}
Once the mean field value is obtained, we can expand partition function Eq. (\ref{eq:pathInt:DeltaPF}) around it ($\Delta(\vr,\tau)=\Delta_{0}+\eta(\vr,\tau)$). The linear order of  expansion is zero because $\Delta_{0}$ is the saddle point.  The second order gives us the bilinear terms on $\eta$, i.e., correlation of bosonic field $\Delta$ (four-fermion correlation).  Note that here the hamiltonian only has an extreme-short-range ($\delta$) potential, therefore it cannot cover the situation of charged system where long-range Columnb interaction cannot be neglected.  We will discuss this later.  Nevertheless, it can be shown that a more realistic short-range potential only renormalizes some parameters in the following calculation while leaves the qualitative result unmodified.  

Notice that we can expand the second term in Eq. \ref{eq:pathInt:DeltaPF} for $\hat{G}{}^{-1}=\hat{G}_{0}^{-1}+\hat{K}$
\begin{equation}
\tr\ln \hat{G}^{-1}=\tr\ln\hat{G_{0}}^{-1}+\tr(\hat{G_{0}}\hat{K})-\nth{2}\tr(\hat{G_{0}}\hat{K}\hat{G_{0}}\hat{K})+\cdots
\end{equation}
In our case,
\begin{equation}
\hat{K}=\begin{pmatrix}
0&\eta\\
\eta^{*}&0
\end{pmatrix}
\end{equation}
Here the linear terms of $\hat{K}$ or $\eta$ ($\eta^{*}$) are zero as the saddle point condition.  So to the second order, the action is 
\begin{equation}\label{eq:pathInt:DeltaActionGaussian}
S[\Delta_{0},\eta,\eta^{*}]=S[\Delta_{0}]+
	\nth{2g}\tr(\hat{K}\hat{K})+\nth{2}\tr(\hat{G_{0}}\hat{K}\hat{G_{0}}\hat{K})
\end{equation}
Write the last term into the momentum representation
\begin{equation}
\tr(\hat{G_{0}}\hat{K}\hat{G_{0}}\hat{K})=\sum_{q,p}\Tr\br{G_{0}({p})K_{q}G_{0}{}({p-q})K_{-q}}
\end{equation}
Notice that the second ``$\Tr$'' and following ``$\Tr$'' in this section only runs in Nambu spinor space and $q={(\vq,q_{l})}$
, $p=(\vp,p_{n})$ are all four momentum, where $q_{l}$ is bosonic Matsubara frequency while $p_{n}$ is fermionic Matsubara frequency.
\begin{equation}
K_{q}=\begin{pmatrix}
0&\eta_{q}\\
\eta^{*}_{-q}&0
\end{pmatrix}
\end{equation}
If we introduce the a new vector 
\begin{equation}
\eta{(q)}=\begin{pmatrix}\eta_{q}\\\eta^{*}_{-q}\end{pmatrix}\qquad
\eta^{\dg}{(q)}=\begin{pmatrix}\eta^{*}_{q}&\eta_{-q}\end{pmatrix}
\end{equation}
the action can be rewritten into a more compact form
\begin{equation}
S[\Delta_{0},\eta,\eta^{*}]=S[\Delta_{0}]+\nth{2}\sum_{q}\Tr\mbr{\eta^{\dg}(q)\mathbf{M(q)}\eta(q)}
\end{equation}
Notice that we can always choose a real $\Delta_{0}$ and therefore $G_{0}{\ _{12}}(p)=G_{0}{\ _{21}}(p)$, we have 
\begin{equation}
\mathbf{M(q)}=
\begin{pmatrix}
\nth{g}+\sum_{p}G_{0}{\ }_{11}(p)G_{0}{\ }_{22}(p-q)&\sum_{p}G_{0}{\ }_{12}(p)G_{0}{\ }_{12}(p-q)\\
\sum_{p}G_{0}{\ }_{12}(p)G_{0}{\ }_{12}(p-q)&\nth{g}+\sum_{p}G_{0}{\ }_{11}(p-q)G_{0}{\ }_{22}(p)
\end{pmatrix}
\end{equation}
The summation over (fermionic) Matsubara frequency of $p_{n}$ can be carried out at zero temperature
\footnote{The summation of Matsubara frequency of function $h(i\omega_{n})$ is carried out by the normal method to multiplying $h(z)$ with fermi distribution function $n_{F}(z)$, and then find all residues with a contour over infinity.  However, due to zero temperature, the  $n_{F}(z)$ only nonzero at the negative singular points of $h(z)$, $-E_{\vk}$ in our case.  (The other singular point $E_{\vk}$ gives $n_{F}(E_{\vk})=0$ for zero temperature.}
\begin{equation}
\begin{split}
M_{11}(q)&=M_{22}(-q)\\
	&=\nth{g}+\sum_{\vp{,}p_{n}}G_{0}{\ }_{11}(p)G_{0}{\ }_{22}(p-q)\\
	&=\nth{g}+\sum_{\vp}\br{\frac{u^{2}u'^{2}}{iq_{l}-E-E'}-\frac{v^{2}v'^{2}}{iq_{l}+E+E'}}
\end{split}
\end{equation}
\begin{equation}
\begin{split}
M_{12}(q)&=M_{21}(q)\\
	&=\sum_{\vp{,}p_{n}}G_{0}{\ }_{12}(p)G_{0}{\ }_{12}(p-q)\\
	&=\sum_{\vp}uvu'v'\br{\nth{iq_{l}+E+E'}-\nth{iq_{l}-E-E'}}
\end{split}
\end{equation}
where $u=u_{\vp}$, $v=v_{\vp}$, $E=E_{\vp}$ and $u'=u_{\vp-\vq}$, $v'=v_{\vk-\vq}$, $E'=E_{\vk-\vq}$.  $u$, $v$, $E$ are as defined usually in BCS literature. 
\begin{equation}
v_{\vk}^{2}=1-u_{\vk}^{2}=\nth{2}\br{1-\frac{\xi_{\vk}}{E_{\vk}}}
\end{equation}
 The $G^{M}=\mathbf{M}^{-1}$ is the correlation function of $\eta$ (or $\Delta$) and its poles give the spectrum of collective mode as every  $\eta_{q}$ (or $\Delta_{q}$) involves many fermions moving in a coherent manner.  So the spectrum of collective modes can be determined by $\det{M(\omega,\vq)}=0$ after we analytically continue for the frequency $iq_{l}\rightarrow\omega+i0^{+}$.  
 
For low energy modes, where $\omega,\,\abs{\vq}^{2}\ll\min\bbr{E_{\vk}}$ both are much smaller than $\Delta_{0}$, we can expand $M$ with $\omega$ and $\vq$.  



 The above two equations only depend on 
