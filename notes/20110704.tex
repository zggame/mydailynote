Tan's universal papers\cite{Tan2008-1,Tan2008-2} basically just match a free part into boundary condition ($\eta$ integral).  And he split the space into short-range part ($I(\epsilon)$) and long-range free part ($D(\epsilon)$).  And finally, many body wave function in free part ($D(\epsilon)$) just need to match the Bethe-Pierels condition, i.e., $1/r-1/a$, the s-wave scattering condition.  If we took the close-channel part is small and put all things into the $I(\epsilon)$, the open-channel match this boundary condition.  How the narrow resonance plays here? This interpretation leads only to the broad resonance. To get to narrow resonance, we have to relax 