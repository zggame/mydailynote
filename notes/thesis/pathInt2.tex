% !TeX root =thesis.tex

\chapter{Path integral approach for two-channel}
For narrow resonance, atoms spend considerable weight in close-channel and the Pauli exclusion between two channels cannot be neglected.  A many-body framework needs to include both channels.  Path integral approach is particular suitable for crossover problem.  With the Hubbard-Stratonovich transformation, the order parameters can be introduced for 

For two-channel problem, we write down the Hamiltonian as
\begin{equation}
\begin{split}
H&=\int{d^{d}r}\bigg\{\sum_{j}\bar\psi_{j}\mbr{\nth{2m}(-i\nabla)^{2}-\mu+\eta_{j}}\psi_{j}\\
	&\qquad-U\bar\psi_{a}(r)\bar\psi_{b}(r)\psi_{b}(r)\psi_{a}(r)-V\bar\psi_{a}(r)\bar\psi_{c}(r)\psi_{c}(r)\psi_{a}(r)\\
	&\qquad-\mbr{Y\bar\psi_{a}(r)\bar\psi_{b}(r)\psi_{b}(r)\psi_{a}(r)+h.c.}
	\bigg\}\\
 &=\int{d^{d}r}\bigg\{\sum_{j}\bar\psi_{j}\mbr{\nth{2m}(-i\nabla)^{2}-\mu+\eta_{j}}\psi_{j}
 	-(\bar\psi\bar\psi)\mtrx{U&Y\\Y^{*}&V}(\psi\psi)
\end{split}
\end{equation}
Here $\eta_{j}$ is the Zeeman energy of the specific hyperfine species.  a is the common species of two channels, (a,b) is the open channel and (a,c) is the close channel.  We can introduce a unitary transformation $Q$ (mixing two channels) to diagonalize the interaction matrix into diagonal matrix $A$.
\begin{equation}
Q^{\dg}AQ=\mtrx{U&Y\\Y^{*}&V}\equiv{}\tilde{U}
\end{equation}
The finite temperature action is 
\begin{equation}\label{eq:pathInt2:actionFermi}
S(\bar\psi,\psi)=\int^{\beta}_{0}d\tau\int{d^{d}r}\mbr{\sum_{j}\bar\psi_{j}(\partial_\tau-\nth{2m}\nabla^{2}-\mu+\eta_{j})\psi_{j}
-(\bar\psi\bar\psi)Q^{\dg}AQ(\psi\psi)}
\end{equation}

Now $(\psi\psi)$ is column vector and $(\bar\psi\bar\psi)$ is row vector
\begin{equation*}
(\bar\psi\bar\psi)=\mtrx{\bar\psi_{a}\bar\psi_{b}&\bar\psi_{a}\bar\psi_{c}}
\qquad(\psi\psi)=\mtrx{\psi_{b}\psi_{a}\\\psi_{c}\psi_{a}}
\end{equation*}

Similar as in Sec. \ref{sec:pathInt}, we can try to work on the two-channel problem.  Here, the bosonic field is a 2-component vector   and start from a fat identity
\begin{equation}\label{eq:pathInt2:indetity}
1=\int{D(\Delta,\bar\Delta)}\exp(-\int{dx}\Delta^{\dg}A^{-1}\Delta)
\end{equation}
here $x$ is four-coordinator,  $\int{dx}=\int^{\beta}_{0}d\tau\int{d^{d}r}$.  All the constant is absorbed into measure of functional integral of $D(\Delta,\bar\Delta)$.

\[
\Delta^{\dg}=(\bar\Delta_{1},\bar\Delta_{2})\qquad\Delta=\begin{pmatrix}\Delta_{1}\\\Delta_{2}\end{pmatrix}
\]
We can make a shift in $\Delta$
\begin{equation}
\Delta\longrightarrow\Delta-A\,Q(\psi\psi)
\end{equation}
Write it into the matrix form
\begin{equation*}
\mtrx{\Delta_{1}\\\Delta_{2}}\longrightarrow
	\mtrx{\Delta_{1}\\\Delta_{2}}-\mtrx{A_{11}&0\\0&A_{22}}\mtrx{Q_{11}&Q_{12}\\Q_{21}&Q_{22}}
	\mtrx{\psi_{b}\psi_{a}\\\psi_{c}\psi_{a}}
\end{equation*}
\begin{equation*}
\mtrx{\bar\Delta_{1},\bar\Delta_{2}}\longrightarrow
	\mtrx{\bar\Delta_{1},\bar\Delta_{2}}-
	\mtrx{\bar\psi_{a}\bar\psi_{b}&\bar\psi_{a}\bar\psi_{c}}
	\mtrx{Q_{11}^{*}&Q_{21}^{*}\\Q_{12}^{*}&Q_{22}^{*}}\mtrx{A_{11}&0\\0&A_{22}}
\end{equation*}

Note that in principle, $\bar{\Delta}_{i}$ is not simple complex conjugate of $\Delta_{i}$ as they are related to fermion field $\psi$ and $\bar\psi$, which are both Grassman numbers and is not relevant to each other.  But in our solution, we actually take them as complex conjugate.  Now the fat identity Eq. \ref{eq:pathInt2:indetity} becomes 
\begin{equation}
1=\int{D(\Delta,\bar\Delta)}\exp\big\{-\int{dx}
	[\Delta^{\dg}A^{-1}\Delta-(\bar\psi\bar\psi)Q^{\dg}\Delta-\bar\Delta{Q}(\psi\psi)+(\bar\psi\bar\psi)Q^{\dg}AQ(\psi\psi)]\big\}
\end{equation}
And we have 
\begin{equation}
\exp[\int{dx}(\bar\psi\bar\psi)Q^{\dg}AQ(\psi\psi)]
=\int{D(\Delta,\bar\Delta)}\exp\big\{-\int{dx}
	[\Delta^{\dg}A^{-1}\Delta-(\bar\psi\bar\psi)Q^{\dg}\Delta-\bar\Delta{Q}(\psi\psi)]\big\}
\end{equation}
This is ready to be apply to the original action in Eq. \ref{eq:pathInt2:actionFermi}, 
\begin{equation}\label{eq:pathInt2:actionMix}
S_{\tau}(\bar\Delta,\Delta,\bar\psi_{i},\psi_{i})=\int^{\beta}_{0}d\tau\int{d^{d}r}\bbr{\sum_{j}\bar\psi_{j}(\partial_\tau-\nth{2m}\nabla^{2}-\mu+\eta_{j})\psi_{j}
+[\Delta^{\dg}A^{-1}\Delta-(\bar\psi\bar\psi)Q^{\dg}\Delta-\bar\Delta{Q}(\psi\psi)}
\end{equation}
We can introduce a spinor space similar to Nambu spinor representation in single-channel superconductivity.  
\begin{equation}
\bar\Psi=\mtrx{\bar\psi_{a}&\psi_{b}&\psi_{c}}\qquad\Psi=\mtrx{\psi_{a}\\\bar\psi_{b}\\\bar\psi_{c}}
\end{equation}
the action can be rewritten in a more compact form
\begin{equation}\label{eq:pathInt2:actionMix}
S(\bar\Delta,\Delta,\bar\psi_{i},\psi_{i})=\int^{\beta}_{0}d\tau\int{d^{d}r}
	\mbr{\bar\Delta{}A^{-1}\Delta-\bar\Psi\mathcal{G}^{-1}\Psi}
\end{equation}
where 
\begin{equation}
\mathcal{G}^{-1}=
\begin{pmatrix}
-\partial_{\tau}+\nth{2m}\nabla^{2}+\mu-\eta_{a}&Q^{*}_{11}\Delta_{1}+Q^{*}_{21}\Delta_{2}&Q^{*}_{12}\Delta_{1}+Q^{*}_{22}\Delta_{2}\\
Q^{}_{11}\bar\Delta_{1}+Q^{}_{21}\bar\Delta_{2}&-\partial_{\tau}-\nth{2m}\nabla^{2}-\mu+\eta_{b}&0\\
Q^{}_{12}\bar\Delta_{1}+Q^{}_{22}\bar\Delta_{2}&0&-\partial_{\tau}-\nth{2m}\nabla^{2}-\mu+\eta_{c}
\end{pmatrix}
\end{equation}
The action is now bilinear to $\Psi$ and we can integrate it out formally
\begin{equation}
S(\bar\Delta,\Delta)=\int{dx}
	\br{\bar\Delta{}A^{-1}\Delta-\tr\ln{G}^{-1}}
\end{equation}
This form can be simplified by introducing mixture within two channels.
\begin{equation}\label{eq:pathInt2:Ddef}
D\equiv\mtrx{D_{1}\\D_{2}}=Q^{\dg}\Delta
\end{equation}

We furthermore assume $\eta_{a}=\eta_{b}=0$, $\eta_{c}=\eta$. In frequency-momentum space, 
\begin{equation}\label{eq:pathInt2:nG}
\mathcal{G}^{-1}=i\omega_{n}I+
\begin{pmatrix}
-\xi_{k}&D_{1}&D_{2}\\
\bar{D}_{1}&+\xi_{k}&0\\
\bar{D}_{2}&0&+\xi_{k}+\eta
\end{pmatrix}
\end{equation}
here $\xi_{k}=\nth{2m}k^{2}-\mu$; and 
\begin{equation*}
\bar\Delta=\bar{D}\,Q^{\dg}\qquad\Delta=Q\,D
\end{equation*}
\[
\bar\Delta{}A^{-1}\Delta=\bar{D}Q^{\dg}A^{-1}QD=\bar{D}\tilde{U}^{-1}D
\]
Now we can change the functional variable into $D(\bar{D})$ 
\begin{equation}\label{eq:pathInt2:actionD}
S(\bar{D},D)=\int{dx}\br{\bar{D}\tilde{U}^{-1}D-\tr\ln{G}^{-1}}
\end{equation}
Comparing Eq. \ref{eq:pathInt2:nG} with Eq. \ref{eq:canonical:M} in Bogoliubov transformation treatment, we  see that $D_{1,2}$ have simple physical meaning in the mean-field expression of $\nG_{0}$ and therefore are more  direct counterpart of order parameter $\Delta$ in single-channel problem. 



\section{Diagonalize Green's function, Fermionic mode and Bogoliubov transform\label{sec:diagonalGreen}}
Here we will use the approach in Sec. \ref{sec:diagonalizeGreen1} for our problem.  In current problem, we need to diagonalize a $3\times3$ matrix Eq. \ref{eq:pathInt2:nG}, in another word, we need to figure out the Bogoliubov canonical transformation and the quasiparticle spectrum for this problem.   This involves solving a cubic equation. An exact solution exists in principle.  However,  it offers little intuition in writing the exact result, instead we content to find the approximated result when spectrum does not change too much from the broad-resonance, where the only effect of close-channel is modifying the effective interaction of open-channel. 


Within the assumption that spectrum  deviates not too much from na\"{i}ve broad-resonance solution, we  break down the unitary transformation into two 
pieces. 
\begin{equation}
B_k=L_k^{\dg}T_k^{\dg}G_k^{-1}T_kL_k
\end{equation} 
Here $T$ and $L$ are both unitary transformation.  We take $T$ as the transformation at lowest order, i.e., when we can ignore Pauli exclusion between channels. 
\begin{equation}
T_k=\mtrx{u_k&v_k&0\\-v_k&u_k&0\\0&0&1}
\end{equation}
Here we define 
\begin{gather}
E_{\vk}=(\xi_{\vk}^{2}+D_{1}^{2})^{1/2}\\
v_{\vk}^{2}=1-u_{\vk}^{2}=\nth{2}\br{1-\frac{\xi_{\vk}}{E_{\vk}}}
\end{gather}

In the broad-resonance, $L$ is simply identity matrix.  Here we will try to approximate it to the first order correction due to Pauli exclusion between two-channel.  
Apply $T_k$ onto $G^{-1}$, we have 
\begin{equation}\label{eq:pathInt2:G2}
T_k^{\dg}G_k^{-1}T_k=i\omega_nI+\mtrx{-E_k&0&u_kD_2\\0&+E_k&v_kD_2\\u_kD_2&v_kD_2&+\xi_k+\eta}
\end{equation}
We regard the off-diagonal elements as perturbation.  This matrix can then be diagonalized with another unitary transformation $L_{\vk}$ within the first order of $D_{i}^{2}/(E\eta)$
\begin{equation}
\begin{split}
B_{\omega_{n},\vk}&=i\omega_{n}I-
	\mtrx{\xi_{1}{}_{\vk}&0&0\\0&-\xi_{2}{}_{\vk}&0\\0&0&-\xi_{3}{}_{\vk}}\\
	&\approx{}i\omega_{n}I+
	\mtrx{-E_{\vk}-\frac{D_{2}^{2}u_{\vk}^{2}}{\eta}&0&0\\
	0&E_{\vk}-\frac{D_{2}^{2}v_{\vk}^{2}}{\eta}&0\\0&0&\xi_{\vk}+\eta+\frac{D_{2}^{2}}{2\eta}}
%	&=
%	\mtrx{i\omega_{n}-E_{\vk}&0&0\\0&i\omega_{n}+E_{\vk}&0\\0&0&i\omega_{n}+\eta}
%	+\mtrx{-\frac{D_{1}^{2}}{\eta}&0&0\\0&-\frac{D_{2}^{2}}{\eta}&0\\0&0&+\frac{D_{1}^{2}+D_{2}^{2}}{2\eta}}\\
%	&\equiv{}B^{(0)}_{\vk}+B^{(1)}_{\vk}
\end{split}	
\end{equation}
Here we choose the sign of $\xi_{1,2,3}$  to make their lowest order in Feshbach resonance positive.  
we have 
\begin{align}\label{eq:pathInt2:xiExpand}
\xi_{1\vk}&\approx{}E_{\vk}+\frac{D_{2}^{2}u_{\vk}^{2}}{\xi_{\vk}+\eta}&\equiv{}&E_{\vk}+\gamma_{1\vk}\\
\xi_{2\vk}&\approx{}E_{\vk}-\frac{D_{2}^{2}v_{\vk}^{2}}{\xi_{\vk}+\eta}&\equiv{}&E_{\vk}+\gamma_{2\vk}\\
\xi_{3\vk}&\approx{}\xi_{\vk}+\eta-\frac{D_{2}^{2}}{2(\xi_{\vk}+\eta)}&\equiv{}&\xi_{\vk}+\eta+\gamma_{3\vk}
\end{align}

\begin{equation}\label{eq:pathInt2:L1}
L_{\vk}\approx{}I+
\mtrx{0&-\frac{D_{1}{}D_{2}{}}{4E^{2}_{\vk}}&u_{\vk}\\
\frac{D_{1}{}D_{2}{}}{4E^{2}_{\vk}}&0&v_{\vk}\\
-u_{\vk}&-v_{\vk}&0
}\frac{D_{2}{}}{\eta}
\equiv{}I+\delta_{k}\qquad
L^{\dg}_{\vk}=I-\delta_{\vk}
\end{equation}
Here we use $uv=D_{1}/2E$.  Please see Appendix \ref{sec:diagonalize} for details.  Note that $L$ and $L^{\dg}$ are unitary only to the first order of $D_{i}/\eta$
And now it is easy to express the Green's function
\begin{equation}
G_{\vk}=T_{\vk}L_{\vk}B_{\vk}^{-1}L_{\vk}^{\dg}T_{\vk}^{\dg}
\end{equation}
This is ready to be expanded over the perturbation in order of  $D_{i}^{2}/(E\eta)$ or $D_{i}/\eta$.  It is easy to see that all $\omega_{n}$ dependence concentrates on $B_{\vk}$, which simplifies the Matsubara frequency summation considerably.   
\begin{equation}\label{eq:pathInt2:Gexpand}
G_{k}=T_{\vk}B_{\omega_{n},\vk}^{-1}T_{\vk}^{\dg}+T_{\vk}\delta_{\vk}B_{\omega_{n},\vk}^{-1}T_{\vk}^{\dg}
	-T_{\vk}B_{\omega_{n},\vk}^{-1}\delta_{\vk}T_{\vk}^{\dg}=G_{\vk}^{(0)}+G_{\vk}^{(1)}
\end{equation}

From another point, we can interprate the above transform $T_\vk{}L_\vk$ as Bogoliubov canonical transformation and $B$ gives us spectrum of fermionic quasi-particle excitation.  
%We only need to keep the zeroth order term for $B_{\vk}^{-1}$ for first order expansion.  
\section{Mean field equation (Excct)}
Use the same techniques as Eq. (\ref{eq:pathInt:diffTr}), we have two equations for $D_{1}$ and $D_{2}$,
 \begin{align}
\frac{\delta}{\delta{}D_{1}}:&\qquad&
(\tilde{U}^{-1})_{11}\bar{D}_{1}+(\tilde{U}^{-1})_{21}\bar{D}_{2}-\tr\mbr{{G_{0}}\cdot\cmtrx{0&1&0\\0&0&0\\0&0&0}}=0
\label{eq:pathInt2:mf01}\\
\frac{\delta}{\delta{}D_{2}}:&\qquad&
(\tilde{U}^{-1})_{12}\bar{D}_{1}+(\tilde{U}^{-1})_{22}\bar{D}_{2}-\tr\mbr{{G_{0}}\cdot\cmtrx{0&0&1\\0&0&0\\0&0&0}}=0
\label{eq:pathInt2:mf02}
 \end{align}
 
 
  If we take $D$ as real constant\footnote{Actually $D_{2}{_{\vk}}$ cannot be constant at high momentum.  However, for the momentum we are interested, i.e. the momentum lower or in the order of Fermi momentum, it slowly varies.  Therefore  it is reasonable to take it as constant.},     we can find the mean field result. Eq. (\ref{eq:pathInt2:nG}) can be inverted to get $G$.  The inversion is quite tedious, but fortunately, we only need two elements of the $G$ matrix ($G_{0\, (21)}$ and $G_{0 \,(31)}$).  The final mean-field equations are (all $D_{i}$'s are taken as real, see Appendix \ref{sec:pathInt2:deriveMF} for detail) 
  \begin{equation}\label{eq:pathInt2:mf}
\mtrx{D_1\\D_2}=\mtrx{U&Y\\Y^{*}&V}\sum_{\vk}\mtrx{h_{1\vk}\\h_{2\vk}}
\end{equation}
  where 
  \begin{gather}
  h_{1\vk}=D_{1}\frac{\xi_{1\,\vk}+\xi_{\vk}+\eta}{(\xi_{1\,\vk}+\xi_{2\,\vk})(\xi_{1\,\vk}+\xi_{3\,\vk})}\label{eq:pathInt2:h1}\\
  h_{2\vk}=D_{2}\frac{\xi_{1\,\vk}+\xi_{\vk}}{(\xi_{1\,\vk}+\xi_{2\,\vk})(\xi_{1\,\vk}+\xi_{3\,\vk})}\label{eq:pathInt2:h2}
  \end{gather}
At high-momentum, both $h_{\vk}$ behaves as $1/\epsilon_{\vk}$ and therefore diverges for summation in 3D.  However, they can be  renormalized as we recognize the interaction is not really contact and $D_{i\vk}$'s decay at high-momentum.  

It is interesting to look at Eq. \ref{eq:pathInt2:h2} more carefully, at low momentum, $\xi_k<0$ and $h_{2\,\vk}$ is close to 0; at higher momentum where $\epsilon_{\vk}>\mu,D_i$, $\xi_{\vk}\approx\xi_{2\,\vk}$ and $  h_{2\vk}\approx\frac{D_2}{(2\epsilon+\eta)}$, this coincides with two-particle wave function for such range (not too high-momentum);  at very high momentum where $\epsilon_{\vk}>\eta$, $D_2$ can no longer be treated as a constant and decays with energy, its specific form is determined by the specific short-range shape of potential. We expect the high-momentum normalization follows the middle-momentum normalization when comparing to two-body bound state.   Only in the low-momentum, it differs from the two-body wave function and that is where 

If we simply take the lowest order of $xi_{i}$, and ignore the close-channel, it is easy to identify $h_{1\vk}\approx{D_{1}}/(2E_{\vk})$ as $F_{\vk}$ in single channel problem.  Therefore, it has a general behavior at low-energy.  

 
\subsection {Renormalization of mean field equation}
One key assumption of the problem is the potential is short-range, and the two-body bound-states of isolated close-channel $\phi$ are much smaller in size than particle-particle distance.  In the short-range, the specific potential determine the specific shape of the wave-functions, however, at long-wave-length, i.e. low momentum, the behave should be quite universal for these bound states except an overall normalization factor.  At the lowest order $\phi_{0\vk}\sim{A}\nth{\kappa^{2}}$ where $\kappa$ is the momentum scale related to binding energy or tuning, $\kappa^{2}/2m=E_{b}$.  This quantity is actually very small as $\kappa{}a_{0}\ll1$ where $a_{0}$ is the particle-particle distance.  Across the crossover, the internal structure of wave-function does not change much, but the overall normalization does vary along tuning.  However, \emph{it is always small across the region interested even when close-channel weight is large.} 
 
Let us look at $h_{2\vk}$ first, it can be rewritten into such form
\begin{equation}\label{eq:pathInt2:h2D2}
 h_{2\vk}\approx\frac{D_{2}}{(\xi_{1\,\vk}+\xi_{3\,\vk})}u_{\vk}^{2}
\end{equation}
In three-species many-body problem, the above conclusion is still valid except at low-momentum where Pauli exclusion between two-channel is severe. We can see the ${D_{2}}/{(\xi_{1\,\vk}+\xi_{3\,\vk})}$ is just the same as two-body wave function $\phi_{0}$, while the extra factor $u_{\vk}^{2}$ describes the Pauli exclusion between two channels.  $D_{2}$ is the product of three factors: 1, the normalization ($A$) factor as in two-body wave function normalized to 1(Appendix \ref{sec:pathInt2:short-range}); 2, total number of particles in close-channel; 3, $u_{\vk}^{2}$ factor related to Pauli exclusion between channels.  At the low-momentum and BCS, open-channel weight is large, phase space for close-channel is limited, this is shown mathematically as small $u_{\vk}^{2}$ factor.   At high-momentum, $u_{\vk}^{2}\approx1$ and close-channel wave-function just follow two-body counterpart with a different normalization factor.   The two-body wave function for isolated close-channel spreads in large momentum space and has small weight in low momentum (below $k_{F}$), and therefore, we expected $u_{\vk}^{2}$ has small correction only.  

It is not hard to see that $D_{2}$ is actually closely related to ``\emph{integrated contact intensity}'', $C$, in Tan's works (\cite{Tan2008-1,Tan2008-2}).  In his work, Tan concluded  the high-end of relative-momentum distribution asymptotically approaches  $C/k^{4}$.  Note that the high-end in his paper means momentum lower than $1/r_{c}$, but higher than any other scale ($1/a_{s}$, $1/a_{0}$).  In such scale, $u_{k}\approx1$ and especially, for close-to-threshold bound-state, $E_{b}\ll{}\frac{\hbar^{2}}{mr_{c}^{2}}$, so $C=D_{2}/2m$ for the close-channel.   

We can rewrite the mean-field equations \ref{eq:pathInt2:mf} as 
\begin{align}
D_{1}&=\sum{}Uh_{1\vk}+\sum{}Yh_{2\vk}\label{eq:pathInt2:mfopen}\\
D_{2}&=\sum{}Yh_{1\vk}+\sum{}Vh_{2\vk}\label{eq:pathInt2:mfclose}
\end{align}
Let us look at the second equation Eq. \ref{eq:pathInt2:mfclose} first, as discussed before, $h_{2}$ is similar to $\phi^{(0)}$ in two-body wave function. More specifically
\begin{equation}
h_{2\vk}=\alpha\phi^{(0)}_{\vk}u_{\vk}^{2}+o(\frac{E_{F}}{\eta})
\end{equation}
 where 
\begin{equation}\label{eq:pathInt2:phi}
-E_{b}^{(0)}\phi_{0\,\vp}=\epsilon_{\vp}\phi_{0\,\vp}-\sum_{\vk}V \phi_{0\,\vk}
\end{equation}
In the interested region, biding energy $E_{b}^{(0)}$ is always close to absolute detuning $\eta$.   
Use Eq. \ref{eq:pathInt2:h2D2}, we can write 
\begin{equation*}
D_{2}\approx{}h_{2\vk}\frac{(\epsilon_{\vk}-2\mu+\eta)}{u_{\vk}^{2}}
\end{equation*}
Combine the above equation with Eq. \ref{eq:pathInt2:mfclose}, we get 
\begin{equation*}
h_{2\vk}\frac{(\xi_{1\vk}+\xi_{3\vk})}{u_{\vk}^{2}}=\sum{}Yh_{1\vk}+\sum{}Vh_{2\vk}
\end{equation*}
Multiply both sides with $\phi^{*}$ and integrate over the momentum,
\begin{equation}
\alpha\br{-E_{b}+\eta-2\mu-\avs{\phi}{(E_{\vk}-\xi_{\vk})}{\phi}+\avs{\phi}{v_{\vk}^{2}V}{\phi}}=\avs{\phi}{Y}{h_{1}}
\end{equation}
Comparing this to the two-body problem, the detuning is shifted by many-body effects ($\mu$ and two average terms on left).  
\begin{equation}\label{eq:pathInt2:alpha}
\alpha=\frac{\avs{\phi}{Y}{h_{1}}}{\br{-E_{b}+\eta-2\mu-\lambda_{1}}}
\end{equation}
\begin{equation}\label{eq:pathInt2:lambda1}
\lambda_{1}\equiv\avs{\phi}{(E_{\vk}-\xi_{\vk})}{\phi}-\avs{\phi}{v_{\vk}^{2}V}{\phi}
\end{equation}

\mycomment{It is not clear that whether $\lambda_{1}$ depends on energy (momentum), but the dependence should be weak even if it does.  }
Put the shift on $\alpha$ aside for now and get back to the other equation Eq. \ref{eq:pathInt2:mfopen}.  With the above equation and Eq. \ref{eq:pathInt2:h1}
\begin{equation*}
D_{1}=U\ket{h_{1\vk}}+\frac{Y\ket{\phi}\bra{\phi}{Y}}{\br{-E_{b}+\eta-2\mu-\lambda_{1}}}u_{\vk}^{2}\ket{{h_{1}}}
\end{equation*}
Comparing this with two-body problem, we can see the detuning part (denominator of the second term) is shifted by $2\mu+\lambda_{1}$ and there is an extra $u_{\vk}^{2}$ term introduced by many-body effect.  Nevertheless, none of these affect the high-momentum behavior, therefore, the equation can be normalized exactly as in two-body problem by introducing the long-wave-length s-scattering length $a_{s}$.  We rewrite the above equation
\begin{equation*}
D_{1}=\br{U+\frac{Y\ket{\phi}\bra{\phi}{Y}}{\br{-E_{b}+\eta-2\mu-\lambda_{1}}}}\ket{{h_{1}}}
	-\frac{Y\ket{\phi}\bra{\phi}{Y}}{\br{-E_{b}+\eta-2\mu-\lambda_{1}}}v_{\vk}^{2}\ket{{h_{1}}}
\end{equation*}
The second term has no divergence at high momentum in 3D due to the extra $v_{k}^{2}$, actually this factor decreases quickly over ``gap'' $D_{1}$, therefore, the summation is essentially only over low-momentum.   Furthermore, considering the short-range nature, this term varies slowly over momentum.  

Multiply both side with $(1+GT)$, 
\begin{equation*}
(1+GT)D_{1}=Th_{1}-\lambda_{2}
\end{equation*}
and 
\begin{equation}
D_{1}=T\sum(h_{1}-GD_{1})-\lambda_{2}
=D_{1}\frac{4\pi{a_{s}}}{m}\sum(\frac{\xi_{1\,\vk}+\xi_{\vk}+\eta}{(\xi_{1\,\vk}+\xi_{2\,\vk})(\xi_{1\,\vk}+\xi_{3\,\vk})}-\nth{2\epsilon_{\vk}})
	-\lambda_{2}
\end{equation}
where 
\begin{equation}\label{eq:pathInt2:lambda2}
\lambda_{2}=(1+GT)\frac{Y\ket{\phi}\bra{\phi}{Y}}{\br{-E_{b}+\eta-2\mu-\lambda_{1}}}v_{\vk}^{2}\ket{{h_{1}}}
\end{equation}
collecting everything, we have the renormalized equation
\begin{equation}
1=\frac{4\pi{\tilde{a}_{s}}}{m}\sum(\frac{\xi_{1\,\vk}+\xi_{\vk}+\eta}{(\xi_{1\,\vk}+\xi_{2\,\vk})(\xi_{1\,\vk}+\xi_{3\,\vk})}-\nth{2\epsilon_{\vk}})
	-\frac{\lambda_{2}}{D_{1}}
\end{equation}
Note that $\tilde{a}_{s}$ corresponds to the two-body s-wave scattering length at detuning shifted by $2\mu+\lambda_{1}$.
Now we can expand the first term in the parentheses, using Eq. \ref{eq:pathInt2:xiExpand}, and keep in mind $E_{\vk}\ll{\eta}$ at low momentum where summation is about, we have
\begin{equation}\label{eq:pathInt2:gapRenorm}
1=\frac{4\pi{\tilde{a}_{s}}}{m}\sum(\nth{2E_{\vk}}-\nth{2\epsilon_{\vk}}-\frac{D_{1}^{2}-D_{2}^{2}}{4\eta{E_{\vk}^{2}}})
	-\frac{\lambda_{2}}{D_{1}}
\end{equation}
The correction term does not have divergence in summation of high-momentum. 
In summary there are several difference of gap equation here comparing to single-channel problems:
\begin{enumerate}
\item\label{item:pathInt2:mu}The shift of $2\mu$ in detuning;
\item The extra shift of $\lambda_{1}$ in detuning;
\item The extra term $\lambda_{2}$ in Eq. \ref{eq:pathInt2:gapRenorm};
\item The extra term  in summation of Eq. \ref{eq:pathInt2:gapRenorm};
\end{enumerate}
Item \ref{item:pathInt2:mu} corresponds to the simple many-body effect one universal; while the rest corrections are unique for Pauli-exclusion between channels in three-species problem.

Furthermore,  close look into $\lambda_1$ and $\lambda_2$ (Appendix \ref{sec:pathInt2:lambda}) reveals that they varies slowly with energy/momentum and is a function of density.  They describe  fundamental many-body effects, and do not have a good correspond in two-body problem.  



\subsection{Number equation}
We can derive two number equations for each channel.  
\begin{gather*}
\sum_{k}G_{22}e^{(-i\omega_n\delta_-)}=N_{open}\qquad
\sum_{k}G_{33}e^{(-i\omega_n\delta_-)}=N_{close}
\end{gather*}
Note that the Matsubara summation is formally diverged and we need to put in a small negative parts into the summation.  It is negative because $\Psi_2=\bar\psi_b$, $\Psi_3=\bar\psi_c$.  In the zero temperature Matsubara summation, we just need to considering the positive root, $\xi_{1\,\vk}$.  It is straightforward to find 
\begin{gather}
N_{open}=\sum_{\vk}\frac{(\xi_{1\,\vk}-\xi_{\vk})(\xi_{1\,\vk}+\xi_{\vk}+\eta)-D_2^2}{(\xi_{1\,\vk}+\xi_{2\,\vk})(\xi_{1\,\vk}+\xi_{3\,\vk})}\\
N_{close}=\sum_{\vk}\frac{(\xi_{1\,\vk}-\xi_{\vk})(\xi_{1\,\vk}+\xi_{\vk})-D_1^2}{(\xi_{1\,\vk}+\xi_{2\,\vk})(\xi_{1\,\vk}+\xi_{3\,\vk})}
=\sum_{\vk}\frac{\xi_{1\,\vk}^2-E_{\vk}^2}{(\xi_{1\,\vk}+\xi_{2\,\vk})(\xi_{1\,\vk}+\xi_{3\,\vk})}\label{eq:pathInt2:numClose}
\end{gather}
Let us look at the equation of close-channel first, if we expand $\xi_{i\,\vk}$, the lowest order is 
\begin{equation}
N_{close}=\sum_{\vk}\frac{\gamma_{1\,\vk}}{(E_{\vk}+\xi_{\vk}+\eta)}=\sum_{\vk}\frac{D_{2}^2u_{\vk}^{2}}{(\xi_{\vk}+\eta)(E_{\vk}+\xi_{\vk}+\eta)}
\end{equation}
\mycomment{This is consistent with Eq. \ref{eq:pathInt2:h2} and \ref{eq:pathInt2:h2D2} if we assume $N_{close}\approx\sum{h_{2}^{2}}$.  The weight spread in a very large range of momentum ($\sim\eta$),  both of them has $D_{2}^{2}/\eta^{2}$ for summand. }
Similar to the argument in gap equation, the summand  above is mostly useful at energy below $\eta$, where the component is small everywhere comparing to 1, however, the summation runs over high-momentum ($\sim\eta$), where $D_2$ should no longer be regarded as constant and summation can gives $N_{close}$ comparable to total number.  \mycomment{ This might not be correct statement.  Maybe it is just fine as it properly converges. } This summation shows that the close-channel state is small and therefore the summation runs up to high momentum, where mostly is determined by two-body physics.  


\begin{equation}
\begin{split}
N_{open}=&\sum_\vk\mbr{\frac{E_\vk-\xi_\vk}{2E_\vk}+\frac{\delta_{1\vk}+\delta_{2\vk}}{2E_\vk}
	-\frac{(E_\vk-\xi_\vk)(\delta_{1\vk}+\delta_{2\vk})}{4E_\vk^2}
	-\frac{(E_\vk-\xi_\vk)(\delta_{1\vk}+\delta_{3\vk})}{2E_\vk(\xi_\vk+E_\vk+\eta)}}\\
	=&\sum_\vk\mbr{\frac{E_\vk-\xi_\vk}{2E_\vk}+\frac{D_1^2-D_2^2}{2E_\vk(\eta+\xi_\vk)}
	-\frac{(E_\vk-\xi_\vk)(D_1^2-D_2^2)}{4\eta{}E_\vk^2}
	+\frac{(E_\vk-\xi_\vk)(D_2^2)}{2E_\vk\eta(\xi_\vk+E_\vk+\eta)}}
\end{split}
\end{equation}
%\section{Mean field result}
 Use the same techniques as Eq. (\ref{eq:pathInt:diffTr}), we have two equations for $D_{1}$ and $D_{2}$,
 \begin{align}
\frac{\delta}{\delta{}D_{1}}:&\qquad&
(\tilde{U}^{-1})_{11}\bar{D}_{1}+(\tilde{U}^{-1})_{21}\bar{D}_{2}-\tr\mbr{{G_{0}}\cdot\cmtrx{0&1&0\\0&0&0\\0&0&0}}=0\\
\frac{\delta}{\delta{}D_{2}}:&\qquad&
(\tilde{U}^{-1})_{12}\bar{D}_{1}+(\tilde{U}^{-1})_{22}\bar{D}_{2}-\tr\mbr{{G_{0}}\cdot\cmtrx{0&0&1\\0&0&0\\0&0&0}}=0
 \end{align}


 If we take $D$ as real constant\footnote{Actually $D_{2}{_{\vk}}$ cannot be constant at high momentum.  However, for the momentum we are interested, i.e. the momentum lower or in the order of Fermi momentum, it slowly varies.  Therefore  it is reasonable to take it as constant.},     we can find the mean field result.  Usting Eq. (\ref{eq:pathInt2:Gexpand}),
 
 \begin{equation}\label{eq:pathInt2:G0}
 \begin{split}
 G_{0}=&
 \begin{pmatrix}
 {g_{1\,\vk}}&
-u_{k}v_{k}{g_{2\,\vk}}&0\\
-u_{k}v_{k}{g_{2\,\vk}}& {g_{3\,\vk}}&0\\
  0&0&\nth{i\omega_{n}-\xi_{3}{}_{\vk}}
 \end{pmatrix}\\
&+\frac{D_{2}}{\eta}
\begin{pmatrix}
\frac{D_{1}^{2}D_{2}}{4E_{\vk}^{3}}g_{2\,\vk}&\frac{D_{1}D_{2}\xi_{\vk}}{4E_{\vk}^{3}}g_{2\,\vk}&-g_{1\,\vk}+\nth{i\omega_{n}-\xi_{3}{}_{\vk}}\\
\frac{D_{1}D_{2}\xi_{\vk}}{4E_{\vk}^{3}}g_{2\,\vk}&-\frac{D_{1}^{2}D_{2}}{4E_{\vk}^{3}}g_{2\,\vk}&u_{k}v_{k}{g_{2\,\vk}}\\
-g_{1\,\vk}+\nth{i\omega_{n}-\xi_{3}{}_{\vk}}&u_{k}v_{k}{g_{2\,\vk}}&0
\end{pmatrix}
\end{split}
 \end{equation}
\begin{gather}
g_{1}{}_{\vk}=\frac{u_{\vk}^{2}}{i\omega_{n}-\xi_{1}{}_{\vk}}+\frac{v_{\vk}^{2}}{i\omega_{n}-\xi_{2}{}_{\vk}}\\
g_{2}{}_{\vk}=\nth{i\omega_{n}-\xi_{1}{}_{\vk}}-\nth{i\omega_{n}-\xi_{2}{}_{\vk}}\\
g_{3}{}_{\vk}=\frac{v_{\vk}^{2}}{i\omega_{n}-\xi_{1}{}_{\vk}}+\frac{u_{\vk}^{2}}{i\omega_{n}-\xi_{2}{}_{\vk}}
\end{gather}

 \begin{align}
\tr\mbr{G_{0}\cdot\cmtrx{0&1&0\\0&0&0\\0&0&0}}&=
\sum_{\vk}\sum_{\omega_{n}}
	\mbr{(\nth{i\omega_{n}-\xi_{1}{}_{\vk}}-\nth{i\omega_{n}-\xi_{2}{_{\vk}}})
	(-\frac{D_{1}}{2E_{\vk}}+\frac{D_{1}D_{2}^{2}\xi_\vk}{4E_{\vk}^{3}\eta})}\\
	&=\sum_{\vk}(\frac{D_{1}}{2E_{\vk}}-\frac{D_{1}D_{2}^{2}\xi_\vk}{4E_{\vk}^{3}\eta})\\
\tr\mbr{G_{0}\cdot\cmtrx{0&0&1\\0&0&0\\0&0&0}}&=
\sum_{\vk}\sum_{\omega_{n}}
\mbr{\nth{i\omega_{n}-\xi_{3}{}_{\vk}}-
\frac{u_{\vk}^{2}}{i\omega_{n}-\xi_{1}{}_{\vk}}-\frac{v_{\vk}^{2}}{i\omega_{n}-\xi_{2}{}_{\vk}}}\frac{D_{2}}{\eta}\label{eq:pathInt2:F20}\\
&=\sum_{\vk}\frac{D_{2}}{\eta}u_{\vk}^2
  \end{align}
Here we take the interest only in $T=0$, so we only need to consider the negative frequencies ($\xi_{2\,\vk}$, $\xi_{3\,\vk}$) for summation of Matsubara frequency. Note that the second summation diverges badly in high-momentum. %this is because we take $D_2$ as constant.  It decreases at high-energy in the scale of $\eta$ and needs to be regularize carefully.  
We notice that this term is controlled by parameter $D_{2}/\eta$, this actually goes back to the fact that we only keep the first order in $L$ expansion (Eq. \eqref{eq:pathInt2:L1}), which is only valid for energy smaller or in order of Fermi energy.  In a more careful study, this term should be like $D_{2}/(\epsilon_{k}+\eta)$, is approximately $D_{2}/\eta$ when the interesting region  is lower or at the Fermi energy.   We can reestablish the $F_{k}\propto1/\epsilon_{k}$ if we retain all terms in the expansion of $L$, i.e. inverting Green's function $G$ exactly.       Indeed it should be just proportional to simple bounded two-body solution of isolated close-channel, $\phi_{0\,\vk}$ at high-momentum, which is not of interest for the many-body problem. 
 Another interesting thing about this term is the $u_\vk$ factor, which is small below chemical potential $\mu$ in BCS side.  This shows the fact that the low momentum is filled mostly by open-channel and close-channel is crowded out.  However, this does not affect close-channel too much as it is much more extended in momentum space and its occupation over each level is low due to the smaller size of close-channel bound state.  With above result, we can rewrite gap equations as 
\begin{equation*}
\tilde{U}^{-1}D-\mtrx{\sum_{\vk}(\frac{D_{1}}{2E_{\vk}}-\frac{D_{1}D_{2}^{2}\xi_\vk}{4E_{\vk}^{3}\eta})\\\sum_{\vk}\frac{D_{2}}{\eta}u_{\vk}^2}=0
\end{equation*}
 We can multiply it with $\tilde{U}$ and we have 
\begin{equation}\label{eq:pathInt2:meanfield}
\mtrx{D_1\\D_2}=\mtrx{U&Y\\Y^{*}&V}\sum_{\vk}\mtrx{\frac{D_{1}}{2E_{\vk}}-\frac{D_{1}D_{2}^{2}\xi_\vk}{4E_{\vk}^{3}\eta}\\
\frac{D_{2}}{\eta}u_{\vk}^2}
\end{equation}
 This equation can be renormalized in a very similar fashion as variation method. Notice that the second term in the first component describes the Pauli exclusion between two channels.  

\subsection{Renormalizing mean-field equation\label{sec:pathIntRenorm}}
We define two quantities for the summand in the mean-field equation Eq. \ref{eq:pathInt2:meanfield}.
\begin{gather}
F_{1\,\vk}=\frac{D_{1}}{2E_{\vk}}-\frac{D_{1}D_{2}^{2}\xi_\vk}{4E_{\vk}^{3}\eta}\\
F_{2\,\vk}=\frac{D_{2}}{\eta}u_{\vk}^2\label{eq:pathInt2:F2k}
\end{gather}
Considering the argument from last section, we modify equation of $F_2$ to the following,
\begin{equation}
\tilde{F}_{2\,\vk}=\frac{D_{2}}{\eta+2\epsilon_{\vk}}u_{\vk}^2\label{eq:pathInt2:F2kMod}
\end{equation}
And now $F_2$ has the same behavior at high momentum as $F_1$, actually this is the behavior we expected for $\epsilon_k<\eta$, it falls off even faster beyond energy scale of $\eta$, which is determined by the specific shape of close-channel potential.

We can rewrite the mean-field equation as
\begin{gather}
D_{1}=\sum_{\vk}(U F_{1\,\vk}+Y \tilde{F}_{2\,\vk})\label{eq:pathInt2:D2}\\
D_{2}=\sum_{\vk}(Y^{*} F_{1\,\vk}+V \tilde{F}_{2\,\vk})\label{eq:pathInt2:D2}
\end{gather}
Here we see the $F_{1\,\vk}$ and $\tilde{F}_{2}$  both go as $1/\epsilon_{\vk}$  at high-momentum.  %To resolve divergence in the summation of $F_{2\,\vk}$ we restore the momentum dependence on $D_{2\,\vk}$ and therefore $F_{2\,\vk}$.  
  And we can see $\frac{D_{2}}{\eta}$ is actually a good approximation at low-momentum, where kinetic energy is much smaller than $\eta$.  We can rewrite $\tilde{F}_{2}=\alpha\phi_{0\,\vk}u_{\vk}^{2}$. Eq. (\ref{eq:pathInt2:D2}) can be rewritten as
\begin{equation*}
\eta{}F_{2\,\vp}=\sum_{\vk}(Y^{*} F_{1\,\vk}+V F_{2\,\vk})\label{eq:pathInt2:D2}
\end{equation*}
comparing this to a two-body \sch equation
\begin{equation}
-E_{0}\phi_{0\,\vp}=\epsilon_{\vp}\phi_{0\,\vp}-\sum_{\vk}V \phi_{0\,\vk}
\end{equation}
where $E_{0}$ is the binding energy of two-body bound state of isolated close-channel.   




\section{Collective mode}
%\subsection{Simple approach for collective mode}
With the above arrangement, it is easy to study the collective mode of the system, which corresponding the second order expansion over $D$. We introduce the deviation over $\nG=G_{0}^{-1}+K_{q}$, where $G_{0}$ is described by real constant $D_{1,2}$
\begin{equation}
K_\vq=\mtrx{0&\theta_{1\,\vq}&\theta_{2\,\vq}\\\theta_{1\,-\vq}^*&0&0\\\theta_{2\,-\vq}^*&0&0}
\end{equation}
Follow the same approach in single-channel (Sec. \ref{sec:collective1}), we need to calculate $\tr(\hat{G_{0}}\hat{K}\hat{G_{0}}\hat{K})$, 
\begin{equation}
\tr({G_{0\,k}}{K_{q}}{G}_{0\,k+q}{K_{-q}})=\Theta_{q}^{\dg}M_{q}\Theta_{q}
\end{equation}
where 
\begin{equation}
\Theta_{q}=\mtrx{\theta^{*}_{1\,\vq}\\\theta^{}_{1\,-\vq}\\\theta^{*}_{2\,\vq}\\\theta^{}_{2\,-\vq}}
\qquad
\Theta_{q}^{\dg}=\mtrx{\theta^{}_{1\,\vq}&\theta^{*}_{1\,-\vq}&\theta^{}_{2\,\vq}&\theta^{*}_{2\,-\vq}}
\end{equation}
At the lowest order, if we take $L_{\vq}=I$,
\begin{equation}
M^{(0)}_{q}=\sum_{\vk}
\begin{pmatrix}
-u_{\vk}^{2}u_{\vk+\vq}^{2}f_{1\,q}+v_{\vk}^{2}v_{\vk+\vq}^{2}f_{2\,q}&u_{\vk}v_{\vk}u_{\vk+\vq}v_{\vk+\vq}(f_{2\,q}-f_{1\,q})&0&0\\
u_{\vk}v_{\vk}u_{\vk+\vq}v_{\vk+\vq}(f_{2\,q}-f_{1\,q})&u_{\vk}^{2}u_{\vk+\vq}^{2}f_{2\,q}-v_{\vk}^{2}v_{\vk+\vq}^{2}f_{1\,q}&0&0\\
0&0&-u_{\vk}^{2}f_{3\,q}&0\\
0&0&0&u_{\vk+\vq}^{2}f_{4\,q}
\end{pmatrix}
\end{equation}
where
\begin{gather}
f_{1\,q}=\nth{i q_{l}+\xi_{1\,\vk}-\xi_{2\,\vk+\vq}}\\
f_{2\,q}=\nth{i q_{l}-\xi_{1\,\vk+\vq}+\xi_{2\,\vk}}\\
f_{3\,q}=\nth{i q_{l}+\xi_{1\,\vk}-\xi_{3\,\vk+\vq}}\\
f_{4\,q}=\nth{i q_{l}-\xi_{1\,\vk+\vq}+\xi_{3\,\vk}}
\end{gather}
The two channels are completely decoupled at this level and the upper quarter is exactly same as single channel case if we ignore the correction over quasi-particle spectrum.   We can also expand each $f_{i\,q}$ to get the first order about $D_{2}/\eta$. (We take $q=0$ for it. See Appendix \ref{sec:expansionM})
\begin{equation}
M^{(1a)}=\sum_{\vk}
\begin{pmatrix}
(1-\frac{D_{1}^{2}}{2E_{\vk}^{2}})\frac{D_{1}^{2}-D_{2}^{2}}{4E_{\vk}^{2}\eta}
&-\frac{D_{1}^{2}}{4E_{\vk}^{2}}\frac{D_{1}^{2}-D_{2}^{2}}{4E_{\vk}^{2}\eta}&0&0\\
-\frac{D_{1}^{2}}{4E_{\vk}^{2}}\frac{D_{1}^{2}-D_{2}^{2}}{4E_{\vk}^{2}\eta}
&(1-\frac{D_{1}^{2}}{2E_{\vk}^{2}})\frac{D_{1}^{2}-D_{2}^{2}}{4E_{\vk}^{2}\eta}&0&0\\
0&0&\nth{2}(1+\frac{\xi_{\vk}}{E_{\vk}})\frac{3D_{1}^{2}+D_{2}^{2}}{2\eta(E_{\vk}+\xi_{\vk}+\eta)}&0\\
0&0&0&\nth{2}(1+\frac{\xi_{\vk}}{E_{\vk}})\frac{3D_{1}^{2}+D_{2}^{2}}{2\eta(E_{\vk}+\xi_{\vk}+\eta)}
\end{pmatrix}
\end{equation}

To be consistent, we also need to expand $L_{q}$ to the first order of  $D_{2}/\eta$ as well.  
\begin{equation}
\begin{split}
M^{(1b)}=&\qquad\frac{D_{2}}{\eta}\sum_{\vk}\\
&\begin{pmatrix}
-\frac{D_{1}^{2}D_{2}\xi_{\vk}}{4E^{5}_{\vk}}&-\frac{D_{1}^{2}D_{2}\xi_{\vk}}{2E^{5}_{\vk}}
&\frac{D_{1}\xi_{\vk}}{4E_{\vk}^{3}}&\frac{D_{1}}{2E_{\vk}}(\nth{E_{\vk}+\xi_{\vk}+\eta}+\nth{2E_{\vk}})\\
-\frac{D_{1}^{2}D_{2}\xi_{\vk}}{4E^{5}_{\vk}}&-\frac{D_{1}^{2}D_{2}\xi_{\vk}}{2E^{5}_{\vk}}
&\frac{D_{1}}{2E_{\vk}}(\nth{E_{\vk}+\xi_{\vk}+\eta}+\nth{2E_{\vk}})&\frac{D_{1}\xi_{\vk}}{4E_{\vk}^{3}}\\
\frac{D_{1}\xi_{\vk}}{4E_{\vk}^{3}}&\frac{D_{1}}{2E_{\vk}}(\nth{E_{\vk}+\xi_{\vk}+\eta}+\nth{2E_{\vk}})
&-\frac{D_{1}^{2}D_{2}}{4E_{\vk}^{3}(E_{\vk}+\xi_{\vk}+\eta)}&0\\
\frac{D_{1}}{2E_{\vk}}(\nth{E_{\vk}+\xi_{\vk}+\eta}+\nth{2E_{\vk}})&\frac{D_{1}\xi_{\vk}}{4E_{\vk}^{3}}
&0&-\frac{D_{1}^{2}D_{2}}{4E_{\vk}^{3}(E_{\vk}+\xi_{\vk}+\eta)}\\
\end{pmatrix}
\end{split}
\end{equation}
The interesting thing here is that if we only takes the first order of $D_{2}/\eta$, the two-channel is still decoupled when we write down the secular equation.  And the correction of the elements at the upper-left corner ($\theta_{1\,\pm{q}}$) is the same and that will gives an small finite value for $\omega_{q}$ at $q=0$. \emph{This conclusion is however problemetic as it violates the f sum rule.  Phase flucturation should be a Goldstone mode without mass.(see Sec. \ref{sec:phaseFluctuation}) }

\subsubsection{Alternative Approach: Phase fluctuation mode \label{sec:phaseFluctuation}}
 Action of $D$, $S(\bar{D_i},D_i)$ Eq. \eqref{eq:pathInt2:nG} and Eq. \eqref{eq:pathInt2:actionD}, is invariant over the phase change of $D_{1\,\vk}$ and $D_{2\,\vk}$ simultaneously. We therefore conclude that we should have a massless (Goldstone) mode.  And particularly, this mode is related to the local phase invariance. 
\begin{equation*}
D_{i}(x)\rightarrow{}D_{i}(x)e^{i2\theta(x)}\qquad{}
\bar{D}_{i}(x)\rightarrow{}\bar{D}_{i}(x)e^{-i2\theta(x)}
\end{equation*}
This corresponding to the translation of fermionic operator $\psi$
\begin{equation*}
\psi_{i}(x)\rightarrow{}\psi_{i}(x)e^{i\theta(x)}\qquad{}
\bar{\psi}_{i}(x)\rightarrow{}\bar{\psi}_{i}(x)e^{-i2\theta(x)}
\end{equation*}
Note that the phase $\theta(x)$ is common for the different components. The rest three degree of freedom of $D_i$, magnitude variation of two $D_i$ and internal phase between two $D_i$, change the action accordingly, while the overall phase $\theta(x)$ leaves action invariant.  We here isolate this special degree of freedom while leave others at their mean-field values. With a phase shift, we can rewrite the action (taken mean-field value $D^{(0)}$) (here we closely follow Nagaosa\cite{Nagaosa})
\begin{subequations}
\begin{align}\label{eq:pathInt2:actionPhase}
S[\theta,\bar\psi_{i},\psi_{i}]=&S_0[\bar\psi_{i},\psi_{i}]+S_1[\theta,\bar\psi_{i},\psi_{i}]+S_2[\theta,\bar\psi_{i},\psi_{i}]\\
S_0[\bar\psi_{i},\psi_{i}]=&\int{dx}
\Big\{\sum_{j}\bar\psi_{j}(\partial_\tau-\nth{2m}\nabla^{2}-\mu+\eta_{j})\psi_{j}\nonumber\\
&\quad+[D^{(0)}{}^{\dg}\tilde{U}^{-1}D^{(0)}-(\bar\psi\bar\psi)D^{(0)}-{D^{(0)}}{}^{\dg}(\psi\psi)\Big\}\\
S_1[\theta,\bar\psi_{i},\psi_{i}]=&\int{dx}\sum_{j}\Big\{
   i\,\bar\psi_{j}(\partial_{\tau}\theta)\psi_{j}+\nabla\theta\cdot\nth{2mi}[\bar\psi_{j}\nabla\psi_{j}-(\nabla\bar{\psi}_{j}\psi_{j})]\Big\}\\
S_2[\theta,\bar\psi_{i},\psi_{i}]=&\int{dx}\sum_{j}\nth{2m}(\nabla\theta)^{2}\bar\psi_{j}\psi_{j}
\end{align}
\end{subequations}
Note that here $D^{(0)}$ is a constant 2-component vector, no longer functional variables.  Here we see that $S_{0}$ has the same form as before except it only takes the mean field value of $D$ and it is described by the same correlation $G_{0}$ (Eq. \eqref{eq:pathInt2:G0}, \eqref{eq:pathInt2:nG}).  We can regard $S_{1}$ and $S_{2}$ as perturbation for the so-called gradient expansion on $\nabla\theta$.  It is then obvious that $S_{1}$ is in the first order while $S_{2}$ is in the second order.  Use the same spinor representation as before, $S$ is bilinear of $\psi$ and therefore we can formally integrate out $\psi$. 
\begin{equation}
S[\theta]=const.+\ln\det\nG(\theta)
\end{equation}
 We write out the formal Green function according to above action (with respect to Nambu-like spinor)
\begin{subequations}
\begin{align}
\nG=&G_{0}^{-1}+K_{1}+K_{2},\\
K_{1\, k,k'}=
	&\nth{({\beta{}V)}^{1/2}}(\omega_n-\omega_{n'})\theta(k-k')\sigma_3+
		\nth{{(\beta{}V)}^{1/2}}i\frac{(\vk-\vk')\cdot(\vk+\vk')}{2m}\theta(k-k')\hat{1}\\
K_{2\, k,k'}=
	&\nth{2m}\sum_{q,q'}\nth{{\beta{}V}}(\vq\cdot\vq')\theta(q)\theta(q')\delta(q+q'+k-k')\sigma_3
\end{align}
\end{subequations}
where $G_{0}$ is the same as (Eq. \eqref{eq:pathInt2:G0}, \eqref{eq:pathInt2:nG}).  And like in the single channel, 
\begin{equation}
\sigma_3=\mtrx{1&0&0\\0&-1&0\\0&0&-1}
\end{equation}
and $\hat{1}$ is $3\times3$ identity matrix.  As the expansion in Eq. \ref{eq:pathInt:expand}, we can look for the expansion of $\nG$ over $K_{1,2}$.  
\begin{equation}\tag{\ref{eq:pathInt:expand}}
\tr\ln \hat{G}^{-1}=\tr\ln\hat{G_{0}}^{-1}+\tr(\hat{G_{0}}\hat{K})-\nth{2}\tr(\hat{G_{0}}\hat{K}\hat{G_{0}}\hat{K})+\cdots
\end{equation}
For the first order, $\tr(\hat{G_{0}}\hat{K})$, 
\begin{align}
\tr(\hat{G_{0}}\hat{K_1})=&\sum_{k}{G_{0\,k}K_{1\,k,k}}=0\\
\tr(\hat{G_{0}}\hat{K_2})=&\sum_{k}{G_{0\,k}K_{2\,k,k}}\nonumber\\
	=&-\nth{2m}\nth{{\beta{}V}}\sum_{k}\tr(\hat{G}_{0\,k}\sigma_3)\sum_{q}q^2\theta{q}\theta{-q}\nonumber\\
	=&-\frac{n}{2m}\sum_{q}q^2\theta{(q)}\;\theta{(-q)}
\end{align}
Here we use the fact $\nth{{\beta{}V}}\sum_{k}\tr(\hat{G}_{0\,k}\sigma_3)=n$. $\tr(\hat{G_{0}}\hat{K_2})$ is already in the second order of $\theta$, and we only need to keep the expansion of $K_2$ to this order. On the other hand, we have to go to the second order of $K_1$ for the second order of $\theta$. 
\begin{align}		
\tr(\hat{G_{0}}{K_1}\hat{G_{0}}{K_1})=&\sum_{k,q}\tr(\hat{G}_{0,k+q}K_{1\,k+q,k}\hat{G}_{0\,k}K_{1\,k,k+q})\\
=&\nth{{\beta{}V}}\sum_{k,q}\theta(q)\theta(-q)\Big[(-\omega_m^2)\tr(\hat{G}_{0,k+q}\sigma_3\hat{G}_{0\,k}\sigma_3)\nonumber\\
&\quad+\nth{m^2}\sum_{i,j}q_iq_j(k_i+\frac{q_i}{2})(k_j+\frac{q_j}{2})\tr(\hat{G}_{0,k+q}\hat{G}_{0\,k})\Big]\nonumber\\
\equiv&\sum_{q}\theta(q)\theta(-q)\big[-\pi^{(0)}(q)\omega_m^2+\sum_{ij}\pi^{(\perp)}_{ij}(q)q_iq_j\big]
\end{align}
\emph{Here we only interested in the low-energy behavior of the mode and therefore we only use $\pi(0)$.} Use the previous result of Sec. \ref{sec:diagonalGreen}, we can calculate the Green's function in the lowest and first order of $D_i/\eta$ in $\pi$.  After some long but straigh-forward algebra (see Sec. \ref{sec:calculatePi}), we find
\begin{gather}
\pi^{(0)}(0)\approx\sum_{\vk}\frac{D_{1}^{2}}{E_{\vk}^{3}}-(\frac{D_{1}^{2}+D_{2}^{2}}{2\eta})\sum_{\vk}\frac{D_{1}^{2}}{E_{\vk}^{4}}\\
\pi^{(\perp)}(0)=0
\end{gather}
Combine all these together
\begin{equation}
S[\theta]=\int{dx}\sum_{q}\theta(q)\theta(-q)\big[\nth{2}\pi^{(0)}(q)\omega_m^2-\frac{n}{2m}q^2\big]
\end{equation}
And this determines the velocity of the Anderson-Bogoliubov collective mode.  We see that its behavior is qualitatively same as  single-channel with some correction in the order of $D_{i}/\eta$.



\begin{subappendices}
% !TeX root =thesis.tex


\section{Diagonalize Matrix Eq. (\ref{eq:pathInt2:G2})\label{sec:diagonalize}}
We need to find a unitary transformation $L$ to diagonalize matrix 
\begin{equation*}
i\omega_{n}I+\mtrx{-E_k&0&u _kD_2\\0&+E_k&v_kD_2\\u_kD_2&v_kD_2&+\xi_k+\eta}
\end{equation*}
We drops all the $k$ subscripts in this section.  We notice that the first term is proportional to identity matrix and does not change by unitary transformation, we only need to concentrate for the second term.  We rescale all elements with $E$, 
\begin{equation*}
R=
\begin{pmatrix}
-1&0&y_1\\
0&1&y_2\\
y_1&y_2&t
\end{pmatrix}
\end{equation*}
The secular equation is 
\begin{equation}\label{eq:pahtApp:secular}
(x^{2}-1)(x-t)-(y_{1}^{2}+y_{2}^{2})x+y_{1}^{2}-y_{2}^{2}=0
\end{equation}
We will assume at the zeroth order, the three eigenvalues are $-1$, $1$ and $t$.  ($t$ has weak dependency on energy as $(\xi_{k}+\eta)/E_{k}$, however, at the low energy region of interest, we ignore $\xi_{k}$.) Both $y_{1,2}$ and t are larger than 1, however, we will verify that given condition $y_{i}^{2}\ll{t}$, the correction is indeed small and the expansion is legit.(See Sec.\ref{sec:pathApp:consistency})  \emph{Indeed, this approximation is not as bad as it seems to be, close-channel component can still be smaller than open-channel at low-k (in order of $k_{F}$)  due to close-channel bound state is much smaller than inter-particle distance even when total close-channel is more than open-channel.  And here all the quantities are about low-k unless specifically noticed.} 
We expand the system to the first order of $y_{i}^{2}/{t}$, and find
\begin{equation}
\begin{array}{ccc}
x^{(0)}&\quad{}x^{(1)}&\quad{}Eigenvector\nonumber\\
-1&-\frac{y_{1}^{2}}{t}&\mtrx{1&\frac{y_{1}y_{2}}{2t}&-\frac{y_{1}}{t}}\\
1&-\frac{y_{2}^{2}}{t}&\mtrx{-\frac{y_{1}y_{2}}{2t}&1&-\frac{y_{2}}{t}}\\
t&\frac{y_{1}^{2}+y_{2}^{2}}{2t}&\mtrx{\frac{y_{1}}{t}&\frac{y_{2}}{t}&1}
\end{array}
\end{equation}
Now it is easy to write down the corresponding diagonal matrix and unitary transformation
\begin{equation}
B=i\omega_{n}I+E\mtrx{-1-\frac{y_{1}^{2}}{t}&0&0\\0&1-\frac{y_{2}^{2}}{t}&0\\0&0&t+\frac{y_{1}^{2}+y_{2}^{2}}{2t}}
\end{equation}
\begin{equation}
L=\mtrx{1&-\frac{y_{1}y_{2}}{2t}&\frac{y_{1}}{t}\\\frac{y_{1}y_{2}}{2t}&1&\frac{y_{2}}{t}\\-\frac{y_{1}}{t}&-\frac{y_{2}}{t}&1}
\end{equation}
Here $L$ is not exactly unitary transformation, it only unitary in the first order of  $y_{i}^{2}/{t}$ (or $D_{i}^{2}/(E\eta)$). We have 
\[
B=L^{\dg}RL+o(\frac{y_{i}^{2}}{t})
\]
Alternatively, we can write $L$ as 
\begin{equation}
L=I+
\mtrx{0&-\frac{D_{1}D_{2}}{4E^{2}}&u\\
\frac{D_{1}D_{2}}{4E^{2}}&0&v\\
-u&v&0
}\frac{D_{2}}{\eta}
\end{equation}
Here we use $uv=D_{1}/2E$


\section{Derive mean-field equation \eqref{eq:pathInt2:mf}\label{sec:pathInt2:deriveMF}}
For a $3\times3$ matrix as in Eq. (\ref{eq:pathInt2:nG}), 
\begin{equation}\tag{\ref{eq:pathInt2:nG}}
\mathcal{G}^{-1}=\begin{pmatrix}
i\omega_{n}-\xi_{k}&D_{1}&D_{2}\\
\bar{D}_{1}&i\omega_{n}+\xi_{k}&0\\
\bar{D}_{2}&0&i\omega_{n}+\xi_{k}+\eta
\end{pmatrix}
\end{equation}
A general $3\times3$ matrix inverted as such, 
  \begin{equation}
  \mtrx{A_{11}&A_{12}&A_{13}\\A_{12}^{*}&A_{22}&0\\A_{13}^{*}&0&A_{33}}^{-1}=
  \nth{|A|}
  \mtrx{A_{22}A_{33}&-A_{12}A_{33}&-A_{13}A_{22}\\
  	-A_{12}^{*}A_{33}&A_{11}A_{33}-A_{13}A_{13}^{*}&A_{12}^{*}A_{13}\\
	-A_{13}^{*}A_{22}&A_{12}A_{13}^{*}&A_{11}A_{22}-A_{12}A_{12}^{*}}
  \end{equation}
where $|A|$ is the determined of $A$ matrix.  Here we work in momentum space, in which the system is nicely decoupled at least to the mean-field order.  And we therefore drop all the $k$ subscript in the rest of section.  For Eq. (\ref{eq:pathInt2:nG}),
\begin{equation}
|A|=(i\omega_{n}-E_{1})(i\omega_{n}+E_{2})(i\omega_{n}+E_{3})
\end{equation}
Now we have $G_{0}$, and we can find the last term in \ref{eq:pathInt2:mf01}, 
\begin{equation}
\begin{split}
\tr\mbr{{G_{0}}\cdot\cmtrx{0&1&0\\0&0&0\\0&0&0}}&=\sum_{\vk\omega_{n}}G_{0\,(21)}\\
&=\sum_{\vk}\sum_{\omega_{n}}\frac{-D_{1}^{*}(i\omega_{n}+\xi+\eta)}{(i\omega_{n}-E_{1})(i\omega_{n}+E_{2})(i\omega_{n}+E_{3})}\\
&=\sum_{\vk}D_{1}^{*}\frac{E_{1}+\xi+\eta}{(E_{1}+E_{2})(E_{1}+E_{3})}\equiv\sum_{\vk}h_{1\,\vk}
\end{split}
\end{equation}
Here we perform Matsubara summation at zero temperature in the third equal sign with the normal trick (see sec. 4.2.1 in \cite{Altland}, sec. 25 in \cite{Fetter}, also refer to Footnote \ref{foot:intro:sum} at Page. \pageref{foot:intro:sum}).  We notice that within three roots, $E_{1}$, $-E_{2}$ and $-E_{3}$, we only need to take into account two negative roots $-E_{2}$ and $-E_{3}$, assuming the correction is small. 
Similarly
\begin{equation}
\begin{split}
\tr\mbr{{G_{0}}\cdot\cmtrx{0&0&1\\0&0&0\\0&0&0}}&=\sum_{\vk\omega_{n}}G_{0\,(31)}
=\sum_{\vk}D_{2}\frac{E_{1}+\xi}{(E_{1}+E_{2})(E_{1}+E_{3})}\equiv\sum_{\vk}h_{2\,\vk}
\end{split}
\end{equation}
And we have 
 \begin{align*}
(\tilde{U}^{-1})_{11}\bar{D}_{1}+(\tilde{U}^{-1})_{21}\bar{D}_{2}-\sum_{\vk}h_{1\,\vk}=0\\
(\tilde{U}^{-1})_{12}\bar{D}_{1}+(\tilde{U}^{-1})_{22}\bar{D}_{2}-\sum_{\vk}h_{2\,\vk}=0
 \end{align*}
Invert the interaction matrix $\tilde{U}$ and we have Eq.  \ref{eq:pathInt2:mf}.


\section{Wave function for short-range potential}\label{sec:pathInt2:short-range}
Here we discuss some possible generalization on the wave function for short-range potential.  This topic has been studied by Shizhong Zhang \cite{shizhongUniv}. We will use some similar ideas.  Outside the range $r_{c}$ of a short-range potential,  atom is free and  \sch equation is very simple.
\begin{equation}
-\frac{\hbar^{2}}{2m}\nabla^{2}\psi=E\psi
\end{equation}
The equation has a simple solution for s-wave, $\psi=A{e^{-\kappa{r}}}/{r}$ ($\kappa$ is imaginary for scattering state), here normalization $A$ is determined  by connecting it with the short-range part of the wave function, $\varphi_0$. 

Let us discuss the bound-state first, $\kappa>0$.  In the momentum space, there is also a universal behavior at low-momentum, where $kr_{c}\ll1$.   
\begin{equation*}
\psi_{k}=\nth{(2\pi)^{3/2}}\int{d\vr}(\varphi_{0}+A\frac{e^{-\kappa{r}}}{r})e^{-i\vk\cdot\vr}
\end{equation*}
The first part for $\varphi_{0}$ has very little $k$ dependence and the second terms give
\begin{equation*}
\psi_{k}=\varphi_{0\,k}+\nth{(2\pi)^{3/2}}\int{d\vr}(A\frac{e^{-\kappa{r}}}{r})e^{-i\vk\cdot\vr}=\varphi_{0\,k}-(\frac{2}{\pi})^{1/2}A\nth{k^{2}+\kappa^{2}}
\end{equation*}

Furthermore, if the bound-state is the one close to threshold, the most weight is outside $r_{c}$, we can neglect the first term and we have universal behavior at low-momentum while the normalization is determined in two-body problem.   Besides  bound-state, if interaction is weak and short-range, the low energy scattering state is well described by s-wave scattering state $\psi\propto1/r-1/a$ (Eq. \ref{eq:intro:Bethe}), and its Fourier transform in momentum space has the similar form $1/k^{2}$.  When considering many-body physics, in the low momentum below or around the scale of Fermi momentum,  wave function  is modified by many-body effect; but in the medium momentum, (still much smaller than $1/r_{c}$), this universal behavior is preserved.  The distribution of particle in such momentum, $k_{F}\ll{k}\ll{1/r_{c}}$, is $1/k^{4}$. This is actually the ``high-momentum'' (medium here) behavior ($C/k^{4}$) described in Tan's work about universality\cite{Tan2008-1,Tan2008-2}. 

On the other hand, at very high momentum ($k\gg1/r_{c}$), the second term in the above is very small.  This is because the smooth tail part of the wave function contributes little in high-oscillation.  The high-frequency Fourier component is solely determined by the wave function within the potential range.   This can be extend beyond two-body wave function to two-body correlation as long as the long-wave-length part is smooth.  In all cases, two-body, or many-body, very high-frequency of two-body correlation follows the two-body wave function.  





\section{$\lambda_{1}$ and $\lambda_{2}$\label{sec:pathInt2:lambda}}
\begin{equation}\tag{\ref{eq:pathInt2:lambda1}}
\lambda_{1}\equiv\avs{\phi}{(E_{\vk}-\xi_{\vk})}{\phi}-\avs{\phi}{v_{\vk}^{2}V}{\phi}
\end{equation}
Use relationship $v_{k}=\frac{E_{\vk}-\xi_{\vk}}{2E_{\vk}}$, we can rewrite the above equation into 
\begin{equation*}
\lambda_{1}=\avs{\phi}{v_{\vk}^{2}(2E_{\vk}-V)}{\phi}
\end{equation*}
$v_{\vk}$ is close to zero when momentum much higher Fermi momentum, on this region, $E_{\vk}$ is much smaller comparing to potential energy $V$.  So $V\ket{\phi}\approx-\eta\ket{\phi}$, and $\phi\approx\frac{A}{\kappa^{2}}$Therefore, we can estimate this term 
\begin{equation}
\lambda_{1}=\frac{A^{2}}{\kappa^{2}}\sum{}v_{\vk}^{2}=n\frac{A^{2}}{\kappa^{2}}\sim{}D_{1}^{2}/\eta
\end{equation}







\begin{equation}\tag{\ref{eq:pathInt2:lambda2}}
\lambda_{2}=(1+GT)\frac{Y\ket{\phi}\bra{\phi}{Y}}{\br{-E_{b}+\eta-2\mu-\lambda_{1}}}v_{\vk}^{2}\ket{{h_{1}}}
\end{equation}
Here the argument is more or less the same as in $\lambda_{1}$, considering the short-range nature of $Y$, and $v_{k}^{2}$ introduces an extra $nA^{2}/\kappa^{2}\sim{}D_{1}^{2}/\eta$ factor.  

We can see, both of them depends on many-body effect through density.  Particularly, they depend on density of open-channel component linearly at the lowest order.  $\lambda_{i}(n_o)=\lambda_{i}^{(0)}n_o$.  When not too close to resonance, $n_o$ is close to total density.  And $\lambda_{i}$ can be measured at different densities,  $\lambda_{i}^{(0)}$ estimated then accordingly. 

In the replacement of $1/(E_{1\vk}+E_{3\vk})$ by $\phi_{k}$ in Eq. \ref{eq:pathInt2:hphi}, certain error is introduced by directly replacement.  We expect the error is in higher order.  They might lead a non-linear relationship between $\lambda_{i}$ and density $n$.  But we expect the non-linearity is weak and we can still include the Pauli exclusion in these two parameters $\lambda_{i}(n)$.




\section{Calculate $\pi^{(0)}$ and $\pi^{\perp}$\label{sec:calculatePi}}
Here we will calculate $\pi^{(0)}$ and $\pi^{\perp}$ to the first order of $D_i/\eta$ using the expansion of Green's function in Sec. \ref{sec:diagonalGreen}.

\begin{equation}\label{eq:pathInt2:pi0}
\begin{split}
\pi^{(0)}(0)=&\sum_k\tr(\hat{G}_{0\,k}\sigma_3\hat{G}_{0\,k}\sigma_3)\\
	\approx&\sum_k\tr\big(T_{\vk}B_{k}^{-1}T_{\vk}^{\dg}\sigma_3T_{\vk}B_{k}^{-1}T_{\vk}^{\dg}\sigma_3\big)\\
	&\quad+\tr\Big(T_{\vk}\delta_{\vk}B_{k}^{-1}T_{\vk}^{\dg}\sigma_3T_{\vk}B_{k}^{-1}T_{\vk}^{\dg}\sigma_3
	-T_{\vk}B_{k}^{-1}\delta_{\vk}T_{\vk}^{\dg}\sigma_3T_{\vk}B_{k}^{-1}T_{\vk}^{\dg}\sigma_3\\
	&\qquad+T_{\vk}B_{k}^{-1}T_{\vk}^{\dg}\sigma_3T_{\vk}\delta_{\vk}B_{k}^{-1}T_{\vk}^{\dg}\sigma_3
	-T_{\vk}B_{k}^{-1}T_{\vk}^{\dg}\sigma_3T_{\vk}B_{k}^{-1}\delta_{\vk}T_{\vk}^{\dg}\sigma_3\Big)\\
	=&\sum_k\tr\big(M_{k}\big)+2\tr\Big(\delta_{\vk}M_{k}-\delta_{\vk}M_{k}\Big)
\end{split}
\end{equation}
where 
\begin{equation}
M_{k}=T_{\vk}^{\dg}\sigma_3T_{\vk}B_{k}^{-1}T_{\vk}^{\dg}\sigma_3T_{\vk}B_{k}^{-1}
\end{equation}
Here we use the cyclical  property of the trace $\tr(AB)=\tr(BA)$.  
It is straightforward to calculate
\begin{equation*}
T_{\vk}^{\dg}\sigma_3T_{\vk}B_{k}^{-1}=
\begin{pmatrix}
{\frac{\xi_{\vk}}{E_{\vk}(i\omega_{k}-E_{1\,\vk})}}&\frac{D_{1}}{E_{\vk}(i\omega_{k}+E_{2\,\vk})}&0\\
{\frac{D_{1}}{E_{\vk}(i\omega_{k}-E_{1\,\vk})}}&-\frac{\xi_{\vk}}{E_{\vk}(i\omega_{k}+E_{2\,\vk})}&0\\
0&0&-\nth{i\omega_{k}+E_{3\,\vk}}\\
\end{pmatrix}
\end{equation*}
Now it is easy to calculate the first term
\begin{equation}
\begin{split}
\sum_k\tr\big(M_{k}\big)=&
\sum_{k}\mbr{
\frac{2D_{1}^{2}}{E_{\vk}^{2}(i\omega_{k}-E_{1\,\vk})(i\omega_{k}+\xi_{2\,\vk})}+
\br{\frac{\xi_{\vk}^{2}}{E_{\vk}^{2}(i\omega_{k}-E_{1\,\vk})^{2}}+\frac{\xi_{\vk}^{2}}{E_{\vk}^{2}(i\omega_{k}+E_{1\,\vk})^{2}}
-\nth{(i\omega_{k}+E_{3\,\vk})^{2}}}
}
\end{split}
\end{equation}
Only root $-\xi_{2\,\vk}$ in the first term contributes in Matrubara frequency summation.
\begin{equation}\label{eq:pathInt2:pi0-1}
\sum_k\tr\big(M_{k}\big)=\sum_{\vk}\frac{2D_{1}^{2}}{E_{\vk}^{2}(E_{1\,\vk}+\xi_{2\,\vk})}
\approx\sum_{\vk}\frac{D_{1}^{2}}{E_{\vk}^{3}}-\sum_{\vk}\frac{D_{1}^{2}D_{2}^{2}\xi_{\vk}}{2E_{\vk}^{5}(\xi_{\vk}+\eta)}
\end{equation}

For the lowest order of the second term in Eq. \eqref{eq:pathInt2:pi0}, we only need to take the lowest order of $B_{k}$
\begin{equation}
B_{k}=\mtrx{i\omega_{k}-E_{\vk}&0&0\\0&i\omega_{k}+E_{\vk}&0\\0&0&i\omega_{k}+\xi_{\vk}+\eta}
\end{equation}
It is easy to verify at this approximation
\begin{equation}\label{eq:pathInt2:pi0-2}
\tr\Big(\delta_{\vk}M_{k}-\delta_{\vk}M_{k}\Big)=0
\end{equation}
Combine Eq. \eqref{eq:pathInt2:pi0-1} and Eq. \eqref{eq:pathInt2:pi0-2}, we have 
\begin{equation}
\pi^{(0)}(0)\approx\sum_{\vk}\frac{D_{1}^{2}}{E_{\vk}^{3}}-\sum_{\vk}\frac{D_{1}^{2}D_{2}^{2}\xi_{\vk}}{2E_{\vk}^{5}(\xi_{\vk}+\eta)}\end{equation}

and it is actually exact for $\pi^{\perp}(0)$
\begin{equation}
\begin{split}
\pi^{\perp}(0)=&\sum_k\tr(\hat{G}_{0\,k}\hat{G}_{0\,k})\\
	=&\sum_k\tr\big(T_{\vk}L_{\vk}B_{k}^{-1}L_{\vk}^{\dg}T_{\vk}^{\dg}T_{\vk}L_{\vk}B_{k}^{-1}L_{\vk}^{\dg}T_{\vk}^{\dg}\big)\\
	=&\sum_k\tr\big(B_{k}^{-1}B_{k}^{-1}\big)\\
	=&\sum_{\vk,i}(\sum_{\omega_{k}}(i\omega_{k}-\xi_{i})^{-2})\\
	=&0
\end{split}
\end{equation}


\section{Check for consistency of expansion\label{sec:pathApp:consistency}}
In our treatment here, one crucial assumption in expansion is the smallness of $D_{2}/\eta$.  Here we check it.  We have 
\begin{equation}
D_{2}=\sum{}Yh_{1\vk}+\sum{}Vh_{2\vk}\tag{\ref{eq:pathInt2:mfclose}}
\end{equation}
The first term on the right is relatively small comparing to the second term.  In an estimation we just keep the second term.  Furthermore,  we assume $h_{2\,\vk}=\sqrt{N_{c}}\phi_{0\,\vk}$, where $\phi_{0\,\vk}$ is the normalized wave function of isolated close-channel potential satisfying \sch equation
\begin{equation}\tag{\ref{eq:pathInt2:phi}}
-E_{b}^{(0)}\phi_{0\,\vp}=\epsilon_{\vp}\phi_{0\,\vp}-\sum_{\vk}V \phi_{0\,\vk}
\end{equation}
rearrange it we have (especially at low-momentum)
\begin{equation*}
\sum_{\vk}V \phi_{0\,\vk}=(\epsilon_{\vp}+E_{b})\phi_{0\,\vp}\approx{\eta}\phi_{0\,\vp}
\end{equation*}
The second approximation is correct at low-momentum (smaller or in the same order of Fermi momentum) as $\epsilon_{\vp}\ll{}E_{b}\approx\eta$.  Put all these together, we have
\begin{equation*}
D_{2}\approx\alpha{}E_{b}\phi
\end{equation*}
If we assume a simple exponentially decayed wave function
\begin{equation*}
\phi_{\vk}=\sqrt{\frac{8\pi\kappa}{V_{0}}}\frac{1}{k^{2}+\kappa^{2}}\approx\sqrt{\frac{8\pi\kappa}{V_{0}}}\frac{1}{\kappa^{2}}
\end{equation*}
Here  $V_{0}$ is the total volume.  The second approximation is for low-momentum as previous equation.  Collect all these together, we have
\begin{equation}
D_{2}\approx\sqrt{N_{c}}\eta\sqrt{\frac{8\pi\kappa}{V_{0}}}\frac{1}{\kappa^{2}}
\sim\eta\sqrt\frac{n_{c}}{\kappa^{3}}
\sim\eta\br{\frac{k_{Fc}}{\kappa}}^{\frac{3}{2}}
\sim\eta\br{\frac{E_{Fc}}{\eta}}^{\frac{3}{4}}
\end{equation}
%:
$k_{Fc}$ is the Fermi momentum corresponding to the particles in close-channel, which is much smaller than the characteristic momentum for bound-state, $\kappa$.   Therefore we have $D_{2}\ll\eta$, even when $n_{c}$ is close to total density $n$. 

Now we check the correction in Fermionic spectrum (Eq. \ref{eq:pathInt2:xiExpand3}-\ref{eq:pathInt2:xiExpand3}), 
is indeed small comparing to the main term.  
\begin{align}\tag{\ref{eq:pathInt2:xiExpand}}
E_{1\vk}&\approx{}E_{\vk}+\frac{D_{2}^{2}u_{\vk}^{2}}{\xi_{\vk}+\eta}&\equiv{}&E_{\vk}+\gamma_{1\vk}\\
\xi_{2\vk}&\approx{}E_{\vk}-\frac{D_{2}^{2}v_{\vk}^{2}}{\xi_{\vk}+\eta}&\equiv{}&E_{\vk}+\gamma_{2\vk}
\tag{\ref{eq:pathInt2:xiExpand2}}\\
E_{3\vk}&\approx{}\xi_{\vk}+\eta-\frac{D_{2}^{2}}{2(\xi_{\vk}+\eta)}&\equiv{}&\xi_{\vk}+\eta+\gamma_{3\vk}
\tag{\ref{eq:pathInt2:xiExpand3}}
\end{align}
Here, we mostly only concern of low-momentum ($k\sim{}k_{F}$).  In Eq. \ref{eq:pathInt2:xiExpand3}, 
\begin{equation*}
\frac{\gamma_{3\vk}}{E_{3\vk}}\sim{}\frac{D_{2}^{2}}{\eta^{2}}\sim\br{\frac{k_{Fc}}{\kappa}}^{2}\ll1
\end{equation*}
Eqs. \ref{eq:pathInt2:xiExpand} and Eqs. \ref{eq:pathInt2:xiExpand2} is slightly more complicated.  Both of them involve $\frac{D_{2}^{2}}{E_{\vk}\eta}$,  at very BCS side, close-channel density is small, $k_{F\,c}$ is small and that makes this ratio small; when close to (narrow) resonance, where $n_{c}$ is comparable to to total density, at low energy, $D_{1}$ is in the order of Fermi energy, so does $E_{\vk}$.   We have (we no longer distinguish $k_{F\,c}$ with $k_{F}$)
 \begin{equation*}
 \frac{\gamma_{i}}{\xi_{i}}<\frac{D_{2}^{2}}{E_{\vk}\eta}\sim\frac{\eta^{2}\frac{k_{Fc}^{3}}{\kappa^{3}}}{k_{F}^{2}\eta}\sim\frac{k_{F}}{\kappa}\ll1
\end{equation*}

More deeply, in the secular equation that leads to spectrum, Eq. \ref{eq:pahtApp:secular}.  We convert it back to normal scale without $E_{\vk}$,  (We drop subscript $\vk$ in the following equations for simplicity)
\begin{equation*}
(x^{2}-E^{2})(x-\xi-\eta)-D_{2}^{2}x+D_{2}^{2}E(u^{2}-v^{2})=0
\end{equation*}
It is not hard to use definition of $u$ and $v$ to find $u^{2}-v^{2}=\xi/E$, and express $E^{2}=\xi^{2}+D_{1}^{2}$, therefore we have
\begin{equation*}
(x-\xi)(x+\xi)(x-\xi-\eta)-D_{1}^{2}(x-\xi-\eta)-D_{2}^{2}(x-\xi)=0
\end{equation*}
Here the first term is for free particles, and let us estimate the size of the last two terms.  For low-momentum solution, we simply use $D_{1}\sim{}E_{F}$, we find
\begin{equation*}
\frac{D_{1}^{2}(x-\xi-\eta)}{D_{2}^{2}(x-\xi)}\sim\frac{E_{F}^{2}\eta}{D_{2}^{2}E_{F}}\sim\frac{\kappa}{k_{F}}\gg1
\end{equation*}
This justifies our choice to neglect the last term when finding the lowest-order solution and use the last term for correction.  

 



 In another word, the above estimation is saying that the total occupation number of close-channel at low-momentum is much smaller than 1 in all region of resonance (narrow or broad) because the close-channel bound-state is much smaller than inter-particle distance.  This factor gives us a small factor, $\frac{r_{c}}{a_{0}}\sim\frac{k_{F}}{\kappa}\sim\sqrt\frac{E_{F}}{\eta}$, upon which we can do the expansion.  

\end{subappendices}






