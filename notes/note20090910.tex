\section{note of 2009.09.10}
\subsection{}
If we start from real space of pair wave function.  There is no obvious $\Delta$, i.e. gap.  Or we can define it as the $\sum{{U_{\vk,\vk'}}F_\vk'}$  and it is relative slowly-varied in k-space because of the slow-change of $U_{\vk,\vk'}$.

\subsection{Pair-Wave-Function in Real Space}

How should pair wave function $F(r)$ looks like?  Let us look in two extremes. In BCS, it falls exponentially  with Piccard Coherence length (Eq. 2.62 in \cite{ZhangThesis},\cite{Leggett}) which should be larger than $1/\vk_F$; while in BEC, it also falls exponentially but with a  characteristic length  smaller than $1/\vk_F$.  And in principle, between potential range $r_0$ and this characteristic length, or $1/\vk_F$ if it is smaller, it has the s-wave scattering like wave-function, $\nth{r}\br{1-\frac{r}{a_s}}$ charactered by $a_s$, which is determined by two-body physics.   

\subsection{What is the Difference between the Many-body Motion of Equation and the Two-Body Schr\"odinger Equation\label{subsec:twoR}}
The equation 5.37 in \cite{ZhangThesis} is the same as the two-body equation.  We know in k-space, the difference is that the Fermi sea is filled for many-body case; in real space, two-body limit means that density approaches 0, so many-body means that potential is modified in the $1/\vk_F$ scale and beyond. So maybe we can breaks the length scale by $\lvk$.  For equation 5.36 \eqref{eq:zhang536} in \cite{ZhangThesis}, we can throw away those two four-operator terms for $r<\lvk$ and must keep them for $r>\lvk$.  For $r>\lvk$, we can choose $r'$ within the range of either $r_1$ or $r_2$, and this might be sufficient to include the many-body physics into the problem.

In Abrikosov's Green function treatment over BCS, it is exactly these four-operator terms work toward the gap equation.  By breaking r into two regions, I might reconcile the two-body solution with the many-body one.  

This complication however prevents a simple Fourier transformation of the equations as there are two different equations for different regions of r.  

One obvious problem is whether the original Abrikosov treatment can be solved directly in real space and therefore can be used here.  The short r region certainly can be solved in real space and real space indeed is more suitable for solving.  But the long r region, i.e. the many-body, is easier to solve in k-space.  

One possible method may be using the Psudopotential $\Delta$ instead, just as in Bogolubov-deGennes equation.  

One odd thing about Shizhong's treatment is this two-operator term, that emerges only because of the equal-time of $\Psi_\alpha(t)\Psi_\beta(t)$.  They do not show in Abrikosov's treatment and needs more careful investigation.  

\subsection{Meet Tony}
Tony has some doubts on Abrikosov's treatment over BCS (\cite{Abrikosov}).  He is not convinced by their throwing the Coulomb interaction and other terms.  But this is also true in the variation method.   Anyway, he agrees that the four-operators thrown away by Shizhong might be important for my problem.  I should explore my idea in \ref{subsec:twoR}. 

My goal is to find out when we take the Pauli principle in the close-channel, open-channel due to the common species seriously, what kind of correction, no matter how small it is, I can get.  In most approach, Pauli principle does not matter as close-channel is treated as independently condensate and no Pauli principle on this.  
