\section{note of 2009.09.30}
\subsection{Connetion between Two-Body Wave Fucntion and Many-Body Wave Function}
I talked with Shizhong about how he reduce the order parameter in the s-wave scattering wave function in short-distance at real space. (equation 2.61 in \cite{ZhangThesis})  He does not get the $a_s$ directly from 2.60, but by split $\nth{E_k}=\br{\nth{E_k}-\nth{\xi_k}}+\nth{\xi_k}$ and the first difference term is in order of $\nth{k^6}$ for large $k$ and gives negligible contribution in integral. We can neglect the large $k$ part, approximate $\sin{kr}$ in the nominator as $kr$ and that gives us $k^2r$ in the nominator, which is the same as the the normalized gap equation. By comparing with the gap equation, we have $-2r/\pi{a_s}$ and the second part $\int\frac{k\sin{kr}}{\xi_k}dk=\frac{2m}{\hbar^2}\frac{\pi}{2}$.  It does have a $2/\pi$ difference,(maybe just my calculation mistake somewhere), but the structure is correct.  However, you cannot get $a_s$ by comparing with the normalized gap equation.  On the other hand, even without that, it still suggests the correct short-range behavior $\nth{r}\br{1-\frac{r}{\widetilde{a_s}}}$.  Presumably, $F(r)$ is a solution valid for all the region, and the above arguments demonstrates that it reduces to the correct short-range form where long-term effects, i.e. the many-body effects, can be neglected. And it gives the normalization factor which cannot be obtained by solving short-range version of equation.  

So the structure of $F(r)$ is correct, and the factor happens to be $a_s$ because the gap equation and that comes from the two-body description of $T$ matrix. On the other hand, if working backward, we can identify the gap equation and $a_s$ by reasoning that the short-range behavior of $F(r)$ must be proportional to the two-body solution. 

Related to this, the question to ask is what type of the many-body solution reduces to the two-body solution for the Feshbach resonance, especially, the close channel, as exponentially decaying solution, $e^{-r/a^{(c)}}$.  Considering that $a^{(c)}$ is much smaller than the open-channel scattering length $a_s$, the region where it can be treated in the form of $1-r/a$ is very small, $r{\ll}a^{(c)}$; while the two-body solution is expected to be correct for the region much larger $r{\ll}1/k_F$.  But we need to find out how is $F(r)$ and how is $F(k,t=0)$.  The formal solution of $F(k,\omega)$ is in principle available although this needs solving third-order equation and probably only hopeless complicated analytic solution, and it needs to be integrated over both $k$ and $\omega$.  

\section{Bare Gap Equation}
%\tableofcontents
We start from a general Hamiltonian, (As usual, the particle number is relaxed and we use grand ensemble.)
\begin{equation}\label{eq:Hamiltonian}
\begin{split}
H-\mu N&=\sum_\alpha\int{d}\vr\psi^\dagger_\alpha\br{\vr}\br{-\hm\nabla^2-\mu_\alpha+\eta_\alpha}\psi_\alpha\br{\vr}\\
&+\nth{2}\sum_{\alpha\beta\gamma\delta}\int{d}\vr_1{d}\vr_2\psi^\dagger_\alpha\br{\vr_1}\psi^\dagger_\beta\br{\vr_2}U_{\alpha\beta\gamma\delta}\br{\vr_1-\vr_2}\psi_\gamma\br{\vr_2}\psi_\delta\br{\vr_1}
\end{split}
\end{equation}
where $\eta_\alpha$ is the Zeeman energy of species $\alpha$ and $\mu_\alpha$ is chemical potential. In the three-species two-channel problem, $\alpha=(a,b,c)$, where $(a,b)$ is the open channel and $(c,b)$ is the close channel. There are four different interaction terms,  $U_{abba}$, $V_{cbbc}$ and $Y_{abbc}$ ($Y^*_{cbba}$). As $a$ and $c$ can interchange, their chemical potential is the same, $\mu_a=\mu_c$ Here we work with the finite-temperature Matsubara formalism. We can define the normal and abnormal Green's functions.
\begin{equation}
\begin{split}
G_\alpha(\vr_1-\vr_2,\tau_1-\tau_2)&=-\avtu{\psi_\alpha(\vr_1,\tau_1)\psi^+_\alpha(\vr_2,\tau_2)}\\
	&=-\avtu{\psi_\alpha(\vr_1-\vr_2,\tau_1-\tau_2)\psi^+_\alpha(0,0)}\\
F^+(\vr_1-\vr_2,\tau_1-\tau_2)&=-\avtu{\psi^+_a(\vr_1,\tau_1)\psi^+_b(\vr_2,\tau_2)}\\
R^+(\vr_1-\vr_2,\tau_1-\tau_2)&=-\avtu{\psi^+_c(\vr_1,\tau_1)\psi^+_b(\vr_2,\tau_2)}	
\end{split}
\end{equation}
Using the relation $\pdiff{}{\tau}\chi(\tau)=\mbr{H-\mu N,\chi(\tau)}$,  we can obtain their equations of motion
\begin{multline}
\br{-\pdiff{}{\tau}+\hm\nabla^2+\mu_b}G_b(x,\tau)-\nth2\int{dr}\mbr{U(r-x)F(r-x,0)+Y(r-x)R(r-x,0)}F^+(r,\tau)\\
	-\nth2\int{dr}\mbr{V(r-x)R(r-x,0)+Y^*(r-x)F(r-x,0)}R^+(r,\tau)=\delta(\tau)\delta(x)\\
\br{-\pdiff{}{\tau}-\hm\nabla^2-\mu_a}F^+(x,\tau)-\nth2\int{dr}\mbr{U(r-x)F(r-x,0)+Y(r-x)R(r-x,0)}G_b(r,\tau)=0\\
\br{-\pdiff{}{\tau}-\hm\nabla^2-\mu_a+\eta}R^+(x,\tau)-\nth2\int{dr}\mbr{V(r-x)R(r-x,0)+Y^*(r-x)F(r-x,0)}G_b(r,\tau)=0	
\end{multline}
Here we take $\eta_a=\eta_b=0$ and $\eta_c=\eta$.  As in the normal BCS approach, we neglect Hatree terms as $\int{dr}U(x-r)n(r)F(x,\tau)$ because such terms can be absorbed into the chemical potential.  These equations can be solved with Fourier transformation
\begin{align*}
f(x,\tau)&=\nth{\beta}\sum{}e^{-i\tau{p_n}}\int{dk}\frac{V}{(2\pi)^3}e^{ikx}f(k,ip_n)\\
f(k,ip_n)&=\int^\beta_0e^{i\tau{p_n}}\int{dx}e^{-ikx}f(x,\tau)	
\end{align*}
\begin{equation}\label{eq:eomk}
\begin{split}
\br{ip_n-\hm{k^2}+\mu_b}G_b(k,ip_n)+{\Delta^F_{-k}}^*F^+(k,ip_n)+{\Delta^R_{-k}}^*R^+(k,ip_n)&=1\\
\br{ip_n+\hm{k^2}-\mu_a}F^+(k,ip_n)+\Delta^F_{k}G_b(k,ip_n)&=0\\
\br{ip_n+\hm{k^2}+\eta-\mu_a}R^+(k,ip_n)+\Delta^R_{k}G_b(k,ip_n)&=0
\end{split}
\end{equation} 
where 
\begin{equation}\begin{split}\label{eq:gap}
\Delta^F_k&=-\nth2\int{dx}e^{-ikx}\mbr{U(x)F^+(x,\tau=0)+Y^*(x)R^+(x,\tau=0)}\\
	&=-\nth2\sum\mbr{U(p+k)F^+(-p,\tau=0)+Y^*(p+k)R^+(-p,\tau=0)}\\
\Delta^R_k&=-\nth2\int{dx}e^{-ikx}\mbr{V(x)R^+(x,\tau=0)+Y(x)F^+(x,\tau=0)}\\
	&=-\nth2\sum\mbr{V(p+k)R^+(-p,\tau=0)+Y(p+k)F^+(-p,\tau=0)}
\end{split}\end{equation}
Equation \eqref{eq:eomk} is a simple algebra equation and can be solved easily.  Here combine the kinetic energy with the chemical potential as $\xi_k=\hm{k^2}-\mu$.
\begin{equation}\begin{split}
G_b(k,ip_n)&=\nth{A_k(ip_n)}\br{ip_n+\xi^a_k}\br{ip_n+\xi^a_k+\eta}\\
F^+(k,ip_n)&=-\nth{A_k(ip_n)}\Delta^F_k\br{ip_n+\xi^a_k+\eta}\\
R^+(k,ip_n)&=-\nth{A_k(ip_n)}\Delta^R_k\br{ip_n+\xi^a_k}
\end{split}\end{equation}
where 
\begin{equation}\label{eq:A}
A_k(ip_n)=\br{ip_n-\xi^b_k}\br{ip_n+\xi^a_k}\br{ip_n+\xi^a_k+\eta}-{\Delta^F_k}^2\br{ip_n+\xi^a_k+\eta}-{\Delta^R_k}^2\br{ip_n+\xi^a_k}
\end{equation}
Now the root for $A(z)=0$ gives the pole of Green's function and the excitation spectrum of the system.      This equation does have analytic solution as a third order algebraic equation. There are three different roots, corresponding three different types of excitation? Unlike in single-channel problem, where there is second-order equation and two roots degenerate ($\pm{E_k}$). However the roots are extremely tedious and serves little practical usage.  One simple observation is that for large $k$, the last two terms in \eqref{eq:A} can be neglected and therefore the pole is close to $\xi^a$,$-\xi^b$ (degenerate with $\xi^a$), and $\xi^c=\eta+\xi^a$.  In principle, we can sum the solution over $p_n$ to get the quantities at $\tau=0$ which is in the gap equation \eqref{eq:gap}. 
\[F(k,\tau=0)=\nth{\beta}\sum_{p_n}F(k,ip_n)\] And the poles of $A(z)=0$ determines this quantity.
\footnote{
$F(k,\tau=0)$ summation can be found using $\oint{F(k,z)n_F(z)\frac{dz}{2\pi{i}}}=0$, where $n_F(z)=\nth{e^{\beta{z}}+1}$ is the Fermi distribution function with poles $p_n=i(2n+1)\pi/\beta$. The residual associated with the poles of $F(k,z)$ gives $F(k,\tau=0)$.  
} 

\section{Renormalization with Two-Channel T-Matrix}
One observation is that both abnormal Green's functions approach ${\Delta}/{2\epsilon}$ for large $k$ ($\epsilon=\hm{k^2}$).  This fact makes it possible to renormalize the gap equation in the similar fashion as in \cite{Leggett}.  

Formally, the gap equation \eqref{eq:gap} can be written in matrix form
\begin{equation}\label{eq:gapMatrix}
\begin{pmatrix}\Delta^F\\\Delta^R\end{pmatrix}=\begin{pmatrix}U&Y\\Y^*&V\end{pmatrix}\begin{pmatrix}F\\R\end{pmatrix}
\end{equation}
 or in a more compact way, 
\begin{equation}\label{eq:mgap}\Delta=UF\end{equation}
Note that there is really summation over $k$ between U and F.
For two-body system, we have 
\begin{equation}\label{eq:TU}T=U+TGU=(1+TG)U\end{equation}
notice $T=\br{\begin{smallmatrix}T^{oo}&T^{oc}\\T^{co}&T^{cc}\end{smallmatrix}},U,G=\br{\begin{smallmatrix}1/2\epsilon&0\\0&1/(2\epsilon+\eta)\end{smallmatrix}}$ are $2\times2$ matrices. Apply $(1+TG)$ on both side of gap equation \eqref{eq:mgap}.
\[(1+TG)\Delta=TF\]
and 
\begin{equation}
\Delta=T(F-G\Delta)
\end{equation}
or in more elaborate form
\begin{equation}\label{eq:renomal}
\begin{pmatrix}\Delta^F\\\Delta^R\end{pmatrix}=\begin{pmatrix}T^{oo}&T^{oc}\\T^{co}&T^{cc}\end{pmatrix}
\begin{pmatrix}\sum{F-\frac{\Delta^F}{2\epsilon}}\\\sum{R--\frac{\Delta^R}{2\epsilon+\eta}}\end{pmatrix}
\end{equation}
Now the summation of $k$ is inside the last vector. T matrix is just four numbers which can be related to 0-energy scattering, therefore experiments.   And the summation has no divergence at high k.  

%The equation obtained previously does not converges for a constant contact-type interaction.  However, it can be renormalized in the similar fashion as in \cite{Leggett}. 

Several problems present here 
\begin{itemize}
	\item How is T-matrix related to the familiar quantities in literature?  $T^{oo}$ is probably directly related to open-channel s-wave scattering length $a_s$.  More subtle is $T^{co}$, $T^{oc}$.  How does that related the observed quantities?
	
	\item Although equation \eqref{eq:renomal} is well-behaving without any divergence, F(,R) lacks the simple form in the single channel case.  How to simplify it is not clear.  
	\item So far, two channels are treated equally.  However, two channels are quite different in reality, especially the close-channel.  In two-body problem, it can be treated as only 1-dimensional, just the bound-state close to threshold. One only needs to determine its normalization.  How to consider that in the many-body solution and simplify the gap equation? 
	\item The high-k is related to the short-range behavior of the system.  And the free-particle type $k$ corresponding the free particles, not include the feature of the short-range bound state feature of the close-channel.  That should be remedy somehow to simplify the computation and make progress. 
\end{itemize}

